%% uctest.tex 11/3/94
%% Copyright (C) 1988-2004 Daniel Gildea, BBF, Ethan Munson.
%
% This work may be distributed and/or modified under the
% conditions of the LaTeX Project Public License, either version 1.3
% of this license or (at your option) any later version.
% The latest version of this license is in
%   http://www.latex-project.org/lppl.txt
% and version 1.3 or later is part of all distributions of LaTeX
% version 2003/12/01 or later.
%
% This work has the LPPL maintenance status "maintained".
% 
% The Current Maintainer of this work is Daniel Gildea.
\newcommand*{\ATLASLATEXPATH}{/Users/sheenaschier/Library/TexShop/texmf/tex/latex/atlaslatex-01-07-01/latex/}
\documentclass[11pt]{ucthesis}
\def\dsp{\def\baselinestretch{2.0}\large\normalsize}
\dsp
\usepackage{\ATLASLATEXPATH atlasphysics}
\usepackage{graphicx}     
\usepackage{amsmath}
\usepackage{tikz}	
\usepackage{placeins}
\usepackage{caption}
\usepackage{subcaption}
\usepackage[font={small}]{caption}
\usepackage{lineno}
\usepackage{booktabs}
\linenumbers

\begin{document}

% Declarations for Front Matter

\title{Searches for Electroweak Production of Compressed Supersymmetry in Events with Soft Leptons Plus Missing Transverse Momentum and Hard Jet Recoil}
\author{Sheena Calie Schier}
\degreeyear{2018}
\degreemonth{June}
\degree{DOCTOR OF PHILOSOPHY}
\chair{Professor Abraham Seiden}
\committeememberone{Professor Jason Neilsen}
\committeemembertwo{Professor Michael Hance}
\numberofmembers{3} 
\deanlineone{Dean Tyrus Miller}
\deanlinetwo{Vice Provost and Dean of Graduate Studies}
\deanlinethree{}
\field{Physics}
\campus{Santa Cruz}

\begin{frontmatter}

\maketitle
\copyrightpage

\tableofcontents
\listoffigures
\listoftables

\makeatletter
\newcommand{\rmnum}[1]{\romannumeral #1}
\newcommand{\Rmnum}[1]{\expandafter\@slowromancap\romannumeral #1@}
\makeatother
%TODO: LINE NUMBERS

\begin{abstract}
Supersymmetry (SUSY) is an extension of the Standard Model that predicts a boson (fermion) partner for each fermion (boson) in the Standard Model. Weak scale SUSY is attractive for reasons like improving gauge coupling unification, reducing fine-tuning in the Higgs sector and providing a dark matter candidate. In this thesis, I present a dedicated search for direct production of new colorless weak scale states in a compressed mass spectra with final states characterized by soft visible decay products. This analysis uses $pp$ collisions at $\sqrt s$ = 13 TeV at the Large Hadron Collider and collected by the ATLAS experiment during 2015 and 2016 corresponding to 36.1 $fb^{-1}$ of integrated luminosity. This analysis selects events with two soft electrons or muons and missing transverse momentum ($E_T^{miss}$) recoiling against hadronic initial state radiation. Backgrounds from $t\bar{t}$, diboson, and other production mechanisms with prompt leptons are estimated with Monte Carlo
simulation while reducible backgrounds with instrumental $E_T^{miss}$ and fake/non-prompt leptons is estimated with a mix of Monte Carlo and data-driven methods. Results are consistent with Standard Model expectation and limits on compressed supersymmetric states are extended for the first time since LEP.

\end{abstract}

\begin{dedication}
\null\vfil
{\large
\begin{center}
To my father,\\\vspace{12pt}
Lecil Charles Schier,\\\vspace{12pt}
the person who taught me at the age of 7 that the grass in not green.
\end{center}}
\vfil\null
\end{dedication}


\begin{acknowledgements}
I want to ``thank'' my committee, without whose ridiculous demands, I
would have graduated so, so, very much faster.
\end{acknowledgements}

\end{frontmatter}

\chapter{Introduction}

Since the world's first particle accelerator went online in the 1930's, colliding protons against a fixed lithium target at the Cavendish Laboratory, particle collisions have been providing physicists with portals into the subatomic realm where quantum physics is the supreme ruler.  Progressively, particle accelerators have become more and more powerful, and the depth at which physicists can peer into the atom, into the structure of particles, and eventually into interactions of the most fundamental, has hastened.  Today, we stand at the energy frontier of particle experiments with the Standard Model of Particle Physics, a theory that could appear as a complete map of fundamental particles and interactions, in hand to guide us through the sea of quantum possibilities, while astronomical observations, for one, give us the distinct sense that we are holding only a small slice of the key.  

Another historic event in the 1930's was the first hint of dark matter in astronomical observation.  J.H. Oort, the namesake of the Oort Cloud, measured the velocities of stars using their Doppler shifts.  Surprisingly, the galactic mass that binds them in gravitational orbit should not be strong enough to overcome their velocities, and the stars should escape.  It wasn't until 40 years later that Vera Rubin actually opened the eyes of science to the alarming possibility of a new kind of matter so different from the electrically interacting matter that makes our universe observable, that we call it 'dark'.  Dark matter is now observed to be so abundant in the universe, we believe there is five times more of it than the matter that constitutes all the stars, planets, gas clouds, and anything else made of Standard Model particles.  But this extraordinary matter might not be completely dark.  It might couple to the Standard Model extremely weakly, and when it does, we hope to be there to witness.

The European Organization for Nuclear Research, or Conseil Europ\'een pour la Recherche Nucl\`eaire (CERN), began as an official scientific union between 12 European countries in 1954, when engineers began digging the first hole near Geneva, Switzerland which marked the beginning of a new era of particle physics collisions.  Experiments at CERN have been heros in electroweak physics, with discoveries of the $W^\pm$ and $Z^0$ bosons and 1983 with the Super Proton Synchrotron (SPS), and the discovery the particle that so far sufficiently looks like the Standard Model Higgs boson in 2012 with the Large Hadron Collider (LHC).  Currently, the LHC is the largest and most powerful accelerator on Earth, colliding protons with a center of mass energy of $13~\TeV$.  With this machine, we step closer to the energy scales of the early universe before thermal freeze-out when the universe became too cold for dark matter production, or any other interactions at the dark matter scale.  There is a chance the LHC will produce dark matter particles if they interact with the electroweak force.  There is also the chance to produce a plethora of other particles that do not account for dark matter, but are motivated by predictive new physics models. 

Some of these models predict a new symmetry, and with this new symmetry, a new particle paired to each the Standard Model particles.  Within these models, a well motivated set of particles are possible that couple to the Higgs and the electroweak gauge bosons at the weak scale and decay to a stable particle that fits the police profile of dark matter.  The LHC is capable of producing these particles if they exist, but if the mass spectrum of the new particles is compressed, meaning the masses are within a few $\GeV$ of each other, the signatures of these events will be hard to resolve in a detector that normally identifies objects above $20 \GeV$.  It takes dedicated teams and a lot of strategy to do physics at the edge of detector limits.  

This thesis presents a search for new compressed electroweak physics marked by "soft', low-momentum leptons and a sufficient amount of energy deduced to have left the detector unseen.  The analysis is broken into three parts.  Part 1 will engage the theories that give context and meaning to this search and also describe the LHC and the particle detector used in the experiment.  Part 2 will describe all the work done in performing the analysis, and Part 3 will overview the uncertainties, results, and interpretations.









\part{Theoretical Motivation and Experimental Setup}
\chapter{Theoretical Background and Motivation}
\label{ch:thy}
To any curious mind staring into the starry deep late in the night, or gazing at pictures from the Hubble Space Telescope, the universe can seem deeply mysterious, as a vast space containing a rich spectra of matter moving and transforming via some set of complex mechanisms.  Although this mysterious sense of the universe rings true even in the mind of the most learned physics scholar, large leaps have been made in understanding the true nature of the matter and forces that make up the observable universe.   In the last century, particle physicists have constructed a theory that incorporates all the directly observed fundamental particles and explains their existence and interactions in simplicity through field equations that describe the fundamental forces in the universe.  This theory is called the Standard Model of Particle Physics (SM) and, apart from gravity being far too weak to be described by particle interactions, is internally complete in that every piece of the SM has been observed according to prediction.

But the story doesn't end here.  There are reasons to think the complete and successful Standard Model is a low-scale approximation of a much larger theory.  Some reasons are philosophical in nature; we want to understand why the SM has its structure, or lack confidence in a theory that is so incredibly fine-tuned as the Standard Model.  Other reasons come from observations that we can not be resolved with SM predictions, like the  abundance of 'dark matter' that drives massive galaxies to rotate contrary to predictive models accounting only for gravity and SM particles and forces.

The proceeding structure of this chapter is as follows: Section~\ref{sec:sm}, summarizes the Standard Model of Particle Physics; %Section~\ref{sec:gauge} , describes the gauge symmetries that give the Standard Model its particular structure and spontaneous electroweak symmetry breaking  that calls for the existence of the Higgs boson.  Next, Section~\ref{sec:higgs} relates electroweak symmetry breaking to the Higgs mechanism and the gauge boson masses.  
Section~\ref{sec:fail} goes over some of the shortcomings of the Standard Model, and supersymmetry is introduced in Section~\ref{sec:susy} as a viable model for physics beyond the Standard Model.  Lastly, Section~\ref{sec:pheno} describes the phenomenology of supersymmetric Higgsinos and sleptons in compressed scenarios.

\section{The Standard Model of Particle Physics}
\label{sec:sm}
The Standard Model of Particle Physics provides a quantum description of three of the four known fundamental forces; the electromagnetic force, the strong force, and the weak force.  It leaves out the gravitational force because the strength of gravitational interactions is several orders of magnitude lower than the others, which leads to intrinsic incompatibilities in a description of quantum gravitational interactions at energies below the Plank scale, $M_P\approx10^{19}\GeV$.  The SM was pieced together throughout the second half of the twentieth century by several progressive discoveries, and we now know that there are only a hand full of fundamental constituents that make up the incredible collection of particles in nature.  The fundamental components separate into two distinct categories: fermions and bosons.  These two types of particles are characterized by their spin and interactions, and ultimately play completely different roles in the state and phenomena of the universe. \cite{tully}

The main ingredients of the Standard Model are a set of Dirac fermion fields having specific muliplet representations in group theory given by the $SU(3)_{C} \times SU(2)_{L} \times U(1)_{Y}$ gauge group.  In SM quantum field theory (QFT), called the "Yang Mills theory" \cite{PhysRev.96.191}, fermion interactions are mediated by gauge bosons.  %The structure of the gauge bosons and the interactions they govern is a consequence of gauge invariance in $SU(n)$ type Lie groups \cite{westra}.  

Gauge invariance in QFT demands the existence of gauge boson fields, which occur in two independent sectors: the electroweak sector, described by quantum electroweak dynamics (QED), and the strong sector, described by quantum chromodynamics (QCD).  Glashow, Salam, and Weinberg first presented the structure for the electroweak model in the 1960's \cite{Glashow:1961tr, PhysRevLett.19.1264, Salam:1968rm}.  The $SU(2)_L\otimes U(1)_Y$ symmetry of QED produces the photon $\gamma$ and the massive bosons, $W^\pm$ and $Z^0$.  $U(1)_Y$ is a mathematical group described by unitary $1\times1$ matrices generated by weak-hypercharge symmetry $Y$, defined as 
\begin{equation}
Y=  2(Q-T_3)
\label{eq:Y}
\end{equation}
, where Q is the electromagnetic charge, and $T_3$ is the z-component of the weak isospin\footnote{Weak isospin is the charge associated with the $SU(2)_L$ symmetry.  $SU(2)_L$ multiplets are often called \textit{isospin multiplets}.}.   This symmetry produces the $B^0$ gauge boson.  Similarly, $SU(2)_L$ represents a group of unitary $2\times2$ matrices with determinant 1.  These are generated by a left-handed chiral symmetry~\cite{koch} that produces the $W^{\pm}$ and $W^0$, or $W_3$ gauge bosons.  If this were a perfect symmetry, these gauge bosons would be mass eigenstates with mass equal to zero. But the observed electroweak gauge bosons are not massless; therefore, the symmetry must be broken.  The mass eigenstates of the photon and the neutral vector boson $Z^0$ are informed by the mixing of the neutral $B$ and $W_3$ states, shown in Eq~\ref{eq:mix}.
\begin{equation}
\begin{pmatrix}
\gamma \\
Z^0 \\
\end{pmatrix}
=
\begin{pmatrix}
\cos\theta_W & \sin\theta_W\\
\sin\theta_W & \cos\theta_W\\
\end{pmatrix}
\begin{pmatrix}
B^0 \\
W_3^0 \\
\end{pmatrix}
\label{eq:mix}
\end{equation}
In Eq~\ref{eq:mix}, $\theta_W$ is the weak mixing angle~\cite{BILENKY198273}.  

In the wake of electroweak symmetry breaking, an external mechanism called the \textit{Higgs mechanism} is needed to provide the masses of the $W^\pm$ and $Z^0$.  To generate the masses of the charged and neutral electroweak bosons, the Higgs Mechanism is expressed as two scalar fields, producing a chiral doublet, as in Equation~\ref{eq:h1}.
\begin{equation}
\Phi= \begin{pmatrix}
\Phi^+ \\
\Phi^0 \\
\end{pmatrix} \Rightarrow
\begin{pmatrix}
H^\pm_u & H^0_u \\
H^\pm_d & H^0_d  \\
\end{pmatrix}
\label{eq:h1}
\end{equation}
The \textit{u} and \textit{d} subscripts in Equation~\ref{eq:h1} mean \textit{up} and \textit{down}, referring to the relative direction of the weak isospin.  The two charged and one neutral boson states provide the longitudinal degrees of freedom to the $W^\pm$ and the $Z^0$ bosons, and the last neutral boson provides the SM Higgs, which, until recently, remained the last missing piece of the Standard Model.  The squared mass of the Higgs, seen in Equation~\ref{eq:h2} is quadratically sensitive to the scale at which particle couplings to the Higgs turn on $\Lambda$, which, in the SM, is the weak-scale at $\sim 100\GeV$. 
 \begin{equation}
 m_H^2 = (m_H^2)_0+\frac{kg^2\Lambda^2}{16\pi^2}
 \label{eq:h2}
 \end{equation}
  Here, $g$ an electroweak coupling, and $k$ a constant that scales the coupling; calculable in the low-energy effective theory, it is expected to be of $\mathcal{O}(1)$ \cite{haber}. 
The $SU(3)_C$ represents a group of unitary $3\times3$ matrices with determinant 1 generated by color symmetry.  The gauge invariance imposed on this symmetry produces a color octet of massless gluons.  The gauge bosons (plus the Higgs) masses and their $SU(3)_{C} \times SU(2)_{L} \times U(1)_{Y}$ multiplet representations are summarized in Table~\ref{tab:boson}.  All bosons are integer-spin particles.      
\begin{table}[!htb]
\centering
\small
\begin{tabular}{|lcrc|}
\hline
State  & Spin & Mass &  $SU(3)_{C}, SU(2)_{L}, U(1)_{Y}$ \\
\hline \hline
$g$ & 1&  $0$ & $(\mathbf{8}, \mathbf{1}, 0)$ \\ 
\hline
$W^\pm$ & 1 & $80.4\GeV$ & $(\mathbf{1}, \mathbf{3}, 0)$  \\  
$Z^0$ & 1 & $91.2\GeV$ & $(\mathbf{1}, \mathbf{3}, 0)$\\ 
$\gamma$ & 1 & $0$ & $(\mathbf{1}, \mathbf{1}, 0)$ \\  
\hline
$H^0$& 0 & $125~\GeV$&$(\mathbf{1}, \mathbf{2}, \pm\frac{1}{2})$\\
\hline 

\hline
\end{tabular}
\caption{Strong and EW boson spin mass, and $SU(3)_{C}, SU(2)_{L}, U(1)_{Y}$ multiplet representations. }
\label{tab:boson}
\end{table} 

Fermions are 1/2-integer-spin particle that fall into two categorizes, leptons and quarks.  Leptons carry electromagnetic and weak isospin charge, but do not carry strong color charge.  The leptons consists of three generations of isospin doublets which contain the electron, muon, and tau-lepton with their associated neutrino partners.  Quarks are strongly charged particles that also carry weak isospin and fractional electromagnetic charge.  Like the leptons, there are three quark families, each forming an isospin doublet and consisting of an up-type and a down-type quark.  The fermion masses and multiplet representations are summarized in Table~\ref{tab:fermion}.
\begin{table}[!htb]
\centering
\small
\begin{tabular}{|cllc|}
\hline
State  & Mass & Q & $SU(3)_{C} \times SU(2)_{L} \times U(1)_{Y}$ \\
\hline \hline
\textbf{leptons}&&&$(\mathbf{1}, \mathbf{2}, -1/2)$\\
\hline
$\begin{pmatrix}
e^- \\
\nu_e\\
\end{pmatrix}$
&$\begin{matrix}
0.511\MeV \\
<2~\mathrm{eV}\\
\end{matrix}$
&$\begin{matrix}
-1\\
~0\\
\end{matrix}$&\\
\hline
$\begin{pmatrix} \mu^-\\ \nu_\mu \end{pmatrix}$
 &$\begin{matrix} 105.7\MeV\\  <0.19\MeV \\ \end{matrix}$
  &$\begin{matrix} -1\\  ~0 \\ \end{matrix}$&\\
\hline
$\begin{pmatrix} \tau^- \\ \nu_\tau \end{pmatrix}$
&$\begin{matrix} 1.78\GeV  \\ <18.2\MeV \end{matrix}$
  &$\begin{matrix} -1\\  ~0 \\ \end{matrix}$&\\  
\hline 

\hline
\textbf{quarks}&&&$(\mathbf{3}, \mathbf{2}, 1/6)$\\
\hline 
$\begin{pmatrix} d\\ u \end{pmatrix}$
 &$\begin{matrix} 5\MeV\\  2\MeV \\ \end{matrix}$
  &$\begin{matrix} -1/3\\  ~2/3 \\ \end{matrix}$&\\  
\hline
$\begin{pmatrix} s\\ c \end{pmatrix}$
&$\begin{matrix}\approx100\MeV \\  \approx1\GeV \\ \end{matrix}$
  &$\begin{matrix} -1/3\\  ~2/3 \\ \end{matrix}$&\\  
  \hline
  $\begin{pmatrix} b\\ t \end{pmatrix}$
&$\begin{matrix} 4.19\GeV\\  <172.0\GeV \\ \end{matrix}$
  &$\begin{matrix} -1/3\\  ~2/3 \\ \end{matrix}$&\\  
\hline

\hline
\end{tabular}
\caption{Description of fermion mass, electric charge Q, and $SU(3)_{C}, SU(2)_{L}, U(1)_{Y}$ multiplet representations. }
\label{tab:fermion}
\end{table}  
\FloatBarrier

\section{Shortcomings of the Standard Model}
\label{sec:fail}
As mentioned before, the Standard Model of Particle Physics in all its glory has limitations.

\begin{itemize}
\item The inability to explain dark matter \cite{BERTONE2005279}
\item The hierarchy problem in relation to $M_W/M_P$
\item Neutrino masses and mixing \cite{1367-2630-16-4-045018}
\item CP-violation in the early universe \cite{Sakharov:1967dj}
 \end{itemize}
Dark matter is proposed to make up about $80\%$ of the matter in the universe, and yet, unlike matter from SM particles, does not interact with the electromagnetic or the strong forces, and possibly not even the weak force.  The fact that the SM only accounts for $20\%$ of the matter in the universe is perplexing, but there are hints to what type of new particles we should be looking for.  First, we know that dark matter does not interact via the electromagnetic or strong forces.  We have no reason to believe it interacts via the weak force, but it could, and for experimental purposes we often assume that it does.  Another important quality of dark matter is that it is stable enough to statically populate the universe.  This also relates to the relic abundance of dark matter, which is the measured abundance of dark matter 'frozen' into existence in the early universe once it cooled to the point that dark matter could no longer be produced.  This puts theorized constraints on the masses of dark matter candidates.  We also know from cosmological dark matter mapping, like from recent Dark Energy Survey~\cite{surveydm}, that it must have the ability to cluster, therefore it should not be extremely light and relativistic.  The lightest supersymmetric particle (LSP) in some SUSY models that conserve what is called R-parity provides a stable, weak-scale, weakly interacting candidate for dark matter. 

The hierarchy between the weak-scale and the Planck-scale is a problem of the SM because the Higgs potential is quite sensitive to new physics in any sensible extension to the SM.  Quantum loop corrections from any particle that couples to the Higgs potential can cause quadratic divergences in the Higgs mass through $\Lambda$, as in Equation~\ref{eq:h2}.  Supersymmetry, introduced in the next section, has the benefit of cancelling these diverging mass corrections by adding new particles to the spectrum with corrections opposite to those from SM particles.  The only other option in extending the SM is to make the rather \textit{ad hoc} assumption that none of the undiscovered high-mass particles or condensates from new physics far above the weak scale couple in any way to the Higgs potential.  


\section{Supersymmetry}
\label{sec:susy}
Supersymmetry offers an extension to the Standard Model by extending the Poincare symmetry of quantum field theory to $SO(10)_{SUSY}$~\cite{Martin:1997ns}.  This extension leads to a boson-fermion symmetry that can be expressed by a supersymmetric transformation operator which carries $1/2$-integer spin angular momentum that transforms boson states to fermion states, and vice versa.  If unbroken, this symmetry generates a supersymmetric partner for all Standard Model particles, with each pair being equivalent in mass and all other quantum numbers, but differing intrinsically by half-integer spin.  So, each SM fermion has a bosonic supersymmetric partner, and each SM boson has a fermionic supersymmetric partner.  

According to this symmetry, assuming it is a perfect symmetry, these new particles should have already been observed with their SM masses, but this is not the case.  In order for this theory to remain viable, the new symmetry must be broken in a way that preserves the fermion-boson symmetry and all observations of the Standard Model while allowing fermion-boson partners to be decoupled in mass.  If the effective scale of supersymmetry breaking is near the weak scale, no unnatural cancellations need to be added to Equation~\ref{eq:h2} to keep the Higgs mass near the electroweak scale and free of quadratic divergences due to quantum corrections.  

A detailed description of the various models for mediating this symmetry-breaking and communicating it the visible sector of observable particles is beyond the scope of this thesis, but a very clear explanation by Howard Haber can be found in the Supersymmetry (Theory) chapter in the PDG \cite{haber}.   This search targets SUSY models that have undergone soft-breaking in the SUSY electroweak sector. 

  \begin{figure}[tbp]
    \centering
% http://cdsweb.cern.ch/record/1095926
 \includegraphics[width=0.7\columnwidth]{/Users/sheenaschier/Documents/LaFiles/figures/thesis/theory/Susy-particles.jpg}
    \caption{Schematic of supersymmetry particle spectrum}
   \label{fig:susy}
 \end{figure}

\subsection{Minimal Supersymmetric Model (MSSM)}

\begin{table}[!htb]
\centering
\small
\begin{tabular}{|c|c|c|c|}
\hline
Spin 0  & Spin $\frac{1}{2}$& Spin 1 &  $SU(3)_{C}, SU(2)_{L}, U(1)_{Y}$ \\
\hline \hline
($\tilde{u}~\tilde{d}$) & ($u~d$)&   & $(\mathbf{3}, \mathbf{2}, 1/6)$ \\ 
\hline
($\tilde{e}~\tilde{\nu}$) & ($e~\nu$)&   & $(\mathbf{1}, \mathbf{2}, -1/2)$ \\
\hline
($H_u^+~H_u^0$)&($\tilde{H}_u^+~\tilde{H}_u^0$) && $(\mathbf{1}, \mathbf{2}, +1/2)$\\
($H_d^0~H_d^-$)&($\tilde{H}_0^+~\tilde{H}_d^-$) & &$(\mathbf{1}, \mathbf{2}, -1/2)$\\
\hline
 &$\tilde{g}$&  $g$ & $(\mathbf{8}, \mathbf{1}, 0)$ \\
\hline
& $\tilde{W}^\pm~ \tilde{W}^0$& $W^\pm~ W^0$ & $(\mathbf{1}, \mathbf{3}, 0)$  \\  
 & $\tilde{B}^0$& $B^0$ & $(\mathbf{1}, \mathbf{1}, 0)$\\ 
\hline

\hline %\hline
\end{tabular}
\caption{SUSY MSSM spectrum in $SU(3)_{C}, SU(2)_{L}, U(1)_{Y}$ multiplet representation.}
\label{tab:susy}
\end{table}
Supersymmetric extensions to the SM are free to include multiple sectors and new sets of supersymmetric partners, and a minimal supersymmetric extension to the Standard Model (MSSM) adds the minimal number of new states needed to complete the theory, and most importantly, just one new Higgs doublet.  The general MSSM has 124 free parameters, many of which are related to each other only through some unknown SUSY breaking mechanism.  Observed or inferred constraints can be placed on many of the 100 plus parameters, reducing this number down to 19.  Among these is the top quark mass \cite{Bechtle2006}.  Table~\ref{tab:susy} shows the particle content in the MSSM.  In this table,  ($e~\nu$) stands for all three generations of SM lepton, and ($u~d$) refers to the three generations of quark.  Both chiral representation of the Higgs fields are shown explicitly.  In the MSSM, these form chiral supermultiplets with their superpartners; three generations of \textit{sleptons} ($\tilde{e}~\tilde{\nu}$), three generations of \textit{squarks} ($\tilde{u}~\tilde{d}$), and four new spin-1 Higgsino fields.  The name for all supersymmetric quark partners and supersymmetric lepton partners is just the SM partner name with an \textit{s} in front.  This \textit{s} does not mean \textit{supersymmetric}; but rather, it means \textit{scalar}, which refers to a particle with spin angular momentum 0, as seen in Table~\ref{tab:susy}.  The names for SM boson partners have the suffix \textit{ino}/

Standard model gauge bosons and their superpartners, typically referred to as \textit{gauginos}, form gauge supermultiplets.  The superpartner to the gluon $g$ is the spin-1/2 color-octet \textit{gluino} $\tilde{g}$.  The spin-1 gauge eigenstates that mix to form the SM vector bosons are the $W^+$, $W^0$, $W^-$, and $B^0$.  Their spin-1/2 superpartners are the \textit{winos} and \textit{binos}: $\tilde{W}^+$, $\tilde{W}^0$, $\tilde{W}^-$, and $\tilde{B}^0$.  Like with SM gauge boson, mass eigenstates are not necessarily pure weak eigenstates.  There can be mixing between the electroweak gauginos and the Higgsinos to form the charged and neutral SUSY mass eigenstates called the \textit{charginos} and \textit{neutralinos}.  There are two charged states ($\tilde\chi_1^\pm$, $\tilde\chi_2^\pm$) and four light neutral states ($\tilde\chi_1^0$, $\tilde\chi_2^0$, $\tilde\chi_3^0$, $\tilde\chi_4^0$), and can be referred to together as \textit{electroweakinos}.
  
The MSSM is defined to conserve R-parity.  All SM particles have R-parity +1, while all SUSY particles have R-parity -1.  R-parity is defined as:
\begin{equation}
 P_R=(-1)^{3(B+L)+2s}
 \end{equation}
 where B and L are the baryon number and lepton number defined in Section~\ref{sec:sm}.  The conservation of R-parity means that, in the collision of two R-parity even SM particles, R-parity odd SUSY particles must be produced in pairs, and the subsequent decay chain of each must end with the lightest SUSY particle (LSP) in the MSSM model.  The LSP must be stable since the only kinematically available decays are to lighter SM particles, which would violate R-parity conservation.  The stability of a weakly interacting LSP in R-parity conserving models can make them good candidates for dark matter.  

Of the 19 free parameters in the constrained MSSM, only a handful determine the chargino and neutralino masses; $M_1$, $M_2$, $\mu$, and $\tan\beta$.  $M_1$ and $M_2$ are the bino and wino mass parameters, $\mu$ is the Higgsino mass parameter, and $\tan\beta$ is the ratio of the vacuum expectation values of the two Higgs doublets:
\begin{equation}
\tan\beta=\nu_u/\nu_d
\end{equation}
The chargino and neutralino mass mixing matrices are shown in Equations~\ref{eq:chargino} and ~\ref{eq:neutralino}.
\begin{equation}
M_{\chi^\pm}=
\begin{pmatrix}
M_2 & \sqrt{2}M_W\sin\beta \\
\sqrt{2}M_W\cos\beta & \mu \\
\end{pmatrix}
\label{eq:chargino}
\end{equation}

\begin{equation}
M_{\chi^0}=
\begin{pmatrix}
M_1 & 0 & -M_Z\cos\beta \sin\theta_W & M_Z\sin\beta \sin\theta_W\\
0 & M_2 & M_Z\cos\beta \cos\theta_W & -M_Z\sin\beta \cos\theta_W\\
-M_Z\cos\beta \sin\theta_W & M_Z\cos\beta \cos\theta_W & 0 & -\mu \\
M_Z\sin\beta \sin\theta_W & -M_Z\sin\beta \cos\theta_W & -\mu & 0 \\
\end{pmatrix}
\label{eq:neutralino}
\end{equation}
In these equations, $\cos\beta$ and $\sin\beta$ are the x- and y-components of $\tan\beta$.  The structure of wino/bino/higgsino mixing and relative mass spectrum of the lightest electroweakinos is governed by the relative magnitudes of the mass parameters $M_1$, $M_2$, and $\mu$ in Equations~\ref{eq:chargino} and~\ref{eq:neutralino}.  When $|\mu|\ll~|M_1|,~|M_2|$, the lightest mass eigenstates of the mass mixing matrices are mostly Higgsino with little or no wino/bino mixing.  In this case, the lightest stable SUSY particle is the Higgsino $\tilde\chi_1^0$, and is called the Higgsino LSP.  When the eigenstates are purely Higgsino, the solution gives a fully degenerate set of electroweakinos\footnote{Small mass-splitting of order 200 MeV occur through radiative corrections.}, and there needs to be some level of wino/bino mixing added to get larger differences between the lightest and next-to-lightest chargino and neutralino masses \cite{PhysRevD.93.063525}.  Another relevant scenario, $|M_1| <|M_2|\ll~|\mu|$, leads to wino dominated $\tilde\chi_1^\pm$ and $\tilde\chi_2^0$ states that are nearly mass degenerate and $\mathcal{O}(10\GeV)$ heavier than the bino LSP $\tilde\chi_1^0$.  This is the order of mixing assumed for the compressed slepton model interpretations where the slepton masses are in between the bino LSP and the heavier winos.  For these scenarios to be compressed means the small mass-splittings between the $\tilde\chi_1^0$ and the sleptons are of $\mathcal{O}(1\GeV)$.  Other scenarios can occur as well, for example: the Higgsino-bino model $|\mu|\sim |M_1|\ll~|M_2|$, the Higgsino-wino model $|\mu|\sim |M_2|\ll~|M_1|$ and the wino-bino model $|M_1|\sim |M_2|\ll~|\mu|$ display mass spectra related to $\Delta(\mu, M_1)$, $\Delta(\mu, M_2)$ and $\Delta(M_1, M_2)$ respectively~\cite{PhysRevD.96.055018}.  


%%%%%%%%%%%%%%%%%%%%%%%%%%%%%%%%%%%%%%%%%%%%%%%%%%%%
\subsection{Phenomenology of Directly Produced Higgsinos and Sleptons in Compressed Scenarios}
\label{sec:pheno}
 \begin{figure}%[h!]
  \begin{center}
  \includegraphics[width=0.64\textwidth]{/Users/sheenaschier/Documents/LaFiles/figures/thesis/theory/C1N2-WZN1N1}
  \includegraphics[width=0.35\textwidth]{/Users/sheenaschier/Documents/LaFiles/figures/thesis/theory/SEW_mass_spectra2}
   \end{center}
 \caption{Feynman diagram of direct Higgsino production (left), and schematic of electroweakino mass spectrum (right)}
 \label{fig:fn1}
 \end{figure}
This analysis targets direct production of electroweakinos that decay to  Higgsino LSPs, as in Figure~\ref{fig:fn1}, and sleptons that decay to bino LSPs , as in Figure~\ref{fig:fn2} in compressed scenarios.  Small mass splittings among the electroweakinos come from the Higgsino scenario with $\mu\ll~M_1,~M_2$, and in order for supersymmetry breaking to occur at the correct scale without any unnatural corrections, the parameter $\mu$ must be near the weak scale $\approx 100~\GeV$.  This sets the Higgsinos masses near the weak scale, while allowing the winos and binos, with masses given by $M_1$ and $M_2$, to still be heavy.  The slepton model assumes $|M_1| <|M_2|\ll~|\mu|$ with the slepton mass just above the LSP mass~\cite{gondolo}.  In the natural scenario, $M_1$ and $M_2$ are near the weak scale, and the bino becomes a valid dark matter candidate, except that it leads to a higher dark matter relic abundance than measured with the WMAP and Planck experiments.  If the slepton has a mass slightly above the LSP mass, then coannihilation could reduce the dark matter abundance~\cite{seckel}.  So far, there no sign of the colored SUSY sector, so we can ignore the colored states all together by assuming there masses are very large.  
   \begin{figure}%[h!]
  \begin{center}
  \includegraphics[width=0.64\textwidth]{/Users/sheenaschier/Documents/LaFiles/figures/thesis/theory/slsl-llN1N1j.pdf}
   \includegraphics[width=0.35\textwidth]{/Users/sheenaschier/Documents/LaFiles/figures/thesis/theory/SEW_mass_spectra_slep}
   \end{center}
 \caption{Feynman diagram of direct slepton production (left) and schematic of electroweakino and slepton mass spectrum}
 \label{fig:fn2}
  \end{figure}

One way to search for Higgsinos is through direct production of squarks that then decay to Higgsinos, but these particles have little effect on the mass of the Higgs, and therefor, may naturally have masses well beyond the reach of the LHC.  Also, Higgsino models are very sensitive to the spectrum of light SUSY particles when trying to observe them through direct squark production.  Direct Higgsino or slepton production allows one to remain fairly agnostic to the spectrum of the SUSY sector, and therefore, retain sensitivity to a large range of weak-scale SUSY models.  Unfortunately, the direct production of electroweakinos, including Higgsinos, is subject to electroweak cross-sections $\sim 1\ipb$, and the slepton cross-sections are even lower, limiting the search sensitivity at the LHC.  Figure~\ref{fig:thy:xsec} shows the cross-sections for the SUSY particles in the MSSM as a function of mass to show how much smaller electroweakino and slepton cross-sections are compared to strongly produced SUSY.

     \begin{figure}%[h!]
  \begin{center}
  \includegraphics[width=0.6\textwidth]{/Users/sheenaschier/Documents/LaFiles/figures/thesis/theory/SUSYx-sec.png}
   \end{center}
 \caption{SUSY cross-sections in LHC pp collisions~\cite{Bechtle:2015nta}}
 \label{fig:thy:xsec}
 \end{figure}

When the electroweakino mass-splittings are close to mass of the $W$ boson, Standard Model $W$ and $Z$ bosons are produced on-shell, or produced at their nominal masses, and about $30\%$ of the time will decay to detectable leptons.  In this case, analyses have been performed in both ATLAS and CMS to search for all three leptons from the $W$ and $Z$, where the $Z$ can be reconstructed from an opposite-sign-same-flavor lepton pair.  These searches also require a substantial amount of missing transverse momentum from the lightest neutral electroweakinos.  When the mass-splittings fall below the $W$ mass, the $W$ and $Z$ bosons are produced off-shell, they are lighter than their nominal $80-90~\GeV$ mass, and the leptons from these decays become less energetic, or \textit{softer}.  When the leptons become very soft, triggering and lepton reconstruction become challenging; therefore, dedicated efforts are needed to probe model-space where the electroweakino mass-splittings are less than $\sim~60\GeV$. 
   \begin{figure}%[h!]
  \begin{center}
  \includegraphics[width=0.8\textwidth]{/Users/sheenaschier/Documents/LaFiles/figures/thesis/theory/Feynman_2}
   \end{center}
 \caption{Feynman diagram of direct Higgsino production in compressed scenario}
 \label{fig:fn3}
 \end{figure}
For final states with soft leptons and \met{}, requiring a hard ISR jet in the event helps sculpt the kinematic signature in a way that makes the decays of the nearly degenerate particles more distinguished from the backgrounds.  Figure~\ref{fig:fn3} points out some of the kinematic features of direct production of compressed electroweakinos with an ISR jet.   The jet boosts the system, increases the \met, and forces a large angular separation between the leading jet in the system and the intermediate amount of \met.  Having more \met associated with the LSPs is also important for triggering, as \met might be the most efficient object on which to trigger.  The characteristic are also relevant for compressed slepton production with hadronic ISR.  

Another important feature is that the dilepton invariant mass ($m_{\ell\ell}$) distribution electroweakino production is linked to the mass-splitting between the chargino and the lightest neutralino through the mass of the very off-shell $Z$.  We can exploit the dilepton invariant mass for the electroweakinos, through what is called the kinematic end-point, which is a strict limit on the dilepton invariant mass set by $m(\tilde\chi_2^0)-m(\tilde\chi_1^0)$. The sleptons do not have the same sensitivity in $m_{\ell\ell}$, but instead show angular correlations between the SM leptons and \met{} coming from the bino LSP.  This relationship is expressed through a variable called the \textit{stransverse mass} ($M_{T2}$), which is defined in Chapter~\ref{ch:sr}, and is subject to kinematic boundaries set by the mass of the LSP and its difference from the slepton masses.

 \iffalse
   \begin{figure}%[h!]
  \begin{center}
  \includegraphics[width=0.7\textwidth]{/Users/sheenaschier/Documents/LaFiles/figures/thesis/theory/SEW_mass_spectra.png}
   \end{center}
 \caption{Schematic of the mass spectrum among lightest electroweakinos, $\tilde\chi_1^0$, $\tilde\chi_1^\pm$, and $\tilde\chi_2^0$.}
 \label{fig:thy:mass}
 \end{figure}
 \fi
 \FloatBarrier

 

\chapter{The LHC and The ATLAS Experiment}
\label{ch:detector}
\section{The Large Hadron Collider Machine}
\label{sec:LHC}

The Large Hadron Collider (LHC) is a circular proton accelerator and collider that has two rings with counter-rotating proton beams circumnavigating the 26.7 km long tunnel that was originally built for the CERN LEP machine.  The oppositely traveling proton beams have separate vacuum beam pipes and are accelerated around the ring to the TeV energy scale with a gigantic semi-conducting magnet system.  To reach LHC energies, the proton beams move through a stream of smaller accelerator structures that increase the kinetic energy of the beam at each step, until the beam is finally injected into the LHC, which is still, at the completion of this thesis, the largest and most powerful accelerator in the world.  There are two transfer tunnels, each about 2.5 km long, that join the LHC to the CERN accelerator complex, now acting as the injector for the LHC.  The LHC tunnel is broken into octets with eight straight sides and eight curves.  This is not an LHC design, but rather an artifact of LEP.   That being said, each octet is considered as a reference point around the ring; for instance, octet 1 is "point 1", octet 2 is "point 2" and so on.  The beams collide at four interaction points located approximately 100m underground, and surrounding each interaction point is a physics detector apparatus to collect data from the proton collisions.  The four different detector experiments are ALICE, LHC-B, CMS, and ATLAS.  Figure~\ref{fig:lhc} depicts the tunnel octets and the beam injection and dump points.  It also shows the placement of the four detectors; ATLAS is located at point 1.

  \begin{figure}[tbp]
    \centering
% http://cdsweb.cern.ch/record/1095926
 \includegraphics[width=0.8\columnwidth]{/Users/sheenaschier/Documents/LaFiles/figures/thesis/detector/lhc-schematic.jpg}
    \caption{Schematic of the LHC layout}
   \label{fig:lhc}
 \end{figure}
The primary objective of the LHC is to expose the physics outside of the Standard Model of Particle Physics; therefor, the accelerator needs to supply a high enough center of mass energy to the collisions to unlock possible new physics interactions.  The initial aim was a center of mass energy of 14 TeV, but only 13 TeV has been successfully achieved.  Many Beyond Standard Model (BSM) theories predict new particle interactions with weak-scale cross-sections or lower, creating the need abundant enough luminosity to measure these low probability events.  The machine luminosity depends only on beam parameters, as expressed in Eq~\ref{eq:lumi}.

\begin{equation}
L=\frac{N_b^2n_bf_{rev}\gamma_r}{4\pi\epsilon_n\beta\star}F
\label{eq:lumi}
\end{equation}
In the numerator of Eq~\ref{eq:lumi}, $N_b$ is the number of particles per bunch, $n_b$ is the number of bunches per beam, $f_{rev}$ is the revolution frequency, and $\gamma_r$ is the relativistic gamma factor of the beam particles, since they are highly relativistic at speeds near the speed of light.  In the denominator of Eq~\ref{eq:lumi}, $\epsilon_n$ is the normalized transverse beam emittance and $\beta\star$ in the beta function at the collision point.  $F$ is the geometric luminosity reduction factor due to the beam crossing at an angle at the interaction points:
\begin{equation}
F=(1+(\frac{\Theta_c\sigma_z}{2\sigma^{\star}})^2)^{-1/2}
\label{eq:reduction}
\end{equation}
$\Theta_c$ is the full crossing angle at the interaction point, $sigma_z$ is the RMS bunch length, and $\sigma^{\star}$ is the transverse RMS beam size at the interaction point.  ATLAS, one of the high luminosity experiments at the LHC, achieved a peak luminosity above $L=10^{34}cm^2s^1$

\textcolor{red}{Say more about the achieved luminosity and all that}
 
TODO: Paragraph about magnet system	

General detector requirements are set by the benchmark physics goals and the experimental environment and constraints of the LHC.  
\begin{itemize}
\item \textcolor{blue}{Detectors require fast, radiation-hard electronics and sensor elements since the LHC is designed for high energy and high luminosity performance.  The large number of interactions per bunch-crossing also creates the need for very granular detectors to reduce the effects of event pileup.}
\item \textcolor{blue}{High acceptance in pseudorapidity with nearly full azimuthal angle coverage is needed for the general purpose detectors.}
\item \textcolor{blue}{High track reconstruction efficiency and high resolution charged-particle momentum measurements are essential.}
\item \textcolor{blue}{Good EM calorimetry for efficient electron and photon identification.}
\end{itemize}
 
\section{The ATLAS Experiment}

The ATLAS experiment is a general purpose detector apparatus that almost completely covers the entire solid angle around one of the LHC beam collision points.  ATLAS recorded its first LHC $pp$ collisions in 2009 at center of mass energy $\sqrt{7}~TeV$, and has since recorded events at several different center of mass energies, including the most extensive reach in the history of particle accelerators at $\sqrt{13}~TeV$.  %(http://inspirehep.net/record/1240374/files/CHARGED%202012_011.pdf). 
ATLAS achieves central coverage from in symmetric cylindrical barrel, and forward-backward detecting capabilities in the end-caps.  The complete detector system is 44m long, 25m in diameter, and weighs 4000 tons.  The ATLAS detector, shown in Figure~\ref{fig:ATLAS} is comprised of several sub-detector systems, each calibrated and optimized for a different observational purpose.  Listed in order from the center of ATLAS outward, the sub-detectors are: the inner tracking detector (ID), the electromagnetic calorimeter (eCAL), the hadronic calorimeter (hCAL), and the muon spectrometer (MS).  Together, these sub-detectors measure the energy and momentum of a variety of particles and reconstruct the dynamics of each recorded event.  The combination of the detector systems provide charged particle measurements and efficient lepton and photon measurements out to $|\eta| < 2.4$.  Jets are MET are reconstructed using the full set of information out to $|\eta| < 4.9$.  
  \begin{figure}[tbp]
  \centering
 \includegraphics[width=0.6\columnwidth]{/Users/sheenaschier/Documents/LaFiles/figures/thesis/detector/ATLAS.pdf}
    \caption{Cut-away view of the complete ALTAS Detector}
   \label{fig:ATLAS}
 \end{figure}

%%%%%%%%%%%%%%%%%%%%%%
\subsection{Inner Tracking Detector}
  \begin{figure}[tbp]
 \includegraphics[width=0.49\columnwidth]{/Users/sheenaschier/Documents/LaFiles/figures/thesis/detector/IDlayout.pdf}
  \includegraphics[width=0.49\columnwidth]{/Users/sheenaschier/Documents/LaFiles/figures/thesis/detector/IDtotal.pdf}
    \caption{Layout of the ALTAS Inner Detector}
   \label{fig:ID}
 \end{figure}
The ATLAS Inner Detector (ID), show in Figure~\ref{fig:ID}, provides position measurements of charged particles passing through the fiducial region $|\eta|~<~2.5$ by combining information from three separate tracking systems; the silicon pixel detector, the silicon microstrip semi-conductor tracker, and the straw-tube transition radiation tracker.  The ID is made of a central cylindrical barrel that covers the region $|\eta|~<~1.5$, and two end-caps that complete the ID range $1.5~<~|\eta|~<~2.5$ . The ID is surrounded by a superconducting solenoid that encases the entire ID in a 2 Tesla magnetic field.  The 2 T magnetic field bends the charged particles traveling through the tracker and the curvature induced is driven by the momentum of the particle. This section will give an overview of the setup and capabilities of each sub-detector of the ID.  
\iffalse
  \begin{figure}[tbp]
 \includegraphics[width=0.48\columnwidth]{/Users/sheenaschier/Documents/LaFiles/figures/thesis/detector/ID.pdf}
  \includegraphics[width=0.48\columnwidth]{/Users/sheenaschier/Documents/LaFiles/figures/thesis/detector/IDendcap.pdf}
    \caption{Layout of the ALTAS Inner Detector}
   \label{fig:IDscematic}
 \end{figure}
\fi
\subsection{Pixel Detector}
Inner most pixelated tracker has the highest granularity sensors in the ID.  There are three pixel layers in the central barrel and three layers in each end cap, providing up to three space-points per track.  The pixel detector has approximately 80.4 million readout channels bonded to pixel sensors segmented in the $R-\phi$ and $z$ directions.The pixel sensors have dimensions $50\mu m \times 400\mu m$ in $R-\phi \times z$, and provide an intrinsic resolution of $10\mu m$ in $R-\phi$ and $115 \mu m$ along $z$.  
\subsection{Semi-Conductor Tracker}
Middle silicon $\mu$strip tracker has overall radial extension $255 mm < R < 549 mm$ in the barrel and $251 mm < R < 610 mm$ in the end-caps.  The Semi-Conductor Tracker (SCT) has eight paired strip layers that provide four space points per track.  In the barrel (end-cap), one set of strips is aligned parallel (perpendicular) the beam axis and is daisy chained to a second set of strips that is misaligned with the its partner by a 40 $mrad$ stereo angle.  The strip pitch is $80\mu m$.  The resulting intrinsic resolution in both the barrel and the end-caps is $17\mu m$ in  $R-\phi$, and in the barrel (end-caps) it is $580 \mu m$ in $z$ ($R$).  There are approximately 6.3 million readout channels.
\subsection{Transition Radiation Tracker}
Outer most straw tube transition radiation tracker provides an average of 36 track position measurements, making it responsible for the large majority of track hits.   Between the barrel and the end-caps, as described in Figure~\ref{fig:ID}(a), the TRT can track charged particles through the region $|\eta| < 2.0$.  All straw tubes in the TRT are 4mm in diameter but vary in length between the barrels and the end-caps.  In the barrel, the straw tubes are 144 cm (37cm) long and positioned parallel to the beam axis; in the end-caps, the tubes are 37 cm long and arranged transverse to the beam axis in the radial direction.  There are approximately 351,000 readout channels.


%%%%%%%%%%%%%%%%%%%%%%
\subsection{Calorimetry}
ATLAS electromagnetic and hadronic calorimeters extend to $|\eta|~<~4.9$  To gain the extra coverage over the ID in the forward regions $3.2~<~|\eta|~<~4.9$, LAr calorimeters are used for both electromagnetic and hadronic measurements.  In the region  $|\eta|~<~2.5$, also secured by the ID, the EM calorimeter is finely segmented for precise measurement of electrons and photons. 

  \begin{figure}[tbp]
  \centering
 \includegraphics[width=0.6\columnwidth]{/Users/sheenaschier/Documents/LaFiles/figures/thesis/detector/cal.pdf}
    \caption{Picture of the ALTAS Calorimeters}
   \label{fig:cal}
 \end{figure}

\subsubsection{Electromagnetic Calorimeter}
The eCAL is divided into a central barrel (pseudorapidity $|\eta|~<~1.475$) and two end-caps enclosing each side of the barrel.  The end-cap regions have an outer wheel corresponding to $1.375~<~|\eta|~<~2.5$, and an inner wheel for $2.5~<~|\eta|~<~3.2$.  The region $|\eta|~<~2.5$, which matches the coverage of the inner detector, is segmented into three layers.  The first layer is the most finely segmented in $\eta$ to aid the discrimination between true photons and neutral pions that have decayed to a pair of pions.  Both objects are trackless in flight and undetectable until they interact with the eCAL.  Closely-spaced photons from a boosted neutral pion decays can not be resolved into two objects without the extremely fine grain of this first layer.  The fine grain also helps improve the resolution of the shower position, shape and direction.  The eCAL is preceded by a pre-sampler at $|\eta| < 1.8$ to correct for upstream energy losses.
Measures energy of electromagnetic objects.

\subsubsection{Hadronic Calorimeter}
The iron-scintillator/tile that makes up the hadronic calorimeter barrel spans the region $|\eta| < 1.7$ and sits just outside the EM calorimeter, extending radially from 2.28 m to 4.25 m.  Shown in Fig~\ref{fig:cal}, the tile barrel is in three sections, the central barrel and two extended barrels. 

The LAr hadronic end-caps cover the $\eta$ range $1.5~<~|\eta|~<~3.2$.  Measures energy of hadronic objects

%%%%%%%%%%%%%%%%%%%%%%
\subsection{Muon System}
%https://arxiv.org/pdf/1603.05598.pdf
The muon spectrometer, a tracking detector dedicated entirely to tracking muons, is the outer most sub-detector in ATLAS.  It is designed to track muons in the pseudorapidity region $|\eta|~<~2.7$ with a central barrel covering $|\eta|~<~1.05$ and two end-caps at $1.05~<~|\eta|~<~2.7$.  A network of three large super-conducting toroidal magnets, each with eight coils, supplies a magnetic to the muon spectrometer with am integral bending power in the barrel of around 2.5 Tm and up to 6Tm in the end caps.  Resistive plate chambers in the central region $|\eta|~<~1.05$ and end gap chambers in the forward-backward region $1.05~<~|\eta|~<~2.7$ impart triggering capabilities to the MS as well as position measurements in $\eta$ and $\phi$ with a spacial resolution of 5-10mm. Monitored drift tube chambers provide precision tracking out to $|\eta| < 2.7$ where each chamber provides 6-8 hits in $\eta$ along the muon flight path. 

%%%%%%%%%%%%%%%%%%%%%%
\subsection{Trigger and DAQ}
\textcolor{blue}{Timing and  trigger control logic, the TDAQ system is complex computing system to acquire and store data.  It is partitioned into sub-systems that are typically affiliated with with sub-detectors that have the same logic components and building blocks.}  Originally a three-level trigger system in Run-1, the trigger was restructured into a two-level system with only a hardware level-1 (L1) trigger and a software-based high-level (HL) trigger.  Each trigger level makes decisions about which events to store and which events to throw away forever.

\subsubsection{Trigger System}
%https://cds.cern.ch/record/2133909/files/ATL-DAQ-PROC-2016-003.pdf
The L1 trigger searches for electrons, muons, photons, jets, hadronically decaying $\tau$-leptons, and missing transverse momentum.  In each event, the L1 trigger defines Regions-of-Interest (ROIs), which are detector regions where interesting activity is identified, then stores the geographical ($\eta$, $\phi$) coordinates, the basic characteristics of the detector response in that region, and the set of criteria that triggered the L1.  This information is subsequently passed to the HL trigger to perform a more refined event selection.

\subsubsection{Readout and Data Acquisition}

Electronic elements called the Readout Drivers (RODs) gather information the detector-specific front-end data streams to multiplex highly concentrated data.  The front-end electronics are these combined sub-systems: the front-end analogue-to-digital converter, the L1 buffer that stores the information the L1 trigger will process, the de-randomising buffer that holds data accepted by the L1 trigger before passing it to the HLT, and the buses that move the front-end data stream to the HLT.  The L1 and the de-randomising buffers are important for decreasing the dead-time of the sensors and the L1 trigger.  The L1 buffer stores the information in buckets pass the information along a chain before passing to the L1 trigger to process.  This sequence produces a large enough latency in the detector output account for the intrinsic L1 trigger latency \textcolor{red}{what is the exact latency?}. The de-randomizing buffer allows the L1 store it decisions and process the next event without waiting for the HLT to finishing processing the previous event.


\part{Method}
\chapter{Data Samples and Event Selection}
%\label{sec:mcdata}
This analysis uses $pp$ collision data at $\sqrt{13}~TeV$ from the LHC, collected by the ATLAS detector in 2015 and 2016. events must pass data quality criteria (good runs list).  Events in data are chosen using inclusive MET triggers.  The MET trigger threshold varies by data taking period, where the lowest unprescaled inclusive MET trigger is used.  (Make table and refer to it). A total integrated luminosity of $36.1~fb^{-1}$ at $\sqrt{13} ~ TeV$ passing data quality criteria and the corresponding inclusive MET triggers was recorded.



The event weights are very important.  Currently I am applying the total weight as:
\begin{equation}
\label{eq:weight}
totalWeight = ttbarNNLOWeight*pileupWeight*eventWeight*leptonWeight*jvtWeight*bTagWeight
\end{equation}

\section{Triggers}
\label{sec:eff}
\subsection{MET Triggers}
\label{sec:met}
Inclusive met trigger efficiencies

\subsection{Combined Trigger Studiies}
Lepton plus jet plus met trigger efficiencies..  
**Talk about the development and study of the new triggers implemented n data starting at run 308084, corresponding to an integrated luminosity of $8.8~fb^{-1}$.

  \begin{figure}[tbp]
   % \centering
     \includegraphics[width=0.48\columnwidth]{/Users/sheenaschier/Documents/LaFiles/figures/thesis/eventselection/eff_MM_signal_110_100.pdf}
       \includegraphics[width=0.48\columnwidth]{/Users/sheenaschier/Documents/LaFiles/figures/thesis/eventselection/eff_MM_mtautau_110_100.pdf}\\
   \caption{Trigger Efficiency as a function of MET after event preselection (left) and in a signal region similar to the analysis signal region (right)}
   \label{fig:TrigEff1}
 \end{figure}
 
   \begin{figure}[tbp]
   % \centering
     \includegraphics[width=0.48\columnwidth]{/Users/sheenaschier/Documents/LaFiles/figures/thesis/eventselection/eff_MM_jet145_110_100.pdf}
       \includegraphics[width=0.48\columnwidth]{/Users/sheenaschier/Documents/LaFiles/figures/thesis/eventselection/eff_MM_jet105_110_100.pdf}\\
   \caption{Trigger efficiency as a function of MET for the combined single muon trigger (left) and the combined dimuon trigger (right)}
   \label{fig:TrigEff2}
 \end{figure}
 


\section{Signal Samples}
Simplified models are used for Monte Carlo simulation of signal events.


\subsection{Higgsino LSP Samples}
  \begin{figure}[tbp]
   % \centering

 \includegraphics[width=0.45\columnwidth]{/Users/sheenaschier/Documents/LaFiles/figures/thesis/signal_samples/ossf_Lep1Pt.pdf}
 \includegraphics[width=0.45\columnwidth]{/Users/sheenaschier/Documents/LaFiles/figures/thesis/signal_samples/ossf_Lep2Pt.pdf}\\
 \includegraphics[width=0.45\columnwidth]{/Users/sheenaschier/Documents/LaFiles/figures/thesis/signal_samples/ossf_Mt_l1met.pdf}
 \includegraphics[width=0.45\columnwidth]{/Users/sheenaschier/Documents/LaFiles/figures/thesis/signal_samples/ossf_Mt_l2met.pdf}\\
  \includegraphics[width=0.45\columnwidth]{/Users/sheenaschier/Documents/LaFiles/figures/thesis/signal_samples/ossf_nJet20.pdf}
 \includegraphics[width=0.45\columnwidth]{/Users/sheenaschier/Documents/LaFiles/figures/thesis/signal_samples/ossf_nLep_signal.pdf}\\
   \caption{Kinematic distributions of signal samples}
   \label{fig:SigSample1}
 \end{figure}
 
   \begin{figure}[tbp]
   % \centering
 \includegraphics[width=0.45\columnwidth]{/Users/sheenaschier/Documents/LaFiles/figures/thesis/signal_samples/ossf_Jet1Pt.pdf}
\includegraphics[width=0.45\columnwidth]{/Users/sheenaschier/Documents/LaFiles/figures/thesis/signal_samples/ossf_Jet2Pt.pdf}\\
\includegraphics[width=0.45\columnwidth]{/Users/sheenaschier/Documents/LaFiles/figures/thesis/signal_samples/ossf_MET.pdf}
\includegraphics[width=0.45\columnwidth]{/Users/sheenaschier/Documents/LaFiles/figures/thesis/signal_samples/ossf_lep_type.pdf}\\
 \includegraphics[width=0.45\columnwidth]{/Users/sheenaschier/Documents/LaFiles/figures/thesis/signal_samples/ossf_dR_l1l2.pdf}
 \includegraphics[width=0.45\columnwidth]{/Users/sheenaschier/Documents/LaFiles/figures/thesis/signal_samples/ossf_dphi_j1met.pdf}\\
 \includegraphics[width=0.45\columnwidth]{/Users/sheenaschier/Documents/LaFiles/figures/thesis/signal_samples/ossf_mll.pdf}
 \includegraphics[width=0.45\columnwidth]{/Users/sheenaschier/Documents/LaFiles/figures/thesis/signal_samples/ossf_ptll.pdf}\\
   \caption{Kinematic distributions of signal samples}
   \label{fig:SigSample2}
 \end{figure}

\subsection{Compressed Slepton Samples}

 \FloatBarrier
 
 \section{Background Simulation}
 \subsection{V+Jets}
 \subsection{Diboson}
 \subsection{Top Quark}
 \subsection{Higgs}


\section{Derivation}
Describe details of the SUSY16 derivation used to select events from data

 \section{Definition of the Measurement}
 \label{sec:event}
\begin{itemize}
\item I don't think I need to say anything about the ntuples or framework.
\item Introduce the ideas of selecting physics objects and events in ATLAS
\item Overview of kinematic phase space that define our signal regions
\item The next chapter will talk about how objects are identified and selected so don't make any references to that stuff
\item Will talk about which events are chosen for this analysis
\item Will talk about how the signal regions are defined for the Higgsino and Slepton analyses.
\end{itemize}
Study signal MonteCarlos samples to understand the phenomenology of compressed higgsino and slepton production during and LHC collision and subsequent decay in the ALTAS detector.  These studies inform our choices choices for signal region cuts for the slepton and higgsino searches. 

\subsection{Discriminating Variables}
\label{sec:discvar}
$\met$, d phi j-met, min d phi jets-met, $\pt(j_i)$, Number of $b$-tagged jets $N_\mathrm{b-jets}$\\
Same flavour lepton pair with opposite charge, $\Delta R_{\ell\ell}$, $m_{\ell\ell}$, $m_{T2}^{m_{\chi}}$,$m_\text{T}^{\ell_1}$, $\met/\HT^\text{leptons}$, $m_{\tau\tau}$

  % https://gitlab.cern.ch/jeliu/atlas-susy-ew-softlepton/blob/master/pyCut/refactor/central/plot1d_signals_only.py
  \begin{figure}[tbp]
   % \centering
     \includegraphics[width=0.48\columnwidth]{/Users/sheenaschier/Documents/LaFiles/figures/thesis/higgsino_slep_signal_Rll_met0.pdf}
  %  \caption{No \met{} requirement (only truth filter).}
       \includegraphics[width=0.48\columnwidth]{/Users/sheenaschier/Documents/LaFiles/figures/thesis/higgsino_slep_signal_Rll_met100.pdf}\\
   % \caption{$\met{} > 100$ GeV.}
     \includegraphics[width=0.48\columnwidth]{/Users/sheenaschier/Documents/LaFiles/figures/thesis/higgsino_slep_signal_Rll_met200.pdf}
 %   \caption{$\met{} > 200$ GeV.}
     \includegraphics[width=0.48\columnwidth]{/Users/sheenaschier/Documents/LaFiles/figures/thesis/higgsino_slep_signal_Rll_met300.pdf}\\
%    \caption{$\met{} > 300$ GeV.}
   \caption{Comparison of Higgisno N2C1p (solid) and slepton (dashed) signals in the $R_{\ell\ell}$ variable for 10 GeV (dark) and 20 GeV (light) mass splittings. The \met{} here acts as a p    roxy for the boost of the system. Only a 2 signal lepton selection is applied.}
   \label{fig:Rll_signals only}
 \end{figure}
 
 \begin{figure}[tbp]
  \centering
  \includegraphics[width=0.48\columnwidth]{/Users/sheenaschier/Documents/LaFiles/figures/thesis/METoverHTLep_mll}
%\caption{Higgsinos}
  \includegraphics[width=0.48\columnwidth]{/Users/sheenaschier/Documents/LaFiles/figures/thesis/METoverHTLep_mT2}
%\caption{Sleptons}
 \caption{Distributions of $\met/H_{T}^{leptons}$ for the Higgsino (left) and Slepton (right) selections, after applying all signal region cuts except those on the $\met/H_{T}^{leptons}$, $m_{ll}$, and $m_{T2}$.  The black dashed line indicates the cut applied in the signal region; events in the region below the black line are rejected.}
 \label{fig:METoverHTLep2D}
 \end{figure}
 
 
  \begin{figure}
  \centering
  \input{/Users/sheenaschier/Documents/LaFiles/figures/thesis/ditau_schematic}
  \caption{Schematic illustrating the fully leptonic $(Z\to\tau\tau)$ + jets system motivating the construction of $m_{\tau\tau}$. }
  \label{fig:ditau_schematic}
  \end{figure}
  \subsection{Signal Regions}
 \begin{figure}[h!]
 \centering
 \includegraphics[scale=0.6]{/Users/sheenaschier/Documents/LaFiles/figures/thesis/cutflow_SF.pdf}
 \caption{Non-normalized cutflow with significance plot, showing how the significance for signal improves as more cuts are added.}
 \label{fig:cutflow_zn}
 \end{figure}
\subsubsection{Slepton Signal Regions}
This signal region based on MT2 cuts
\subsubsection{Higgsino Signal Regions}
This signal region based on Mll cuts




\chapter{Physics Object Reconstruction and Identification}
\label{ch:obj}
The term \textit{reconstruction} describes the process of interpreting signal output from the detector and using that information to make measurements associated with actual physics objects.  The ATLAS detector and its reconstruction algorithms are designed for efficient particle identification and precise energy and momentum measurement.  Reliable tracking and vertexing are the building blocks for efficient reconstruction and identification of most objects.  In this chapter, the assembly of tracks and vertices will first be described in Section~\ref{sec:obj:bb}.  Next, reconstruction are identification variables are defined for directly and indirectly observable objects in Section~\ref{sec:obj:reco}.  In ATLAS, these objects are; electrons, muons, jets, photons, and missing energy and momentum.  Lastly, Section~\ref{sec:obj:treat} describes the techniques of overlap removal and isolation correction of closely-spaced leptons as subsequent treatment of reconstructed objects before analysis.  

\section{The Building Blocks}
\label{sec:obj:bb}
%1704.07983.pdf - 1
%Salzburger_2015_J._Phys.__Conf._Ser._664_072042.pdf - 2

%luminosities $2*10^{33}cm^{-2}s^{-1}$ in 2008 -> $10^{34}cm^{-2}s^{-1}$
Track reconstruction, also called \textit{tracking}, provides the important information needed for primary and secondary vertex reconstruction, charged particle reconstruction, jet flavor tagging, and photon conversions; therefore, track reconstruction algorithms must be swift, concise, and perform with high efficiency, low fake rates and with proper resolution on tracking parameters.  In 2015, at the start of Run 2, the LHC extended the center of mass energy in proton-proton collisions to 13 TeV, and over the duration of Run 2, ramped-up the instantaneous luminosity pushing the average interactions per bunch crossing ($\mu$) to above 40 by the end of Run 2.  This extension of center of mass energy and instantaneous luminosity enhances the outlook of discovery while simultaneously slowing down track reconstruction and degrading its efficiency.  Events with jet showers in the TeV range and $\tau$ leptons and B-hadrons that traverse multiple ID layers before decaying, occur at rates high enough to be considered in optimizing track and cluster reconstruction in Run 2 \cite{aad}.  In the core of boosted hadronic jets and $\tau$ lepton decays, particles in flight do not scatter much as they traverse the inner tracking layers, making separate energy deposits in the discrete sensors hard to resolve and near-by tracks hard to distinguish from each other.  If tracking efficiency is low in events with high track density, mismeasurements are expected in identifying long-lived b-hadron and hadron $\tau$ decays and in calibrating the energy and mass of jets.  These mismeasurements will also cause induced \met, which is an important quantity for this search and many other beyond standard model searches.  

The first step in track reconstruction involves preprocessing Pixel, SCT, and TRT information. Event by event charged track reconstruction in the pixel and SCT detectors starts with clustering groups of pixels and strips in each sensor that respond to an energy deposition above a set threshold and share a common edge or corner.  These clusters form three-dimensional space-points that measure where a particle intersects the active material in the ID.  In the pixel detector, each particle corresponds to one space-point, while in the SCT, clusters must be combined from both sides of a strip layer to obtain a three-dimensional position measurement.  %Charge in the pixel sensors is often collected in more than one adjoining pixels \textcolor{red}{Maybe connect here hit resolution to cluster size and use plots from the IBL paper.. remembering to explain how impact parameter resolution is driven by the resolution of the hit closest to the primary vertex.} 

The next step in tracking is called track finding.  This involves combining Pixel and SCT hits into tracks seeds.  Three consecutive hits are required for a track seed, and seeds with an additional compatible cluster are sent to a Kalman filter.  In the last step, hits from all three of the tracking detectors are fit to make tracks using a global $\chi^2$ function.  These tracks are then given a score based on the fit quality and the number of holes and shared clusters.  Tracks that fall below the minimum allowable score are rejected.


Reconstructed tracks are characterized using five \textit{perigee} parameters at the point of closest approach to the beam axis.  
\begin{itemize}
\item \textbf{\textit{transverse impact parameter}} $d_0$ - track distance to the $z$-axis at the point of closest approach in the $x-y$ plane.
\item \textbf{\textit{longitudinal impact parameter}} $z_0$ - track coordinate along $z$ at the point of closest approach.
\item \textbf{\textit{azimuthal angle}} $\phi_0\equiv\tan^{-1} p_y/p_x$ - track angle to the $x$-axis in the $x$-$y$ plane.
\item \textbf{\textit{polar angle}} $\theta_0$ - track angle to the $z$-axis in the $R-z$ plane.
\item \textbf{\textit{charge over momentum}} $q/p$ - electric charge divided by the track momentum.
\end{itemize}
  \begin{figure}[tbp]
       \includegraphics[width=0.8\columnwidth]{/Users/sheenaschier/Documents/LaFiles/figures/thesis/eventselection/TrkPerigee.png}\\
   \caption{Sketch of ATLAS tracking parameters at the perigee in the $x-y$ plane (left) and the $R-z$ plane (right). }
   \label{fig:trkParam}
 \end{figure}
 
The primary vertex is defined as space position in the detector of the initial pp interaction.  Primary vertices are identified using inner detector tracks that satisfy a set of requirements.  For a track to be considered in the construction of a primary vertex, it must have $ \pt{} > 400\GeV$, $|\eta| < 2.5$, between 9 ($|\eta| \leq 1.65$) and 11 ($|\eta| > 1.65$) silicon hits, at least 1 hit in the IBL or B-Layer, a maximum of one shared pixel hit or two shared SCT hits, no holes in the pixel layers, and no more that one hole in the SCT layers.  Any primary vertex must have at least two associated tracks for reconstruction~\cite{1742-6596-898-4-042056}.  %The primary vertex track selection criteria is summarized in Table~\ref{tab:pvtrk}.
\iffalse
\begin{table}
\tiny
\centering
\begin{tabular}{l|l}
%pg 2 Boutle
  \small Track Kinematics & Track Hit Criteria  \\
  \hline
  $\pt > 400~\MeV$ & \\
  $|d_0|<4~\mathrm{mm}$ & \\
  \hline
\end{tabular}
\caption{Summary of primary vertex track selection}
\label{tab:pvtrk}
\end{table}
\fi
\FloatBarrier

\section{Particle Identification and Reconstruction}
\label{sec:obj:reco}

Reconstructed and identified particles in ATLAS are leptons($e, \mu, \tau$), photons, hadronic jets, which can further be identified as b-jets, and missing transverse momentum \met.  This analysis does not use $\tau$ reconstruction.  There are two categories of reconstructed objects; \textit{baseline}, which is the most inclusive definition of an object and is typically used for preliminary event selection and background modeling, and \textit{signal}, a more exclusive object definition that is a subset of \textit{baseline} and is typically used in defining signal events.  A summary of all the signal and baseline object definitions is given in Table~\ref{tab:objdef}.

    \begin{figure}[tbp]
       \includegraphics[width=.8\columnwidth]{/Users/sheenaschier/Documents/LaFiles/figures/thesis/eventselection/idSketch.png}\\
   \caption{Schematic of an electron in the ATLAS}
   \label{fig:idsketch}
 \end{figure}
Electron likelihood identification is a multivariate technique that uses signal and background probability density functions of discriminating variables to give an overall likelihood of being signal or background.  Figure~\ref{fig:idsketch} depicts an electron in ATLAS moving through the ID detectors and into the calorimeters.  Likelihood variables that discriminate on the tracking include: number of hits on the inner-most pixel layer, hits in the Pixel detector, hits in the SCT+Pixel detectors, transverse impact parameter $d_0$, transverse impact parameter significance ($|d_0/\sigma_{d_0}$), and fractional momentum lost in the detector, likelihood probability based on the transition radiation in the TRT, and track-cluster matching variables.  Likelihood variables that discriminate on calorimeter measurements include: the ratio of transverse energy in the TileCal to the energy in the LAr, the ratio of energy in the last LAr layer to the energy in the full LAr\footnote{This variable is only used for electrons with $\pt<80\GeV$.}, the lateral electromagnetic shower shape in the second LAr layer, shower width in the LAr strip layers.  Signal and background probabilities combine into a single discriminant on which a cut is applied to define a likelihood-based operating point.  Operating points in the electron likelihood identification menu are \textit{VeryLoose, Loose, LooseAndBLayer, Medium}, \textit{Tight}.  \textit{LooseAndBLayer} uses the same likelihood as \textit{Loose} and also requires a hit in the inner-most Pixel layer.  All operating points use the same discriminating variables to ensure tighter operating points are subsets of the more loose operating points.  The electron efficiencies for the \textit{Loose}, \textit{Medium}, and \textit{Tight} LH working points are compared in Figure~\ref{fig:lepEff}.
\begin{figure}[h!]
 \centering
 \includegraphics[width=0.49\columnwidth]{/Users/sheenaschier/Documents/LaFiles/figures/thesis/fakes/fig_01.pdf}
  \includegraphics[width=0.49\columnwidth]{/Users/sheenaschier/Documents/LaFiles/figures/thesis/fakes/fig_02.pdf}
 \caption{Electron efficiency in $\eta$ (left) and \et (right) in the \textit{Loose}, \textit{Medium}, and \textit{Tight} LH identification algorithms}
 \label{fig:lepEff}
 \end{figure}
 
 Muons in this analysis use a cut-based identification technique that first identifies muon tracks in the ID and MS and combines them to form complete muon tracks.  Identification working points are provided based on the muon reconstruction efficiency and background rejection they provide.  Muon ID \textit{Medium} is the default working point used by physics analyses in ATLAS \cite{muonid}.  The \textit{Medium} ID achieves over $95\%$ muon efficiency for muons $4\GeV<\pt<20\GeV$, and over $60\%$ background rejection.  Muon identification efficiencies measured versus muon \pt by the Muon Combined Performance Group in ATLAS are shown in Figure~\ref{fig:emuon}. 
 \begin{figure}[h!]
 \centering
 \includegraphics[width=0.8\columnwidth]{/Users/sheenaschier/Documents/LaFiles/figures/thesis/eventselection/muonEff}
 \caption{Reconstruction efficiency for \textit{Medium} muon identification working point as a function of muon \pt, in the region $0.1<|\eta|<2.5$ \cite{muon}.}
 \label{fig:emuon}
 \end{figure}
  
Lepton isolation is quantified by two main variables, track isolation and calorimeter isolation.  Track isolation is determined by the transverse momenta of tracks in some cone around the track with a radius determined by the lepton \pt.  Calorimeter isolation is dictated by the sum of the transverse energy in the topological clusters (topo clusters), which are cell clusters seeded by calorimeter cells with energy more than four times greater than the noise threshold in the cell.  Topo clusters are then expanded to neighboring cells with energy more than twice above the noise threshold, and finally a last layer of excited calorimeter cells are added to the cluster.  To measure the isolation energy, the lepton energy in the isolation topo cluster is removed and the topo cluster is corrected for pileup and any lepton energy that was not subtracted away.  Final isolation cuts using the track and calorimeter based isolation variables are are classified as either \textit{fixed cut} or \textit{gradient}.  Fixed cut means the working point provides fixed efficiencies across the $\eta-\pt$ plane.  Gradient means the efficiencies are \pt-variant, but still flat in $\eta$.  Like isolation working points are provided for for three grades of isolation: \textit{Loose, Medium,} and \textit{Tight}, and can be based on track isolation, calorimeter isolation, or both.  The \textit{Tight} working points will provide the best rejection of backgrounds, but the lowest efficiencies.
 
Baseline electrons are seeded from energy deposits in the EM calorimeter and reconstructed with algorithms using EM calorimeter clusters that are matched to inner detector tracks.  Baseline electrons must pass \pt{} threshold of 4.5\GeV~and exclusively travel through the central detector region $|\eta | < 2.47$.  A longitudinal impact parameter requirement of $|z_0sin\theta| < 0.5~\mathrm{mm}$ is also applied.  This analysis uses likelihood based identification criteria only.  Baseline electrons are required to satisfy \textit{VeryLooseLLH} identification while signal electrons must pass \textit{Tight} identification plus \textit{GradientLoose} isolation criteria.  Signal electrons also require transverse impact parameter significance $|d_0/\sigma(d_0)| < 5$.  Electron energy deposits in the LAr are generally narrow in $\eta$ and $\phi$ and mostly concentrated in the first two sampling layers.
\begin{figure}[h!]
 \centering
 \includegraphics[width=0.8\columnwidth]{/Users/sheenaschier/Documents/LaFiles/figures/thesis/eventselection/figaux_11.pdf}
 \caption{Signal lepton efficiencies for electrons and muons, averaged over all Higgsino and slepton samples. Efficiencies are shown for leptons within detector acceptance, and with lepton \pt within a factor of 3 of $\Delta~m(\tilde l\tilde\chi_1^0)$ for sleptons samples or within a factor of 3 of $\Delta~m(\tilde \chi_2^0\tilde\chi_1^0)/2$ for Higgsino samples. Uncertainty bands represent the range of efficiencies observed across all signal samples for the given \pt bin.  The $\eta$ dependance consistent with values reported in ATLAS combined performance papers.}
 \label{fig:sigeff}
 \end{figure}
 
Muon information primarily comes from tracks in the muon spectrometer that are often matched charged tracks in the inner detector.  Baseline muons are reconstructed with algorithms that combine tracks from the inner detector and muon spectrometer to form muon candidates.  They must pass \pt{} threshold of 4 \GeV and be in fiducial region $|\eta | < 2.5$. Like with electrons, muon likelihood identification is used, and the discriminating variables are extended to include information from tracks in the muon spectrometer.  Baseline muons are also expected to satisfy \textit{Medium} identification standards  and have a transverse impact parameter $|z_0sin\theta| < 0.5~\mathrm{mm}$.  Signal muons must also satisfy \textit{FixedCutTightTrackOnly} isolation criteria and a transverse impact parameter significance of $|d_0/\sigma(d_0)| < 3$.

Lepton identification, isolation, impact parameter cuts, fiducial acceptance and \pt threshold all effect the lepton efficiencies and result in the efficiencies shown in Figure~\ref{fig:sigeff} that range from roughly $50\%$ for low \pt muons and up to $90\%$ for higher \pt.  For electrons the efficiencies are roughly $20\%$ for low \pt electrons, and increase up to $\sim65\%$.  This is the average over signal samples that fall within some range, where the most compressed signal samples used to evaluate the low \pt leptons and so on.

Baseline jets are built from locally-calibrated three-dimensional topologically clustered calorimeter cells.  Topological clustering here is the same as described in the discussion of lepton isolation.  Jets are constructed using anti-$K_t$ clustering algorithms~\cite{antikt} with radius parameter R = 0.4.   Baseline jets must pass \pt{} threshold of 20 \GeV and be in fiducial region $|\eta | < 4.5$.   Also, jets within $|\eta | < 2.5$ originating from b-hadrons are tagged with the 2-dimensional multivariate b-tagging algorithm \textit{MV2c10} with an 85\% working point.  Signal jets are further restricted to fiducial region $|\eta | < 2.8$, and pileup jets are removed using the jet vertex tagger (JVT) with \textit{Medium} working point efficiency applied to jets with $\pt>60\GeV$ and $|\eta|<2.4$. 

\begin{table}
\tiny
 \centering
  \begin{tabular}{l||c|c|c}
 \hline
\small Selection Criteria & \small \textbf{Electrons} & \small \textbf{Muons} & \small \textbf{Jets}  \\
 \hline
 \hline
\small \textbf{Baseline} &  & & \\ 
 \hline
\small Reco Algorithm &\small \textit{author 16 veto}  &&\\
\small Kinematic&\small $\pt{} > 4.5$ \GeV,  &\small $\pt{} > 4$ \GeV,  &\small $\pt{} > 20$ \GeV,\\
&\small $|\eta | < 2.47$&\small $|\eta | < 2.5$& $|\eta | < 4.5$\\
\small Impact Parameter &\small $|z_0sin\theta|< 0.5$ mm &\small $|z_0sin\theta|< 0.5$ mm &\\
& -- & -- &\\
\small Identification &\small \textit{VeryLooseLLH}  &\small \textit{Medium}  &                 \\
\small Isolation & --    & --  &   \\
\small Clustering & & &\small Anti-$K_t$ R = 0.4 EMTopo\\
\small Jet Vertex Tagging &&& -- \\
\small \textit{b}-tagging &&& -- \\
 \hline
 \hline
 \small \textbf{Signal} &  & \\ 
 \hline
 \small Reco Algorithm &\small \textit{author 16 veto}  &&\\
\small Kinematic&\small $\pt{} > 4.5 \GeV$, &\small $\pt{} > 4 \GeV$,  &\small $\pt{} > 30$ \GeV,\\
&\small $|\eta | < 2.47$&\small $|\eta | < 2.5$&\small $|\eta | < 2.8$\\
\small Impact Parameter &\small $|z_0sin\theta|< 0.5~mm$,&\small $|z_0sin\theta|< 0.5~mm$, &\\
&\small $|d_0/\sigma(d_0)|< 5$&\small $|d_0/\sigma(d_0)|< 3$&\\
\small Identification &\small \textit{Tight} &\small \textit{Medium}   &                 \\
\small Isolation &\small \textit{GradientLoose}     & \small \textit{FixedCutTightTrackOnly} &    \\
\small Clustering & & &\small Anti-$K_t$ R = 0.4 EMTopo\\
\small Jet Vertex Tagging &&&\small \textit{JVT Medium}\\
\small \textit{b}-tagging &&&\small $\pt{} > 20$, $|\eta | < 2.5$ \\
&&& \small MV2c10 FixedCutBeff 85\% \\
  \end{tabular}
  \caption{Summary of object definitions}
  \label{tab:objdef}
\end{table}

Well calibrated energy and momentum measurements of the directly observable objects is important for construction of the particles that traverse the detector without interacting.  These "missing" particles carry away energy and momentum which is recovered with energy and momentum conservation in the plane transverse to the beam pipe.  The vector quantity missing transverse momentum \pt{} is the negative vector sum of the transverse momentum of all the identified physics objects (electrons, muons, jets, photons) plus an additional soft term.  The scalar magnitude of the missing transverse momentum vector gives the missing transverse energy \met. The soft term is constructed from all the tracks not associated with any physics object, but are associated with the primary vertex.  Therefore, \met{} is adjusted for the best possible calibration of the jets and other identified physics objects and still independent of pileup in the soft term.  Pileup jets are removed with a jet vertexing technique that matches jets to primary vertices with track-vertex tagging.


\section{Special Treatment of Reconstructed Objects}
\label{sec:obj:treat}
Once objects are reconstructed and identified, special algorithms often need to be performed before these objects can be used.  For this analysis, these final steps were the removal of overlapping objects and the isolation correction of closely-spaced leptons.

Overlap removal is performed to prevent double counting of physics objects by removing objects based in their separation $\Delta R$ in detector coordinates $\eta$ and $\phi$, given by:
\begin{equation}
\Delta R_{p_1p_2} = \sqrt{(\eta_{p_1}-\eta_{p_2})^2+(\phi_{p_1}-\phi_{p_2})^2}
\end{equation} 
First, jet-electron overlap removal is performed.  If $\Delta R_{jet, electron}$ is less than 0.2 and the the jet is not tagged as a b-jet, the jet is removed and the electron is kept.  If the jet is identified as a b-jet, then the jet is kept and the electron object is removed since the electron is most likely from the semi-leptonic decay of a B-hadron.  If $\Delta R_{jet, electron}$ is less than 0.4, we remove the electron and keep the jet.  Similarly, if the $\Delta R_{jet, muon}$ is less than 0.4, we remove the muon and keep the jet unless the jet has less than three tracks; in which case, the muon will be kept and the jet is discarded.  Lastly, we perform overlap removal on photons and other objects.  It is common that electron and muon objects will also be included in the photon container since they pass the Ecal shower requirements, so typically, overlapping photons and leptons will result in the photon object being removed from the photon container.  If $\Delta R_{photon, electron}$ is less than 0.4 we remove the photon and keep the electron.  If $\Delta R_{photon, muon}$ is less than 0.4, we remove the photon and keep the muon.  If $\Delta R_{photon, jet}$ is less than 0.4, we keep the photon and remove the jet.  

 Soft leptons in a boosted system often have small angular separation, especially when they are products of a low-mass $Z^*$ decay.  These boosted leptons often lie within each others isolation cones, leading to efficiency loss for very small mass-splittings.  The top row of Figures~ \ref{fig:EffRll_ISOCorr} and ~\ref{fig:EffMll_ISOCorr} illustrate the efficiency loss for nearby leptons within $\Delta R < 0.4$ and dilepton invariant mass ($m_{\ell\ell}$) $<5\GeV$ using an electroweakino signal sample with $m_{\tilde\chi_2^0}-m_{\tilde\chi_1^0}=10\GeV$.  This loss is corrected for using a dedicated tool that checks for baseline leptons that fail the isolation criteria to due to another nearby lepton within it's isolation cone and removes tracks associated with the nearby lepton from the track isolation sum.  If the nearby lepton is an electron, the topocluster $E_T$ is also removed from the calorimeter isolation sum.  The corrected isolation variables are then reanalyzed using the original isolation working point.  The bottom rows of Figures~ \ref{fig:EffRll_ISOCorr} and ~\ref{fig:EffMll_ISOCorr} exhibits the recovered dilepton efficiency in simulation after applying the isolation correction tool.  Figure~\ref{fig:nearbylepiso} shows the effect of this correction on low invariant mass dilepton pairs in data.  The data are chosen such that $\Delta\phi(\met, p_{t}^{j_1})<1.5$ to avoid the signal region, which selects $\Delta\phi(\met, p_{t}^{j_1})>2.0$, as explained in Chapter~\ref{ch:sr}.   
  \begin{figure}[tbp]
   % \centering
     \includegraphics[width=0.48\columnwidth]{/Users/sheenaschier/Documents/LaFiles/figures/thesis/eventselection/eff_EE_Rll_110_100_NoOS_NoISO.pdf}
       \includegraphics[width=0.48\columnwidth]{/Users/sheenaschier/Documents/LaFiles/figures/thesis/eventselection/eff_MM_Rll_110_100_GradLoose_NoOS_NoISO.pdf}\\
     \includegraphics[width=0.48\columnwidth]{/Users/sheenaschier/Documents/LaFiles/figures/thesis/eventselection/eff_EE_Rll_110_100_NoOS.pdf}
     \includegraphics[width=0.48\columnwidth]{/Users/sheenaschier/Documents/LaFiles/figures/thesis/eventselection/eff_MM_Rll_110_100_GradLoose_NoOS.pdf}\\
   \caption{Dilepton $\Delta$ R distribution before LepIsoCorrection (top) and after LepIsoCorrection (bottom) for the $ee$-channel (left) and $\mu\mu$-channel (right), using electroweakino signal samples with m($\tilde\chi_2^0\tilde\chi_1^0$) $(110, 100)\GeV$.}
   \label{fig:EffRll_ISOCorr}
 \end{figure}

  \begin{figure}[tbp]
   % \centering
     \includegraphics[width=0.48\columnwidth]{/Users/sheenaschier/Documents/LaFiles/figures/thesis/eventselection/eff_EE_Mll_110_100_NoOS_NoISO.pdf}
       \includegraphics[width=0.48\columnwidth]{/Users/sheenaschier/Documents/LaFiles/figures/thesis/eventselection/eff_MM_Mll_110_100_GradLoose_NoOS_NoISO.pdf}\\
     \includegraphics[width=0.48\columnwidth]{/Users/sheenaschier/Documents/LaFiles/figures/thesis/eventselection/eff_EE_Mll_110_100_NoOS.pdf}
     \includegraphics[width=0.48\columnwidth]{/Users/sheenaschier/Documents/LaFiles/figures/thesis/eventselection/eff_MM_Mll_110_100_GradLoose_NoOS.pdf}\\
   \caption{Dilepton invarient mass distribution before LepIsoCorrection (top) and after LepIsoCorrection (bottom) for the $ee$-channel (left) and $\mu\mu$-channel (right), using electroweakino signal samples with m($\tilde\chi_2^0\tilde\chi_1^0$) $(110, 100)\GeV$.}
   \label{fig:EffMll_ISOCorr}
 \end{figure}
 \begin{figure}[tbp]
 \centering
  \includegraphics[width=0.8\columnwidth,trim=1.2cm 0cm 1.9cm 0cm,clip]{/Users/sheenaschier/Documents/LaFiles/figures/thesis/nearbylepiso.pdf}
 %  \includegraphics[width=0.48\columnwidth]{/Users/sheenaschier/Documents/LaFiles/figures/thesis/nearbylepiso_signal.pdf}
  \caption{Impact of the \texttt{NearbyLepIsoCorrection} tool on the efficiency of low-mass dilepton pairs in data.  The data are shown in a region with $\Delta\phi(\met, p_{t}^{j1})<1.5$ to avoid the signal region.  Events are triggered with the inclusive-\met{} trigger.  The red trend shows events with two baseline leptons without applying any isolation; the green shows the impact of applying \texttt{GradientLoose} isolation; the blue shows the result of the \texttt{NearbyLepIsoCorrection} applied to the \texttt{GradientLoose} sample.  %(right) Impact of the correction on a Higgsino LSP signal sample with $\Delta m(\chi,\chi)=3\GeV$.
  }
 \label{fig:nearbylepiso}
 \end{figure}
 

 
 



\chapter{Signal Region Optimization}
 \label{ch:sr}
This analysis relies on external predictions of signal and background processes in data to help interpret observations, and for observations to be meaningful, it is imperative to search for new physics where its presence is not excessively drowned out by SM backgrounds.  To achieve this, a signal enriched region in phase space, called a \textit{signal region} (SR), is defined through a series of selection cuts on kinematic variables targeting events where predicted signal yields display a significant excess over the estimated backgrounds, which are discussed in Chapter~\ref{ch:bkg}.   
 
 In the chapter, the discriminating variables that define the Higgsino and slepton signal regions are expounded first in Section~\ref{sec:sr:discvar}, then the signal regions are defined in Section~\ref{sec:sr:srdef}.  To exploit the Higgsino and sleptons models fully, they are treated by separate analyses in independent signal regions, but the compressed nature of these models makes many of their SR cuts overlap.  Section~\ref{sec:sr:srdef} is broken into sections, first detailing the common SR selection cuts in Section~\ref{sec:sr:commom}:, then the Higgsino SR specific cuts and the slepton SR specific cuts in Section~\ref{sec:sr:mll} and~\ref{sec:sr:mt2}. 
 
\section{Discriminating Variables}
\label{sec:sr:discvar}
This section will define all the discriminating variables used to define the signal regions, then the next section will detail how they are applied to the SRs, and what benefits or limitations they present.  These discriminating variables are presented in terms of three classifications, those that exploit the lepton information, those that exploit the topology of the jets and the \met{}, and those that exploit both.  

The variables that depend only on lepton information are: lepton flavor, lepton charge, the distance between a lepton pair ($\Delta R_{\ell\ell}$), and the invariant mass of a lepton pair ($m_{\ell\ell}$).  Lepton flavor refers to it being an electron or a muon, and the lepton charge is its positive or negative electric charge.  $\Delta R_{\ell\ell}$ is defined in terms of detector angles $\eta$ and $\phi$, as:
\begin{equation}
\Delta R_{\ell\ell} = \sqrt{(\eta_{\ell_1}-\eta_{\ell_2})^2+(\phi_{\ell_1}-\phi_{\ell_2})^2}
\end{equation} 
The invariant mass is taken from the energy-momentum 4-vector in Equation~\ref{eq:invarm}, and the invariant mass of two leptons is the magnitude of the summed lepton energy-momentum vectors, as in Equation~\ref{eq:mll}.
 \begin{equation}
m^2 = E^2-\mathbf{p}^2
\label{eq:invarm}
\end{equation} 
 \begin{equation}
m_{\ell\ell} = \sqrt{(E_{\ell_1}+E_{\ell_2})^2 - (\mathbf{p}_{\ell_1}+\mathbf{p}_{\ell_2})^2}
\label{eq:mll}
\end{equation} 

The variables that exploit the jet and \met topology are: \met{}, the \pt of the leading\footnote{In reference to particle objects, the term \textit{leading} always refers that type of object in an event with the highest measured \pt{}.  \textit{Subleading} always refers to the second highest \pt{} object in the event.} jet ($\pt(j_1)$), the number of $b$-tagged jets ($N_\mathrm{b-jets}$), the angular separation between missing transverse momentum and the leading jet ($|\Delta\phi(j_1, \pt^{miss})|$), and the minimum angular separation between missing transverse momentum and the nearest reconstructed jet ($min|\Delta\phi(jets, \pt^{miss})|$).  The angular separation between two objects in ATLAS is measured in terms of the azimuthal $\phi$, so $|\Delta\phi(j_1, \pt^{miss})|$ is simply the difference in the $\phi$ coordinates of the leading jet and \met{} in the interval [-$\pi$, $\pi$].  Similarly, to calculate the minimum separation between the \met and the reconstructed jets, $|\Delta\phi(j, \pt^{miss})|$ is measured for each jet and the minimum value is selected.

The variables that use combined information from the leptons, jets, and \met{} are: the transverse mass of the leading lepton and the missing transverse momentum ($m_\text{T}^{\ell_1}$), the ratio of \met{} over the scalar sum of the lepton transverse momenta ($\met/\HT^\text{lep}$), the di-tau invariant mass ($m_{\tau\tau}$), and the stransverse mass ($m_{T2}^{m_{\chi}}$).  The transverse mass of the combined leading lepton and missing transverse momentum is defined by the energy-momentum 4-vector using the transverse quantities:
\begin{equation}
\label{eq:mt}
m_\text{T}^{\ell_1} = \sqrt{2(E^{\ell_1}_TE^{miss}_T-\pt^{\ell_1}\pt^{miss})} 
\end{equation}
$m_{T2}^{m_{\chi}}$ is similar to $m_\text{T}^{\ell_1}$ in that it relates lepton transverse momentum and \met{}, but it is a bit more complicated.  To understand the $m_{T2}^{m_{\chi}}$ variable, one must consider a process like in Figure~\ref{fig:fn1} where a pp collision produces a slepton pair  which immediately decay to a visible lepton and and invisible LSP.  $m_{T2}^{m_{\chi}}$, detailed in Eq~\ref{eq:mt2}, essentially determines a bound on the masses of the invisible particles as a function of the \pt of the two leading leptons and the measured missing transverse momentum.  It is mathematically defined by the minimum value of $q_T$ for the maximum of the transverse mass of the leptons and invisible particles for some set value of $m_\chi$.  
\begin{equation}
\label{eq:mt2}
m^{m_\chi}_{T2}(\pt^{\ell_1}, \pt^{\ell_2}, \pt^{miss})  = \underaccent{\mathbf{q}_T}{\text{min}}\big(\text{max}[m_T(\pt^{\ell_1}, q_T; m_\chi), m_T(\pt^{\ell_2}, \pt^{miss}-q_T; m_\chi)]\big)
\end{equation}
Here, $q_T$ is the sum of the transverse momentum vectors of each of the invisible particles, as in Eq~\ref{eq:qt}.  The transverse mass of the leptons and invisible particles is shown explicitly in Eq~\ref{eq:mtchi}.
\begin{equation}
\label{eq:qt}
q_T = \pt^{\chi,1}+\pt^{\chi,2}
\end{equation}
\begin{equation}
 m_T\left(\mathbf{p}_T^{\ell}, \mathbf{q}_T, m_\chi\right)= \sqrt{m_\ell^2 + m_\chi^2 + 2\left(E_T^\ell E_T^q -\mathbf{p}_T\cdot \mathbf{q}_T\right)}
 \label{eq:mtchi}
 \end{equation}
   \begin{figure}
  \centering
  \input{/Users/sheenaschier/Documents/LaFiles/figures/thesis/ditau_schematic}
  \caption{Schematic illustrating the fully leptonic $(Z\to\tau\tau)$ + jets system motivating the construction of $m_{\tau\tau}$. }
  \label{fig:ditau_schematic}
  \end{figure}
 Lastly, the di-tau invariant mass, $m_{\tau\tau}\left(p_{\ell_1}, p_{\ell_2}, \mathbf{p}_\mathrm{T}^\mathrm{miss}\right) $ is used by this analysis to veto the $Z\rightarrow\tau\tau$ background.   The purpose of this variable is to reconstruct the di-tau invariant mass of the fully leptonic $Z\rightarrow\tau\tau$ process from the measurable quantities in the event, which are the 4-momenta of the two leptons and the missing transverse momentum.  A ($Z\rightarrow\tau\tau$) + jets event within the signal region relies on the $Z$ boson recoiling off the jet activity, boosting the decaying di-tau system oppositely along the jet axis.  A schematic of this process is displayed in Figure~\ref{fig:ditau_schematic}.  This kick from the jets causes the leptons and neutrinos to remain close to a single axis, so the 4-momentum of the invisible neutrino system $p_{\nu_i}$, for the $i_{th}$ $\tau$ in the event, can be well approximated by a simple rescaling of the lepton 4-momentum. The di-tau invariant mass is defined in Eq~\ref{eq:mtt}.
 \begin{equation}
 \label{eq:mtt}
 m^2_{\tau\tau}\left(p_{\ell_1}, p_{\ell_2}, \mathbf{p}_\mathrm{T}^\mathrm{miss}\right) \equiv 2p_{\ell_1}\cdot p_{\ell_2}(1+\xi_1)(1+\xi_2)
 \end{equation}
 where $\xi_1$ and $\xi_2$ are determined by solving Eq~\ref{eq:xi}, and the sign of $m^2_{\tau\tau}$ is given by Eq~\ref{eq:mtt2}.
  \begin{equation}
   \label{eq:xi}
  \mathbf{p}_\mathrm{T}^\mathrm{miss} = \xi_1\mathbf{p}_\mathrm{T}^\mathrm{\ell_1}+\xi_2\mathbf{p}_\mathrm{T}^\mathrm{\ell_2}
   \end{equation}
 \begin{equation}
 \label{eq:mtt2}
 m_{\tau\tau}\left(p_{\ell_1}, p_{\ell_2}, \mathbf{p}_\mathrm{T}^\mathrm{miss}\right) =
\begin{cases}
\hphantom{-}\sqrt{m_{\tau\tau}^2}~;               & m_{\tau\tau}^2 \geq 0,\\
 -\sqrt{\left| m_{\tau\tau}^2\right|}~; & m_{\tau\tau}^2 < 0.
\end{cases} 
 \end{equation} 
 %Z. Han, G. D. Kribs, A. Martin and A. Menon, Hunting quasidegenerate Higgsinos, Phys. Rev. D 89 (2014) 075007, arXiv: 1401.1235 [hep-ph].
 % H. Baer, A. Mustafayev and X. Tata, Monojet plus soft dilepton signal from light higgsino pair production at LHC14, Phys. Rev. D 90 (2014) 115007, arXiv: 1409.7058 [hep-ph].
 %A. Barr and J. Scoville, A boost for the EW SUSY hunt: monojet-like search for compressed sleptons at LHC14 with 100 fb?1, JHEP 04 (2015) 147, arXiv: 1501.02511 [hep-ph].
  
\section{Signal Region Definitions}
\label{sec:sr:srdef}
 There are two types of signal region used in this analysis: Higgsino SRs and slepton SRs.  Within each of the kinds of signal region, both inclusive and exclusive SR are defined, and these will be detailed in Sections~\ref{sec:sr:mt2} and~\ref{sec:sr:mt2}.  Among the selection cuts that define the Higgsino and slepton SRs are many that overlap, and these will be laid out in Section~\ref{sec:sr:commom}.

\subsection{Common Signal Regions}
\label{sec:sr:commom}
SR events are required to contain two signal leptons, and intermediate amount of \met, and at least one jet.  The leading lepton is required to have $\pt > 5~\GeV$ and the subleading lepton is required to have $\pt > 4.5~\GeV$, or $\pt > 4.5~\GeV$ for muons.  Furthermore, the two leptons are required to make a same-flavor-opposite-sign\footnote{\textit{Sign} is another term for positive or negative electric charge} (SFOS) pair.  For Higgsino signals, this prefers the dominant leptonic decay mode of the Higgsino, via an off-shell $Z^*$.  In slepton signals, light flavor sleptons always decay to two oppositely charged leptons of the same flavor.  Also, selecting OSSF pairs allows the SR to target the decays of this analysis and leaves different flavor or same signed lepton pairs to be exploited in the control and validation regions.  Collinear leptons from photon conversions are filtered out with a restriction on the minimum $\Delta R_{\ell\ell}$ between the leptons of $0.05$ and an invariant mass cut of $m_{\ell\ell} > 1 ~\GeV$.  

\met{} is an important variable when there are heavy invisible particle in the final state, and also because this analysis uses inclusive \met{} triggers.  A \met{} threshold of $200~\GeV$ is imposed to be fully efficient in the \met{} trigger, even though the optimal cut to achieve the best signal over background discrimination might be lower.  This limitation on the trigger is an unfortunate consequence of the increasing luminosity.  The \met is also correlated to $\pt(j_1)$.  Since the leptons are so light compared to the mass of the LSP, the boost from the hadronic recoil is mostly given to the \met{}. So, if the $\pt(j_1)$ threshold is too high, it will reduce the sensitivity in \met{}, but if the $\pt(j_1)$ threshold is too low, other subleading jets may contribute significantly to the recoil of the system.  The leading jet \pt threshold is set to $100~\GeV$.  The intermediate \met{} requirement sculpts the topology of the signal to prefer events where the direction of the \met and the direction of the leading jet are opposite each other in the transverse plane.  Because of the small mass-splittings between the electroweakinos or the sleptons and the LSP, the LSPs will typically only produce significant enough \met{} to pass the $\met{} > 200~\GeV$  cut when they are aligned opposite to the hadronic initial state radiation in the transverse plane.  A cut on $\Delta\phi(j, \pt^{miss}) > 2.0$ is established to take advantage of this topology and cut away backgrounds that are more agnostic to it.  

\met from jet mismeasurements tends to align the the $\pt^{miss}$ with some of the jets, leading to a small $\Delta\phi$ between them.  This mostly occurs in QCD and $Z$+jets events.  $min\Delta\phi(jets, \pt^{miss})$ considers the minimum angular separation between $\pt^{miss}$ and the nearest reconstructed jet, so this variable should have a minimum requirement to reduce the induced \met{}.  The cut is set at $min\Delta\phi(jets, \pt^{miss}) > 0.4$.  Top quark backgrounds are significantly enhanced in b-tagged jets while the Higgsino and slepton signals are not; therefore, a b-jet veto is set to discriminate against these processes.  Lastly, the region $m_{\tau\tau}^2$ = [0, 160] GeV is vetoed to reduce the Z($\rightarrow\tau\tau$)+jets background.  All of the common SR selection cuts are summarized in Table~\ref{tab:cSR}.
 \begin{table}[tbp]
 \centering
 \renewcommand{\arraystretch}{1.1}
 \begin{tabular}{ll}
 \hline
 Variable                                                & Requirement    \\
 \hline
  $N_\mathrm{leptons}$                                    & Exactly two signal leptons\\
 Lepton charge and flavor                               & $e^\pm e^\mp$ or $\mu^\pm \mu^\mp$\\
 %Author 16 Electrons (ambiguous conversions)             & Veto\\
 Leading electron (muon) $\pt^{\ell_1}$                  & $>5 (5)$ GeV             \\
 Subleading electron (muon) $\pt^{\ell_2}$               & $>4.5 (4)$ GeV             \\
  $m_{\ell\ell}$                                          &   [1, 3] or [3.2, 60] \GeV  \\
 $\Delta R_{\ell\ell}$                                   & $> 0.05$           \\
 \met                                                    & $>200$~GeV                 \\
 Leading jet $\pt(j_1)$                                  & $>100$ GeV              \\
 $|\Delta\phi(j_{1},\met)|$                                           & $>2.0$                     \\
 min$|\Delta\phi(all~jets,\met)|$                                       & $>0.4$                   \\
 $N_\mathrm{b-jet}^{20}$, 85\% WP                        & Exactly zero               \\
 $m_{\tau\tau}$                                          & $<0$ or $>160$ \GeV          \\
 \hline
 \end{tabular}
 \caption{Summary of common Higgsino and slepton SR cuts}
 \label{tab:cSR}
 \end{table}
 \FloatBarrier
 %%%%%%%%%%%%%%%%%%%%%%%%%%%%%%%%%%%%%%%%%%%%%%%%
 \subsection{Higgsino Signal Regions}
\label{sec:sr:mll}
For electroweakino signals, the leading lepton and the $\pt^{miss}$ are likely to have a small separation, and therefore a small $m_T^{\ell_1}$.  In background events with W bosons, the peak of the $m_T^{\ell_1}$ distribution is near the mass of the W, so cutting on $m_T(\pt^{\ell_1} < 70~\GeV$ can reduce the contribution from $t\bar{t}$, $WW/WZ$, and $W(\rightarrow\ell\nu)$+jets backgrounds.  The leptons in compressed electroweakino signals are also likely to have small separation, while most backgrounds do not.  For this reason, $\Delta R_{\ell\ell}$ tends to be a powerful discriminator for Higgsinos and a cut of $\Delta R_{\ell\ell} < 2.0$ is added to Higgsino SR selection.  Slepton SRs do not include this cut because the lepton topology is quite different.  Figure~\ref{fig:Rll_signals only} compares the $\Delta R_{\ell\ell}$ distributions of the Higgsino $\chi_2^0\chi_1^+$ and the slepton signals for $10~\GeV$ and $20~\GeV$ mass-splittings.  When no \met cut is applied, the Higgsino and slepton samples seem to show orthogonal response in $\Delta R_{\ell\ell}$.  The is because, unlike with Higgsino decays, the leptons come from separate sleptons, and without any boost to the system, which increases the \met, the sleptons are back-to-back.  Once \met is required, the sleptons become more collimated and $\Delta R_{\ell\ell}$ flattens, while for Higgsino samples, the shape in $\Delta R_{\ell\ell}$ becomes more pronounced. 
  \begin{figure}[tbp]
   % \centering
     \includegraphics[width=0.48\columnwidth]{/Users/sheenaschier/Documents/LaFiles/figures/thesis/higgsino_slep_signal_Rll_met0.pdf}
  %  \caption{No \met{} requirement (only truth filter).}
 %      \includegraphics[width=0.48\columnwidth]{/Users/sheenaschier/Documents/LaFiles/figures/thesis/higgsino_slep_signal_Rll_met100.pdf}\\
   % \caption{$\met{} > 100$ GeV.}
     \includegraphics[width=0.48\columnwidth]{/Users/sheenaschier/Documents/LaFiles/figures/thesis/higgsino_slep_signal_Rll_met200.pdf}\\
 %   \caption{$\met{} > 200$ GeV.}
 %    \includegraphics[width=0.48\columnwidth]{/Users/sheenaschier/Documents/LaFiles/figures/thesis/higgsino_slep_signal_Rll_met300.pdf}\\
%    \caption{$\met{} > 300$ GeV.}
   \caption{Comparison of Higgisno N2C1p (solid) and slepton (dashed) signals in the $R_{\ell\ell}$ variable for 10 GeV (dark) and 20 GeV (light) mass splittings. The \met{} here acts as a proxy for the boost of the system. Only a 2 signal lepton selection is applied.}
   \label{fig:Rll_signals only}
 \end{figure}

 For intermediate values of \met, SM diboson and $t\bar{t}$ background processes produce hard leptons, likewise diminishing the values of $\met/H_T^{lep}$.  In compressed electroweakino and slepton events, the \met{} is mostly from the boost of the hadronic recoil.  The recoiling jet affects the heavier invisible particle much more than it effects the lighter leptons; therefore, these signal events prefer larger values of $\met/H_T^{lep}$.  Figure~\ref{fig:METoverHTmll} shows the $\met/H_{T}^{lep}$ distribution for Higgsino samples after applying all the signal region cuts except $\met/H_{T}^{leptons}$ and $m_{ll}$
 \begin{figure}[tbp]
  \centering
  \includegraphics[width=0.7\columnwidth]{/Users/sheenaschier/Documents/LaFiles/figures/thesis/METoverHTLep_mll}
 \caption{Distributions of $\met/H_{T}^{lep}$ for the Higgsino selections, after applying all signal region cuts except those on the $\met/H_{T}^{lep}$ and $m_{ll}$.  The red solid line indicates the cut applied in the signal region; events in the region below the red line are rejected.}
 \label{fig:METoverHTmll}
 \end{figure}

Inclusive and exclusive Higgsino SRs are binned in $m_{ll}$. The dilepton mass $m_{\ell\ell}$ can both suppress backgrounds as well as exploit special features of the Higgsino model.  \textcolor{red}{Continue on about the kinematic endpoint..}  
 \begin{table}[]
 \tiny
\centering
\resizebox{\linewidth}{!}{
\begin{tabular}{llllllll}
\hline
Variable                  & Selection cut\\%\multicolumn{7}{l}{\textbf{\textcolor{red}{Selections optimised for Higgsinos}}}                                \\
\hline
$\met/\HT^\text{leptons}$ & \multicolumn{7}{l}{$> \text{Max}\left(5.0, 15 - 2 \cdot m_{\ell\ell}/\text{~GeV} \right)$}\\
$\Delta R_{\ell\ell}$     & \multicolumn{7}{l}{$<2.0$} \\
$m_\text{T}^{\ell_1}$     & \multicolumn{7}{l}{$<70$ GeV}                                                      \\
\hline
SRee-, SRmm-              & eMLLa   & eMLLb   & eMLLc    & eMLLd      & eMLLe      & eMLLf      & eMLLg     \\
$m_{\ell\ell}$ [GeV]      & $[1,3]$ & $[3.2,5]$ & $[5,10]$ & $[10, 20]$ & $[20, 30]$ & $[30, 40]$ & $[40,60]$ \\
\hline
SRSF-                     & iMLLa   & iMLLb   & iMLLc    & iMLLd      & iMLLe      & iMLLf      & iMLLg     \\
$m_{\ell\ell}$ [GeV]      & $<3$    & $<5$    & $<10$    & $<20$      & $<30$      & $<40$      & $<60$     \\
\hline
\end{tabular}
}
\caption{Higgsino specific SR cuts.}
\end{table}
\FloatBarrier
 
\subsection{Slepton Signal Regions}
\label{sec:sr:mt2}
Figure~\ref{fig:METoverHTmt2} shows the $\met/H_{T}^{lep}$ distribution for slepton samples after applying all the signal region cuts except $\met/H_{T}^{leptons}$ and $m_{T2}$.
 \begin{figure}[tbp]
  \centering
  \includegraphics[width=0.7\columnwidth]{/Users/sheenaschier/Documents/LaFiles/figures/thesis/METoverHTLep_mT2}
%\caption{Sleptons}
 \caption{Distributions of $\met/H_{T}^{lep}$ for the slepton selections, after applying all signal region cuts except those on the $\met/H_{T}^{leptons}$ and $m_{T2}$.  The red solid line indicates the cut applied in the signal region; events in the region below the red line are rejected. \textcolor{red}{Change to updated plot.}}
 \label{fig:METoverHTmt2}
 \end{figure}
 
 Inclusive and exclusive slepton SRs are binned $m^{m_\chi}_{T2}$.  Slepton signals a kinematic endpoint defined by the 'stransverse' mass $m^{m_\chi}_{T2}$, which is a function of the measures momentum of the leading two leptons $p_{\ell_1}$, $p_{\ell_2}$, the measured \pt, and the hypothesized invisible particle mass $m_\chi$.  \textcolor{red}{explain how the $m^{m_\chi}_{T2}$ is actually restricted in signal samples.. and more on all of this in section such and such}.    For the pair of semi-invisible particles in the slepton signal \textcolor{red}{is this the slepton pair that decay to leptons and neutralinos?}, $m^{m_\chi}_{T2}$ is always less than the parent slepton mass $m_{\tilde\ell}$ when the hypothesized $m_\chi$ mass is set to the neutralino mass in the underlying process.  This defines the lower kinematic endpoint in $m^{m_\chi}_{T2}$ for slepton signals.  Requiring $m^{m_\chi}_{T2} < m_{\tilde\ell}$, various mass scenarios can be probed in the slepton-neutrino mass plane.  Standard Model backgrounds not display this kind of feature since the invisible particles are massless neutrinos, therefor there is not such enhancement in background when making this requirement.  In fact, in the compressed region of the slepton-neutrino mass  plane, events populate a narrower region in $m_{T2}$, giving this variable more discriminating power.  

\begin{table}[]
 \tiny
\centering
\resizebox{\linewidth}{!}{
\begin{tabular}{llllllll}
\hline
Variable                  & \multicolumn{7}{l}{\textbf{\textcolor{red}{Selections optimised for sleptons}}}                           \\
\hline
$\met/\HT^\text{leptons}$ & \multicolumn{7}{l}{$> \text{Max}\left(3.0, 15 - 2 \cdot \left[m_\text{T2}^{100} / \text{~GeV}-100\right] \right)$}\\
\hline
SRee-, SRmm-              & eMT2a        & eMT2b       & eMT2c       & eMT2d        & eMT2e        & eMT2f      & \\
$m_\text{T2}^{100}$ [GeV] & $[100,102]$ & $[102,105]$ & $[105,110]$ & $[110, 120]$ & $[120, 130]$ & $\geq 130$ &\\
\hline
SRSF-                     & iMT2a       & iMT2b       & iMT2c       & iMT2d        & iMT2e       &  iMT2f      &\\
$m_\text{T2}^{100}$ [GeV] & $<102$      & $<105$      & $<110$      & $<120$       & $<130$      &  $\geq 100$ & \\
\hline
\end{tabular}
}
\caption{Slepton specific signal region cuts}
\label{tab:SRMLLMT2}
\end{table}

\iffalse
\section{Conclusion}
\label{sec:concl}
Here say some final summarizing remarks and reference these final cutflow plots.  \textcolor{red}{Say something about non-normalized cutflow with significance plot showing how the significance for signal improves as more cuts are added.}

\begin{figure}[h!]
\centering
\begin{subfigure}[b]{0.47\textwidth}
\includegraphics[width=\textwidth]{/Users/sheenaschier/Documents/LaFiles/figures/thesis/signal_regions/cutflow_bkg_SF.pdf}
\caption{Normalized cutflow, background-only}
\end{subfigure}
 \begin{subfigure}[b]{0.47\textwidth}
\includegraphics[width=\textwidth]{/Users/sheenaschier/Documents/LaFiles/figures/thesis/signal_regions/cutflow_norm_SF.pdf}
 \caption{Normalized cutflow, with signal}
\end{subfigure}
\begin{subfigure}[b]{0.58\textwidth}
\includegraphics[width=\textwidth]{/Users/sheenaschier/Documents/LaFiles/figures/thesis/signal_regions/cutflow_SF.pdf}
 \caption{Non-normalized cutflow with significance plot.}
\end{subfigure}

 \caption{Normalized Cutflow for background-only, signal-inclusive, and  non-normalized cutflow with significance plots. }
 \label{fig:cutflow_norm}
\end{figure}

\textcolor{red}{Show signal region plots?}
\fi




\chapter{Backgrounds}
%\label{sec:bkg}

\section{Fake Lepton Background}
Show the calculations and results of the fake estimate.  The method will be explained in a dedicated chapter/section.

\section{$t\bar{t}$ Background}
\label{sec:top}
Mainly b-jet requirement

\section{Drell-Yan Background}
\label{sec:mettrigger}
Off-shell $z\rightarrow ll$ events.

 \begin{figure}
 \centering
    \includegraphics[width=0.6\columnwidth]{/Users/sheenaschier/Documents/LaFiles/figures/thesis/backgrounds/dataCR_mll_MuMu_pre.pdf}
  % \caption{Di-muon.}
 \includegraphics[width=0.6\columnwidth]{/Users/sheenaschier/Documents/LaFiles/figures/thesis/backgrounds/dataCR_mll_ElEl_pre.pdf}
%  \caption{Di-electron.}
  \includegraphics[width=0.6\columnwidth]{/Users/sheenaschier/Documents/LaFiles/figures/thesis/backgrounds/dataCR_mll_DF_pre.pdf}
%  \caption{Different flavour ($e\mu+\mu e$).}
  \caption{Data events passing inclusive \met{} triggers with opposite sign baseline leptons in the dilepton invariant mass $m_{\ell\ell}$ spectrum. The $\Delta\phi(j_1, \mathbf{p}_\mathrm{    T}^\mathrm{miss})$ variable is inverted to ensure this is orthogonal to the signal region.}
  \label{fig:mll_data}
 \end{figure}
\FloatBarrier

\section{Z+jets Background}
\label{sec:elements}


 \input{/Users/sheenaschier/Documents/LaFiles/figures/thesis/control_plots}

\chapter{Fake Factor Method}
%\label{sec:ff}
There are two main types of backgrounds, irreducible and reducible. Irreducible backgrounds are Standard Model processes that produce the same particle final state as our BSM final state.  In this case, Monte Carlo simulation is robust enough to model these background processes so their rates can be estimated in the data.  Reducible backgrounds arise from Standard Model processes that should not produce the same final state as the signal; and yet, because of mismeasurements inside the detector, these events can still pass signal selection cuts.  For low pt dilepton signals, the reducible fake background dominates and primarily comes from W+jets events where one jets is misidentified as a lepton.  Monte Carlo simulation does not model the detector shortcomings that lead to these mismeasurements very well, so the best estimate of this background must comes from data.  The "fake factor" method is a data driven approach to modeling backgrounds from particle misidentification in the detector by estimating the lepton fake rate with a set of data kinematically enriched in events producing fake leptons.  The background estimate is validated in an orthogonal control region before it is estimated in the signal region. 

The rest of this chapter goes as follows:  In Section~\ref{sec:FFintro}, I will introduce fake leptons backgrounds more in detail, then in Section~\ref{sec:FFdesc} I will give a general overview of the method used to estimate the fake lepton background for this analysis.   The fake factor method applied to low \pt{} di-lepton events is explained for electrons and muons separately in Section~\ref{sec:FFmethod} and the results are summarized in Section~\ref{sec:FFcon}.

%%%%%%%%%%%%%%%%%%%%
\section{Introduction}
\label{sec:FFintro}
\begin{itemize}
\item lepton identification and misidentification
\item Compare production cross-sections of signal and W+jets processes
\item Sources of electron and muon misidentification
\item How to model backgrounds from misidentification (can't use MC, must choose data driven method)
\item Concept of fake factor method
\item Primary fake background is W+jets (multi-jet is miniscule...  how do I qualify this?)
\item Rest of chapter describes FF method in the context of my analysis
\end{itemize}
Efficient lepton identification techniques make leptons powerful discriminators in ATLAS physics searches with large background rejection and heavily suppressed QCD multi-jets.  Jet suppression is very high in the range of lepton $\pt > 20 GeV$ but degrades at lower lepton $\pt$.  Misidentified electrons can be true but non-prompt electrons from photon conversions and heavy-flavor decays, where there is a real electron in the event that does not originate at the primary vertex like true, prompt electrons or they can be charged hadrons where the  hadronic jet activity in the detector fakes an electron.  
\begin{figure}[h!]
 \centering
 \includegraphics[width=0.6\columnwidth]{/Users/sheenaschier/Documents/LaFiles/figures/thesis/fakes/fig_01.pdf}
  \includegraphics[width=0.6\columnwidth]{/Users/sheenaschier/Documents/LaFiles/figures/thesis/fakes/fig_02.pdf}
 \caption{Electron identification efficiency}
 \label{fig:electronID}
 \end{figure}

**Make plot comparing production cross-sections, at least for W+jets and Higgsino/Slepton, and maybe even include other reducible background production cross-sections.
 \FloatBarrier
 
 %%%%%%%%%%%%%%%%%%%%%%%%%
\section{Description of Fake Factor Method}
\label{sec:FFdesc}
\begin{itemize}
\item General description of fake factor method (measurement in control region then extrapolated to signal region)
\item We know what the signal region is already (described in Chapter ..)
\item Control region, meant to select events with misidentified leptons, is defined by signal region cuts but with one lepton chosen to satisfy a selection criteria that is more likely to include more misidentified particles than that used in the analysis signal region.  A control region designed to capture W+jets events where a jet is misidentified as a lepton would be the same as the signal region requiring two leptons, but only one lepton is defined as an analysis lepton, while the other has at least one orthogonal selection criteria cut that makes it easier to include jets in the container of lepton identified in this particular way.
\item Electrons and muons are treated separately
\item Fake factor is the ratio of leptons passing analysis lepton identification criteria to the leptons passing anti-identification criteria, measured in a region of kinematic phase space contrived to be enriched in fake leptons.  This will be considered as the fake factor measurement region in this thesis
\item Fake background contribution estimated by scaling the number of selected events in the control region by the fake factor.
\item Separate samples are used to measure the fake factor and count the number of events in the control region
\item control region and anti-ID lepton definition have contamination from sources that are not from the background of interest
\end{itemize}

Explain signal and control regions as well as fake factor measurement and application regions. 
\begin{figure}
\centering
 \input{/Users/sheenaschier/Documents/LaFiles/figures/thesis/fakes/fakefactor_schematic.tex}
 \caption{Schematic illustrating the fake factor method to estimate the fake lepton contribution in the signal region.}
 \label{fig:fake_schematic}
 \end{figure}
 
The fake factors are computed from events  with $m_{\mathrm{T}}<40~\GeV$, using the distributions in Fig.~\ref{fig:elec_FF_dists_pt}, as:
\begin{equation}
  F(\pt) = \frac{\mathrm{Numerator}_{\mathrm{data}} - \mathrm{Numerator}_{\mathrm{MC}}}{\mathrm{Denominator}_{\mathrm{data}} - \mathrm{Denominator}_{\mathrm{MC}}}
\end{equation}
  \FloatBarrier
  
  %%%%%%%%%%%%%%%%%%%%%%%%%%%%%%%%%%%
  \section{Fake Factor Method Applied to Low-$\pt$ Di-lepton Events}
  \label{sec:FFmethod}
  \begin{itemize}
  \item Describe data samples used for FF measurement
  \item Describe data samples used for fake background estimate
  \item Describe the single lepton triggers and how the pre-scales are unfolded to normalize the entire 2015+2016 dataset to $10pb^{-1}$
\end{itemize}
\begin{table}[tbp]
  \centering
  \begin{tabular}{lll}
    \hline
    Trigger                             &\multicolumn{2}{c}{Prescaled Luminosity [\ipb]}\\
                                        &2015           &2016\\
    \hline
    \texttt{HLT\_e5\_lhvloose}            &0.1               &0.1    \\
    \texttt{HLT\_e10\_lhvloose\_L1EM7}     &0.5               &0.8    \\
    \texttt{HLT\_e15\_lhvloose\_L1EM13VH}  &5.5               &9    \\
    \texttt{HLT\_e20\_lhvloose}           &10                &17    \\
    \hline
    \texttt{HLT\_mu4}                    &0.5               &0.5    \\
    \texttt{HLT\_mu10}                   &2.3               &2.5    \\
    \texttt{HLT\_mu14}                   &25                &14    \\
    \texttt{HLT\_mu18}                   &26                &48    \\
    \hline
  \end{tabular}
  \caption{Pre-scaled single-lepton triggers from 2015 and 2016 used to compute the lepton fake factors. The pre-scaled luminosities shown are taken from \texttt{LumiCalc}.}
  \label{tab:prescaledtrigs}
\end{table}

  \FloatBarrier
  %%%%%%%%%%%%%%%%%%%
  \subsection{Fake Lepton Composition}
 Monte Carlo studies of fake and non-prompt lepton composition is done separately for events with opposite sign lepton pairs and events with same-sign lepton pairs.  In the MC samples, there is a variable MCTruthClassifier that determines lepton categories based on their source.  Real prompt leptons fall into two categories: \textit{isolated} and $lep\rightarrow \gamma \rightarrow lep$, which refers to truth matched leptons that arise from a Bremsstrahlung to photon conversion process. Fake and non-prompt leptons occupy the remaining categories: \textit{non-isolated}, which are mostly from heavy flavor decays, \textit{photons}, which are either photons faking leptons or actualy leptons from photon conversions, \textit{hadron}, which are from light flavor decays, and \textit{unknown, unknown electron, or unknown muon}, which are primarily from pile-up.\\
(\textbf{Describe Figure~\ref{fig:elMC} and Figure~\ref{fig:muMC}}). Sources of fake leptons in the di-muon and di-electron signal regions mostly come from heavy flavor decays, and fake leptons in the  di-muonand di-electron control regions are primarily from light flavor decays.  One important result from this study is the similarity of fake lepton contribution between the opposite sign lepton pain and same sign lepton pair events.  This gives confidence that the same-sign validation region can be successfully used to validate our fake background predictions in the data without accidentally unblinding our signal region and biasing the results.  

  
\begin{figure}[htb]
        \centering
        \includegraphics[width=.48\textwidth]{/Users/sheenaschier/Documents/LaFiles/figures/thesis/fakes/fakeLeptonComposition/626_cdsComments_mu_SR2_lep2Pt.pdf}
       \includegraphics[width=.48\textwidth]{/Users/sheenaschier/Documents/LaFiles/figures/thesis/fakes/fakeLeptonComposition/626_cds_noIso_mu_QCR2_lep2Pt.pdf}
      \includegraphics[width=.48\textwidth]{/Users/sheenaschier/Documents/LaFiles/figures/thesis/fakes/fakeLeptonComposition/626_cds_ss_wIso_mu_SR2_lep2Pt.pdf}
        \includegraphics[width=.48\textwidth]{/Users/sheenaschier/Documents/LaFiles/figures/thesis/fakes/fakeLeptonComposition/626_cds_ss_mu_QCR2_lep2Pt.pdf}
        \caption{Fake lepton composition as a function of leading and subleading lepton $p_{T}$, with and without prompt (``Isolated'' plus ``lep$\to$gamma$\to$lep'') leptons, for opposite sign muon pairs in the signal region.}
        \label{fig:muMC}
\end{figure}

 
\begin{figure}[htb]
        \centering
        \includegraphics[width=.48\textwidth]{/Users/sheenaschier/Documents/LaFiles/figures/thesis/fakes/fakeLeptonComposition/725_cdsComments_el_SR2_lep2Pt.pdf}
        \includegraphics[width=.48\textwidth]{/Users/sheenaschier/Documents/LaFiles/figures/thesis/fakes/fakeLeptonComposition/725_cds_noIso_el_QCR2_lep2Pt.pdf}
        \includegraphics[width=.48\textwidth]{/Users/sheenaschier/Documents/LaFiles/figures/thesis/fakes/fakeLeptonComposition/725_cds_ss_wIso_el_SR2_lep2Pt.pdf}
          \includegraphics[width=.48\textwidth]{/Users/sheenaschier/Documents/LaFiles/figures/thesis/fakes/fakeLeptonComposition/725_cds_ss_el_QCR2_lep2Pt.pdf}
        \caption{Fake lepton composition in opposite sign signal and control region as a function of leading and subleading lepton $p_{T}$, with and without prompt (``Isolated'' plus ``lep$\to$gamma$\to$lep'') leptons, for opposite sign electron pairs in the signal region.}
        \label{fig:elMC}
\end{figure}


 \FloatBarrier
 


%%%%%%%%%%%%%%%%%%%%%%%
 \subsection{Anti-identified Lepton Definitions}
 \begin{itemize}
 \item Anti-ID definition chosen to enhance fake and non-prompt leptons while suppressing real prompt leptons.
 \item Enhancement is obtained by easing or inverting identification cuts used to suppress lepton misidentifiaction
 \item Tighter anti-ID cuts reduces systematic uncertainties on the fake background prediction.
 \item Tighter anti-ID cuts increases the statistical uncertainty on the fake background prediction.
 \end{itemize}

 %%%%%%%%%% 
\subsubsection{Anti-ID Muons}


ID muons used for the fake factor calculation are the same as signal muons defined in Chapter~\ref{sec:event}, which are baseline muons that must pass \texttt{FixedCutTightTrackOnly} isolation and $|d_0/\sigma(d_0)|<3.0$.  Anti-ID muons are also baseline muons, but instead of requiring they pass the isolation and $d_0$ significance requirements of the ID muons, they instead must fail the \texttt{FixedCutTightTrackOnly} isolation or $|d_0/\sigma(d_0)|<3.0$ criteria\footnote{Failing both the isolation and the $d_0$ significance cut still satisfies the anti-ID definition.}. Both the ID and anti-ID muons are required to pass the $|z_0\sin\theta| < 0.5$~mm requirement to reduce the impact of pileup.  One notable difference with respect to the signal muon requirements is that the muon-jet overlap removal is relaxed when performing the fake factor measurement\footnote{This enhances the statistics used for deriving the fake factors, and is motivated by the observation that the muon-jet overlap removal is primarily designed to reduce the number of heavy flavor decays which are inadvertently being classified as signal muons.}.  A summary of the ID and anti-ID muon definitions are summarized in Table~\ref{tab:AllMuDefs}

The decomposition of anti-ID muons in all events according to which set of ID criteria failed is shown in Fig~\ref{fig:muDeco}. The  $m_{T}$ distribution is plotted over the entire $m_{T}$ range, while the $\met$, \pt{} and $\eta$ distributions are all shown for $m_{T} <40~\GeV$.  Note that these distributions are separated into categories: one for events with exactly zero $b$-jets, and and another for events with one or more $b$-jets. \textbf{Here explain more about the 2 categories mentioned}

\begin{table}[!htb]
\begin{center}
\begin{tabular}{c|c}
\hline
Signal Muon Definition  & Anti-ID Muon Definition \\
\hline \hline
\multicolumn{2}{c}{$\pt > 4~\GeV$}      \\
\multicolumn{2}{c}{$\abseta < 2.5$ }     \\
\multicolumn{2}{c}{$|z_0\sin\theta| < 0.5$~mm} \\
\multicolumn{2}{c}{Pass \textit{Medium} Identification}     \\
$|d_0/\sigma(d_0)| < 3$  &   \textbf{(}$|d_0/\sigma(d_0)| > 3$ \textbf{\textit{or}}\\
Pass \textit{FixedCutTightTrackOnly} Isolation  & Fail \textit{FixedCutTightTrackOnly} Isolation\textbf{)} \\   \\
\hline
\end{tabular}
\caption{Summary of muon definitions.}
\label{tab:AllMuDefs}
\end{center}
\end{table}

\begin{figure}
        \centering
        \includegraphics[width=.4\textwidth]{/Users/sheenaschier/Documents/LaFiles/figures/thesis/fakes/FF_muon/AID_deco_Mt_b0}
                \includegraphics[width=.4\textwidth]{/Users/sheenaschier/Documents/LaFiles/figures/thesis/fakes/FF_muon/AID_deco_Mt_b1}
        \includegraphics[width=.4\textwidth]{/Users/sheenaschier/Documents/LaFiles/figures/thesis/fakes/FF_muon/AID_deco_MET_b0}
        \includegraphics[width=.4\textwidth]{/Users/sheenaschier/Documents/LaFiles/figures/thesis/fakes/FF_muon/AID_deco_MET_b1}\\
                \includegraphics[width=.4\textwidth]{/Users/sheenaschier/Documents/LaFiles/figures/thesis/fakes/FF_muon/AID_deco_AntiIDmuPt_b0}
                        \includegraphics[width=.4\textwidth]{/Users/sheenaschier/Documents/LaFiles/figures/thesis/fakes/FF_muon/AID_deco_AntiIDmuPt_b1}
        \includegraphics[width=.4\textwidth]{/Users/sheenaschier/Documents/LaFiles/figures/thesis/fakes/FF_muon/AID_deco_AntiIDmuEta_b0}
        \includegraphics[width=.4\textwidth]{/Users/sheenaschier/Documents/LaFiles/figures/thesis/fakes/FF_muon/AID_deco_AntiIDmuEta_b1}\\
        \caption{Anti-ID muon composition in events with exactly zero $b$-jets(left) and one or more $b$-jets(right) as a function of $m_{T}$, $\met$, muon \pt{}, and muon $\eta$. All but the $m_{T}$ distribution corresponds to events with $m_{T} < 40~GeV$.}
        \label{fig:muDeco}
\end{figure}


  \FloatBarrier

%%%%%%%%
\subsubsection{Anti-ID Electrons}


ID electrons are the same as signal electrons defined in Section~\ref{sec:event}, which are baseline electrons that also pass \texttt{TightLLH} PID, \texttt{GradientLoose} isolation, and $|d_0/\sigma(d_0)|<5.0$ . Anti-ID electrons are also baseline electrons that pass \texttt{LooseAndBLayerLLH} PID but fail one of the signal selection criteria, i.e.~they are required to fail at least one of the \texttt{TightLLH}, \texttt{GradientLoose}, or $|d_0/\sigma(d_0)|<5.0$ requirements. Studies motivating the definition of the anti-ID electrons were performed and are documented in this section. All ID and anti-ID electrons are required to pass the $|z_0\sin\theta| < 0.5$~mm requirement to reduce the impact of pileup. The composition of anti-ID electrons in the fake factor signal samples according to which set of ID criteria failed is shown in Fig~\ref{fig:elDeco}. The $m_{T}$ distribution of this decomposition is plotted over the entire $m_{T}$ spectrum, while the $\met$, \pt{} and $\eta$ distributions are all shown for $m_{T} <40~\GeV$.


\begin{table}[!htb]
\begin{center}
\begin{tabular}{c|c}
\hline
Signal Electron Definition  & Anti-ID Electron Definition \\
\hline \hline
\multicolumn{2}{c}{$\pt > 4.5~\GeV$}      \\
\multicolumn{2}{c}{$\abseta < 2.47$ }     \\
\multicolumn{2}{c}{$|z_0\sin\theta| < 0.5$~mm} \\
\multicolumn{2}{c}{Electron \textit{author} $!= 16$}\\
Pass \textit{Tight} Identification & Pass \textit{LooseAbdBLayer} Identification\\
%Pass \textit{Tight} Identification}  &  Pass \textit{LooseAndBLayer} Identification\\
      &             \textbf{(}Fail \textit{Tight} Identification \textbf{\textit{or}} \\
Pass \textit{GradientLoose} Isolation  & Fail \textit{GradientLoose} Isolation \textbf{\textit{or}} \\   
$|d_0/\sigma(d_0)| < 5$  &   $|d_0/\sigma(d_0)| > 5$\textbf{)} \\
\hline
\end{tabular}
\caption{Summary of electron definitions.}
\label{tab:AllElDefs}
\end{center}
\end{table}

\begin{figure}[htb]
        \centering
        \includegraphics[width=.4\textwidth]{/Users/sheenaschier/Documents/LaFiles/figures/thesis/fakes/FF_electron/AID_deco_Mt}
        \includegraphics[width=.4\textwidth]{/Users/sheenaschier/Documents/LaFiles/figures/thesis/fakes/FF_electron/AID_deco_MET}
         \includegraphics[width=.4\textwidth]{/Users/sheenaschier/Documents/LaFiles/figures/thesis/fakes/FF_electron/AID_deco_AntiIDelPt}
        \includegraphics[width=.4\textwidth]{/Users/sheenaschier/Documents/LaFiles/figures/thesis/fakes/FF_electron/AID_deco_AntiIDelEta}
        \caption{Fake electron composition as a function of $m_{T}$, $\met$, electron \pt{}, electron $\eta$. All distributions corresponds to events with $m_{T} < 40~GeV$, excluding the $m_{T}$ distribution. }
        \label{fig:elDeco}
\end{figure}


\begin{itemize}
\item Dedicated study was done to find anti-ID electron definition that well models the source of fake electron background. (Struggling with where to place this in this section.. before or after introducing the anti-ID definitions used)
\item Goal is to reduce the systematic uncertainty of the fake electron estimate
\item Tighter electron identification enhances fraction of heavy flavor decays
\item Requiring tracks to have a hit in the b-layer reduces fraction of fakes from conversions
\item Loose or Medium isolation requirement narrows source of fakes towards heavy and light hadronic decays
\item Requiring a large $d_0/\sigma_{d_0}$ can increase the fraction of heavy flavor decays and conversions
\item Deciding which anti-ID definitions best models electron fake backgrounds is a balancing act.
\item \textbf{Need help thinking through how in depth to go with this part. }

\end{itemize}

 \begin{figure}
 \centering
 \begin{subfigure}[b]{0.47\textwidth}
    \includegraphics[width=\textwidth]{/Users/sheenaschier/Documents/LaFiles/figures/thesis/fakes/antiIDStudies/AllMC_ee_SR_lep2Pt.pdf}
    \caption{Signal lepton;\\ 9.99 MC W+jet events.}
    \end{subfigure}
     \begin{subfigure}[b]{0.47\textwidth}
  \includegraphics[width=\textwidth]{/Users/sheenaschier/Documents/LaFiles/figures/thesis/fakes/antiIDStudies/AllMC_ee_VeryLoose_FailSignal_lep2Pt.pdf}
 \caption{ VeryLoose \& !signal;\\ 433.02 MC W+jet events.}
 \end{subfigure}
  \begin{subfigure}[b]{0.47\textwidth}
     \includegraphics[width=\textwidth]{/Users/sheenaschier/Documents/LaFiles/figures/thesis/fakes/antiIDStudies/AllMC_ee_VeryLooseBL_FailSignal_lep2Pt.pdf}
      \caption{VeryLoose \& PassBL \& !signal;\\ 313.38 MC W+jet events.}
 \end{subfigure}
   \begin{subfigure}[b]{0.47\textwidth}
     \includegraphics[width=\textwidth]{/Users/sheenaschier/Documents/LaFiles/figures/thesis/fakes/antiIDStudies/AllMC_ee_LooseBL_FailSignal_lep2Pt.pdf}
      \caption{Loose \& PassBL \&\& !signal;\\ 168.85 MC W+jet events.}
 \end{subfigure}       
    \begin{subfigure}[b]{0.46\textwidth}
     \includegraphics[width=\textwidth]{/Users/sheenaschier/Documents/LaFiles/figures/thesis/fakes/antiIDStudies/AllMC_ee_Medium_FailSignal_lep2Pt.pdf}
      \caption{Loose \&\& Medium \& !signal;\\ 75.28 MC W+jet events.}
 \end{subfigure}
    \begin{subfigure}[b]{0.46\textwidth}
     \includegraphics[width=\textwidth]{/Users/sheenaschier/Documents/LaFiles/figures/thesis/fakes/antiIDStudies/AllMC_ee_LooseBL_D0SigGt_OR_Medium_lep2Pt.pdf}
      \caption{Medium$||$(Loose \& PassBL \& $d_0/\sigma_{d_0}> 1.5$) \& !signal; 86.28 MC W+jet events.}
 \end{subfigure} 
 
    \caption{Fake lepton composition as a function of the subleading lepton \pt. }
 \label{fig:LeptonIDComposition}
\end{figure}


  \FloatBarrier
  
 \subsection{Fake Factor Measurement}
 
 
 \subsubsection{Muon Fake Factors}
 Both data and MC contributions to the numerator and denominator samples in the single-muon trigger sample are normalized to 10~\ipb, to remove the effects of the prescales in the data.  The MC is then re-scaled to the data in events with $\met{}>200$~\GeV, a kinematic region expected to pure in prompt leptons.  For events with exactly 0 $b$-jets, the MC re-scaling factor for numerator muons is $1.01 \pm 0.13$, for denominator muons it is $1.20\pm 0.29$. For events with one or more $b$-jets, the MC re-scaling factor for numerator muons is $1.24 \pm 0.20$, for denominator muons it is $7.34\pm 5.00$. If instead, the MC is re-scaled to match the data for events with $m_{T} > 100$~\GeV, a region that should also be pure in prompt leptons, the re-scaling factors for events with exactly 0 $b$-jets are $2.37 \pm 0.10$ for numerator muons and $11.68 \pm 2.28$ for denominator muons; events with one or more $b$-jets have re-scale factors $1.60 \pm 0.06$ for numerator muons and $10.41 \pm 6.34$ for denominator muons. The re-scaling factors vary significantly between the two methods but the fake factors themselves exhibit small changes between the two methods and can be used as a systematic uncertainty.

Distributions of \met{} and $m_{T}$ for numerator and denominator muons for events with exactly zero $b$-jets are shown in Fig.~\ref{fig:muon_FF_dists_b0}, and for events with one or more $b$-jets in Fig.~\ref{fig:muon_FF_dists_b1}.  Muon \pt{} distributions for events with exactly zero $b$-jet are shown in Fig.~\ref{fig:muon_FF_dists_pt_b0}, and for events with one or more $b$-jets in Fig.~\ref{fig:muon_FF_dists_pt_b1}.

% mT, MET, and lepton pT plots for ID, anti-ID
\begin{figure}[tbp]
  \centering
  \includegraphics[width=0.48\columnwidth]{/Users/sheenaschier/Documents/LaFiles/figures/thesis/fakes/FF_muon/IDb0_CR_MET}
  \includegraphics[width=0.48\columnwidth]{/Users/sheenaschier/Documents/LaFiles/figures/thesis/fakes/FF_muon/IDb0_CR_Mt}\\
  \includegraphics[width=0.48\columnwidth]{/Users/sheenaschier/Documents/LaFiles/figures/thesis/fakes/FF_muon/AIDb0_CR_MET}
  \includegraphics[width=0.48\columnwidth]{/Users/sheenaschier/Documents/LaFiles/figures/thesis/fakes/FF_muon/AIDb0_CR_Mt}
  \caption{The \met{} (left) and  $m_{T}$ (right) distributions for numerator (top) and denominator (bottom) muons in the prescaled single-lepton-trigger sample for events with exactly zero $b$-jets.  MC has been scaled to the data in the $\met > 200~\GeV$ region.}
  \label{fig:muon_FF_dists_b0}
\end{figure}

\begin{figure}[tbp]
  \centering
  \includegraphics[width=0.48\columnwidth]{/Users/sheenaschier/Documents/LaFiles/figures/thesis/fakes/FF_muon/IDb1_CR_MET}
  \includegraphics[width=0.48\columnwidth]{/Users/sheenaschier/Documents/LaFiles/figures/thesis/fakes/FF_muon/IDb1_CR_Mt}\\
  \includegraphics[width=0.48\columnwidth]{/Users/sheenaschier/Documents/LaFiles/figures/thesis/fakes/FF_muon/AIDb1_CR_MET}
  \includegraphics[width=0.48\columnwidth]{/Users/sheenaschier/Documents/LaFiles/figures/thesis/fakes/FF_muon/AIDb1_CR_Mt}
  \caption{The \met{} (left) and $m_{T}$ (right) distributions for numerator (top) and denominator (bottom) muons in the prescaled single-lepton-trigger sample for events with one or more $b$-jets.  MC has been scaled to the data in the $\met > 200~\GeV$ region.}
  \label{fig:muon_FF_dists_b1}
\end{figure}

\begin{figure}[tbp]
  \centering
  \includegraphics[width=0.48\textwidth]{/Users/sheenaschier/Documents/LaFiles/figures/thesis/fakes/FF_muon/IDb0_SR_IDmuPt}
  \includegraphics[width=0.48\textwidth]{/Users/sheenaschier/Documents/LaFiles/figures/thesis/fakes/FF_muon/AIDb0_SR_AntiIDmuPt}
  \caption{Muon \pt{} for numerator (left) and denominator (right) objects in the prescaled single-muon trigger sample for events with $m_{T} < 40~ GeV$.  MC has been scaled to the data in the $m_{T} > 100~\GeV$ region. Distributions from~\cite{Boerner:2231917}.}
  \label{fig:muon_FF_dists_pt_b0}
\end{figure}

\begin{figure}[tbp]
  \centering
  \includegraphics[width=0.48\textwidth]{/Users/sheenaschier/Documents/LaFiles/figures/thesis/fakes/FF_muon/IDb1_SR_IDmuPt}
  \includegraphics[width=0.48\textwidth]{/Users/sheenaschier/Documents/LaFiles/figures/thesis/fakes/FF_muon/AIDb1_SR_AntiIDmuPt}
  \caption{Muon \pt{} for numerator (left) and denominator (right) objects in the prescaled single-muon trigger sample for events with $m_{T}< 40~ GeV$.  MC has been scaled to the data in the $m_{T} > 100~\GeV$ region. Distributions from~\cite{Boerner:2231917}.}
  \label{fig:muon_FF_dists_pt_b1}
\end{figure}


% actual fake factors
The fake factors are computed using events with $m_{\mathrm{T}}<40~\GeV$, using the distribution in Figs.~\ref{fig:muon_FF_dists_pt_b0} and \ref{fig:muon_FF_dists_pt_b1}, as
\begin{equation}
  F(\pt) = \frac{\mathrm{Numerator}_{\mathrm{data}} - \mathrm{Numerator}_{\mathrm{MC}}}{\mathrm{Denominator}_{\mathrm{data}} - \mathrm{Denominator}_{\mathrm{MC}}}
\end{equation}
where the fake factor $F$ is computed in discrete \pt{} bins with different single-muon triggers applied. The specific trigger applied to each range in lepton \pt{} was chosen to reduce the effect of the trigger turn on and maintain good statistics. Muon \pt{} distributions for the prescaled triggers shown in Fig.~\ref{fig:mu_triggers} are arbitrarily normalized to 1~\ipb.  HLT\_mu4 trigger is required for muon \pt{} $4 - 11~ \GeV$, HLT\_mu10 is required for muon \pt{} $11- 15~\GeV$, HLT\_mu14 is required for muon \pt{} $15-20~\GeV$, and HLT\_mu18 is required for muon \pt{} $>20~\GeV$. A table of these triggers and corresponding \pt{} range is shown in Table~\ref{tab:muon_trigger_range}  %The final fake factors are shown in Table~\ref{fig:muon_FF_values}.

% Trigger distributions in lepton pt
\begin{figure}[tbp]
  \centering
  \includegraphics[width=0.48\columnwidth]{/Users/sheenaschier/Documents/LaFiles/figures/thesis/fakes/FF_muon/IDmuonTriggers}
  \includegraphics[width=0.48\columnwidth]{/Users/sheenaschier/Documents/LaFiles/figures/thesis/fakes/FF_muon/AntiIDmuonTriggers}\\
  \caption{The numerator muon (left) and denominator denominator (right) \pt{} distributions for prescaled single-muon triggers, normalized to 1~\ipb{}. Blue curve: HLT\_mu4, red curve: HLT\_mu10, purple curve: HLT\_mu14, green curve: HLT\_mu18.}
  \label{fig:mu_triggers}
\end{figure}
\begin{table}[tbp]
  \centering
  \begin{tabular}{|c|c|}
    \hline
    el trigger  & \pt{} range [\GeV]\\
    \hline
    HLT\_mu4 &4 --11  \\
    HLT\_mu10 & 11--15  \\
    HLT\_mu14 & 18--20  \\
    HLT\_mu18 & $>$ 20  \\
    \hline
  \end{tabular}
  \caption{Single-muon triggers used for fake factor computation and their corresponding \pt{} range.}
  \label{tab:muon_trigger_range}
\end{table}

Muon fake factors depend strongly on muon \pt, but also display a systematic dependence on the leading jet \pt{}.  Unlike the electron fake factors, there is also a separate dependence on $b$-jet multiplicity.  Fig.~\ref{fig:muon_FF_hist_noCut} shows the muon fake factors as functions of muon \pt{}, leading jet \pt{}, and $b$-jet multiplicity before any hard jet requirement.  Similar to the electron fake factor calculation, the fake factor measurement region requires a hard jet of \pt{} greater than $100~GeV$, but unlike the electron fake factors, the muon fake factros are also separated into two $b$-jet multiplicity bins: exactly zero $b$-jets, and one or more $b$-jets.  The bin with exactly zero $b$-jets is used to estimate the fake contribution in the signal region, and the bin with one or more $b$-jets is used to estimate the fake contribution in the $t\bar{t}$ control region.

\begin{figure}[tbp]
  \centering
  \includegraphics[width=0.48\columnwidth]{/Users/sheenaschier/Documents/LaFiles/figures/thesis/fakes/FF_muon/FakeFactor_mu_pt}
  \includegraphics[width=0.48\columnwidth]{/Users/sheenaschier/Documents/LaFiles/figures/thesis/fakes/FF_muon/FakeFactor_mu_j1pt}\\
  \includegraphics[width=0.48\columnwidth]{/Users/sheenaschier/Documents/LaFiles/figures/thesis/fakes/FF_muon/FakeFactor_mu_nbjet}\\
  \caption{Muon fake factors \textit{before} requiring a hard jet of $\pt{}> 100~GeV$, computed from single-muon prescaled triggers as a function of muon \pt{} (top-left), as a function of leading jep \pt{} (top-right), and as a function of $b$-jet multiplicity (bottom). A red line denotes the average muon fake factor over all muon \pt{}.}
  \label{fig:muon_FF_hist_noCut}
\end{figure}

The final fake factors are shown in Fig.~\ref{fig:muon_FF_hist} as a functions of muon \pt{} for each of the $b$-jet multiplicity bins.  In addition to the final fake factors binned in \pt, fake factors binned in other variables are also inspected to check for significant trends:
\begin{itemize}
\item Fake factors as a function of muon $\eta$ are shown in Fig.~\ref{fig:muon_FF_hist_eta},
\item Fake factors as a function of $\Delta\phi_{jet1-met}$ are shown in Fig.~\ref{fig:muon_FF_dphij1},
\item Fake factors as a function of jet multiplicity are shown in Fig.~\ref{fig:muon_FF_njet},
%\item Fake factors as a function of $b$-jet multiplicity are shown in Fig.~\ref{fig:muon_FF_nbjet},
\item Fake factors as a function of average interactions per bunch crossing are shown in Fig.~\ref{fig:muon_FF_mu},
\item Fake factors as a function of the number of primary vertices are shown in Fig.~\ref{fig:muon_FF_npv}.
\end{itemize}
The relative uncertianties on the muons fake factors versus muon \pt{} for the separate $b$-jet multiplicity bins are show in Fig.~\ref{fig:muon_FF_rel_uncert}.

\begin{figure}[tbp]
  \centering
  \includegraphics[width=0.48\columnwidth]{/Users/sheenaschier/Documents/LaFiles/figures/thesis/fakes/FF_muon/FakeFactor_mu_ptb0}
  \includegraphics[width=0.48\columnwidth]{/Users/sheenaschier/Documents/LaFiles/figures/thesis/fakes/FF_muon/FakeFactor_mu_ptb1}\\
  \caption{Muon fake factors computed from single-muon prescaled triggers as a function of muon \pt{} in events with exactly zero $b$-jets (left) and one or more $b$-jets (right). A red line denotes the average muon fake factor over all muon \pt{}.}
  \label{fig:muon_FF_hist}
\end{figure}

\begin{figure}[tbp]
  \centering
  \includegraphics[width=0.48\columnwidth]{/Users/sheenaschier/Documents/LaFiles/figures/thesis/fakes/FF_muon/FakeFactor_mu_etab0}
  \includegraphics[width=0.48\columnwidth]{/Users/sheenaschier/Documents/LaFiles/figures/thesis/fakes/FF_muon/FakeFactor_mu_etab1}\\
  \caption{Muon fake factors computed from single-muon prescaled triggers as a function of muon $\eta$ in events with exactly zero $b$-jets (left) and one or more $b$-jets (right). A red line denotes the average muon fake factor over all muon \pt{}.}
  \label{fig:muon_FF_hist_eta}
\end{figure}

\begin{figure}[tbp]
  \centering
  \includegraphics[width=0.48\columnwidth]{/Users/sheenaschier/Documents/LaFiles/figures/thesis/fakes/FF_muon/FakeFactor_mu_dphijb0}
  \includegraphics[width=0.48\columnwidth]{/Users/sheenaschier/Documents/LaFiles/figures/thesis/fakes/FF_muon/FakeFactor_mu_dphijb1}
  \caption{Muon fake factors computed from single-muon prescaled triggers as a function of $\Delta\phi_{jet-\met}$ in events with exactly zero $b$-jets (left) and one or more $b$-jets (right).  A red line denotes the average muon fake factor over all muon \pt{}}
  \label{fig:muon_FF_dphij1}
\end{figure}

\begin{figure}[tbp]
  \centering
  \includegraphics[width=0.48\columnwidth]{/Users/sheenaschier/Documents/LaFiles/figures/thesis/fakes/FF_muon/FakeFactor_mu_njetb0}
  \includegraphics[width=0.48\columnwidth]{/Users/sheenaschier/Documents/LaFiles/figures/thesis/fakes/FF_muon/FakeFactor_mu_njetb1}\\
  \caption{Muon fake factors computed from single-muon prescaled triggers as a function of the jet multiplicity in events with exactly zero $b$-jets (left) and one or more $b$-jets (right).  A red line denotes the average muon fake factor over all muon \pt{}}
  \label{fig:muon_FF_njet}
\end{figure}

\begin{figure}[tbp]
  \centering
  \includegraphics[width=0.48\columnwidth]{/Users/sheenaschier/Documents/LaFiles/figures/thesis/fakes/FF_muon/FakeFactor_mu_mub0}
  \includegraphics[width=0.48\columnwidth]{/Users/sheenaschier/Documents/LaFiles/figures/thesis/fakes/FF_muon/FakeFactor_mu_mub1}\\
  \caption{Muon fake factors computed from single-muon prescaled triggers as a function of the average number of interactions per bunch crossing in events with exactly zero $b$-jets (left) and one or more $b$-jets (right).  A red line denotes the average muon fake factor over all muon \pt{}}
  \label{fig:muon_FF_mu}
\end{figure}

\begin{figure}[tbp]
  \centering
  \includegraphics[width=0.48\columnwidth]{/Users/sheenaschier/Documents/LaFiles/figures/thesis/fakes/FF_muon/FakeFactor_mu_npvb0}
  \includegraphics[width=0.48\columnwidth]{/Users/sheenaschier/Documents/LaFiles/figures/thesis/fakes/FF_muon/FakeFactor_mu_npvb1}\\
  \caption{Muon fake factors computed from single-muon prescaled triggers as a function of the number of primary vertices in events with exactly zero $b$-jets (left) and one or more $b$-jets (right).  A red line denotes the average muon fake factor over all muon \pt{}}
  \label{fig:muon_FF_npv}
\end{figure}

\begin{figure}[tbp]
  \centering
  \includegraphics[width=0.48\columnwidth]{/Users/sheenaschier/Documents/LaFiles/figures/thesis/fakes/FF_muon/FakeFactor_mu_ptb0_uncert}
  \includegraphics[width=0.48\columnwidth]{/Users/sheenaschier/Documents/LaFiles/figures/thesis/fakes/FF_muon/FakeFactor_mu_ptb1_uncert}\\
  \caption{Relative uncertianties on muon fake factors versus muon \pt{} in zero $b$-jets bin (left) and one or more $b$-jets bin (right).}
  \label{fig:muon_FF_rel_uncert}
\end{figure}

 \FloatBarrier

 
 
  \subsubsection{Electron Fake Factors}
\begin{table}[tbp]
  \centering
  \begin{tabular}{|c|c|}
    \hline
    el trigger  & \pt{} range [\GeV]\\
    \hline
    HLT\_e5\_lvhloose & 5--11  \\
    HLT\_e10\_lvhloose\_L1EM7 & 11--18  \\
    HLT\_e15\_lvhloose\_L1EM13VH & 18--23  \\
    HLT\_e20\_lvhloose & $>$ 23  \\
    \hline
  \end{tabular}
  \caption{Single-Electron triggers used for fake factor computation and their corresponding \pt{} range.}
  \label{tab:elec_trigger_range}
\end{table}


Electron fake factors show the largest dependance on electron \pt{}, but also display a dependence on the leading jet \pt{}, which is evident in Fig.~\ref{fig:elec_FF_hist_noCut} that shows electron fake factors as a function of electron \pt{} and leading jet \pt{} separately. Given this trend, and the fact that all signal regions used in this analysis require a hard jet with \pt{} greater than 100~\GeV, we design the fake factor measurement region to also require a hard jet of \pt{} greater than 100~\GeV.  Fake factors as a function of other kinematic variables are also studied as a cross-check and for understanding systematic uncertainties.



\begin{figure}[tbp]
  \centering
  \includegraphics[width=0.48\columnwidth]{/Users/sheenaschier/Documents/LaFiles/figures/thesis/fakes/FF_electron/ID_CR_MET}
  \includegraphics[width=0.48\columnwidth]{/Users/sheenaschier/Documents/LaFiles/figures/thesis/fakes/FF_electron/ID_CR_Mt}\\
  \includegraphics[width=0.48\columnwidth]{/Users/sheenaschier/Documents/LaFiles/figures/thesis/fakes/FF_electron/AID_CR_MET}
  \includegraphics[width=0.48\columnwidth]{/Users/sheenaschier/Documents/LaFiles/figures/thesis/fakes/FF_electron/AID_CR_Mt}
  \caption{The \met{} (left) and $m_{T}$ (right) distributions for numerator (top) and denominator (bottom) electrons in the pre-scaled single-lepton-trigger sample.  MC has been scaled to the data in the $\met > 200~\GeV$ region.}
  \label{fig:elec_FF_dists_1}
\end{figure}

\begin{figure}[tbp]
  \centering
  \includegraphics[width=0.48\columnwidth]{/Users/sheenaschier/Documents/LaFiles/figures/thesis/fakes/FF_electron/ID_SR_IDelPt}
  \includegraphics[width=0.48\columnwidth]{/Users/sheenaschier/Documents/LaFiles/figures/thesis/fakes/FF_electron/AID_SR_AntiIDelPt}\\
  \caption{Electron \pt{} for numerator (left) and denominator (right) objects in the pre-scaled single-lepton-trigger sample for events with $m_{T} < 40 GeV$.  MC has been scaled to the data in the $\met > 200~\GeV$ region.}
  \label{fig:elec_FF_dists_pt}
\end{figure}

% actual fake factors
% Trigger distributions in lepton pt
\begin{figure}[tbp]
  \centering
  \includegraphics[width=0.48\columnwidth]{/Users/sheenaschier/Documents/LaFiles/figures/thesis/fakes/FF_electron/electronTriggers}
  \includegraphics[width=0.48\columnwidth]{/Users/sheenaschier/Documents/LaFiles/figures/thesis/fakes/FF_electron/AIDelectronTriggers}\\
  \caption{The numerator electron (left) and denominator electron (right) \pt{} distributions for pre-scaled single-lepton-trigger, normalized to 1~\ipb{}. Blue curve: HLT\_e5\_lvhloose, red curve: HLT\_e10\_lvhloose\_L1EM7, purple curve: HLT\_e15\_lvhloose\_L1EM13, green curve: HLT\_e20\_lvhloose.}
  \label{fig:triggers}
\end{figure}


\begin{figure}[tbp]
  \centering
  \includegraphics[width=0.48\columnwidth]{/Users/sheenaschier/Documents/LaFiles/figures/thesis/fakes/FF_electron/FakeFactor_el_pt_noCut}
  \includegraphics[width=0.48\columnwidth]{/Users/sheenaschier/Documents/LaFiles/figures/thesis/fakes/FF_electron/FakeFactor_el_j1pt_noCut}\\
  \caption{Electron fake factors \textit{before} requiring a hard jet of $\pt{} > 100~GeV$, computed from single-electron prescaled triggers as a function of electron \pt{} (left) and leading jet \pt{} (right). Fake factors for electron $\pt{}~ 4.5-5~\GeV$ are taken to be the same as electron $\pt{}~5-6~\GeV$.  A red line denotes the average electron fake factor over all electron \pt{} of 0.267. }
  \label{fig:elec_FF_hist_noCut}
\end{figure}


Final fake factors computed as a function of electron \pt{} are shown in Fig.~\ref{fig:elec_FF_hist}a.  In addition, fake factors as functions of other variables are also inspected to check for significant trends:
\begin{itemize}
\item the dependence of the fake factors on $|\eta|$ is shown in Fig.~\ref{fig:elec_FF_hist}b,
\item fake factors as a function of leading jet \pt{} and  $\Delta\phi_{jet-\met}$ are shown in Fig.~\ref{fig:elec_FF_hadronic},
\item fake factors as a function of jet multiplicity and $b$-jet multiplicity are shown in Fig.~\ref{fig:elec_FF_njet},
\item fake factors as a function of pile up variables, such as average interaction per bunch crossing and number of primary vertices, are also shown in Fig.~\ref{fig:elec_FF_pileup}.
\end{itemize}
The relative uncertianties on the final electron fake factors versus electron \pt{} are shown in Fig.~\ref{fig:elec_FF_rel_uncert}.

%The relative statistical uncertaintiess are shown in Fig.~\ref{fig:elec_FF_2D} and  will be incorporated into the total systematic uncertainty on the electron fake factors.
\begin{figure}[tbp]
  \centering
  \includegraphics[width=0.48\columnwidth]{/Users/sheenaschier/Documents/LaFiles/figures/thesis/fakes/FF_electron/FakeFactor_el_pt}
  \includegraphics[width=0.48\columnwidth]{/Users/sheenaschier/Documents/LaFiles/figures/thesis/fakes/FF_electron/FakeFactor_el_eta}
  \caption{Electron fake factors computed from single-electron prescaled triggers as a function of electron \pt{} (left) and electron $\eta$ (right) in the kinematic region with leading jet$ \pt{}>100GeV$  Fake factors for electron $\pt{}~ 4.5-5~\GeV$ are taken to be the same as electron $\pt{}~5-6~\GeV$.  A red line denotes the average electron fake factor over all electron \pt{} of 0.211. }
  \label{fig:elec_FF_hist}
\end{figure}

\begin{figure}[tbp]
  \centering
  \includegraphics[width=0.48\columnwidth]{/Users/sheenaschier/Documents/LaFiles/figures/thesis/fakes/FF_electron/FakeFactor_el_j1pt}
  \includegraphics[width=0.48\columnwidth]{/Users/sheenaschier/Documents/LaFiles/figures/thesis/fakes/FF_electron/FakeFactor_el_dphij}\\
  \caption{Electron fake factors computed from single-electron prescaled triggers as a function of leading jet \pt{} (left) and $\Delta\phi_{jet-\met}$ (right). A red line denotes the average electron fake factor over all electron \pt{} of 0.211.}
  \label{fig:elec_FF_hadronic}
\end{figure}

\begin{figure}[tbp]
  \centering
  \includegraphics[width=0.48\columnwidth]{/Users/sheenaschier/Documents/LaFiles/figures/thesis/fakes/FF_electron/FakeFactor_el_njet}
  \includegraphics[width=0.48\columnwidth]{/Users/sheenaschier/Documents/LaFiles/figures/thesis/fakes/FF_electron/FakeFactor_el_nbjet}\\
  \caption{Electron fake factors computed from single-electron prescaled triggers as a function of the jet multiplicity (left) and the $b$-jet multiplicity (right). A red line denotes the average electron fake factor over all electron \pt{} of 0.211.}
  \label{fig:elec_FF_njet}
\end{figure}


\begin{figure}[tbp]
  \centering
  \includegraphics[width=0.48\columnwidth]{/Users/sheenaschier/Documents/LaFiles/figures/thesis/fakes/FF_electron/FakeFactor_el_mu}
  \includegraphics[width=0.48\columnwidth]{/Users/sheenaschier/Documents/LaFiles/figures/thesis/fakes/FF_electron/FakeFactor_el_npv}\\
  \caption{Electron fake factors computed from single-electron prescaled triggers as a function of the average interaction per bunch crossing (left) and the number of primary vertices (right). A red line denotes the average electron fake factor over all electron \pt{} of 0.211.}
  \label{fig:elec_FF_pileup}
\end{figure}

\begin{figure}[tbp]
  \centering
  \includegraphics[width=0.48\columnwidth]{/Users/sheenaschier/Documents/LaFiles/figures/thesis/fakes/FF_electron/FakeFactor_el_pt_uncert}\\
  \caption{Relative uncertainties on electron fake factors binned electron \pt{}.}
  \label{fig:elec_FF_rel_uncert}
\end{figure}

 \FloatBarrier

 
 \section{Conclusion}
 \label{sec:FFcon}
 This chapter went over a lot of material and I think somehow I have to reiterate the important points here..

\part{Analysis and Results}
\chapter{Systematic Uncertianties}
\label{sec:syst}
Systematic uncertainties are split into two categories: experimental and theoretical.  The major sources of experimental uncertainties are the modeling of particle reconstruction in detector simulation, luminosity and pileup measurements, and systematic effects from data-driven estimates.  The main theoretical uncertainties emerge from the modeling of Standard Model background processes.  Simulation of these processes relies on cross-section measurements, parton distribution functions, and renormalization and factorization scale assumptions. Systematic uncertainties propagate to the final expected yields of signal to background, and limit the resolution of predictions. 

This chapter is organized as follows: experimental uncertainties are described in Section~\ref{sec:sys:exp}, where first CP Group uncertainties on measurements of pile-up re-weighting, luminosity, jets, electrons, muons, and missing transverse energy are summarized in Section~\ref{sec:sys:expCP}, and next fake factor uncertainties are described in Section~\ref{sec:sys:expFF}.  Finally, theoretical uncertainties on SM background modeling are dissected in Section~\ref{sec:sys:thy}.

\section{Experimental Uncertainties}
\label{sec:sys:exp}
This chapter will cover uncertainties from CP group recommendations and fake factor measurements.  
\subsection{CP Group Uncertainties}
\label{sec:sys:expCP}
Combined Performance (CP) groups are dedicated teams in ATLAS that work to optimize the characteristic measurements of certain classes of particle.  These groups make recommendations to analysis teams about pile-up re-weighting, luminosity measurements, and which jet, electron, muon, and missing transverse energy definitions to use.  The uncertainties associated with these objects and measurements are discussed in this section.

Multiple pile-up interactions need to be modeled well in Monte Carlo so that the simulated detector response and particle reconstruction conditions match the actual data.  The distribution of the average number of interactions per bunch crossing applied to Monte Caro events, the $\mu$ profile, is based on relevant assumptions and does not always agree with the $\mu$ profile observed in data.  To resolve these disagreements, the $\mu$ profile for Monte Carlo is reweighted to better match the shape in data.  This is typically called pile-up reweighting.  Studies of the data/MC agreement for the number of primary vertices versus $\mu$ suggest an additional rescaling of the $\mu$ distribution in data of $1/1.16$.  A systematic uncertainty for the pile-up reweighting scheme is assigned by varying the scaling factor assigned to data between 1.00 and 1.21 and assessing the change in event yields.  An uncertainty on the luminosity measurement is also examined.  For the 2015+2016 combined datasets, the luminosity uncertainty is observed as $3.2\%$.

Uncertainties on the jet energy scale and jet energy resolution are measured using five parameters varied up and down for the energy uncertainty estimate, and one parameter varied up and down for the uncertainty on the resolution.  A separate uncertainty is assigned to account for the differences in the jet-vertex tagging and b-jet tagging efficiencies between Monte Carlo and data. Uncertainties on the electron energy and momentum scale and resolution are also considered, along with uncertainties on the electron and muon scale factors applied to Monte Carlo events that ensure the simulated reconstruction, identification, isolation, and track-to-vertex association efficiencies match the data.  Furthermore, uncertainties on the missing transverse energy and momentum arise from the propagation of error in the transverse momentum measurements of hard physics objects.  Additional uncertainties on the \met propagate from the scale and resolution of the track-based soft term, described in Chapter~\ref{sec:obj:reco}.  The dominant CP group systematic is from the jet energy scale and resolution.

\subsection{Fake Factor Uncertainties}
\label{sec:sys:expFF}
Fake and non-prompt lepton backgrounds are estimated with a data-driven fake factor method, as described in Chapter~\ref{ch:fakefactor}.  Uncertainties arise from several sources, but are mainly from: kinematic dependencies, non-closure in the same-sign validation region, statistical uncertainties on the applied fake factors, and prompt lepton subtraction using Monte Carlo.

The primary fake factor uncertainty comes from kinematic dependancies on variables that are not included in the fake factor binning.  Fake factors are measured as a function of electron \pt for the electrons, and as a function of muon \pt and $N_{b-jet}$ for the muons.  These choices are motivated by the strong correlation of the fake factors and these variables, but other, smaller kinematic dependencies are present.  The fake factor vulnerabilities are not large enough to consider binning them in every variable, so they are accounted for as a systematic. Figure~\ref{fig:elec_FF_all} presents electron fake factors, and Figures~\ref{fig:muon_FF_hist_eta} -~\ref{fig:muon_FF_npv} present muon fake factors binned in alternative variables.  We consider the largest, statistically meaningful variation of the fake factors binned in the alternative relevant variables and subtract it from the average fake factor for the electron and muon samples separately.  The resulting uncertainty is $25\%$ for each, both driven by the variation in lepton $\eta$.

The relationship between the fake lepton estimate and the data in VR-SS is another source of systematic uncertainty.  This is quantified by comparing data in a version of the VR-SS that does not require an $\met/H_T$ cut in the envelope containing the systematic variations described above.  The root mean square of the variations is compared with the data and the quadrature difference is interpreted as the closure systematic.  This uncertainty is determined to be $38\%$ for electrons with $\pt < 7\GeV$, $97\%$ for muons with \pt 7-10 GeV, and $0\%$ everywhere else. %This $0\%$ is assigned because the fake lepton estimate and the data agree within their uncertainties in VR-SS for the other pT bins considered

Statistical uncertainties on the fake factors are due to the limited size of the samples used to derive them.  These samples use pre-scaled single lepton triggers to select events in data, which are further scrutinized based on the identification, isolation, and impact parameter of the reconstructed leptons to be determined tas either an "ID" or "anti-ID" lepton event.  It is possible that there are overlapping events in these two categories, but it is a rare occurrence since less than $10\%$ of the events have more than one lepton, and both the "ID" and the "anti-ID" leptons would need to fall in the \pt range associated with highest lepton \pt trigger that fired.  Figures~\ref{fig:elec_FF_rel_uncert} and~\ref{fig:muon_FF_rel_uncert} show the relative systematic uncertainties on the electron and muon fake factors per lepton \pt bin.   For electrons, statistical uncertainties range from about $32\%$ in the lowest \pt bin to about $58\%$ in the highest \pt bin.  For muons, the uncertainties on fake factors used to estimate fake backgrounds in the signal regions vary between $12\%$ in the lowest \pt bin to about $32\%$ in the highest \pt bin, and uncertainties on fake factors used to estimate fake backgrounds in the $t\bar{t}$ control region vary between $16\%$ and $38\%$.

Fake factors are measured in regions of data enriched with fake leptons, but prompt lepton contamination is still present.  In the measurement region $m_T<40~\GeV$, prompt lepton events are subtracted from the \pt distributions using SM Monte Carlo that have been rescaled to match data in the high \met region.  To calculate the systematic uncertainty on this method of prompt subtraction, the change in the binned fake factors is studied as three key parameters are varied.  The \met region, where the scale factor for the prompt subtraction is computed, is varied up and down by $20~\GeV$ from the nominal $\met>200~\GeV$ selection, the region where the fake factors are measured is varied up and down by $10~\GeV$ from the nominal $m_T<40~\GeV$ selection, and the scale factor that is applied to the subtracted Monte Carlo is varied up and down by $20\%$.  Uncertainty contributions in the prompt subtraction are assed further by recomputing the Monte Carlo scale factor in the region $m_T>100~\GeV$ and assessing the change in the fake factors.  All together, the resulting uncertainties on both electron and muon scale factors are less that $10\%$, but for one exception in the muon \pt bin above $20~\GeV$, where the uncertainty is $19\%$.  The overall contribution from prompt subtraction is minute compared to the other sources.

\section{Theoretical Uncertainties}
\label{sec:sys:thy}
Theoretical uncertainties from signal and background simulation arise from the uncertainties on the underlying parameters in the Monte Carlo generation.


\subsection{Uncertainty on Simulated Signal Events}
Statistical uncertainties on Higgsino and slepton simulated signal events dominantly arise from the next-to-leading order calculations of the hadronic initial state radiation (ISR), factorization and renormalization scale (FSR), and the underlying event.  ISR/FSR/EU are all around $20\%$.  PDF uncertainties on signal acceptances are also estimated to be around $10\%$.  Uncertainties on signal cross-section are around $5\%$.

\subsection{Uncertainty on Simulated Background Events}
Diboson, $Z(\rightarrow\tau\tau)$+jets, and $t\bar{t}$ are the dominant background processes estimated with Monte Carlo simulation.  There are three main sources of uncertainty: choice of QCD renormalization and factorization scales $\mu_R$ and $\mu_F$, choice of strong coupling constant $\alpha_s$, and choice of PDF set.  To calculate the uncertainties, each of these is varied symmetrically around some parameter, or, in the case of the PDF uncertainty, varied by PDF set.  The effect of the variations on the predicted yield from each of the dominant background processes is evaluated in the signal, control, and validation regions. $\mu_R$ and $\mu_F$ are deviated up and down by a factor of 2 and $\alpha_s$ is varied within its uncertainty of 0.001, and the range of impact on the expected yields are evaluated as the uncertainties.  PDF uncertainties are obtained from the envelope of symmetrized variations within acceptance of the MMHT2014, CT14, NNPDF PDF sets.  Figures~\ref{fig:theoryUncsVV}, \ref{fig:theoryUncsZtt}, and \ref{fig:theoryUncsttbar} show the assortment of event yields in the Higgsino and slepton SRs for the diboson, $Z(\rightarrow\tau\tau)$+jets, and $t\bar{t}$ predictions.  The final uncertainty in each region is calculated as the quadrature sum of all the individual contributions and adds up to $\sim10\%$ relative uncertainty on the Monte Carlo background prediction.

 \begin{figure}
  \centering
  \includegraphics[width=0.4\columnwidth]{/Users/sheenaschier/Documents/LaFiles/figures/thesis/systematics/scaleVars_diboson2L_mll_SR_hg_SFDF_shape.pdf}
 %\caption{$\mu_{F}$ and $\mu_{R}$ uncertainties on the $m_{\ell\ell}$ distribution in the Higgsino signal region.}
  \includegraphics[width=0.4\columnwidth]{/Users/sheenaschier/Documents/LaFiles/figures/thesis/systematics/scaleVars_diboson2L_mt2leplsp_100_SR_sl_SFDF_shape.pdf}
% \caption{$\mu_{F}$ and $\mu_{R}$ uncertainties on the $m_{\text{T}2}$ distribution in the slepton signal region.}
 \includegraphics[width=0.4\columnwidth]{/Users/sheenaschier/Documents/LaFiles/figures/thesis/systematics/alphaVars_diboson2L_mll_SR_hg_SFDF_shape.pdf}
 % \caption{$\alpha_{s}$ uncertainties on the $m_{\ell\ell}$ distribution in the Higgsino signal region.}
 \includegraphics[width=0.4\columnwidth]{/Users/sheenaschier/Documents/LaFiles/figures/thesis/systematics/alphaVars_diboson2L_mt2leplsp_100_SR_sl_SFDF_shape.pdf}
% \caption{$\alpha_{s}$ uncertainties on the $m_{\text{T}2}$ distribution in the slepton signal region.}
  \includegraphics[width=0.4\columnwidth]{/Users/sheenaschier/Documents/LaFiles/figures/thesis/systematics/PDFVars_diboson2L_mll_SR_hg_SFDF_shape_allPDFs.pdf}
 % \caption{PDF uncertainties on the $m_{\ell\ell}$ distribution in the Higgsino signal region.}
  \includegraphics[width=0.4\columnwidth]{/Users/sheenaschier/Documents/LaFiles/figures/thesis/systematics/PDFVars_diboson2L_mt2leplsp_100_SR_sl_SFDF_shape_allPDFs.pdf}
%\caption{PDF uncertainties on the $m_{\text{T}2}$ distribution in the slepton signal region.}
 \caption{QCD scale, $\alpha_{s}$ and PDF uncertainties on the shape and normalization of the diboson background in the Higgsino (left) and slepton (right) signal regions, but with no lepton flavor requirement.}
\label{fig:theoryUncsVV}
 \end{figure}
 
  \begin{figure}
  \centering
   \includegraphics[width=0.4\columnwidth]{/Users/sheenaschier/Documents/LaFiles/figures/thesis/systematics/scaleVars_Zttjets_mll_SR_hg_SFDF_shape.pdf}
 %\caption{$\mu_{F}$ and $\mu_{R}$ uncertainties on the $m_{\ell\ell}$ distribution in the Higgsino signal region.}
  \includegraphics[width=0.4\columnwidth]{/Users/sheenaschier/Documents/LaFiles/figures/thesis/systematics/scaleVars_Zttjets_mt2leplsp_100_SR_sl_SFDF_shape.pdf}
% \caption{$\mu_{F}$ and $\mu_{R}$ uncertainties on the $m_{\text{T}2}$ distribution in the slepton signal region.}
 \includegraphics[width=0.4\columnwidth]{/Users/sheenaschier/Documents/LaFiles/figures/thesis/systematics/alphaVars_Zttjets_mll_SR_hg_SFDF_shape.pdf}
 % \caption{$\alpha_{s}$ uncertainties on the $m_{\ell\ell}$ distribution in the Higgsino signal region.}
 \includegraphics[width=0.4\columnwidth]{/Users/sheenaschier/Documents/LaFiles/figures/thesis/systematics/alphaVars_Zttjets_mt2leplsp_100_SR_sl_SFDF_shape.pdf}
% \caption{$\alpha_{s}$ uncertainties on the $m_{\text{T}2}$ distribution in the slepton signal region.}
  \includegraphics[width=0.4\columnwidth]{/Users/sheenaschier/Documents/LaFiles/figures/thesis/systematics/PDFVars_Zttjets_mll_SR_hg_SFDF_shape_allPDFs.pdf}
 % \caption{PDF uncertainties on the $m_{\ell\ell}$ distribution in the Higgsino signal region.}
  \includegraphics[width=0.4\columnwidth]{/Users/sheenaschier/Documents/LaFiles/figures/thesis/systematics/PDFVars_Zttjets_mt2leplsp_100_SR_sl_SFDF_shape_allPDFs.pdf}
%\caption{PDF uncertainties on the $m_{\text{T}2}$ distribution in the slepton signal region.}
\caption{QCD scale, $\alpha_{s}$ and PDF uncertainties on the shape and normalization of the $Z\to\tau\tau$ background in the Higgsino (left) and slepton (right) signal regions, but with no lepton flavor requirement.}
\label{fig:theoryUncsZtt}
 \end{figure}
 
  \begin{figure}
  \centering 
   \includegraphics[width=0.4\columnwidth]{/Users/sheenaschier/Documents/LaFiles/figures/thesis/systematics/scaleVars_alt_ttbar_PowPy8_dilep_hdamp258p75_mll_SR_hg_SFDF_shape.pdf}
 %\caption{$\mu_{F}$ and $\mu_{R}$ uncertainties on the $m_{\ell\ell}$ distribution in the Higgsino signal region.}
  \includegraphics[width=0.4\columnwidth]{/Users/sheenaschier/Documents/LaFiles/figures/thesis/systematics/scaleVars_alt_ttbar_PowPy8_dilep_hdamp258p75_mt2leplsp_100_SR_sl_SFDF_shape.pdf}
% \caption{$\mu_{F}$ and $\mu_{R}$ uncertainties on the $m_{\text{T}2}$ distribution in the slepton signal region.}
  \includegraphics[width=0.4\columnwidth]{/Users/sheenaschier/Documents/LaFiles/figures/thesis/systematics/PDFVars_alt_ttbar_PowPy8_dilep_hdamp258p75_mll_SR_hg_SFDF_shape_allPDFs.pdf}
 % \caption{PDF uncertainties on the $m_{\ell\ell}$ distribution in the Higgsino signal region.}
  \includegraphics[width=0.4\columnwidth]{/Users/sheenaschier/Documents/LaFiles/figures/thesis/systematics/PDFVars_alt_ttbar_PowPy8_dilep_hdamp258p75_mt2leplsp_100_SR_sl_SFDF_shape_allPDFs.pdf}
%\caption{PDF uncertainties on the $m_{\text{T}2}$ distribution in the slepton signal region.}
\caption{QCD scale and PDF uncertainties on the shape and normalization of the $t\bar{t}$ background in the Higgsino (left) and slepton (right) signal regions, but with no lepton flavor requirement.}
\label{fig:theoryUncsttbar}
 \end{figure}

\chapter{Statistical Analysis}
\label{ch:statanal}
The model dependent analysis is performed by a shape fit of signal to background in the $m_{\ell\ell}$ and $m_{T2}$ distributions.  Analysis channels are defined as multi-binned distributions of $m_{\ell\ell}$ and $m_{T2}$ and a channel object can represent a CR, SR, or VR. The statistical combination of multiple channels is based on a profile likelihood method implemented in the HistFitter package that builds probability density functions, fits them to data, and interprets them with statistical tests.  In this method, a likelihood is constructed as the product of the Poisson probability distributions that describe the total number of events observed in each channel.  The mean is taken as the nominal MC yield in a given region and systematic uncertainties are treated as nuisance parameters in the fit.  

\section{Test Statistics and p-values}
\label{sec:statanal:pval}
The test statistic that provides the most powerful test is the likelihood ratio function, given by Equation~\ref{eq:likelihood}.
\begin{equation}
L(\mu,\vec{\theta})=\prod_c\prod_iPois\big(n_{ci}^{obs}|n_{ci}^{sig}(\mu\vec{\theta})+n_{ci}^{bkg}(\vec{\theta})\big)\prod_kf_k(\theta'_k|\theta_k)
\label{eq:likelihood}
\end{equation}
In Equation~\ref{eq:likelihood}, $\mu$ and $\vec{\theta}$ represent the signal strength and the set of nuisance parameters.  The values of these parameters that maximize $L(\mu,\vec{\theta})$, or equivalently, minimize -$\ln L(\mu,\vec{\theta})$ are called maximum likelihood estimates (MLEs) and denoted as $\hat{\mu}$ and $\hat{\vec{\theta}}$.  There is also a conditional maximum likelihood estimate, $\hat{\hat{\vec{\theta}}}$, which is the value of $\vec{\theta}$ that maximizes $L(\mu,\vec{\theta})$ for a fixed $\mu$.  These are all used with the likelihood function $L(\mu,\vec{\theta})$ to construct the profile likelihood ratio:
\begin{equation}
\lambda(\mu)=\bigg(\frac{L\big(\mu,\hat{\hat{\vec{\theta}}}(\mu)\big)}{L(\hat{\mu},\hat{\vec{\theta}})}\bigg)
%t=-2\ln\lambda(\mu)=-2\ln\bigg(\frac{L\big(\mu,\hat{\hat{\vec{\theta}}}(\mu)\big)}{L(\hat{\mu},\hat{\vec{\theta}})}\bigg)
\end{equation}
In a physical theory, the true signal strength $\mu$ is a non-negative value, and a negative value of $\hat{\mu}$ implies a shortage of signal-like events in the background.  The boundary at $\mu=0$ convolutes the asymptotic distributions in $\lambda(\mu)$, so $\mu$ is free to occupy positive and negative values while the full profile likelihood ratio is defined as:
\begin{equation}
\tilde{\lambda}(\mu)=
 \begin{cases} 
      \frac{L\big(\mu,\hat{\hat{\vec{\theta}}}(\mu)\big)}{L(\hat{\mu},\hat{\vec{\theta}})} & \hat{\mu}\geq 0 \\
      \frac{L\big(\mu,\hat{\hat{\vec{\theta}}}(\mu)\big)}{L(0,\hat{\hat{\vec{\theta}}}(0))} & \hat{\mu}< 0 \\
   \end{cases}
\end{equation}
As stated before, maximizing the likelihood is equivalent to minimizing the negative-log likelihood, which is more convenient for visualization.  The test statistic $\tilde{q}$ is defined separately for discovery and limit-setting using the negative-log likelihood ratio (NLLR).  

For discovery, the test statistic $\tilde{q}_0$ is built to distinguish the background only hypothesis $\mu=0$ from the alternative hypothesis $\mu>0$, where there is an excess above background.  When the MLE $\hat{\mu}$ is positive, the test statistic is the NLLR, otherwise it is zero, as shown in Equation~\ref{eq:disc}.
\begin{equation}
\tilde{q}_0=
 \begin{cases} 
      -2\ln\lambda(\mu) & \hat{\mu}> 0 \\
      0 & \hat{\mu}\leq 0 \\
   \end{cases}
   \label{eq:disc}
\end{equation}

When setting limits, the test statistic $\tilde{q}_\mu$ is meant to distinguish the signal hypothesis, where signal events are produced above background at some rate $\mu$, from the alternative hypothesis with signal events produced at some rate less than or equal to $\mu$.  In this case, when the MLE $\hat{\mu}$ is less than $\mu$, $\tilde{q}_\mu$ equals the NLLR, otherwise, it is set to zero.  This is shown in Equation~\ref{eq:lim}
\begin{equation}
\tilde{q}_\mu=
 \begin{cases} 
      -2\ln\lambda(\mu) & \hat{\mu}\leq\mu  \\
      0 & \hat{\mu}>\mu \\
   \end{cases}
   \label{eq:lim}
\end{equation}

Through the test statistic, the data is mapped to a single real-valued number that represents the outcome of the experiment.  If the experiment was performed many times, the test profile likelihood ratio function would output a different value each time, making a distribution of real-valued discriminating variables.  In practice, Monte Carlo simulation \textcolor{red}{\textit{get the right language here, talk about throwing the toys and such}} is used to generate numerous pseudo experiments, and while the test statistic $\tilde{q}$ is a function of $\mu$, the distribution of $\tilde{q}$ becomes explicitly a function of the nuisance parameters $\vec{\theta}$, denoted as $f(\tilde{q}|\mu,\vec{\theta})$.  The p-value for any given hypothesis represents the probability to observe an equal or more extreme outcome given that hypothesis as the integral of the test statistic distribution from $\tilde{q}_{\mu,obs}$ to $\infty$.  
\begin{equation}
p_{\mu,\vec{\theta}}=\int_{\tilde{q}_{\mu,obs}}^\infty f(\tilde{q}_\mu|\mu,\vec{\theta}) d\tilde{q}_\mu
\label{eq:p0}
\end{equation}

Conventionally in high energy particle physics experiments, a standard one-sided frequentist confidence interval defines an upper limit on the parameter of interest at $95\%$ confidence level.  The p-value can be used to measure how well the data agrees with a signal hypothesis of signal strength $\mu$, given in Equation~\ref{eq:pmu}, or it can be used to measure how consistent the data is with the background only hypothesis, as in Equation~\ref{eq:pb}.
\begin{equation}
p_\mu=\int_{\tilde{q}_{\mu,obs}}^\infty f(\tilde{q}_\mu|\mu,\hat{\hat{\vec{\theta}}}(\mu,obs)) d\tilde{q}_\mu
\label{eq:pmu}
\end{equation}
\begin{equation}
p_b=1-\int_{\tilde{q}_{\mu,obs}}^\infty f(\tilde{q}_\mu|0,\hat{\hat{\vec{\theta}}}(\mu=0,obs)) d\tilde{q}_\mu
\label{eq:pb}
\end{equation}
The $CL_s$ upper limit on $\mu$ comes from solving as a function of $\mu$ for $p'_\mu=0.05$, where $p'_\mu$ is the ratio of p-values in Equation~\ref{eq:pp}.
\begin{equation}
p_\mu ' = \frac{p_\mu}{1-p_b}
\end{equation}

Confidence intervals will come up again in Chapter~\ref{ch:results} when the actual model interpretations are discussed.

%\textcolor{red}{things are okay up to here..}. This likelihood maxima are found by the HISTFITTER package, %M. Baak et al., HistFitter software framework for statistical data analysis, Eur. Phys. J. C 75 (2015) 153, arXiv: 1410.1280 [hep-ex].  
%which constrains the values and uncertainties on $\mu_{top}$ and $\mu_{\tau\tau}$, the variables used to extrapolate the background prediction into the validation regions.   The concept of signal, control, and validation regions are woven into the structure of HistFitter and are used to constrain, extrapolate, and test data model predictions with statistically rigorous built-in methods.  HistFitter is capable of working with multiple data models at once, keeping track of all input histograms, both before and after renormalization.  This allows for straightforward bookkeeping, control, and testing of large collections of signal hypotheses.  HistFitter also includes methods to \textcolor{blue}{ determine the statistical significance  of signal hypotheses, estimate the quality of likelihood fits, and produce high-quality tables and plots for publications}.  HistFitter is written with a Python configuration wrapper around  CPU intensive C++ algorithms.



\chapter{Results}
\label{sec:results}
Model independent interpretations are shown as well as compressed higgsino and compressed slepton interpretations.
\begin{figure}
 \centering
\includegraphics[width=0.6\columnwidth]{/Users/sheenaschier/Documents/LaFiles/figures/thesis/results/pull_plot_summary_yields}
   \caption{Summary of Monte Carlo yields in control, validation and signal regions in a background-only fit using data only in the two CRs to constrain the fit.}
  \label{fig:pull_plot_summary_yields}
 \end{figure}
\FloatBarrier

\section{Compressed Higgsino}
 \begin{figure}
 \centering
 \includegraphics[width=0.6\columnwidth]{/Users/sheenaschier/Documents/LaFiles/figures/thesis/results/exclusion_contour_higgsino.pdf}
  \caption{
 Expected 95\% CL exclusion sensitivity (blue dashed line) with $\pm 1 \sigma_\text{exp}$ (yellow band) from experimental systematics
   and observed limits (red solid) with $\pm 1 \sigma_\text{theory}$ (dotted red) from signal cross section uncertainties.
A shape fit of Higgsino signals to the $m_{\ell\ell}$ spectrum is used to derive
 the limit is displayed in the $m(\tilde{\chi}^0_2) - m(\tilde{\chi}^0_1)$ vs $m(\tilde{\chi}^0_2)$ plane.
 The chargino $\tilde{\chi}^\pm_1$ mass is assumed to be half way between the two lightest neutralinos.
  The grey region denotes the lower chargino mass limit from LEP~\cite{LEPlimits}.}
   \label{fig:exclusion_contour_higgsino}
 \end{figure}
 \FloatBarrier
 
 \section{Compressed Slepton}
  \begin{figure}
 \centering
 \includegraphics[width=0.6\columnwidth]{/Users/sheenaschier/Documents/LaFiles/figures/thesis/results/exclusion_contour_slepton.pdf}
  \caption{
Expected 95\% CL exclusion sensitivity (blue dashed line) with $\pm 1 \sigma_\text{exp}$ (yellow band) from experimental systematics
and observed limits (red solid) with $\pm 1 \sigma_\text{theory}$ (dotted red) from signal cross section uncertainties.
A shape fit of slepton signals to the $m_\text{T2}^{100}$ spectrum is used to derive
the limit projected into the $m(\tilde{\ell}) - m(\tilde{\chi}^0_1)$ vs $m(\tilde{\ell})$ plane.
The slepton $\tilde{\ell}$ refers to a 4-fold mass degenerate system of left- and right-handed selectron and smuon.
The grey region denotes a conservative right-handed smuon $\tilde{\mu}_R$ mass limit from LEP~\cite{LEPlimits},
while the blue region is the 4-fold mass degenerate slepton limit from ATLAS Run 1~\cite{SUSY-2013-11}.}
   \label{fig:exclusion_contour_slepton}
 \end{figure}
 

\chapter{Interpretations}
\label{ch:interpretations}

In absence of any significant excesses over backgrounds, the results are interpreted as constraints on the SUSY models presented in Chapter~\ref{ch:thy} using the exclusive, multi-binned Higgsino and slepton signal regions.  The background only fit is extended to allow for a signal model with a corresponding signal strength parameter in a simultaneous fit of all CRs and relevant SRs, this is referred to as the exclusion fit.  In the previous chapter, background-level estimates obtained from a background-only fit in the CRs only were presented.  When electroweakino simplified models are assumed, the results are interpreted in the 14 exclusive Higgsino signal regions, binned in $m_{\ell\ell}$ and split evenly between the $ee$ and $\mu\mu$ channels.  By statistically combining these signal regions, the signal shape of the $m_{\ell\ell}$ spectrum can be exploited to improve the sensitivity. When slepton simplified models are assumed, the results are interpreted in 12 slepton signal regions, binned in $m_{T2^{100}}$ with 6 SRs the $ee$-channel and 6 in the $\mu\mu$ channel are used for the fit.

\section{Compressed Higgsino}
Hypothesis tests are performed to set limits on simplified model scenarios using the $CL_s$ prescription.  Figure~\ref{fig:exclusion_contour_higgsino} shows the $95\%$ confidence interval limits set on the Higgsino simplified model projected onto the plane defined by the mass difference between the lightest and next-to-lightest neutralino as a function of the next-to-lightest neutralino mass.  These limits are based on an exclusion fit that exploits the shape of the dilepton invariant mass spectrum from the exclusive electroweakino signal regions and exclude next-to-lightest neutralino masses up to $130~\GeV$ for mass splittings between $5$ and $10~\GeV$.  For mass splittings down to $3~\GeV$ next-to-lightest neutralino masses are excluded up to $100~\GeV$. 

 \begin{figure}
 \centering
 \includegraphics[width=0.95\columnwidth]{/Users/sheenaschier/Documents/LaFiles/figures/thesis/results/exclusion_contour_higgsino.pdf}
  \caption{
 Expected 95\% CL exclusion sensitivity (blue dashed line) with $\pm 1 \sigma_\text{exp}$ (yellow band) from experimental systematics
   and observed limits (red solid) with $\pm 1 \sigma_\text{theory}$ (dotted red) from signal cross section uncertainties.
A shape fit of Higgsino signals to the $m_{\ell\ell}$ spectrum is used to derive
 the limit is displayed in the $m(\tilde{\chi}^0_2) - m(\tilde{\chi}^0_1)$ vs $m(\tilde{\chi}^0_2)$ plane.
 The chargino $\tilde{\chi}^\pm_1$ mass is assumed to be half way between the two lightest neutralinos.
  The grey region denotes the lower chargino mass limit from LEP~\cite{LEPlimits}.}
   \label{fig:exclusion_contour_higgsino}
 \end{figure}
% \FloatBarrier
 
 \section{Compressed Wino}
 The $95\%$ confidence level intervals for the wino-bino simplified model are shown in Figure~\ref{fig:exclusion_contour_wino}.  Just like in the Higgsino exclusion plot, these limits are based on an exclusion fit that exploits the shape of the dilepton invariant mass spectrum from the exclusive electroweakino signal regions.  Exclusion limits are projected onto the mass difference $\Delta m(\tilde{\chi}^0_2, \tilde{\chi}^0_1)$ plane as a function of the $\tilde{\chi}^0_2$ mass.  For wino-bino simplified models, next-to-lightest neutralino masses are excluded up to $170~\GeV$ for mass splittings above $10~\GeV$, and excluded up to $100~\GeV$ for mass splittings down to $2.5~\GeV$. 
   \begin{figure}
 \centering
 \includegraphics[width=0.95\columnwidth]{/Users/sheenaschier/Documents/LaFiles/figures/thesis/results/exclusion_contour_wino.pdf}
  \caption{Expected 95\% CL exclusion sensitivity (blue dashed line) with $\pm1\sigma$ exp (yellow band) from experimental systematic uncertainties and observed limits (red solid line) with pm1?theory (dotted red line) from signal cross-section uncertainties for simplified models direct wino production. 
  A shape fit of wino signals to the $m_{\ell\ell}$ spectrum is used to derive
 the limit is displayed in the $m(\tilde{\chi}^0_2) - m(\tilde{\chi}^0_1)$ vs $m(\tilde{\chi}^0_2)$ plane.
 The chargino $\tilde{\chi}^\pm_1$ mass is assumed equal to the $m(\tilde{\chi}^0_2)$ mass.
  The grey region denotes the lower chargino mass limit from LEP~\cite{LEPlimits}, and the blue region in the lower plot indicates the limit from the 2$\ell$+3$\ell$ combination of ATLAS Run 1.} 
     \label{fig:exclusion_contour_wino}
 \end{figure}
 
  \section{Compressed Slepton}
 Figure~\ref{fig:exclusion_contour_slepton} shows the $95\%$ confidence interval limits set on the slepton simplified model projected onto the plane defined by the mass difference between the slepton and lightest neutralino as a function of the slepton mass.  These limits are based on an exclusion fit that exploits the shape of the $m_{T2}$ spectrum from the exclusive slepton signal regions and exclude slepton masses up to $180~\GeV$ for mass splittings down to $5~\GeV$.  For mass splittings down to $1~\GeV$ slepton masses are excluded up to $70~\GeV$.  In slepton simplified models, a fourfold degeneracy is assumed between the left and right-handed selectrons and smuons: $\tilde{e}_R=\tilde{e}_L=\tilde{\mu}_R=\tilde{\mu}_L$.
  \begin{figure}
 \centering
 \includegraphics[width=0.95\columnwidth]{/Users/sheenaschier/Documents/LaFiles/figures/thesis/results/exclusion_contour_slepton.pdf}
  \caption{
Expected 95\% CL exclusion sensitivity (blue dashed line) with $\pm 1 \sigma_\text{exp}$ (yellow band) from experimental systematics
and observed limits (red solid) with $\pm 1 \sigma_\text{theory}$ (dotted red) from signal cross section uncertainties.
A shape fit of slepton signals to the $m_\text{T2}^{100}$ spectrum is used to derive
the limit projected into the $m(\tilde{\ell}) - m(\tilde{\chi}^0_1)$ vs $m(\tilde{\ell})$ plane.
The slepton $\tilde{\ell}$ refers to a 4-fold mass degenerate system of left- and right-handed selectron and smuon.
The grey region denotes a conservative right-handed smuon $\tilde{\mu}_R$ mass limit from LEP~\cite{LEPlimits},
while the blue region is the 4-fold mass degenerate slepton limit from ATLAS Run 1~\cite{SUSY-2013-11}.}
   \label{fig:exclusion_contour_slepton}
 \end{figure}
 % \FloatBarrier
% \subsection{NUHM2}
 

\chapter{Conclusion}
\label{ch:conclusion}
A search for supersymmetry in scenarios with compressed mass spectra was performed using ATLAS data collected in 2015 and 2016 at $\sqrt(s)$ 13 TeV, corresponding to $36.1 fb^{-1}$. % \textcolor{red}{Motivate the searches for compressed electroweakinos and sleptons.}  
We searched for directly produced electroweakinos and sleptons in events containing two soft, oppositely signed and same flavored leptons and include missing transverse energy recoiling against hadronic initial state radiation.  Signal event characteristics are studied with Higgsino and slepton simplified models.  The directly produced electroweakinos and sleptons subsequently decay to their Standard Model partners and the lightest SUSY particle which is nearly degenerate in mass.  The energy given to the visible leptons is related to the mass-splitting between the neutral electroweakinos $\tilde\chi_2^0$ and $\tilde\chi_1^0$ or between the sleptons $\tilde\ell_{L,R}$ and the lightest neutral electroweakino $\tilde\chi_1^0$.  The relationship between lepton momentum and the mass-splittings provides discriminating variables unique to the electroweakino and slepton decays.  Electroweakinos signals are sensitive to the invariant mass of the dilepton system $m_{\ell\ell}$, and slepton signals are sensitive to the stransverse mass of the leptons and \met $m_{T2}^{100}$.  Inclusive and exclusive signal regions are binned in $m_{\ell\ell}$ for the search targeting electroweakino production, and in $m_{T2}^{100}$ for the search targeting sleptons.
 
The dominant backgrounds to signal event with soft leptons and \met are from jets faking leptons in the detector.  These are estimated with a data-driven fake factor technique and tested in a same-sign validation region that includes $ee+\mu e$ events in the electron channel, and $\mu\mu+e\mu$ events in the muon channel.  Irreducible backgrounds from $t\bar t$, $tW$, and $Z(\rightarrow\tau\tau)$+jets processes were estimated with Monte Carlo and normalized in data-driven control regions.  Irreducible diboson backgrounds were estimated with Monte Carlo and tested in a dedicated diboson validation region.  Low mass Drell Yan, Higgs, triboson, and multi-top backgrounds were estimated with Monte Carlo only.

Background only fits were performed on the CR-top and CR-tau to obtain background normalization parameters $\mu_{top} =0.72\pm0.13$ and $\mu_{\tau\tau}=1.02\pm0.09$, respectively.  The accuracy of the background prediction was tested in each of the validation regions and is consistently within $1.5~\sigma$ of the observed data.  Model independent upper limits were set at $95\%$ CL on the observed and expected upper limits on the number of signal events in the inclusive SRs were set with simultaneous fits in each SR and the CRs, assuming the background only hypothesis.  No significant excess in data over Standard Model background was found; therefore, results were consistent with Standard Model prediction. 

For model dependent interpretations, shape fits in $m_{\ell\ell}$ and $m_{T2}^{100}$ were performed.  These are full simultaneous fits over the exclusive, multi-binned SRs and the CRs including both signal and backgrounds predictions.  In the absence of significant excesses in data over background, result were interpreted as constraints on SUSY electroweakino and slepton models.  Higgsino models are excluded for next-to-lightest neutralino masses up to 130 GeV for mass splittings between 5 and 10 GeV. For mass splittings down to 3 GeV next-to-lightest neutralino masses are excluded up to 100 GeV.  For wino-bino simplified models, next-to-lightest neutralino masses are excluded up to 170 GeV for mass splittings above 10 GeV, and excluded up to 100 GeV for mass splittings down to 2.5 GeV.  For slepton simplified models, slepton masses are excluded up to 180 GeV for mass splittings down to 5 GeV. For mass splittings down to 1 GeV slepton masses are excluded up to 70 GeV.

Future prospects: more data and better triggers. better fake background estimates.  Way to model Drell-Yan and not cut away valuable low invariant mass phase-space.  Use of razor variables






\nocite{*}
\bibliographystyle{plain}
\bibliography{myThesis.bib}
%\printbibliography

\appendix
\chapter{Appendix A}
\label{ch:apndx:a}
Signal acceptance, efficiency, efficiency within acceptance, and signal leakage plots are all shown in this appendix.

\section{Acceptance}
Signal acceptance $\alpha$ is defined as the ratio of truth events that pass all signal region cuts over the total number of truth events in the TRUTH3 signal sample.  Both the numerator and denominator events are weighted by the event weight and the numerator events are also weighted by the $Z\rightarrow ll$ branching ratio and filter efficiency, which is mostly driven by the $\met > 50~\GeV$ requirement.  Signal acceptance is described in equation \ref{eq:acceptance}.\\


\begin{itemize}
\item Slepton \& Higgsino acceptances include branching fraction times filter efficiency $\text{BF} \times \epsilon_\text{filt}$ scale factor from SUSYTools.
\item Slepton acceptances have stau veto applied to the denominator using a global 1.5 scale factor.
\item Ran over p3135 TRUTH3 derivations of Higgsino and slepton samples.
\item The $z$-axis scale is fixed between $[0, 25] \times 10^{-3}$ for sleptons and $[0, 11]\times 10^{-4}$ for Higgsino grids.
\end{itemize}


  \begin{equation}
  \alpha = \frac{N_{truth,selected}\times{BR_{Z\rightarrow ll}}\times{\epsilon_{filter}}}{N_{truth,total}}
 \label{eq:acceptance}
  \end{equation}
\begin{figure}
        \centering
    \begin{subfigure}[b]{0.49\textwidth}
        \includegraphics[width=\textwidth]{/Users/sheenaschier/Documents/LaFiles/figures/thesis/acc_eff/acceptance/Slep_SRSF_iMT2a}
    \caption{$SR\ell\ell-m^{100}_{T2}[100,102]$}
    \end{subfigure}
    \begin{subfigure}[b]{0.49\textwidth}
        \includegraphics[width=\textwidth]{/Users/sheenaschier/Documents/LaFiles/figures/thesis/acc_eff/acceptance/Slep_SRSF_iMT2b}
    \caption{$SR\ell\ell-m^{100}_{T2}[100,105]$}
    \end{subfigure}
    \begin{subfigure}[b]{0.49\textwidth}
        \includegraphics[width=\textwidth]{/Users/sheenaschier/Documents/LaFiles/figures/thesis/acc_eff/acceptance/Slep_SRSF_iMT2c}
    \caption{$SR\ell\ell-m^{100}_{T2}[100,110]$}
    \end{subfigure}
    \begin{subfigure}[b]{0.49\textwidth}
        \includegraphics[width=\textwidth]{/Users/sheenaschier/Documents/LaFiles/figures/thesis/acc_eff/acceptance/Slep_SRSF_iMT2d}
    \caption{$SR\ell\ell-m^{100}_{T2}[100,120]$}
    \end{subfigure}
    \begin{subfigure}[b]{0.49\textwidth}
        \includegraphics[width=\textwidth]{/Users/sheenaschier/Documents/LaFiles/figures/thesis/acc_eff/acceptance/Slep_SRSF_iMT2e}
    \caption{$SR\ell\ell-m^{100}_{T2}[100,130]$}
    \end{subfigure}
    \begin{subfigure}[b]{0.49\textwidth}
        \includegraphics[width=\textwidth]{/Users/sheenaschier/Documents/LaFiles/figures/thesis/acc_eff/acceptance/Slep_SRSF_iMT2f}
    \caption{$SR\ell\ell-m^{100}_{T2}[100,\infty]$}
    \end{subfigure}
    \caption{\label{fig:slepton_truth_acceptance}\textbf{Slepton}.}
\end{figure}

\begin{figure}
        \centering
    \begin{subfigure}[b]{0.49\textwidth}
        \includegraphics[width=\textwidth]{/Users/sheenaschier/Documents/LaFiles/figures/thesis/acc_eff/acceptance/N2N1_SRSF_iMLLa}
    \caption{$SR\ell\ell-m_{\ell\ell} [1, 3]$}
    \end{subfigure}
    \begin{subfigure}[b]{0.49\textwidth}
        \includegraphics[width=\textwidth]{/Users/sheenaschier/Documents/LaFiles/figures/thesis/acc_eff/acceptance/N2N1_SRSF_iMLLb}
    \caption{$SR\ell\ell-m_{\ell\ell} [1, 5]$}
    \end{subfigure}
    \begin{subfigure}[b]{0.49\textwidth}
        \includegraphics[width=\textwidth]{/Users/sheenaschier/Documents/LaFiles/figures/thesis/acc_eff/acceptance/N2N1_SRSF_iMLLc}
    \caption{$SR\ell\ell-m_{\ell\ell} [1, 10]$}
    \end{subfigure}
    \begin{subfigure}[b]{0.49\textwidth}
        \includegraphics[width=\textwidth]{/Users/sheenaschier/Documents/LaFiles/figures/thesis/acc_eff/acceptance/N2N1_SRSF_iMLLd}
    \caption{$SR\ell\ell-m_{\ell\ell} [1, 20]$}
    \end{subfigure}
    \begin{subfigure}[b]{0.49\textwidth}
        \includegraphics[width=\textwidth]{/Users/sheenaschier/Documents/LaFiles/figures/thesis/acc_eff/acceptance/N2N1_SRSF_iMLLe}
    \caption{$SR\ell\ell-m_{\ell\ell} [1, 30]$}
    \end{subfigure}
    \begin{subfigure}[b]{0.49\textwidth}
        \includegraphics[width=\textwidth]{/Users/sheenaschier/Documents/LaFiles/figures/thesis/acc_eff/acceptance/N2N1_SRSF_iMLLf}
    \caption{$SR\ell\ell-m_{\ell\ell} [1, 40]$}
    \end{subfigure}
    \begin{subfigure}[b]{0.49\textwidth}
        \includegraphics[width=\textwidth]{/Users/sheenaschier/Documents/LaFiles/figures/thesis/acc_eff/acceptance/N2N1_SRSF_iMLLg}
    \caption{$SR\ell\ell-m_{\ell\ell} [1, 60]$}
    \end{subfigure}
    \caption{\textbf{N2N1}.}
\end{figure}

\begin{figure}
        \centering
    \begin{subfigure}[b]{0.49\textwidth}
        \includegraphics[width=\textwidth]{/Users/sheenaschier/Documents/LaFiles/figures/thesis/acc_eff/acceptance/N2C1p_SRSF_iMLLa}
    \caption{$SR\ell\ell-m_{\ell\ell} [1, 3]$}
    \end{subfigure}
    \begin{subfigure}[b]{0.49\textwidth}
        \includegraphics[width=\textwidth]{/Users/sheenaschier/Documents/LaFiles/figures/thesis/acc_eff/acceptance/N2C1p_SRSF_iMLLb}
    \caption{$SR\ell\ell-m_{\ell\ell} [1, 5]$}
    \end{subfigure}
    \begin{subfigure}[b]{0.49\textwidth}
        \includegraphics[width=\textwidth]{/Users/sheenaschier/Documents/LaFiles/figures/thesis/acc_eff/acceptance/N2C1p_SRSF_iMLLc}
    \caption{$SR\ell\ell-m_{\ell\ell} [1, 10]$}
    \end{subfigure}
    \begin{subfigure}[b]{0.49\textwidth}
        \includegraphics[width=\textwidth]{/Users/sheenaschier/Documents/LaFiles/figures/thesis/acc_eff/acceptance/N2C1p_SRSF_iMLLd}
    \caption{$SR\ell\ell-m_{\ell\ell} [1, 20]$}
    \end{subfigure}
    \begin{subfigure}[b]{0.49\textwidth}
        \includegraphics[width=\textwidth]{/Users/sheenaschier/Documents/LaFiles/figures/thesis/acc_eff/acceptance/N2C1p_SRSF_iMLLe}
    \caption{$SR\ell\ell-m_{\ell\ell} [1, 30]$}
    \end{subfigure}
    \begin{subfigure}[b]{0.49\textwidth}
        \includegraphics[width=\textwidth]{/Users/sheenaschier/Documents/LaFiles/figures/thesis/acc_eff/acceptance/N2C1p_SRSF_iMLLf}
    \caption{$SR\ell\ell-m_{\ell\ell} [1, 40]$}
    \end{subfigure}
    \begin{subfigure}[b]{0.49\textwidth}
        \includegraphics[width=\textwidth]{/Users/sheenaschier/Documents/LaFiles/figures/thesis/acc_eff/acceptance/N2C1p_SRSF_iMLLg}
    \caption{$SR\ell\ell-m_{\ell\ell} [1, 60]$}
    \end{subfigure}
    \caption{\textbf{N2C1p}.}
\end{figure}

\begin{figure}
        \centering
    \begin{subfigure}[b]{0.49\textwidth}
        \includegraphics[width=\textwidth]{/Users/sheenaschier/Documents/LaFiles/figures/thesis/acc_eff/acceptance/N2C1m_SRSF_iMLLa}
    \caption{$SR\ell\ell-m_{\ell\ell} [1, 3]$}
    \end{subfigure}
    \begin{subfigure}[b]{0.49\textwidth}
        \includegraphics[width=\textwidth]{/Users/sheenaschier/Documents/LaFiles/figures/thesis/acc_eff/acceptance/N2C1m_SRSF_iMLLb}
    \caption{$SR\ell\ell-m_{\ell\ell} [1, 5]$}
    \end{subfigure}
    \begin{subfigure}[b]{0.49\textwidth}
        \includegraphics[width=\textwidth]{/Users/sheenaschier/Documents/LaFiles/figures/thesis/acc_eff/acceptance/N2C1m_SRSF_iMLLc}
    \caption{$SR\ell\ell-m_{\ell\ell} [1, 10]$}
    \end{subfigure}
    \begin{subfigure}[b]{0.49\textwidth}
        \includegraphics[width=\textwidth]{/Users/sheenaschier/Documents/LaFiles/figures/thesis/acc_eff/acceptance/N2C1m_SRSF_iMLLd}
    \caption{$SR\ell\ell-m_{\ell\ell} [1, 20]$}
    \end{subfigure}
    \begin{subfigure}[b]{0.49\textwidth}
        \includegraphics[width=\textwidth]{/Users/sheenaschier/Documents/LaFiles/figures/thesis/acc_eff/acceptance/N2C1m_SRSF_iMLLe}
    \caption{$SR\ell\ell-m_{\ell\ell} [1, 30]$}
    \end{subfigure}
    \begin{subfigure}[b]{0.49\textwidth}
        \includegraphics[width=\textwidth]{/Users/sheenaschier/Documents/LaFiles/figures/thesis/acc_eff/acceptance/N2C1m_SRSF_iMLLf}
    \caption{$SR\ell\ell-m_{\ell\ell} [1, 40]$}
    \end{subfigure}
    \begin{subfigure}[b]{0.49\textwidth}
        \includegraphics[width=\textwidth]{/Users/sheenaschier/Documents/LaFiles/figures/thesis/acc_eff/acceptance/N2C1m_SRSF_iMLLg}
    \caption{$SR\ell\ell-m_{\ell\ell} [1, 60]$}
    \end{subfigure}
    \caption{\textbf{N2C1m}.}
\end{figure}


\begin{figure}
        \centering
    \begin{subfigure}[b]{0.49\textwidth}
        \includegraphics[width=\textwidth]{/Users/sheenaschier/Documents/LaFiles/figures/thesis/acc_eff/acceptance/C1C1_SRSF_iMLLa}
    \caption{$SR\ell\ell-m_{\ell\ell} [1, 3]$}
    \end{subfigure}
    \begin{subfigure}[b]{0.49\textwidth}
        \includegraphics[width=\textwidth]{/Users/sheenaschier/Documents/LaFiles/figures/thesis/acc_eff/acceptance/C1C1_SRSF_iMLLb}
    \caption{$SR\ell\ell-m_{\ell\ell} [1, 5]$}
    \end{subfigure}
    \begin{subfigure}[b]{0.49\textwidth}
        \includegraphics[width=\textwidth]{/Users/sheenaschier/Documents/LaFiles/figures/thesis/acc_eff/acceptance/C1C1_SRSF_iMLLc}
    \caption{$SR\ell\ell-m_{\ell\ell} [1, 10]$}
    \end{subfigure}
    \begin{subfigure}[b]{0.49\textwidth}
        \includegraphics[width=\textwidth]{/Users/sheenaschier/Documents/LaFiles/figures/thesis/acc_eff/acceptance/C1C1_SRSF_iMLLd}
    \caption{$SR\ell\ell-m_{\ell\ell} [1, 20]$}
    \end{subfigure}
    \begin{subfigure}[b]{0.49\textwidth}
        \includegraphics[width=\textwidth]{/Users/sheenaschier/Documents/LaFiles/figures/thesis/acc_eff/acceptance/C1C1_SRSF_iMLLe}
    \caption{$SR\ell\ell-m_{\ell\ell} [1, 30]$}
    \end{subfigure}
    \begin{subfigure}[b]{0.49\textwidth}
        \includegraphics[width=\textwidth]{/Users/sheenaschier/Documents/LaFiles/figures/thesis/acc_eff/acceptance/C1C1_SRSF_iMLLf}
    \caption{$SR\ell\ell-m_{\ell\ell} [1, 40]$}
    \end{subfigure}
    \begin{subfigure}[b]{0.49\textwidth}
        \includegraphics[width=\textwidth]{/Users/sheenaschier/Documents/LaFiles/figures/thesis/acc_eff/acceptance/C1C1_SRSF_iMLLg}
    \caption{$SR\ell\ell-m_{\ell\ell} [1, 60]$}
    \end{subfigure}
    \caption{\label{fig:c1c1_truth_acceptance}\textbf{C1C1}.}
\end{figure}

\FloatBarrier

\subsection{Efficiency}

\begin{itemize}
\item Slepton \& Higgsino efficiencies are derived using reconstructed events passing signal region cuts as the numerator and truth events passing signal region cuts as denominator.
\item Slepton efficiencies have stau veto applied to the denominator using a global 1.5 scale factor.
\item Ran over p3135 TRUTH3 derivations of Higgsino and slepton samples for truth events passing signal region cuts.
\item Ran over p2952 SUSY16 derivations of Higgsino and slepton samples for reconstructed events passing signal region cuts.
\end{itemize}

Signal efficiency, $\epsilon$, is defined as the ratio of reconstructed events that pass all signal region cuts to the total number of truth events that pass all signal region cuts.  Signal efficiency is described in equation \ref{eq:efficiency}.\\
\begin{equation}
\epsilon = \frac{N_{reco,selected}}{N_{truth,selected}}
\label{eq:efficiency}
\end{equation}
\FloatBarrier

\begin{figure}
        \centering
    \begin{subfigure}[b]{0.44\textwidth}
        \includegraphics[width=\textwidth]{/Users/sheenaschier/Documents/LaFiles/figures/thesis/acc_eff/efficiency/eff_iMT2a_slepton}
    \caption{$SR\ell\ell-m^{100}_{T2}[100,102]$}
    \end{subfigure}
    \begin{subfigure}[b]{0.44\textwidth}
        \includegraphics[width=\textwidth]{/Users/sheenaschier/Documents/LaFiles/figures/thesis/acc_eff/efficiency/eff_iMT2b_slepton}
    \caption{$SR\ell\ell-m^{100}_{T2}[100,105]$}
    \end{subfigure}
    \begin{subfigure}[b]{0.44\textwidth}
        \includegraphics[width=\textwidth]{/Users/sheenaschier/Documents/LaFiles/figures/thesis/acc_eff/efficiency/eff_iMT2c_slepton}
    \caption{$SR\ell\ell-m^{100}_{T2}[100,110]$}
    \end{subfigure}
    \begin{subfigure}[b]{0.44\textwidth}
        \includegraphics[width=\textwidth]{/Users/sheenaschier/Documents/LaFiles/figures/thesis/acc_eff/efficiency/eff_iMT2d_slepton}
    \caption{$SR\ell\ell-m^{100}_{T2}[100,120]$}
    \end{subfigure}
    \begin{subfigure}[b]{0.44\textwidth}
        \includegraphics[width=\textwidth]{/Users/sheenaschier/Documents/LaFiles/figures/thesis/acc_eff/efficiency/eff_iMT2e_slepton}
    \caption{$SR\ell\ell-m^{100}_{T2}[100,130]$}
    \end{subfigure}
    \begin{subfigure}[b]{0.44\textwidth}
        \includegraphics[width=\textwidth]{/Users/sheenaschier/Documents/LaFiles/figures/thesis/acc_eff/efficiency/eff_iMT2f_slepton}
    \caption{$SR\ell\ell-m^{100}_{T2}[100,\infty]$}
    \end{subfigure}
    \caption{\label{fig:slepton_efficiency}\textbf{Slepton Efficiency}.}
\end{figure}

\begin{figure}
        \centering
    \begin{subfigure}[b]{0.44\textwidth}
        \includegraphics[width=\textwidth]{/Users/sheenaschier/Documents/LaFiles/figures/thesis/acc_eff/efficiency/eff_iMLLa_N2N1}
    \caption{$SR\ell\ell-m_{\ell\ell} [1, 3]$}
    \end{subfigure}
    \begin{subfigure}[b]{0.44\textwidth}
        \includegraphics[width=\textwidth]{/Users/sheenaschier/Documents/LaFiles/figures/thesis/acc_eff/efficiency/eff_iMLLb_N2N1}
    \caption{$SR\ell\ell-m_{\ell\ell} [1, 5]$}
    \end{subfigure}
    \begin{subfigure}[b]{0.44\textwidth}
        \includegraphics[width=\textwidth]{/Users/sheenaschier/Documents/LaFiles/figures/thesis/acc_eff/efficiency/eff_iMLLc_N2N1}
    \caption{$SR\ell\ell-m_{\ell\ell} [1, 10]$}
    \end{subfigure}
    \begin{subfigure}[b]{0.44\textwidth}
        \includegraphics[width=\textwidth]{/Users/sheenaschier/Documents/LaFiles/figures/thesis/acc_eff/efficiency/eff_iMLLd_N2N1}
    \caption{$SR\ell\ell-m_{\ell\ell} [1, 20]$}
    \end{subfigure}
    \begin{subfigure}[b]{0.44\textwidth}
        \includegraphics[width=\textwidth]{/Users/sheenaschier/Documents/LaFiles/figures/thesis/acc_eff/efficiency/eff_iMLLe_N2N1}
    \caption{$SR\ell\ell-m_{\ell\ell} [1, 30]$}
    \end{subfigure}
    \begin{subfigure}[b]{0.44\textwidth}
        \includegraphics[width=\textwidth]{/Users/sheenaschier/Documents/LaFiles/figures/thesis/acc_eff/efficiency/eff_iMLLf_N2N1}
    \caption{$SR\ell\ell-m_{\ell\ell} [1, 40]$}
    \end{subfigure}
    \begin{subfigure}[b]{0.44\textwidth}
        \includegraphics[width=\textwidth]{/Users/sheenaschier/Documents/LaFiles/figures/thesis/acc_eff/efficiency/eff_iMLLg_N2N1}
    \caption{$SR\ell\ell-m_{\ell\ell} [1, 60]$}
    \end{subfigure}
    \caption{\textbf{N2N1 Efficiency}.}
\end{figure}

\begin{figure}
        \centering
    \begin{subfigure}[b]{0.44\textwidth}
        \includegraphics[width=\textwidth]{/Users/sheenaschier/Documents/LaFiles/figures/thesis/acc_eff/efficiency/eff_iMLLa_N2C1p}
    \caption{$SR\ell\ell-m_{\ell\ell} [1, 3]$}
    \end{subfigure}
    \begin{subfigure}[b]{0.44\textwidth}
        \includegraphics[width=\textwidth]{/Users/sheenaschier/Documents/LaFiles/figures/thesis/acc_eff/efficiency/eff_iMLLb_N2C1p}
    \caption{$SR\ell\ell-m_{\ell\ell} [1, 5]$}
    \end{subfigure}
    \begin{subfigure}[b]{0.44\textwidth}
        \includegraphics[width=\textwidth]{/Users/sheenaschier/Documents/LaFiles/figures/thesis/acc_eff/efficiency/eff_iMLLc_N2C1p}
    \caption{$SR\ell\ell-m_{\ell\ell} [1, 10]$}
    \end{subfigure}
    \begin{subfigure}[b]{0.44\textwidth}
        \includegraphics[width=\textwidth]{/Users/sheenaschier/Documents/LaFiles/figures/thesis/acc_eff/efficiency/eff_iMLLd_N2C1p}
    \caption{$SR\ell\ell-m_{\ell\ell} [1, 20]$}
    \end{subfigure}
    \begin{subfigure}[b]{0.44\textwidth}
        \includegraphics[width=\textwidth]{/Users/sheenaschier/Documents/LaFiles/figures/thesis/acc_eff/efficiency/eff_iMLLe_N2C1p}
    \caption{$SR\ell\ell-m_{\ell\ell} [1, 30]$}
    \end{subfigure}
    \begin{subfigure}[b]{0.44\textwidth}
        \includegraphics[width=\textwidth]{/Users/sheenaschier/Documents/LaFiles/figures/thesis/acc_eff/efficiency/eff_iMLLf_N2C1p}
    \caption{$SR\ell\ell-m_{\ell\ell} [1, 40]$}
    \end{subfigure}
    \begin{subfigure}[b]{0.44\textwidth}
        \includegraphics[width=\textwidth]{/Users/sheenaschier/Documents/LaFiles/figures/thesis/acc_eff/efficiency/eff_iMLLg_N2C1p}
    \caption{$SR\ell\ell-m_{\ell\ell} [1, 60]$}
    \end{subfigure}
    \caption{\textbf{N2C1p Efficiency}.}
\end{figure}

\begin{figure}
        \centering
    \begin{subfigure}[b]{0.44\textwidth}
        \includegraphics[width=\textwidth]{/Users/sheenaschier/Documents/LaFiles/figures/thesis/acc_eff/efficiency/eff_iMLLa_N2C1m}
    \caption{$SR\ell\ell-m_{\ell\ell} [1, 3]$}
    \end{subfigure}
    \begin{subfigure}[b]{0.44\textwidth}
        \includegraphics[width=\textwidth]{/Users/sheenaschier/Documents/LaFiles/figures/thesis/acc_eff/efficiency/eff_iMLLb_N2C1m}
    \caption{$SR\ell\ell-m_{\ell\ell} [1, 5]$}
    \end{subfigure}
    \begin{subfigure}[b]{0.44\textwidth}
        \includegraphics[width=\textwidth]{/Users/sheenaschier/Documents/LaFiles/figures/thesis/acc_eff/efficiency/eff_iMLLc_N2C1m}
    \caption{$SR\ell\ell-m_{\ell\ell} [1, 10]$}
    \end{subfigure}
    \begin{subfigure}[b]{0.44\textwidth}
        \includegraphics[width=\textwidth]{/Users/sheenaschier/Documents/LaFiles/figures/thesis/acc_eff/efficiency/eff_iMLLd_N2C1m}
    \caption{$SR\ell\ell-m_{\ell\ell} [1, 20]$}
    \end{subfigure}
    \begin{subfigure}[b]{0.44\textwidth}
        \includegraphics[width=\textwidth]{/Users/sheenaschier/Documents/LaFiles/figures/thesis/acc_eff/efficiency/eff_iMLLe_N2C1m}
    \caption{$SR\ell\ell-m_{\ell\ell} [1, 30]$}
    \end{subfigure}
    \begin{subfigure}[b]{0.44\textwidth}
        \includegraphics[width=\textwidth]{/Users/sheenaschier/Documents/LaFiles/figures/thesis/acc_eff/efficiency/eff_iMLLf_N2C1m}
    \caption{$SR\ell\ell-m_{\ell\ell} [1, 40]$}
    \end{subfigure}
    \begin{subfigure}[b]{0.44\textwidth}
        \includegraphics[width=\textwidth]{/Users/sheenaschier/Documents/LaFiles/figures/thesis/acc_eff/efficiency/eff_iMLLg_N2C1m}
    \caption{$SR\ell\ell-m_{\ell\ell} [1, 60]$}
    \end{subfigure}
    \caption{\textbf{N2C1m Efficiency}.}
\end{figure}


\begin{figure}
        \centering
    \begin{subfigure}[b]{0.44\textwidth}
        \includegraphics[width=\textwidth]{/Users/sheenaschier/Documents/LaFiles/figures/thesis/acc_eff/efficiency/eff_iMLLa_C1C1}
    \caption{$SR\ell\ell-m_{\ell\ell} [1, 3]$}
    \end{subfigure}
    \begin{subfigure}[b]{0.44\textwidth}
        \includegraphics[width=\textwidth]{/Users/sheenaschier/Documents/LaFiles/figures/thesis/acc_eff/efficiency/eff_iMLLb_C1C1}
    \caption{$SR\ell\ell-m_{\ell\ell} [1, 5]$}
    \end{subfigure}
    \begin{subfigure}[b]{0.44\textwidth}
        \includegraphics[width=\textwidth]{/Users/sheenaschier/Documents/LaFiles/figures/thesis/acc_eff/efficiency/eff_iMLLc_C1C1}
    \caption{$SR\ell\ell-m_{\ell\ell} [1, 10]$}
    \end{subfigure}
    \begin{subfigure}[b]{0.44\textwidth}
        \includegraphics[width=\textwidth]{/Users/sheenaschier/Documents/LaFiles/figures/thesis/acc_eff/efficiency/eff_iMLLd_C1C1}
    \caption{$SR\ell\ell-m_{\ell\ell} [1, 20]$}
    \end{subfigure}
    \begin{subfigure}[b]{0.44\textwidth}
        \includegraphics[width=\textwidth]{/Users/sheenaschier/Documents/LaFiles/figures/thesis/acc_eff/efficiency/eff_iMLLe_C1C1}
    \caption{$SR\ell\ell-m_{\ell\ell} [1, 30]$}
    \end{subfigure}
    \begin{subfigure}[b]{0.44\textwidth}
        \includegraphics[width=\textwidth]{/Users/sheenaschier/Documents/LaFiles/figures/thesis/acc_eff/efficiency/eff_iMLLf_C1C1}
    \caption{$SR\ell\ell-m_{\ell\ell} [1, 40]$}
    \end{subfigure}
    \begin{subfigure}[b]{0.44\textwidth}
        \includegraphics[width=\textwidth]{/Users/sheenaschier/Documents/LaFiles/figures/thesis/acc_eff/efficiency/eff_iMLLg_C1C1}
    \caption{$SR\ell\ell-m_{\ell\ell} [1, 60]$}
    \end{subfigure}
    \caption{\label{fig:c1c1_efficiency}\textbf{C1C1 Efficiency}.}
\end{figure}

\FloatBarrier
\subsection{Efficiency within Acceptance}

Figures~\ref{fig:slepton_efficiency_in_acceptance} to \ref{fig:c1c1_efficiency_in_acceptance} shows
the efficiencies for the Higgsino signals split by process and slepton signal.

\begin{itemize}
\item Slepton \& Higgsino efficiencies within acceptance are derived using truth events passing signal region cuts as the  denominator and reconstructed events passing signal region cuts that are matched to denominator events for the numerator.
\item Slepton efficiencies have stau veto applied to the denominator using a global 1.5 scale factor.
\item Ran over p3135 TRUTH3 derivations of Higgsino and slepton samples for truth events passing signal region cuts.
\item Ran over p2952 SUSY16 derivations of Higgsino and slepton samples for reconstructed events passing signal region cuts.
\end{itemize}

Efficiency within acceptance, denoted $\epsilon_{\alpha}$, is defined as the ratio of reconstructed events that pass all signal region cuts to the total number of truth events that pass all signal region cuts, but in this case, each reconstructed event in the numerator must match a truth event in the denominator. Efficiency within acceptance is described in equation \ref{eq:eff_win_acc}.\\
\begin{equation}
\epsilon_{\alpha} = \frac{N_{reco,selected,matched}}{N_{truth,selected}}
\label{eq:eff_win_acc}
\end{equation}
\FloatBarrier

\begin{figure}
        \centering
    \begin{subfigure}[b]{0.44\textwidth}
        \includegraphics[width=\textwidth]{/Users/sheenaschier/Documents/LaFiles/figures/thesis/acc_eff/efficiency/eff_win_acc_iMT2a_slepton}
    \caption{$SR\ell\ell-m^{100}_{T2}[100,102]$}
    \end{subfigure}
    \begin{subfigure}[b]{0.44\textwidth}
        \includegraphics[width=\textwidth]{/Users/sheenaschier/Documents/LaFiles/figures/thesis/acc_eff/efficiency/eff_win_acc_iMT2b_slepton}
    \caption{$SR\ell\ell-m^{100}_{T2}[100,105]$}
    \end{subfigure}
    \begin{subfigure}[b]{0.44\textwidth}
        \includegraphics[width=\textwidth]{/Users/sheenaschier/Documents/LaFiles/figures/thesis/acc_eff/efficiency/eff_win_acc_iMT2c_slepton}
    \caption{$SR\ell\ell-m^{100}_{T2}[100,110]$}
    \end{subfigure}
    \begin{subfigure}[b]{0.44\textwidth}
        \includegraphics[width=\textwidth]{/Users/sheenaschier/Documents/LaFiles/figures/thesis/acc_eff/efficiency/eff_win_acc_iMT2d_slepton}
    \caption{$SR\ell\ell-m^{100}_{T2}[100,120]$}
    \end{subfigure}
    \begin{subfigure}[b]{0.44\textwidth}
        \includegraphics[width=\textwidth]{/Users/sheenaschier/Documents/LaFiles/figures/thesis/acc_eff/efficiency/eff_win_acc_iMT2e_slepton}
    \caption{$SR\ell\ell-m^{100}_{T2}[100,130]$}
    \end{subfigure}
    \begin{subfigure}[b]{0.44\textwidth}
        \includegraphics[width=\textwidth]{/Users/sheenaschier/Documents/LaFiles/figures/thesis/acc_eff/efficiency/eff_win_acc_iMT2f_slepton}
    \caption{$SR\ell\ell-m^{100}_{T2}[100,\infty]$}
    \end{subfigure}
    \caption{\label{fig:slepton_efficiency_in_acceptance}\textbf{Slepton Efficiency within acceptance}.}
\end{figure}

\begin{figure}
        \centering
    \begin{subfigure}[b]{0.44\textwidth}
        \includegraphics[width=\textwidth]{/Users/sheenaschier/Documents/LaFiles/figures/thesis/acc_eff/efficiency/eff_win_acc_iMLLa_N2N1}
    \caption{$SR\ell\ell-m_{\ell\ell} [1, 3]$}
    \end{subfigure}
    \begin{subfigure}[b]{0.44\textwidth}
        \includegraphics[width=\textwidth]{/Users/sheenaschier/Documents/LaFiles/figures/thesis/acc_eff/efficiency/eff_win_acc_iMLLb_N2N1}
    \caption{$SR\ell\ell-m_{\ell\ell} [1, 5]$}
    \end{subfigure}
    \begin{subfigure}[b]{0.44\textwidth}
        \includegraphics[width=\textwidth]{/Users/sheenaschier/Documents/LaFiles/figures/thesis/acc_eff/efficiency/eff_win_acc_iMLLc_N2N1}
    \caption{$SR\ell\ell-m_{\ell\ell} [1, 10]$}
    \end{subfigure}
    \begin{subfigure}[b]{0.44\textwidth}
        \includegraphics[width=\textwidth]{/Users/sheenaschier/Documents/LaFiles/figures/thesis/acc_eff/efficiency/eff_win_acc_iMLLd_N2N1}
    \caption{$SR\ell\ell-m_{\ell\ell} [1, 20]$}
    \end{subfigure}
    \begin{subfigure}[b]{0.44\textwidth}
        \includegraphics[width=\textwidth]{/Users/sheenaschier/Documents/LaFiles/figures/thesis/acc_eff/efficiency/eff_win_acc_iMLLe_N2N1}
    \caption{$SR\ell\ell-m_{\ell\ell} [1, 30]$}
    \end{subfigure}
    \begin{subfigure}[b]{0.44\textwidth}
        \includegraphics[width=\textwidth]{/Users/sheenaschier/Documents/LaFiles/figures/thesis/acc_eff/efficiency/eff_win_acc_iMLLf_N2N1}
    \caption{$SR\ell\ell-m_{\ell\ell} [1, 40]$}
    \end{subfigure}
    \begin{subfigure}[b]{0.44\textwidth}
        \includegraphics[width=\textwidth]{/Users/sheenaschier/Documents/LaFiles/figures/thesis/acc_eff/efficiency/eff_win_acc_iMLLg_N2N1}
    \caption{$SR\ell\ell-m_{\ell\ell} [1, 60]$}
    \end{subfigure}
    \caption{\textbf{N2N1 Efficiency within acceptance}.}
\end{figure}

\begin{figure}
        \centering
    \begin{subfigure}[b]{0.44\textwidth}
        \includegraphics[width=\textwidth]{/Users/sheenaschier/Documents/LaFiles/figures/thesis/acc_eff/efficiency/eff_win_acc_iMLLa_N2C1p}
    \caption{$SR\ell\ell-m_{\ell\ell} [1, 3]$}
    \end{subfigure}
    \begin{subfigure}[b]{0.44\textwidth}
        \includegraphics[width=\textwidth]{/Users/sheenaschier/Documents/LaFiles/figures/thesis/acc_eff/efficiency/eff_win_acc_iMLLb_N2C1p}
    \caption{$SR\ell\ell-m_{\ell\ell} [1, 5]$}
    \end{subfigure}
    \begin{subfigure}[b]{0.44\textwidth}
        \includegraphics[width=\textwidth]{/Users/sheenaschier/Documents/LaFiles/figures/thesis/acc_eff/efficiency/eff_win_acc_iMLLc_N2C1p}
    \caption{$SR\ell\ell-m_{\ell\ell} [1, 10]$}
    \end{subfigure}
    \begin{subfigure}[b]{0.44\textwidth}
        \includegraphics[width=\textwidth]{/Users/sheenaschier/Documents/LaFiles/figures/thesis/acc_eff/efficiency/eff_win_acc_iMLLd_N2C1p}
    \caption{$SR\ell\ell-m_{\ell\ell} [1, 20]$}
    \end{subfigure}
    \begin{subfigure}[b]{0.44\textwidth}
        \includegraphics[width=\textwidth]{/Users/sheenaschier/Documents/LaFiles/figures/thesis/acc_eff/efficiency/eff_win_acc_iMLLe_N2C1p}
    \caption{$SR\ell\ell-m_{\ell\ell} [1, 30]$}
    \end{subfigure}
    \begin{subfigure}[b]{0.44\textwidth}
        \includegraphics[width=\textwidth]{/Users/sheenaschier/Documents/LaFiles/figures/thesis/acc_eff/efficiency/eff_win_acc_iMLLf_N2C1p}
    \caption{$SR\ell\ell-m_{\ell\ell} [1, 40]$}
    \end{subfigure}
    \begin{subfigure}[b]{0.44\textwidth}
        \includegraphics[width=\textwidth]{/Users/sheenaschier/Documents/LaFiles/figures/thesis/acc_eff/efficiency/eff_win_acc_iMLLg_N2C1p}
    \caption{$SR\ell\ell-m_{\ell\ell} [1, 60]$}
    \end{subfigure}
    \caption{\textbf{N2C1p Efficiency within acceptance}.}
\end{figure}

\begin{figure}
        \centering
    \begin{subfigure}[b]{0.44\textwidth}
        \includegraphics[width=\textwidth]{/Users/sheenaschier/Documents/LaFiles/figures/thesis/acc_eff/efficiency/eff_win_acc_iMLLa_N2C1m}
    \caption{$SR\ell\ell-m_{\ell\ell} [1, 3]$}
    \end{subfigure}
    \begin{subfigure}[b]{0.44\textwidth}
        \includegraphics[width=\textwidth]{/Users/sheenaschier/Documents/LaFiles/figures/thesis/acc_eff/efficiency/eff_win_acc_iMLLb_N2C1m}
    \caption{$SR\ell\ell-m_{\ell\ell} [1, 5]$}
    \end{subfigure}
    \begin{subfigure}[b]{0.44\textwidth}
        \includegraphics[width=\textwidth]{/Users/sheenaschier/Documents/LaFiles/figures/thesis/acc_eff/efficiency/eff_win_acc_iMLLc_N2C1m}
    \caption{$SR\ell\ell-m_{\ell\ell} [1, 10]$}
    \end{subfigure}
    \begin{subfigure}[b]{0.44\textwidth}
        \includegraphics[width=\textwidth]{/Users/sheenaschier/Documents/LaFiles/figures/thesis/acc_eff/efficiency/eff_win_acc_iMLLd_N2C1m}
    \caption{$SR\ell\ell-m_{\ell\ell} [1, 20]$}
    \end{subfigure}
    \begin{subfigure}[b]{0.44\textwidth}
        \includegraphics[width=\textwidth]{/Users/sheenaschier/Documents/LaFiles/figures/thesis/acc_eff/efficiency/eff_win_acc_iMLLe_N2C1m}
    \caption{$SR\ell\ell-m_{\ell\ell} [1, 30]$}
    \end{subfigure}
    \begin{subfigure}[b]{0.44\textwidth}
        \includegraphics[width=\textwidth]{/Users/sheenaschier/Documents/LaFiles/figures/thesis/acc_eff/efficiency/eff_win_acc_iMLLf_N2C1m}
    \caption{$SR\ell\ell-m_{\ell\ell} [1, 40]$}
    \end{subfigure}
    \begin{subfigure}[b]{0.44\textwidth}
        \includegraphics[width=\textwidth]{/Users/sheenaschier/Documents/LaFiles/figures/thesis/acc_eff/efficiency/eff_win_acc_iMLLg_N2C1m}
    \caption{$SR\ell\ell-m_{\ell\ell} [1, 60]$}
    \end{subfigure}
    \caption{\textbf{N2C1m Efficiency within acceptance}.}
\end{figure}


\begin{figure}
        \centering
    \begin{subfigure}[b]{0.44\textwidth}
        \includegraphics[width=\textwidth]{/Users/sheenaschier/Documents/LaFiles/figures/thesis/acc_eff/efficiency/eff_win_acc_iMLLa_C1C1}
    \caption{$SR\ell\ell-m_{\ell\ell} [1, 3]$}
    \end{subfigure}
    \begin{subfigure}[b]{0.44\textwidth}
        \includegraphics[width=\textwidth]{/Users/sheenaschier/Documents/LaFiles/figures/thesis/acc_eff/efficiency/eff_win_acc_iMLLb_C1C1}
    \caption{$SR\ell\ell-m_{\ell\ell} [1, 5]$}
    \end{subfigure}
    \begin{subfigure}[b]{0.44\textwidth}
        \includegraphics[width=\textwidth]{/Users/sheenaschier/Documents/LaFiles/figures/thesis/acc_eff/efficiency/eff_win_acc_iMLLc_C1C1}
    \caption{$SR\ell\ell-m_{\ell\ell} [1, 10]$}
    \end{subfigure}
    \begin{subfigure}[b]{0.44\textwidth}
        \includegraphics[width=\textwidth]{/Users/sheenaschier/Documents/LaFiles/figures/thesis/acc_eff/efficiency/eff_win_acc_iMLLd_C1C1}
    \caption{$SR\ell\ell-m_{\ell\ell} [1, 20]$}
    \end{subfigure}
    \begin{subfigure}[b]{0.44\textwidth}
        \includegraphics[width=\textwidth]{/Users/sheenaschier/Documents/LaFiles/figures/thesis/acc_eff/efficiency/eff_win_acc_iMLLe_C1C1}
    \caption{$SR\ell\ell-m_{\ell\ell} [1, 30]$}
    \end{subfigure}
    \begin{subfigure}[b]{0.44\textwidth}
        \includegraphics[width=\textwidth]{/Users/sheenaschier/Documents/LaFiles/figures/thesis/acc_eff/efficiency/eff_win_acc_iMLLf_C1C1}
    \caption{$SR\ell\ell-m_{\ell\ell} [1, 40]$}
    \end{subfigure}
    \begin{subfigure}[b]{0.44\textwidth}
        \includegraphics[width=\textwidth]{/Users/sheenaschier/Documents/LaFiles/figures/thesis/acc_eff/efficiency/eff_win_acc_iMLLg_C1C1}
    \caption{$SR\ell\ell-m_{\ell\ell} [1, 60]$}
    \end{subfigure}
    \caption{\label{fig:c1c1_efficiency_in_acceptance}\textbf{C1C1 Efficiency with acceptance}.}
\end{figure}

\FloatBarrier
\subsection{Signal Leakage}

Figures~\ref{fig:slepton_leakage} to \ref{fig:c1c1_leakage} shows
the leakage for the Higgsino signals split by process and slepton signal.

\begin{itemize}
\item Slepton \& Higgsino signal leakages are derived using truth events passing signal region cuts as the denominator and reconstructed events passing signal region cuts that do not match denominator events as the numerator.
\item Slepton efficiencies have stau veto applied to the denominator using a global 1.5 scale factor.
\item Ran over p3135 TRUTH3 derivations of Higgsino and slepton samples for truth events passing signal region cuts.
\item Ran over p2952 SUSY16 derivations of Higgsino and slepton samples for reconstructed events passing signal region cuts.
\end{itemize}


Signal leakage, $\lambda$, is defined as the ratio of reconstructed events in the signal region that are not matched to a truth event in the signal region divided by the total number of truth events in the signal region. Signal leakage is described in equation \ref{eq:leakage}.  Studies of the migrating truth quantities during recontruction reveal the \met{} and $M_\tau\tau$ and the two variables responsible for the majority of the leakage events.\\
\begin{equation}
\lambda = \frac{N_{reco,selected,unmatched}}{N_{truth,selected}}
\label{eq:leakage}
\end{equation}
%\begin{equation}
%\lambda_2 = \frac{N_{reco,sel,unmatched}}{N_{reco,sel}}
%\end{equation}
\FloatBarrier

\begin{figure}
        \centering
    \begin{subfigure}[b]{0.44\textwidth}
        \includegraphics[width=\textwidth]{/Users/sheenaschier/Documents/LaFiles/figures/thesis/acc_eff/efficiency/leakage_win_acc_iMT2a_slepton}
    \caption{$SR\ell\ell-m^{100}_{T2}[100,102]$}
    \end{subfigure}
    \begin{subfigure}[b]{0.44\textwidth}
        \includegraphics[width=\textwidth]{/Users/sheenaschier/Documents/LaFiles/figures/thesis/acc_eff/efficiency/leakage_win_acc_iMT2b_slepton}
    \caption{$SR\ell\ell-m^{100}_{T2}[100,105]$}
    \end{subfigure}
    \begin{subfigure}[b]{0.44\textwidth}
        \includegraphics[width=\textwidth]{/Users/sheenaschier/Documents/LaFiles/figures/thesis/acc_eff/efficiency/leakage_win_acc_iMT2c_slepton}
    \caption{$SR\ell\ell-m^{100}_{T2}[100,110]$}
    \end{subfigure}
    \begin{subfigure}[b]{0.44\textwidth}
        \includegraphics[width=\textwidth]{/Users/sheenaschier/Documents/LaFiles/figures/thesis/acc_eff/efficiency/leakage_win_acc_iMT2d_slepton}
    \caption{$SR\ell\ell-m^{100}_{T2}[100,120]$}
    \end{subfigure}
    \begin{subfigure}[b]{0.44\textwidth}
        \includegraphics[width=\textwidth]{/Users/sheenaschier/Documents/LaFiles/figures/thesis/acc_eff/efficiency/leakage_win_acc_iMT2e_slepton}
    \caption{$SR\ell\ell-m^{100}_{T2}[100,130]$}
    \end{subfigure}
    \begin{subfigure}[b]{0.44\textwidth}
        \includegraphics[width=\textwidth]{/Users/sheenaschier/Documents/LaFiles/figures/thesis/acc_eff/efficiency/leakage_win_acc_iMT2f_slepton}
    \caption{$SR\ell\ell-m^{100}_{T2}[100,\infty]$}
    \end{subfigure}
    \caption{\label{fig:slepton_leakage}\textbf{Slepton Signal Leakage}.}
\end{figure}

\begin{figure}
        \centering
    \begin{subfigure}[b]{0.44\textwidth}
        \includegraphics[width=\textwidth]{/Users/sheenaschier/Documents/LaFiles/figures/thesis/acc_eff/efficiency/leakage_win_acc_iMLLa_N2N1}
    \caption{$SR\ell\ell-m_{\ell\ell} [1, 3]$}
    \end{subfigure}
    \begin{subfigure}[b]{0.44\textwidth}
        \includegraphics[width=\textwidth]{/Users/sheenaschier/Documents/LaFiles/figures/thesis/acc_eff/efficiency/leakage_win_acc_iMLLb_N2N1}
    \caption{$SR\ell\ell-m_{\ell\ell} [1, 5]$}
    \end{subfigure}
    \begin{subfigure}[b]{0.44\textwidth}
        \includegraphics[width=\textwidth]{/Users/sheenaschier/Documents/LaFiles/figures/thesis/acc_eff/efficiency/leakage_win_acc_iMLLc_N2N1}
    \caption{$SR\ell\ell-m_{\ell\ell} [1, 10]$}
    \end{subfigure}
    \begin{subfigure}[b]{0.44\textwidth}
        \includegraphics[width=\textwidth]{/Users/sheenaschier/Documents/LaFiles/figures/thesis/acc_eff/efficiency/leakage_win_acc_iMLLd_N2N1}
    \caption{$SR\ell\ell-m_{\ell\ell} [1, 20]$}
    \end{subfigure}
    \begin{subfigure}[b]{0.44\textwidth}
        \includegraphics[width=\textwidth]{/Users/sheenaschier/Documents/LaFiles/figures/thesis/acc_eff/efficiency/leakage_win_acc_iMLLe_N2N1}
    \caption{$SR\ell\ell-m_{\ell\ell} [1, 30]$}
    \end{subfigure}
    \begin{subfigure}[b]{0.44\textwidth}
        \includegraphics[width=\textwidth]{/Users/sheenaschier/Documents/LaFiles/figures/thesis/acc_eff/efficiency/leakage_win_acc_iMLLf_N2N1}
    \caption{$SR\ell\ell-m_{\ell\ell} [1, 40]$}
    \end{subfigure}
    \begin{subfigure}[b]{0.44\textwidth}
        \includegraphics[width=\textwidth]{/Users/sheenaschier/Documents/LaFiles/figures/thesis/acc_eff/efficiency/leakage_win_acc_iMLLg_N2N1}
    \caption{$SR\ell\ell-m_{\ell\ell} [1, 60]$}
    \end{subfigure}
    \caption{\textbf{N2N1 Signal Leakage}.}
\end{figure}

\begin{figure}
        \centering
    \begin{subfigure}[b]{0.44\textwidth}
        \includegraphics[width=\textwidth]{/Users/sheenaschier/Documents/LaFiles/figures/thesis/acc_eff/efficiency/leakage_win_acc_iMLLa_N2C1p}
    \caption{$SR\ell\ell-m_{\ell\ell} [1, 3]$}
    \end{subfigure}
    \begin{subfigure}[b]{0.44\textwidth}
        \includegraphics[width=\textwidth]{/Users/sheenaschier/Documents/LaFiles/figures/thesis/acc_eff/efficiency/leakage_win_acc_iMLLb_N2C1p}
    \caption{$SR\ell\ell-m_{\ell\ell} [1, 5]$}
    \end{subfigure}
    \begin{subfigure}[b]{0.44\textwidth}
        \includegraphics[width=\textwidth]{/Users/sheenaschier/Documents/LaFiles/figures/thesis/acc_eff/efficiency/leakage_win_acc_iMLLc_N2C1p}
    \caption{$SR\ell\ell-m_{\ell\ell} [1, 10]$}
    \end{subfigure}
    \begin{subfigure}[b]{0.44\textwidth}
        \includegraphics[width=\textwidth]{/Users/sheenaschier/Documents/LaFiles/figures/thesis/acc_eff/efficiency/leakage_win_acc_iMLLd_N2C1p}
    \caption{$SR\ell\ell-m_{\ell\ell} [1, 20]$}
    \end{subfigure}
    \begin{subfigure}[b]{0.44\textwidth}
        \includegraphics[width=\textwidth]{/Users/sheenaschier/Documents/LaFiles/figures/thesis/acc_eff/efficiency/leakage_win_acc_iMLLe_N2C1p}
    \caption{$SR\ell\ell-m_{\ell\ell} [1, 30]$}
    \end{subfigure}
    \begin{subfigure}[b]{0.44\textwidth}
        \includegraphics[width=\textwidth]{/Users/sheenaschier/Documents/LaFiles/figures/thesis/acc_eff/efficiency/leakage_win_acc_iMLLf_N2C1p}
    \caption{$SR\ell\ell-m_{\ell\ell} [1, 40]$}
    \end{subfigure}
    \begin{subfigure}[b]{0.44\textwidth}
        \includegraphics[width=\textwidth]{/Users/sheenaschier/Documents/LaFiles/figures/thesis/acc_eff/efficiency/leakage_win_acc_iMLLg_N2C1p}
    \caption{$SR\ell\ell-m_{\ell\ell} [1, 60]$}
    \end{subfigure}
    \caption{\textbf{N2C1p Signal Leakage}.}
\end{figure}

\begin{figure}
        \centering
    \begin{subfigure}[b]{0.44\textwidth}
        \includegraphics[width=\textwidth]{/Users/sheenaschier/Documents/LaFiles/figures/thesis/acc_eff/efficiency/leakage_win_acc_iMLLa_N2C1m}
    \caption{$SR\ell\ell-m_{\ell\ell} [1, 3]$}
    \end{subfigure}
    \begin{subfigure}[b]{0.44\textwidth}
        \includegraphics[width=\textwidth]{/Users/sheenaschier/Documents/LaFiles/figures/thesis/acc_eff/efficiency/leakage_win_acc_iMLLb_N2C1m}
    \caption{$SR\ell\ell-m_{\ell\ell} [1, 5]$}
    \end{subfigure}
    \begin{subfigure}[b]{0.44\textwidth}
        \includegraphics[width=\textwidth]{/Users/sheenaschier/Documents/LaFiles/figures/thesis/acc_eff/efficiency/leakage_win_acc_iMLLc_N2C1m}
    \caption{$SR\ell\ell-m_{\ell\ell} [1, 10]$}
    \end{subfigure}
    \begin{subfigure}[b]{0.44\textwidth}
        \includegraphics[width=\textwidth]{/Users/sheenaschier/Documents/LaFiles/figures/thesis/acc_eff/efficiency/leakage_win_acc_iMLLd_N2C1m}
    \caption{$SR\ell\ell-m_{\ell\ell} [1, 20]$}
    \end{subfigure}
    \begin{subfigure}[b]{0.44\textwidth}
        \includegraphics[width=\textwidth]{/Users/sheenaschier/Documents/LaFiles/figures/thesis/acc_eff/efficiency/leakage_win_acc_iMLLe_N2C1m}
    \caption{$SR\ell\ell-m_{\ell\ell} [1, 30]$}
    \end{subfigure}
    \begin{subfigure}[b]{0.44\textwidth}
        \includegraphics[width=\textwidth]{/Users/sheenaschier/Documents/LaFiles/figures/thesis/acc_eff/efficiency/leakage_win_acc_iMLLf_N2C1m}
    \caption{$SR\ell\ell-m_{\ell\ell} [1, 40]$}
    \end{subfigure}
    \begin{subfigure}[b]{0.44\textwidth}
        \includegraphics[width=\textwidth]{/Users/sheenaschier/Documents/LaFiles/figures/thesis/acc_eff/efficiency/leakage_win_acc_iMLLg_N2C1m}
    \caption{$SR\ell\ell-m_{\ell\ell} [1, 60]$}
    \end{subfigure}
    \caption{\textbf{N2C1m Signal Leakage}.}
\end{figure}


\begin{figure}
        \centering
    \begin{subfigure}[b]{0.44\textwidth}
        \includegraphics[width=\textwidth]{/Users/sheenaschier/Documents/LaFiles/figures/thesis/acc_eff/efficiency/leakage_win_acc_iMLLa_C1C1}
    \caption{$SR\ell\ell-m_{\ell\ell} [1, 3]$}
    \end{subfigure}
    \begin{subfigure}[b]{0.44\textwidth}
        \includegraphics[width=\textwidth]{/Users/sheenaschier/Documents/LaFiles/figures/thesis/acc_eff/efficiency/leakage_win_acc_iMLLb_C1C1}
    \caption{$SR\ell\ell-m_{\ell\ell} [1, 5]$}
    \end{subfigure}
    \begin{subfigure}[b]{0.44\textwidth}
        \includegraphics[width=\textwidth]{/Users/sheenaschier/Documents/LaFiles/figures/thesis/acc_eff/efficiency/leakage_win_acc_iMLLc_C1C1}
    \caption{$SR\ell\ell-m_{\ell\ell} [1, 10]$}
    \end{subfigure}
    \begin{subfigure}[b]{0.44\textwidth}
        \includegraphics[width=\textwidth]{/Users/sheenaschier/Documents/LaFiles/figures/thesis/acc_eff/efficiency/leakage_win_acc_iMLLd_C1C1}
    \caption{$SR\ell\ell-m_{\ell\ell} [1, 20]$}
    \end{subfigure}
    \begin{subfigure}[b]{0.44\textwidth}
        \includegraphics[width=\textwidth]{/Users/sheenaschier/Documents/LaFiles/figures/thesis/acc_eff/efficiency/leakage_win_acc_iMLLe_C1C1}
    \caption{$SR\ell\ell-m_{\ell\ell} [1, 30]$}
    \end{subfigure}
    \begin{subfigure}[b]{0.44\textwidth}
        \includegraphics[width=\textwidth]{/Users/sheenaschier/Documents/LaFiles/figures/thesis/acc_eff/efficiency/leakage_win_acc_iMLLf_C1C1}
    \caption{$SR\ell\ell-m_{\ell\ell} [1, 40]$}
    \end{subfigure}
    \begin{subfigure}[b]{0.44\textwidth}
        \includegraphics[width=\textwidth]{/Users/sheenaschier/Documents/LaFiles/figures/thesis/acc_eff/efficiency/leakage_win_acc_iMLLg_C1C1}
    \caption{$SR\ell\ell-m_{\ell\ell} [1, 60]$}
    \end{subfigure}
    \caption{\label{fig:c1c1_leakage}\textbf{C1C1 Signal Leakage}.}
\end{figure}

\FloatBarrier
\subsection{Acceptance*Efficiency}

%Figures~\ref{fig:slepton_efficiency_in_acceptance} to \ref{fig:c1c1_efficiency_in_acceptance} shows
Figures~\ref{fig:c1c1_efficiency_in_acceptance} shows
the acceptance*efficiencies for the Higgsino signals split by process and slepton signal.

\begin{itemize}
\item Slepton \& Higgsino acceptance times efficiencies are derived using truth events passing signal region cuts as the denominator and reconstructed events passing signal region cuts that are matched to denominator events for the numerator.
\item Slepton efficiencies have stau veto applied to the denominator using a global 1.5 scale factor.
\item Ran over p3135 TRUTH3 derivations of Higgsino and slepton samples for truth events passing signal region cuts.
\item Ran over p2952 SUSY16 derivations of Higgsino and slepton samples for reconstructed events passing signal region cuts.
\end{itemize}

Acceptance times efficiency is defined as the ratio of reconstructed events that pass all signal region cuts over the total number of truth events in the TRUTH3 signal sample, but in this case, the denominator events are weighted by the event weight and the numerator events are weighted by the event weight, the detector weights, the filter efficiency, and the $Z\rightarrow ll$ branching ratio.  Acceptance multiplied by efficiency is described in equation \ref{eq:effacc}.\\
\begin{equation}
    \alpha\times{\epsilon} = \frac{\Sigma \textit{w}_{reco,sel}\times{BR_{Z\rightarrow ll}}\times{\epsilon_{filter}}}{\Sigma \textit{evt\_w}_{truth, tot}}
\label{eq:effacc}
\end{equation}
%\begin{equation}
%    w_{gen} = \frac{\sigma \times{BR_{Z\rightarrow ll}}\times{\epsilon_{filter}}}{\Sigma \textit{evt\_w}_{truth, tot}}
%\label{eq:effacc}
%\end{equation}
%\begin{equation}
%    \alpha\times{\epsilon} = \frac{N_{reco,sel}\times{w_{det}}\times{w_{evt}}\times{w_{gen}}}{{\sigma}}
%\label{eq:effacc}
%\end{equation}
\FloatBarrier

\begin{figure}
        \centering
    \begin{subfigure}[b]{0.44\textwidth}
        \includegraphics[width=\textwidth]{/Users/sheenaschier/Documents/LaFiles/figures/thesis/acc_eff/efficiency/effacc_iMT2a_slepton}
    \caption{$SR\ell\ell-m^{100}_{T2}[100,102]$}
    \end{subfigure}
    \begin{subfigure}[b]{0.44\textwidth}
        \includegraphics[width=\textwidth]{/Users/sheenaschier/Documents/LaFiles/figures/thesis/acc_eff/efficiency/effacc_iMT2b_slepton}
    \caption{$SR\ell\ell-m^{100}_{T2}[100,105]$}
    \end{subfigure}
    \begin{subfigure}[b]{0.44\textwidth}
        \includegraphics[width=\textwidth]{/Users/sheenaschier/Documents/LaFiles/figures/thesis/acc_eff/efficiency/effacc_iMT2c_slepton}
    \caption{$SR\ell\ell-m^{100}_{T2}[100,110]$}
    \end{subfigure}
    \begin{subfigure}[b]{0.44\textwidth}
        \includegraphics[width=\textwidth]{/Users/sheenaschier/Documents/LaFiles/figures/thesis/acc_eff/efficiency/effacc_iMT2d_slepton}
    \caption{$SR\ell\ell-m^{100}_{T2}[100,120]$}
    \end{subfigure}
    \begin{subfigure}[b]{0.44\textwidth}
        \includegraphics[width=\textwidth]{/Users/sheenaschier/Documents/LaFiles/figures/thesis/acc_eff/efficiency/effacc_iMT2e_slepton}
    \caption{$SR\ell\ell-m^{100}_{T2}[100,130]$}
    \end{subfigure}
    \begin{subfigure}[b]{0.44\textwidth}
        \includegraphics[width=\textwidth]{/Users/sheenaschier/Documents/LaFiles/figures/thesis/acc_eff/efficiency/effacc_iMT2f_slepton}
    \caption{$SR\ell\ell-m^{100}_{T2}[100,\infty]$}
    \end{subfigure}
    \caption{\label{fig:slepton_efficiency_in_acceptance}\textbf{Slepton Acceptance*Efficiency}.}
\end{figure}

\begin{figure}
        \centering
    \begin{subfigure}[b]{0.44\textwidth}
        \includegraphics[width=\textwidth]{/Users/sheenaschier/Documents/LaFiles/figures/thesis/acc_eff/efficiency/effacc_iMLLa_N2N1}
    \caption{$SR\ell\ell-m_{\ell\ell} [1, 3]$}
    \end{subfigure}
    \begin{subfigure}[b]{0.44\textwidth}
        \includegraphics[width=\textwidth]{/Users/sheenaschier/Documents/LaFiles/figures/thesis/acc_eff/efficiency/effacc_iMLLb_N2N1}
    \caption{$SR\ell\ell-m_{\ell\ell} [1, 5]$}
    \end{subfigure}
    \begin{subfigure}[b]{0.44\textwidth}
        \includegraphics[width=\textwidth]{/Users/sheenaschier/Documents/LaFiles/figures/thesis/acc_eff/efficiency/effacc_iMLLc_N2N1}
    \caption{$SR\ell\ell-m_{\ell\ell} [1, 10]$}
    \end{subfigure}
    \begin{subfigure}[b]{0.44\textwidth}
        \includegraphics[width=\textwidth]{/Users/sheenaschier/Documents/LaFiles/figures/thesis/acc_eff/efficiency/effacc_iMLLd_N2N1}
    \caption{$SR\ell\ell-m_{\ell\ell} [1, 20]$}
    \end{subfigure}
    \begin{subfigure}[b]{0.44\textwidth}
        \includegraphics[width=\textwidth]{/Users/sheenaschier/Documents/LaFiles/figures/thesis/acc_eff/efficiency/effacc_iMLLe_N2N1}
    \caption{$SR\ell\ell-m_{\ell\ell} [1, 30]$}
    \end{subfigure}
    \begin{subfigure}[b]{0.44\textwidth}
        \includegraphics[width=\textwidth]{/Users/sheenaschier/Documents/LaFiles/figures/thesis/acc_eff/efficiency/effacc_iMLLf_N2N1}
    \caption{$SR\ell\ell-m_{\ell\ell} [1, 40]$}
    \end{subfigure}
    \begin{subfigure}[b]{0.44\textwidth}
        \includegraphics[width=\textwidth]{/Users/sheenaschier/Documents/LaFiles/figures/thesis/acc_eff/efficiency/effacc_iMLLg_N2N1}
    \caption{$SR\ell\ell-m_{\ell\ell} [1, 60]$}
    \end{subfigure}
    \caption{\textbf{N2N1 Acceptance*Efficiency}.}
\end{figure}

\begin{figure}
        \centering
    \begin{subfigure}[b]{0.44\textwidth}
        \includegraphics[width=\textwidth]{/Users/sheenaschier/Documents/LaFiles/figures/thesis/acc_eff/efficiency/effacc_iMLLa_N2C1p}
    \caption{$SR\ell\ell-m_{\ell\ell} [1, 3]$}
    \end{subfigure}
    \begin{subfigure}[b]{0.44\textwidth}
        \includegraphics[width=\textwidth]{/Users/sheenaschier/Documents/LaFiles/figures/thesis/acc_eff/efficiency/effacc_iMLLb_N2C1p}
    \caption{$SR\ell\ell-m_{\ell\ell} [1, 5]$}
    \end{subfigure}
    \begin{subfigure}[b]{0.44\textwidth}
        \includegraphics[width=\textwidth]{/Users/sheenaschier/Documents/LaFiles/figures/thesis/acc_eff/efficiency/effacc_iMLLc_N2C1p}
    \caption{$SR\ell\ell-m_{\ell\ell} [1, 10]$}
    \end{subfigure}
    \begin{subfigure}[b]{0.44\textwidth}
        \includegraphics[width=\textwidth]{/Users/sheenaschier/Documents/LaFiles/figures/thesis/acc_eff/efficiency/effacc_iMLLd_N2C1p}
    \caption{$SR\ell\ell-m_{\ell\ell} [1, 20]$}
    \end{subfigure}
    \begin{subfigure}[b]{0.44\textwidth}
        \includegraphics[width=\textwidth]{/Users/sheenaschier/Documents/LaFiles/figures/thesis/acc_eff/efficiency/effacc_iMLLe_N2C1p}
    \caption{$SR\ell\ell-m_{\ell\ell} [1, 30]$}
    \end{subfigure}
    \begin{subfigure}[b]{0.44\textwidth}
        \includegraphics[width=\textwidth]{/Users/sheenaschier/Documents/LaFiles/figures/thesis/acc_eff/efficiency/effacc_iMLLf_N2C1p}
    \caption{$SR\ell\ell-m_{\ell\ell} [1, 40]$}
    \end{subfigure}
    \begin{subfigure}[b]{0.44\textwidth}
        \includegraphics[width=\textwidth]{/Users/sheenaschier/Documents/LaFiles/figures/thesis/acc_eff/efficiency/effacc_iMLLg_N2C1p}
    \caption{$SR\ell\ell-m_{\ell\ell} [1, 60]$}
    \end{subfigure}
    \caption{\textbf{N2C1p Acceptance*Efficiency}.}
\end{figure}

\begin{figure}
        \centering
    \begin{subfigure}[b]{0.44\textwidth}
        \includegraphics[width=\textwidth]{/Users/sheenaschier/Documents/LaFiles/figures/thesis/acc_eff/efficiency/effacc_iMLLa_N2C1m}
    \caption{$SR\ell\ell-m_{\ell\ell} [1, 3]$}
    \end{subfigure}
    \begin{subfigure}[b]{0.44\textwidth}
        \includegraphics[width=\textwidth]{/Users/sheenaschier/Documents/LaFiles/figures/thesis/acc_eff/efficiency/effacc_iMLLb_N2C1m}
    \caption{$SR\ell\ell-m_{\ell\ell} [1, 5]$}
    \end{subfigure}
    \begin{subfigure}[b]{0.44\textwidth}
        \includegraphics[width=\textwidth]{/Users/sheenaschier/Documents/LaFiles/figures/thesis/acc_eff/efficiency/effacc_iMLLc_N2C1m}
    \caption{$SR\ell\ell-m_{\ell\ell} [1, 10]$}
    \end{subfigure}
    \begin{subfigure}[b]{0.44\textwidth}
        \includegraphics[width=\textwidth]{/Users/sheenaschier/Documents/LaFiles/figures/thesis/acc_eff/efficiency/effacc_iMLLd_N2C1m}
    \caption{$SR\ell\ell-m_{\ell\ell} [1, 20]$}
    \end{subfigure}
    \begin{subfigure}[b]{0.44\textwidth}
        \includegraphics[width=\textwidth]{/Users/sheenaschier/Documents/LaFiles/figures/thesis/acc_eff/efficiency/effacc_iMLLe_N2C1m}
    \caption{$SR\ell\ell-m_{\ell\ell} [1, 30]$}
    \end{subfigure}
    \begin{subfigure}[b]{0.44\textwidth}
        \includegraphics[width=\textwidth]{/Users/sheenaschier/Documents/LaFiles/figures/thesis/acc_eff/efficiency/effacc_iMLLf_N2C1m}
    \caption{$SR\ell\ell-m_{\ell\ell} [1, 40]$}
    \end{subfigure}
    \begin{subfigure}[b]{0.44\textwidth}
        \includegraphics[width=\textwidth]{/Users/sheenaschier/Documents/LaFiles/figures/thesis/acc_eff/efficiency/effacc_iMLLg_N2C1m}
    \caption{$SR\ell\ell-m_{\ell\ell} [1, 60]$}
    \end{subfigure}
    \caption{\textbf{N2C1m Acceptance*Efficiency}.}
\end{figure}


\begin{figure}
        \centering
    \begin{subfigure}[b]{0.44\textwidth}
        \includegraphics[width=\textwidth]{/Users/sheenaschier/Documents/LaFiles/figures/thesis/acc_eff/efficiency/effacc_iMLLa_C1C1}
    \caption{$SR\ell\ell-m_{\ell\ell} [1, 3]$}
    \end{subfigure}
    \begin{subfigure}[b]{0.44\textwidth}
        \includegraphics[width=\textwidth]{/Users/sheenaschier/Documents/LaFiles/figures/thesis/acc_eff/efficiency/effacc_iMLLb_C1C1}
    \caption{$SR\ell\ell-m_{\ell\ell} [1, 5]$}
    \end{subfigure}
    \begin{subfigure}[b]{0.44\textwidth}
        \includegraphics[width=\textwidth]{/Users/sheenaschier/Documents/LaFiles/figures/thesis/acc_eff/efficiency/effacc_iMLLc_C1C1}
    \caption{$SR\ell\ell-m_{\ell\ell} [1, 10]$}
    \end{subfigure}
    \begin{subfigure}[b]{0.44\textwidth}
        \includegraphics[width=\textwidth]{/Users/sheenaschier/Documents/LaFiles/figures/thesis/acc_eff/efficiency/effacc_iMLLd_C1C1}
    \caption{$SR\ell\ell-m_{\ell\ell} [1, 20]$}
    \end{subfigure}
    \begin{subfigure}[b]{0.44\textwidth}
        \includegraphics[width=\textwidth]{/Users/sheenaschier/Documents/LaFiles/figures/thesis/acc_eff/efficiency/effacc_iMLLe_C1C1}
    \caption{$SR\ell\ell-m_{\ell\ell} [1, 30]$}
    \end{subfigure}
    \begin{subfigure}[b]{0.44\textwidth}
        \includegraphics[width=\textwidth]{/Users/sheenaschier/Documents/LaFiles/figures/thesis/acc_eff/efficiency/effacc_iMLLf_C1C1}
    \caption{$SR\ell\ell-m_{\ell\ell} [1, 40]$}
    \end{subfigure}
    \begin{subfigure}[b]{0.44\textwidth}
        \includegraphics[width=\textwidth]{/Users/sheenaschier/Documents/LaFiles/figures/thesis/acc_eff/efficiency/effacc_iMLLg_C1C1}
    \caption{$SR\ell\ell-m_{\ell\ell} [1, 60]$}
    \end{subfigure}
    \caption{\label{fig:c1c1_efficiency_in_acceptance}\textbf{C1C1 Acceptance*Efficiency}.}
\end{figure}

\chapter{Appendix B}
Auxiliary CR and VR material

\section{Control Region Plots}


 \input{/Users/sheenaschier/Documents/LaFiles/figures/thesis/control_plots}
Ancillary material should be put in appendices, which appear after the
bibliography. 

\end{document}
