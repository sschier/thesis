%% uctest.tex 11/3/94
%% Copyright (C) 1988-2004 Daniel Gildea, BBF, Ethan Munson.
%
% This work may be distributed and/or modified under the
% conditions of the LaTeX Project Public License, either version 1.3
% of this license or (at your option) any later version.
% The latest version of this license is in
%   http://www.latex-project.org/lppl.txt
% and version 1.3 or later is part of all distributions of LaTeX
% version 2003/12/01 or later.
%
% This work has the LPPL maintenance status "maintained".
% 
% The Current Maintainer of this work is Daniel Gildea.
\newcommand*{\ATLASLATEXPATH}{/Users/sheenaschier/Library/TexShop/texmf/tex/latex/atlaslatex-01-07-01/latex/}
\documentclass[11pt]{ucthesis}
\def\dsp{\def\baselinestretch{2.0}\large\normalsize}
\dsp
\usepackage{\ATLASLATEXPATH atlasphysics}
\usepackage{graphicx}     
\usepackage{amsmath}
\usepackage{accents}
\usepackage{tikz}	
\usepackage{placeins}
\usepackage{caption}
\usepackage{subcaption}
\usepackage[font={small}]{caption}
\usepackage{lineno}
\usepackage{booktabs}
\linenumbers

\begin{document}

% Declarations for Front Matter

\title{Searches for Electroweak Production of Compressed Supersymmetry in Events with Soft Leptons, Missing Transverse Momentum, and a Hard Jet}
\author{Sheena Calie Schier}
\degreeyear{2018}
\degreemonth{June}
\degree{DOCTOR OF PHILOSOPHY}
\chair{Professor Abraham Seiden}
\committeememberone{Professor Jason Neilsen}
\committeemembertwo{Professor Michael Hance}
\numberofmembers{3} 
\deanlineone{Dean Tyrus Miller}
\deanlinetwo{Vice Provost and Dean of Graduate Studies}
\deanlinethree{}
\field{Physics}
\campus{Santa Cruz}

\begin{frontmatter}

\maketitle
\copyrightpage

\tableofcontents
\listoffigures
\listoftables

\makeatletter
\newcommand{\rmnum}[1]{\romannumeral #1}
\newcommand{\Rmnum}[1]{\expandafter\@slowromancap\romannumeral #1@}
\makeatother
%TODO: LINE NUMBERS

\begin{abstract}
Supersymmetry (SUSY) is an extension of the Standard Model that predicts a boson (fermion) partner for each fermion (boson) in the Standard Model. Weak-scale SUSY is attractive for reasons like improving gauge coupling unification, reducing fine-tuning in the Higgs sector and providing a dark matter candidate. In this thesis, I present a dedicated search for direct production of new, colorless, weak-scale states with a compressed mass spectra in final states characterized by soft visible decay products. This analysis uses $pp$ collisions at $\sqrt s$ = 13 TeV at the Large Hadron Collider and collected by the ATLAS experiment during 2015 and 2016 corresponding to 36.1 $\mathrm{fb}^{-1}$ of integrated luminosity. This analysis selects events with two soft electrons or muons and missing transverse momentum (\met{}) recoiling against hadronic initial state radiation. Backgrounds with two prompt leptons are estimated with Monte Carlo simulation, while reducible backgrounds are estimated with a mix of Monte Carlo and data-driven methods. Results are consistent with Standard Model expectations and used to put limits on compressed supersymmetric states.  Limits are extended on compressed electroweak SUSY model for the first time since LEP.

\end{abstract}

\begin{dedication}
\null\vfil
{\large
\begin{center}
To my father,\\\vspace{12pt}
Lecil Charles Schier,\\\vspace{12pt}
the person who taught me at the age of 7 that the grass in not green.
\end{center}}
\vfil\null
\end{dedication}


\begin{acknowledgements}
I want to ``thank'' my committee, without whose ridiculous demands, I
would have graduated so, so, very much faster.
\end{acknowledgements}

\end{frontmatter}

\chapter{Introduction}

Since the 1930's, when the world's first particle accelerator went online at the Cavendish Laboratory in Cambridge, England, colliding protons against a fixed lithium target, high energy collisions have been proving physicists with portals into the subatomic realm where quantum physics is the supreme ruler.  Progressively, particle accelerators have become more and more powerful, and the depth at which physicists can peer into the atom, into the structure of particles, and eventually into interactions of the most fundamental, has hastened.  Today, we stand at the energy frontier of particle experiments with a complete map of fundamental particles and interactions in hand to guide us through the sea of quantum possibilities, while astronomical observations, for one, give us the distinct sense that we are holding only a small slice of the truth.  

Currently, the Large Hadron Collider (LHC) is the largest and most powerful accelerator on Earth, colliding protons with a center of mass energy of $13~\TeV$.  With this machine, we step into the realm of Big Bang physics, where all possibilities 





\part{Theoretical Motivation and Experimental Setup}
\chapter{Theoretical Background and Motivation}
To any curious mind staring into the starry deep late in the night or gazing at pictures from the Hubble Space Telescope, the universe can seem deeply mysterious as a vast space containing a rich spectra of matter moving and transforming via some set of complex mechanisms.  Although this mysterious sense of the universe rings true even in the mind of the most learned physics scholar, large leaps in understanding the true nature of the matter and forces that make up the observable universe have been made in human history.   In the last century, particle physicists have constructed a theory that incorporates all the directly observed fundamental particles and explains their existence and interactions in simplicity through the field equations that describe the fundamental forces in the universe.  This theory is called the Standard Model of Particle Physics (SM) and, apart from gravity being far too weak to be described by particle interactions, is fundamentally complete.  \textcolor{red}{mention here to discovery of the Higgs boson in 2012, and it confirming the theory of electroweak symmetry breaking, which remained the last undiscovered piece of the SM puzzle.}

But the story doesn't end here.  There are reasons to think the complete and successful Standard Model is a lower-order version of a much larger theory.  Some of these reasons are philosophical in nature; we want to understand why the SM has its structure, or lack confidence in a theory that is so incredibly fine-tuned as the Standard Model.  Other reasons come from observations that we can not resolve with the SM, like the lack of CP-violation in Standard Model mechanics to account for the baryon-antibaryon asymmetry in the early universe, or the abundance of 'dark matter' that drives massive galaxies to rotate contrary to predictions by models accounting only for the known matter and forces of the Standard Model.

The proceeding structure of this chapter is as follows: Section~\ref{sec:forces}, summarizes all the forces and particles in the standard model, then Section~\ref{sec:gauge} , describes the gauge symmetries that give the Standard Model its particular structure and spontaneous electroweak symmetry breaking  that calls for the existence of the Higgs boson.  %Next, Section~\ref{sec:higgs} relates electroweak symmetry breaking to the Higgs mechanism and the gauge boson masses.  
Section~\ref{sec:fail} goes over some of the shortcomings of the Standard Model, and supersymmetry is introduced in Section~\ref{sec:susy} as a viable model for physics beyond the Standard Model.  Lastly, Section~\ref{sec:pheno} describes the phenomenology of supersymmetric Higgsinos and sleptons in compressed scenarios.

\iffalse
By the early 1930's, particle physics theory came out of nuclear physics studies of electrons and atomic nuclei.  At this time, of the observed particle phenomena could be explained by a very small set of particles consisting of the electron and positron, proton and neutron, the photon, and the neutrino and anti-neutrino.  Electromagnetism was still understood through Maxwell's Equations and weak force interactions were newly enlightened by Enrico Fermi's development of contact interactions.  Soon, our knowledge of particles and interactions was proven to be insufficient as relativistic calculations were being attempted and new particle discoveries were emerging from cloud and bubble chamber experiments.
\fi

\section{Forces and Particles}
\label{sec:forces}
The Standard Model of Particle Physics provides a quantum description of three of the four known fundamental forces; the electromagnetic force, the strong force, and the weak force.  It leaves out the gravitational force because the energy scale at which gravity does its business is many orders of magnitude below the other forces, which leads to intrinsic incompatibilities in a description of quantum gravitational interactions.   The SM was pieced together throughout the second half of the twentieth century by several progressive discoveries, and we now know that there are only a hand full of fundamental particles that make up the incredible collection of particles in nature.  The fundamental particles separate into two distinct categories: fermions and bosons.  These two types of particles are characterized by their spin and interactions, and ultimately play completely different roles in the state and phenomena of the universe. \cite{tully}


Fermions fall into two categorized, leptons and quarks.  Leptons are spin $\frac{1}{2}$ particles that do not carry strong color charge, but they do carry electromagnetic charge and weak isospin.  The lepton group consists of three generations of electron and anti-electron ($e^\pm, \mu^\pm, \tau^\pm$) and their associated neutrino and anti-neutrino partners\\ ($\nu_e/\tilde{\nu_e}, \nu_\mu/\tilde{\nu_\mu}, \nu_\tau/\tilde{\nu_\tau}$).  Quarks are strongly charged particles that also carry weak isospin and fractional electromagnetic charge.  Like the leptons, there are three quark families, each forming an isospin doublet and consisting of an up-type and a down-type quark: {(u,d), (c, s), (t,b)}.

Bosons are integer spin particles that are often called force-carriers because they are responsible for mediating interactions between particles. 

\textcolor{red}{\textit{Include one table that summarizes and categorizes all fermions and bosons.}}


\iffalse
\section{Mathematical Formalism of the Standard Model}
\label{sec:math}
Field equations and fermion and boson interactions.

A major push in particle physics in the twentieth century was to describe the strong and weak nuclear forces by renormalized quantum fields.  The related field quanta for the strong force are the eight massless gluons, and for the weak force we have the $W^\pm$ and the $Z^0$ bosons.  The interactions of these force carriers with particles carrying the corresponding color and isospin charge describe by field theories. 

Group theory plays an important role in describing the forces and interactions of fundamental particles.  It provides a mathematical structure that exploits the underlying symmetries behind the fundamental forces.  The structure of the quark combinations that give rise to a slew of mesons and baryons is given by the $SU(3)_{color}$ symmetry.  It contains the $SU(2)$ isospin symmetry, where sets of quark combinations either form SU(2) doublets or triplets. \textit{singlets?}. Group theory is also used to describe the unification of the electromagnetic and weak interactions to form the electroweak theory of the Standard Model. 

Feynman diagrams provide a visual representing the matrix calculations of particle interactions.  Feynman rules make up the mathematical expression relating to each element of a Feynman diagram derived by quantum field theory.  \textcolor{red}{\textit{Give example?}}
\fi
\section{Gauge Symmetries and Spontaneous Symmetry Breaking}
\label{sec:gauge}
The main ingredients of the Standard Model are a set of Dirac fermion fields having specific muliplet representations in group theory given by the $U(1)_{Y} \times SU(2)_{L} \times SU(3)_{S}$ gauge group.  In SM quantum field theory (QFT), called the "Yang Mills theory" \textcolor{red}{reference here}, fermion interactions are mediated by gauge bosons.  The structure of the gauge bosons and the interactions they govern is a consequence of gauge invariance in $SU(n)$ type Lie groups \textcolor{red}{find reference}.  

Gauge invariance in QFT demands the existence of gauge boson fields, which occur in two independent sectors: the electroweak sector described by quantum electroweak dynamics (QED), and the strong sector, described by quantum chromodynamics (QCD).  The $U(1)\otimes SU(2)$ symmetry of QED produces the photon and the weak gauge bosons, $W^\pm$ and $Z^0$, and the $SU(3)$ symmetry produces a color octet of massless gluons.  For the symmetries to be exact, all the force carriers must be massless, and an external mechanism called the \textit{Higgs mechanism} induces masses in the electroweak gauge bosons, $W^\pm$ and $Z^0$.  To quantize this exchange, an extra boson called the Higgs boson must exist, and it's discovery at the LHC was recently announced in June 2012.

Electrodynamics is 'gauge-invariant', meaning one obtains the same solutions to Maxwell's equations under a transformation of the electromagnetic 4-vector potential: $A_\mu \rightarrow A_\mu - \partial_\mu\Lambda$, where $\Lambda$ is some scalar function.  Under this gauge transformation, the wave function changes by a phase $\psi\rightarrow\psi e^{-ie\Lambda}$, revealing a $U(1)$ symmetry.

\iffalse
\section{Higgs Mechanism and Gauge Boson Masses}
\label{sec:higgs}
Explain local SU(2) gauge symmetry breaking, the production of the Higgs boson and how this allows for massive weakly interacting gauge bosons.
\fi

\section{Shortcomings of the Standard Model}
\label{sec:fail}
As mentioned before, the Standard Model of Particle Physics in all its glory has some deficiencies. \textcolor{red}{List and briefly explain.}

One alarming problem with the Standard Model is its incapability to explain dark matter.

\section{Supersymmetry}
\label{sec:susy}
Supersymmetry offers an extension to the Standard Model by extending the Poincare symmetry of quantum field theory.  This extension leads to boson-fermion symmetry which predicts a supersymmetric partner for all standard model partners that are equivalent in mass and all quantum characteristics but differ intrinsically by half-integer spin.  So, each SM fermion has a bosonic supersymmetric partner, and each SM boson has a fermionic supersymmetric partner.  According to this symmetry, assuming it is a perfect symmetry, these particles should have already been observed with their SM masses, but this is not the case.  In order for this theory to remain true, the new symmetry must be broken in a way that preserves the fermion-boson symmetry and all observations of the Standard Model while allowing fermion-boson partners to be decoupled in mass.  A description of the various models for mediating this symmetry-breaking and communicating it the visible sector of observable particles is beyond the scope of this thesis, but I will say a few words about electroweak symmetry breaking models since they are the focus of this thesis.

Add a SUSY picture and make sure I have at least alluded to all the particles.

Explain about electroweak symmetry breaking

Talk about winos, binos, and Higgsinos and how they mix to make up the charginos and neutralinos.

Explain about naturalness in electroweak SUSY and how this would make the Higgsino light, at the electroweak scale, and how the winos and binos, with masses given by $M_1$ and $M_2$, can still be heavy.

\textit{Try} to explain how the electroweakino mass spectrum works and that the more Higgsino like they are the smaller the mass-splittings become, until pure Higgsino states are completely degenerate

\section{Phenomenology of Directly Produced Higgsinos and Sleptons in Compressed Scenarios}
\label{sec:pheno}
Say here a word or two about how these two searches are so similar they are combined into one search effort with minor differences in their search strategies.
\subsection{Higgsino Simplified Models}
Higgsinos are the superpartners of the Standard Model Higgs doublets, the masses of which are controlled by the $\mu$ parameter, which, in supersymmetry, enters directly into the Higgs mass mixing matrix for calculating the squared Higgs mass $M_H^2$. \cite{han}  \textcolor{red}{Naturalness of the Higgsinos refers to the fact that in order for electroweak symmetry breaking to occur at the correct scale without any unnatural corrections, the parameter $\mu$ must be near the weak scale $\approx 100~\GeV$.}. Other supersymmetric particles enter the mass matrix indirectly through quantum loop corrections, but the Higgsinos are the only particle to have a direct effect on the Higgs mass.  This make Higgsinos a powerful tool in understanding electroweak symmetry breaking in SUSY.

Could search for Higgsinos through direct production of squarks that then decay to Higgsinos, but these particles have little effect on the mass of the Higgs and therefor, may naturally have masses well beyond the reach of the LHC.  Also, Higgsino models are very sensitive to the spectrum of light SUSY particles when trying to observe them through direct squark production.

 \begin{figure}%[h!]
  \begin{center}
  \includegraphics[width=0.5\textwidth]{/Users/sheenaschier/Documents/LaFiles/figures/thesis/theory/C1N2-WZN1N1}
   \end{center}
 \caption{Feynman diagram of direct Higgsino production}
 \label{fig:fn1}
 \end{figure}
 
Direct Higgsino production allows one to remain fairly agnostic to the spectrum of the SUSY sector, and therefore, retain sensitivity to a large range of EWSB SUSY models.  Unfortunately, the direct production of electroweakinos, including Higgsinos, is subject to electroweak cross-sections, limiting the search sensitivity at the LHC.  \textit{Refer to the Feynman diagram in Figure~\ref{fig:fn1}.}  When the mass differences between the electroweakinos are close to mass of the $W$ boson, Standard Model $W$ and $Z$ bosons are produced on-shell, or produced at their nominal masses, and about $30\%$ of the time will decay leptonically, subsequently giving birth to detectable leptons.  In this case, analyses have been performed in both ATLAS and CMS to search for all three leptons from the $W$ and $Z$, where the $Z$ can be reconstructed from an opposite-sign-same-flavor lepton pair.  The searches also require a substantial amount of missing transverse momentum from the lightest neutral electroweakinos.

When the mass-splittings fall below the $W$ mass, the $W$ and $Z$ bosons are produced off-shell, they are lighter than their nominal $80-90~\GeV$ mass, and the leptons from these decays become softer, or less energetic.  This type of analysis is limited by how well these events are recorded, and how efficiently leptons are reconstructed at low energies.  Until recently, no experiment at the LHC has been able to search for these models with electroweakino mass-splittings below $\approx 60~\GeV$.  \textcolor{red}{Introduce the 2 lepton search and the Feynman diagram.}

\textit{Say something before here about using simplified models}.
One important kinematic feature in these Higgsino simplified models is the dilepton invariant mass distribution and how it is linked to the mass-splitting between the chargino and the lightest neutralino through the mass of the very off-shell $Z$.

 \begin{figure}%[h!]
  \begin{center}
  \includegraphics[width=0.7\textwidth]{/Users/sheenaschier/Documents/LaFiles/figures/thesis/theory/Feynman_2}
   \end{center}
 \caption{Feynman diagram of direct Higgsino production in compressed scenario}
 \label{fig:fn1}
 \end{figure}
 
 \subsection{Slepton Simplified Models}
 Explain slepton models and direct production of sleptons
 
 Talk about searches for models with large mass-splittings and how there is redundancy with the Higgsino models in why these searches have not been available for models with small mass-splitting between the sleptons, whcih assume a fourfold degeneracy, and the lightest neutral electroweakino.
 
 Talk about the $m{T2}$ variable and why is s great discriminator for these signals.
 
  \begin{figure}%[h!]
  \begin{center}
  \includegraphics[width=0.5\textwidth]{/Users/sheenaschier/Documents/LaFiles/figures/thesis/theory/slsl-llN1N1j.pdf}
   \end{center}
 \caption{Feynman diagram of direct slepton production}
 \label{fig:fn1}
 \end{figure}
\chapter{ATLAS Experiment}
%\label{sec:detector}
The Large Hadron Collider (LHC) is a 27 km long circular proton accelerator with proton beams moving in opposite directions around the ring at speeds near 99.99\% the speed of light.  The beams travel around the ring in separate vacuum beam pipes and are accelerated and directed around the ring using gigantic semi-conducting magnets.  To reach LHC energies the proton beams are accelerated in smaller accelerator structures gradually increasing in size until they are injected into the LHC, which is still the largest and most powerful accelerator in the world.  The beams are made to collide at 4 different interaction points.  At each of the interaction points, located approximately 100m underground, are 4 different detector experiments: ALICE, LHC-B, CMS, and ATLAS.  The first LHC $pp$ collisions were recorded in 2009 at center of mass energy $\sqrt{7}~TeV$, and since operated at several different center of mass energies, including the most recent and highest of $\sqrt{13}~TeV$.

%(http://inspirehep.net/record/1240374/files/CHARGED%202012_011.pdf)
ATLAS is a large multi-purpose particle physics detector with forward-backward detecting capabilities in the end-caps and the symmetric cylindrical barrel.  The complete detector system is 44m long, 25m in diameter, and weighs 4000 tons.  The ALTAS detector is comprised of several sub-detector systems, each calibrated and optimized for a different observational purpose.  Listed in order from the center of ATLAS outward, the sub-detectors are: the inner tracking detector (ID), the electromagnetic calorimeter (eCAL), the hadronic calorimeter (hCAL), and the muon spectrometer (MS).Together, these sub-detectors measure the energy and momentum of a variety of particles and reconstruct the dynamics of each recorded event.

The combination of the detector systems provide charged particle measurements and efficient lepton and photon measurements out to $|\eta| < 2.4$.  Jets are MET are reconstructed using the full set of information out to $|\eta| < 4.9$.  

%%%%%%%%%%%%%%%%%%%%%%
\section{Inner Tracking Detector}
The ID is composed of three separate tracking systems; the silicon pixel detector, the silicon microstrip semi-conductor tracker, and the straw-tube transition radiation tracker.  This section will overview the setup and capabilities of each component of the ID.  The ID is surrounded by a superconducting solenoid that encases the entire ID in a 2 Tesla magnetic field.  The material in each sub-tracker only interacts with charged particles.  The 2 T magnetic field bends the charged particles traversing the tracker with a curvature related to each particle's momentum.
\subsection{Pixel Detector}
Inner most pixelated tracker.
\subsection{Semi-Conductor Tracker}
Middle silicon microstrip tracker.
\subsection{Transition Radiation Tracker}
Outer most straw tube transition radiation tracker.

%%%%%%%%%%%%%%%%%%%%%%
\section{Calorimetry}
**Extra coverage in the forward regions $3.2~<~|\eta|~<~4.9$ with LAr calorimeters for electromagnetic and hadronic measurements.
\subsection{Electromagnetic Calorimeter}
The eCAL is divided into a central barrel (pseudorapidity $|\eta|~<~1.475$) and two end-caps enclosing each side of the barrel.  The end-cap regions have an outer wheel corresponding to $1.375~<~|\eta|~<~2.5$, and an inner wheel for $2.5~<~|\eta|~<~3.2$.  The region $|\eta|~<~2.5$, which matches the coverage of the inner detector, is segmented into three layers.  The first layer is the most finely segmented in $\eta$ to aid the discrimination between true photons and neutral pions that have decayed to a pair of pions.  Both objects are trackless in flight and undetectable until they interact with the eCAL.  Closely-spaced photons from a boosted neutral pion decays can not be resolved into two objects without the extremely fine grain of this first layer.  The fine grain also helps improve the resolution of the shower position, shape and direction.  The eCAL is preceded by a pre-sampler at $|\eta| < 1.8$ to correct for upstream energy losses.
Measures energy of electromagnetic objects.

\subsection{Hadronic Calorimeter}
Iron-scintillator/tile makes up the central hadronic calorimeter for $|\eta| < 1.7$ while the LAr hadronic end-caps cover the $\eta$ range $1.5~<~|\eta|~<~3.2$.  
Measures energy of hadronic objects

%%%%%%%%%%%%%%%%%%%%%%
\section{Muon System}
%https://arxiv.org/pdf/1603.05598.pdf
The muon spectrometer, a tracking detector dedicated entirely to tracking muons, is the outer most sub-detector in ATLAS.  It is designed to track muons in the pseudorapidity region $|\eta|~<~2.7$ with a central barrel covering $|\eta|~<~1.05$ and two end-caps at $1.05~<~|\eta|~<~2.7$.  A network of three large super-conducting toroidal magnets, each with eight coils, supplies a magnetic to the muon spectrometer with am integral bending power in the barrel of around 2.5 Tm and up to 6Tm in the end caps.  Resistive plate chambers in the central region $|\eta|~<~1.05$ and end gap chambers in the forward-backward region $1.05~<~|\eta|~<~2.7$ impart triggering capabilities to the MS as well as position measurements in $\eta$ and $\phi$ with a spacial resolution of 5-10mm. Monitored drift tube chambers provide precision tracking out to $|\eta| < 2.7$ where each chamber provides 6-8 hits in $\eta$ along the muon flight path. 

%%%%%%%%%%%%%%%%%%%%%%
\section{DAQ and Trigger}
Complex computing system to acquire and store data.


\part{Method}
\chapter{Data Collection and Simulated Events}
\label{sec:data}
The chapter will describe the nature of the datasets and events selected from the datasets for analysis.  LHC data is subjected to an analysis of events in the search for compressed electroweak SUSY.  Monto Carlo simulated events generated with a Higgsino simplified model were studied to for a kinematic and topological understanding of compressed electroweak signals of this type and inform our search region targeting LHC data.  Monte Carlo event generation is also used for Standard Model background modeling in the optimized signal region.  All LHC data and simulated events are required to pass event triggers based a \met{} threshold.  In ATLAS data, these triggers are hardware and software based as described in Chapter~\ref{ch:detector}.  In simulation, the triggers are emulated at reconstruction level with ATLAS reconstruction software \textcolor{red}{I know this isn't right, but figure out what is}
 \textcolor{red}{Maybe mention the derivation and the framework?}
 


\section{Triggers}
\label{sec:eff}

\begin{table}[!htb]
\begin{center}
\begin{tabular}{cc}
\hline
Data Period  & Lowest Unprescaled \met Trigger \\
\hline \hline
\textbf{2015} &  \\
%\hline
All & HLT\_xe70\_mht\_L1E50\\   
\hline \hline
\textbf{2016}  &   \\
%\hline
A1-D3 & HLT\_xe90\_mht\_L1E50\\   
D4-L11 & HLT\_xe110\_mht\_L1E50\\   
\hline
\end{tabular}
\caption{Evolution of lowest unprescaled \met trigger from the start of 2015 to the end of 2016. \textcolor{red}{add run numbers and integrated lumi for each trigger}}
\label{tab:trigevol}
\end{center}
\end{table}

The MET trigger threshold varies by data taking period, where the lowest unprescaled inclusive MET trigger is used.  (Make table and refer to it)
\subsection{MET Triggers}
\label{sec:met}
Inclusive met trigger efficiencies

\subsection{Combined Trigger Studiies}
Near the end of period I, starting at run 308084, two new triggers went on the ATLAS trigger menu.
\begin{itemize}
\item \texttt{HLT\_mu4\_j125\_xe90\_mht} (seeded from \texttt{L1\_MU4\_J50\_XE40}),
 \item \texttt{HLT\_2mu4\_j85\_xe50\_mht} (seeded from \texttt{L1\_2MU4\_J40\_XE20}).
\end{itemize}
Lepton plus jet plus met trigger efficiencies..  
**Talk about the development and study of the new triggers implemented n data starting at run 308084, corresponding to an integrated luminosity of $8.8~fb^{-1}$.

  \begin{figure}[tbp]
   % \centering
     \includegraphics[width=0.48\columnwidth]{/Users/sheenaschier/Documents/LaFiles/figures/thesis/eventselection/eff_MM_signal_110_100.pdf}
       \includegraphics[width=0.48\columnwidth]{/Users/sheenaschier/Documents/LaFiles/figures/thesis/eventselection/eff_MM_mtautau_110_100.pdf}\\
   \caption{Trigger Efficiency as a function of MET after event preselection (left) and in a signal region similar to the analysis signal region (right)}
   \label{fig:TrigEff1}
 \end{figure}
 
   \begin{figure}[tbp]
   % \centering
     \includegraphics[width=0.48\columnwidth]{/Users/sheenaschier/Documents/LaFiles/figures/thesis/eventselection/eff_MM_jet145_110_100.pdf}
       \includegraphics[width=0.48\columnwidth]{/Users/sheenaschier/Documents/LaFiles/figures/thesis/eventselection/eff_MM_jet105_110_100.pdf}\\
   \caption{Trigger efficiency as a function of MET for the combined single muon trigger (left) and the combined dimuon trigger (right)}
   \label{fig:TrigEff2}
 \end{figure}
 
\section{Data}
In 2015 and 2016, ATLAS recorded a combined $36.1~fb^{-1}$ total integrated luminosity of LHC $pp$ collision data at $\sqrt{13}~TeV$ that passed data quality cuts (\textcolor{red}{Can you describe what these are?}), empowering numerous new physics searches that were not possible in Run-1. \textcolor{red}{Maybe discern Run-1 and Run-2 at the LHC so this has context?}  Over 90\% of Run-2 data came from 2016.  Peak instantaneous luminosity progressed from $5\times10^{33}~cm^{-2}~s^{-1}$ in 2015, to $13.8\times10^{33}~cm^{-2}~s^{-1}$ in 2016.  \textcolor{red}{How did $\mu$ evolve?} The number of interactions per event averaged ($\mu$) was 13.5 in 2015 and 25 in 2016, with a peak $\mu$ just over 40 near the end of 2016.  %This analysis only uses events that fired the lowest unprescaled inclusive MET trigger according to its data period.  

\section{Simulation}
\textcolor{blue}{ATLAS generated fully simulated Monte Carlo samples that behave like the raw data in the detector}. Monte Carlo samples used in this analysis were part of the mc15 production campaign.
\subsection{Signal samples}
Data simulated with Monte Carlo based event generation techniques by the \textcolor{blue}{ATLAS simulation infrastructure} is used in the analysis for background and signal modeling. 
Refer to the cross-sections for Higgsino and Slepton processes in Fig.~\ref{fig:xsection}
  \begin{figure}[tbp]
   \includegraphics[width=0.6\columnwidth]{/Users/sheenaschier/Documents/LaFiles/figures/thesis/signal_samples/slepton_higgsino_13TeV_xsect.pdf}
   \label{fig:xsection}
  \end{figure}
\subsubsection{Higgsino LSP Samples}
Higgsino simplified model samples include four processes: $\tilde\chi_2^0\chi_1^+$, $\tilde\chi_2^0\chi_1^-$ , $\tilde\chi_2^0\chi_1^0$, and $\tilde\chi_1^+\chi_1^-$.
\begin{itemize}
\item Explain the mass grid: The $\tilde\chi_1^\pm$ masses were fixed to (), while the $\tilde\chi_2^0$ and $\chi_1^0$ masses varied between such and such.  
\item Talk about the cross-sections of these processes
\item Talk about how the signal is produced from $Z^*$ decays and so the $\tilde\chi_1^+\chi_1^-$ samples don't really play a role in the optimization.  The cross-sections are too low and the lepton pairs come from two $W$-boson decays.
\item Explain that radiative corrections give rise to mass-splittings of pure Higgsino states of order MeV, and some level of wino or bino mixing is needed for larger mass splittings.  The models used to generate the signal samples use cross-sections according to electroweak mixing matrices that assume purely Higgsino states for all mass combinations of $\tilde\chi_2^0, \chi_1^0$, $\tilde\chi_1^+$, and $\chi_1^-$.
\item Branching ratios for $\tilde\chi_2^0 \rightarrow Z^*\tilde\chi_1^0$ and $\tilde\chi_1^\pm \rightarrow W^* \tilde\chi_1^0$ are fixed at 100\%
\item $Z^*\rightarrow \ell^+\ell^-$ modeled with SUSY-HIT v1.5b, which correctly treats the finite b-hadron and $\tau$-lepton masses.  %A. Djouadi, M. M. Muhlleitner, and M. Spira, Decays of supersymmetric particles: The Program SUSY-HIT (SUspect-SdecaY-Hdecay-InTerface), Acta Phys. Polon. B 38 (2007) 635, arXiv: hep-ph/0609292. 
\item The branching ratio $Z^* \rightarrow \ell^+\ell^-$ depends on the invariant mass of the $Z^*$, which is driven by the mass-splitting between $\chi^0_2$ and $\chi^0_1$.  For example, the $Z^* \rightarrow \ell^+\ell^-$ branching ratio for a 60 GeV mass-splitting is lower than for a mass-splitting of 2 GeV by  $46\%$ in $Z^* \rightarrow e^+e^-$ and by $40\%$ for $Z^* \rightarrow \mu^+\mu^-$.  This happens as the $Z^*$ mass falls below the threshold needed to produce a pair of heavy quarks or $\tau$ leptons.
\item Branching ratio for $W^* \rightarrow \bar{\nu}_\ell \ell$ also increases as the mass-splitting becomes sufficiently low to suppress decay widths to heavy quarks and $tau$ leptons. ($11\%$ for large $\Delta m$ changes to $20\%$ for a $\Delta m$ of 3 GeV.
\item Events are generated at leading order with up to two extra partons in the matrix element using MG5\_aMC@NLO v2.4.2 event generator % J. Alwall et al., The automated computation of tree-level and next-to-leading order differential cross sections, and their matching to parton shower simulations, JHEP 07 (2014) 079, arXiv: 1405.0301 [hep-ph].
and the NNPDF23LO PDF set. %R. D. Ball et al., Parton distributions with LHC data, Nucl. Phys. B 867 (2013) 244, arXiv: 1207.1303 [hep-ph].  
\item Electroweakinos decayed using MadSpin with a two-lepton events filter.  This means only events were stored in the signal samples if there were at least two final state leptons, even if one or more of the leptons came from a leptonic $\tau$ decay.
\item Resulting events interfaced with Pythia v8.186 using the A14 set of tuned parameters to model the parton shower, hadronization, and underlying event.
\item ME-CS matching done with CKKWL-scheme, with the merging scale set to 15 GeV.  \textcolor{red}{OH BOY, I have to explain all this in plain english!}
\item Add wino rescaling here?  It does use the same samples but the cross-section is rescaled.
\item Must talk about how the relative sign of the $chi_1$ and $\chi_2$ mass parameters affects the model. %U. De Sanctis, T. Lari, S. Montesano, and C. Troncon, Perspectives for the detection and measurement of supersymmetry in the focus point region of mSUGRA models with the ATLAS detector at LHC, Eur. Phys. J. C 52 (2007) 743, arXiv: 0704.2515 [hep-ex].
\end{itemize}

\begin{figure}[tbp]
    \centering
 \includegraphics[width=0.6\columnwidth]{/Users/sheenaschier/Documents/LaFiles/figures/thesis/signal_samples/mll_theory.pdf}
\label{fig:samples:invMass}
\end{figure}


  \begin{figure}[tbp]
   % \centering
 \includegraphics[width=0.45\columnwidth]{/Users/sheenaschier/Documents/LaFiles/figures/thesis/signal_samples/ossf_Lep1Pt.pdf}
 \includegraphics[width=0.45\columnwidth]{/Users/sheenaschier/Documents/LaFiles/figures/thesis/signal_samples/ossf_Lep2Pt.pdf}\\
 \includegraphics[width=0.45\columnwidth]{/Users/sheenaschier/Documents/LaFiles/figures/thesis/signal_samples/ossf_Mt_l1met.pdf}
 \includegraphics[width=0.45\columnwidth]{/Users/sheenaschier/Documents/LaFiles/figures/thesis/signal_samples/ossf_Mt_l2met.pdf}\\
  \includegraphics[width=0.45\columnwidth]{/Users/sheenaschier/Documents/LaFiles/figures/thesis/signal_samples/ossf_nJet20.pdf}
 \includegraphics[width=0.45\columnwidth]{/Users/sheenaschier/Documents/LaFiles/figures/thesis/signal_samples/ossf_nLep_signal.pdf}\\
   \caption{Kinematic distributions of signal samples}
   \label{fig:SigSample1}
 \end{figure}
 
   \begin{figure}[tbp]
   % \centering
 \includegraphics[width=0.45\columnwidth]{/Users/sheenaschier/Documents/LaFiles/figures/thesis/signal_samples/ossf_Jet1Pt.pdf}
\includegraphics[width=0.45\columnwidth]{/Users/sheenaschier/Documents/LaFiles/figures/thesis/signal_samples/ossf_Jet2Pt.pdf}\\
\includegraphics[width=0.45\columnwidth]{/Users/sheenaschier/Documents/LaFiles/figures/thesis/signal_samples/ossf_MET.pdf}
\includegraphics[width=0.45\columnwidth]{/Users/sheenaschier/Documents/LaFiles/figures/thesis/signal_samples/ossf_lep_type.pdf}\\
 \includegraphics[width=0.45\columnwidth]{/Users/sheenaschier/Documents/LaFiles/figures/thesis/signal_samples/ossf_dR_l1l2.pdf}
 \includegraphics[width=0.45\columnwidth]{/Users/sheenaschier/Documents/LaFiles/figures/thesis/signal_samples/ossf_dphi_j1met.pdf}\\
 \includegraphics[width=0.45\columnwidth]{/Users/sheenaschier/Documents/LaFiles/figures/thesis/signal_samples/ossf_mll.pdf}
 \includegraphics[width=0.45\columnwidth]{/Users/sheenaschier/Documents/LaFiles/figures/thesis/signal_samples/ossf_ptll.pdf}\\
   \caption{Kinematic distributions of signal samples}
   \label{fig:SigSample2}
 \end{figure}

\subsubsection{Compressed Slepton Samples}
Slepton simplified models exploit the direct pair productions of the selectron $\tilde{e}_{L,R}$ and smuon $\tilde{\mu}_{L,R}$, where the L and R subscripts denote the left and right chirality of the partner electron or muon.  The fur sleptons are assumed to be mass degenerate \textcolor{red} {I know I have a reference for this}.  Sleptons decay to their Standard Model lepton partner a $\chi_1^0$ $100\%$ of the time.  Events were generated at tree level with MG5\_aMC@NLO v2.2.3 with the NNPDF23LO PDF set with up to two additional partons in the mixing matrix.  The MadGraph generation was interfaced with PYTHIA v8.186.  ME-PS (\textcolor{red}{what does this mean?} matching done with the CKKW-L prescripton.  Merging scale was set to one quarter the slepton mass.
 \FloatBarrier
 
 \subsection{Background Simulation}
 Standard Model background processes were generated with multiple generators \textcolor{red}{can you give a reason for this}.  $Z^{(*)}/\gamma^*$ + jets, diboson, and triboson samples were made using the SHERPA version 2.1.1, 2.2.1, and 2.2.2.  Matrix elements were calculated for up to two additional partons at NLO and four additional partons and LO. depending on the process \textcolor{red}{Say specifically which for which processes}. This is done with COMIX %T. Gleisberg and S. H�che, Comix, a new matrix element generator, JHEP 12 (2008) 039, arXiv: 0808.3674 [hep-ph].
 and OpenLoops %F. Cascioli, P. Maierhofer, and S. Pozzorini, Scattering Amplitudes with Open Loops, Phys. Rev. Lett. 108 (2012) 111601, arXiv: 1111.5206 [hep-ph]. 
 \textcolor{red}{I have never heard of either of these}, and merged with the SHERPA parton shower according to the ME-PS\@ NLO prescription.  The $Z{(*)}/\gamma^*$ + jets samples exploit invariant masses down to 0.5 GeV for $Z{(*)}/\gamma^* \rightarrow e^+e^-/\mu^+\mu^-$, and down to 3.8 GeV for $Z{(*)}/\gamma^* \rightarrow \tau^+\tau^-$.  Dilepton invariant mass in diboson samples covers down to 0.5 GeV.  Singletop and $t\bar{t}$ samples generated at NLO in the matrix element calculations with POWHEG-BOX v1 and v2 interfaced with PYTHIA 6.428 with the PERUGIA 2012 tune.  Higgs+$V$, $t$ + $V$, $t\bar{t}$ +$V/h/\gamma^*$ and $t\bar{t}$ + diboson production simulated with POWHEG-BOX v2 interfaced with PYTHIA 6.428 and 8.184 and the ATLAS A14 tune.  These are all generated with NLO matrix elements, except the $t$ + $Z$, $t\bar{t}$ + $WW$, three-top and four-top samples, which were calculated to LO. \textcolor{red}{make table}
 
 \textcolor{red}{Introduce the extensions of the background samples, then possibly show plots in the sections they correspond to}
 \begin{itemize}
 \item Detector simulation done with GEANT4.  GEANT4 models the ATLAS detector geometry, material interactions, and magnetic field potentials. 
 \item Pileup is modeled \textcolor{red}{how is it modeled}  
 \item \textcolor{blue}{Monte Carlo processed by sub-detector specific digitization algorithms, which translate the particle signatures in the detector into raw byte-stream data of the form that comes from the ATLAS detector.  Finally, fully simulated RDOs are reconstructed with release ?? of the ATLAS	 Athena reconstruction software, just like when processing real data.}
 \item \textcolor{red}{Do I want to make a schematic that illustrates the process of producing ATLAS MC simulation?}
 \end{itemize}
 \subsubsection{V+Jets}
 Model of leptonically decaying W or Z boson done with SHERPA with NNPDF30NNLO PDF set.  The matrix element is calculated with up to four additional partons in the shower.  Merging the parton shower is done with the ME+PS\@ NLO prescription.  \textcolor{blue}{The samples are sliced in maxHTPTV and quark flavor content} \textcolor{red}{I don't know what this means}.  Kinematics of the sample are as follows.  The dilepton invariant mass of the on shell Z+jets samples ir required to be above $50~\GeV$, and the $Z^*$+jets samples are restricted to dilepton invariant mass between $10~\GeV$ and $40~\GeV$ with the leading and subleading leptons having \pt above $5~\GeV$.  The $Z$+jets samples were extended down to very off-shell $Z$ production for dilepton invariant mass below $10~\GeV$, but no less than twice the mass of the leading lepton in the system.  \textcolor{blue}{The samples are inclusive in quark flavor and only available for maxHTPTV $>~280~\GeV$ slice.}
 \subsubsection{Multiboson}
 Multiboson refers to two and three vector boson production modes.  SHERPA is used diboson and fully leptonic triboson processes.  The NNPDF30NNLO PDF set was used for most samples, but for the few were this was not an option CT10 PDF set was used.  The initial samples are made of events with same flavor and oppositely signed leptons, where the invariant mass of the dilepton system is above $4~\GeV$ and the leading and subleading leptons have masses above $5~\GeV$.  The samples were later extended to lower lepton masses, where the leading two leptons have masses above $2~\GeV$ and their invariant mass must be below $10~\GeV$ and can be as low as twice the mass of the leading lepton.  W and Z production in association with an energetic photon is also modeled with SHERPA and in the same kinematic space, but samples were generated exclusively with the CT10 PDF set.
 \subsubsection{Top Quark}
 Single top production (t- and s- channel), $tW$, and $t\bar{t}$ events were generated with POWHEG and interfaced with PYTHIA 6 for parton showering.  \textcolor{blue}{The have various lepton filters...}, for the $tZ$ process, which is filtered to have at least one lepton, MADGRAPH5 calculated the matrix elements while PYTHIA 6 still handles the parton showering.  Rare events with three and four top quarks or $t\bar{t}$ in association with a $Z$, $W$, or $WW$ bosons have matrix elements calculated with MADGRAPH5 and showered with PYTHIA8 according to PFD set NNPDF30NNLO.
 \subsubsection{Higgs}
 \textcolor{blue}{Single Higgs production via gluon-gluon fusion (ggF) and vector boson fusion (VBF) processes decaying via fully leptonic WW or directly into two leptons are modeled using POWHEG, interfaced with PYTHIA 8 for parton showing and hadronization using the NLOCTEQ6L1 PDF set.  Processes involving a single Higgs in association with $W$ or $Z$ boson id modeled just using PYTHIA 8 and the NNPDF23LO PDF set.}
 
 \section{Derivation}
Describe details of the SUSY16 derivation used to select events from data

 


\chapter{Physics Object Reconstruction and Identification}
\label{ch:obj}
The term \textit{reconstruction} describes the process of interpreting signal output from the detector and using that information to make measurements associated with actual physics objects.  The ATLAS detector and its reconstruction algorithms are designed for efficient particle identification and precise energy and momentum measurement.  Reliable tracking and vertexing are the building blocks for efficient reconstruction and identification of most objects.  In this chapter, the assembly of tracks and vertices will first be described in Section~\ref{sec:obj:bb}.  Next, reconstruction and identification variables are defined for directly and indirectly observable objects in Section~\ref{sec:obj:reco}.  In ATLAS, these objects are; electrons, muons, jets, photons, and missing transverse energy and momentum.  Lastly, Section~\ref{sec:obj:treat} describes the techniques of overlap removal and isolation correction of closely-spaced leptons as subsequent treatment of reconstructed objects before analysis.  

\section{The Building Blocks}
\label{sec:obj:bb}
%1704.07983.pdf - 1
%Salzburger_2015_J._Phys.__Conf._Ser._664_072042.pdf - 2

%luminosities $2*10^{33}cm^{-2}s^{-1}$ in 2008 -> $10^{34}cm^{-2}s^{-1}$
Track reconstruction, also called \textit{tracking}, provides the important information needed for primary and secondary vertex reconstruction, charged particle reconstruction, jet flavor tagging, and photon conversions; therefore, track reconstruction algorithms must be swift, concise, and perform with high efficiency, low fake rates and with proper resolution on tracking parameters.  In 2015, at the start of Run 2, the LHC extended the center-of-mass energy in proton-proton collisions to 13 TeV, and over the duration of Run 2, ramped up the instantaneous luminosity, pushing the average interactions per bunch crossing ($\mu$) to above 40 by the end of Run 2.  This extension of center-of-mass energy and instantaneous luminosity enhances the outlook of discovery while simultaneously slowing down track reconstruction and degrading its efficiency.  Events with jet showers in the TeV range and $\tau$ leptons and $b$-hadrons that traverse multiple ID layers before decaying, occur at rates high enough to be considered in optimizing track and cluster reconstruction in Run 2 \cite{aad}.  In the core of boosted hadronic jets and $\tau$ lepton decays, particles in flight are not very separated as they traverse the inner tracking layers, making separate energy deposits in the discrete sensors hard to resolve and near-by tracks hard to distinguish from each other.  If tracking efficiency is low in events with high track density, mismeasurements are expected in identifying long-lived $b$-hadron and hadron $\tau$ decays and in calibrating the energy and mass of jets.  These mismeasurements will also cause induced \met, which is an important quantity for this search and many other Beyond Standard Model (BSM) searches.  

The first step in track reconstruction involves preprocessing Pixel, SCT, and TRT information. Event-by-event charged track reconstruction in the pixel and SCT detectors starts with clustering groups of pixels and strips in each sensor that respond to an energy deposition above a set threshold and share a common edge or corner.  These clusters form three-dimensional space-points that measure where a particle intersects the active material in the ID.  In the pixel detector, each particle corresponds to one space-point, while in the SCT, clusters must be combined from both sides of a strip layer to obtain a three-dimensional position measurement.  %Charge in the pixel sensors is often collected in more than one adjoining pixels \textcolor{red}{Maybe connect here hit resolution to cluster size and use plots from the IBL paper.. remembering to explain how impact parameter resolution is driven by the resolution of the hit closest to the primary vertex.} 

The next step in tracking is called track finding.  This involves combining Pixel and SCT hits into tracks seeds.  Three consecutive hits are required for a track seed, and seeds with an additional compatible cluster are sent to a Kalman filter.  In the last step, hits from all three of the tracking detectors are fit to make tracks using a global $\chi^2$ function.  These tracks are then given a score based on the fit quality and the number of holes and shared clusters.  Tracks that fall below the minimum allowable score are rejected.


Reconstructed tracks are characterized using five \textit{perigee} parameters at the point of closest approach to the beam axis.  
\begin{itemize}
\item \textbf{\textit{transverse impact parameter}} $d_0$ - track distance to the $z$-axis at the point of closest approach in the $x-y$ plane.
\item \textbf{\textit{longitudinal impact parameter}} $z_0$ - track coordinate along $z$ at the point of closest approach.
\item \textbf{\textit{azimuthal angle}} $\phi_0\equiv\tan^{-1} p_y/p_x$ - track angle to the $x$-axis in the $x$-$y$ plane.
\item \textbf{\textit{polar angle}} $\theta_0$ - track angle to the $z$-axis in the $R-z$ plane.
\item \textbf{\textit{charge over momentum}} $q/p$ - electric charge divided by the track momentum.
\end{itemize}
  \begin{figure}[tbp]
       \includegraphics[width=0.8\columnwidth]{/Users/sheenaschier/Documents/LaFiles/figures/thesis/eventselection/TrkPerigee.png}\\
   \caption{Sketch of ATLAS tracking parameters at the perigee in the $x-y$ plane (left) and the $r-z$ plane (right). }
   \label{fig:trkParam}
 \end{figure}
 
The primary vertex is defined as space position in the detector of the initial $pp$ interaction.  Primary vertices are identified using inner detector tracks that satisfy a set of requirements.  For a track to be considered in the construction of a primary vertex, it must have $ \pt{} > 400\MeV$, $|\eta| < 2.5$, between 9 ($|\eta| \leq 1.65$) and 11 ($|\eta| > 1.65$) silicon hits, at least 1 hit in the IBL or B-Layer, a maximum of one shared pixel hit or two shared SCT hits, no holes in the pixel layers, and no more that one hole in the SCT layers.  Any primary vertex must have at least two associated tracks for reconstruction~\cite{1742-6596-898-4-042056}.  %The primary vertex track selection criteria is summarized in Table~\ref{tab:pvtrk}.
\iffalse
\begin{table}
\tiny
\centering
\begin{tabular}{l|l}
%pg 2 Boutle
  \small Track Kinematics & Track Hit Criteria  \\
  \hline
  $\pt > 400~\MeV$ & \\
  $|d_0|<4~\mathrm{mm}$ & \\
  \hline
\end{tabular}
\caption{Summary of primary vertex track selection}
\label{tab:pvtrk}
\end{table}
\fi
\FloatBarrier

\section{Particle Identification and Reconstruction}
\label{sec:obj:reco}

Reconstructed and identified particles in ATLAS are leptons($e, \mu, \tau$), photons, hadronic jets, which can further be identified as $b$-jets, and missing transverse momentum \met.  This analysis does not use $\tau$ reconstruction.  There are two categories of reconstructed objects: \textit{baseline}, which is the most inclusive definition of an object and is typically used for preliminary event selection and background modeling, and \textit{signal}, a more exclusive object definition that is a subset of \textit{baseline} and is typically used in defining signal events.  A summary of all the signal and baseline object definitions is given in Table~\ref{tab:objdef}.

    \begin{figure}[tbp]
       \includegraphics[width=.8\columnwidth]{/Users/sheenaschier/Documents/LaFiles/figures/thesis/eventselection/idSketch.png}\\
   \caption{Schematic of an electron's flight in the ATLAS}
   \label{fig:idsketch}
 \end{figure}
Electron likelihood identification is a multivariate technique that uses signal and background probability density functions of discriminating variables to give an overall likelihood of being signal or background.  Figure~\ref{fig:idsketch} depicts an electron in ATLAS moving through the ID detectors and into the calorimeters.  Likelihood variables related to tracking include: number of hits on the inner-most pixel layer, hits in the Pixel detector, hits in the SCT+Pixel detectors, transverse impact parameter $d_0$, transverse impact parameter significance ($|d_0/\sigma_{d_0}|$), and fractional momentum lost in the detector, likelihood probability based on the transition radiation in the TRT, and track-cluster matching variables.  Likelihood variables that discriminate on calorimeter measurements include: the ratio of transverse energy in the TileCal to the energy in the LAr, the ratio of energy in the last LAr layer to the energy in the full LAr\footnote{This variable is only used for electrons with $\pt<80\GeV$.}, the lateral electromagnetic shower shape in the second LAr layer, shower width in the LAr strip layers.  Signal and background probabilities combine into a single discriminant on which a cut is applied to define a likelihood-based operating point.  Operating points in the electron likelihood identification menu are \textit{VeryLoose, Loose, LooseAndBLayer, Medium}, \textit{Tight}.  \textit{LooseAndBLayer} uses the same likelihood as \textit{Loose} and also requires a hit in the inner-most Pixel layer.  All operating points use the same discriminating variables to ensure tighter operating points are subsets of the more loose operating points.  The electron efficiencies for the \textit{Loose}, \textit{Medium}, and \textit{Tight} LH working points are compared in Figure~\ref{fig:lepEff}.
\begin{figure}[h!]
 \centering
 \includegraphics[width=0.49\columnwidth]{/Users/sheenaschier/Documents/LaFiles/figures/thesis/fakes/fig_01.pdf}
  \includegraphics[width=0.49\columnwidth]{/Users/sheenaschier/Documents/LaFiles/figures/thesis/fakes/fig_02.pdf}
 \caption{Electron efficiency as a function of $\eta$ (left) and \et (right) in the \textit{Loose}, \textit{Medium}, and \textit{Tight} LH identification algorithms}
 \label{fig:lepEff}
 \end{figure}
 
 Muons in this analysis use a cut-based identification technique that first identifies muon tracks in the ID and MS and combines them to form complete muon tracks.  Identification working points are provided based on the muon reconstruction efficiency and background rejection they provide.  Muon ID \textit{Medium} is the default working point used by physics analyses in ATLAS \cite{muonid}.  The \textit{Medium} ID achieves over $95\%$ muon efficiency for muons $4\GeV<\pt<20\GeV$, and over $60\%$ background rejection.  Muon identification efficiencies measured versus muon \pt by the Muon Combined Performance Group in ATLAS are shown in Figure~\ref{fig:emuon}. 
 \begin{figure}[h!]
 \centering
 \includegraphics[width=0.8\columnwidth]{/Users/sheenaschier/Documents/LaFiles/figures/thesis/eventselection/muonEff}
 \caption{Reconstruction efficiency for \textit{Medium} muon identification working point as a function of muon \pt, in the region $0.1<|\eta|<2.5$ \cite{muon}.}
 \label{fig:emuon}
 \end{figure}
  
Lepton isolation is quantified by two main variables, track isolation and calorimeter isolation.  Track isolation is determined by the transverse momenta of tracks in some cone around the track with a radius determined by the lepton \pt.  Calorimeter isolation is dictated by the sum of the transverse energy in the topological clusters (topo clusters), which are cell clusters seeded by calorimeter cells with energy more than four times greater than the noise threshold in the cell.  Topo clusters are then expanded to neighboring cells with energy more than twice above the noise threshold, and finally a last layer of excited calorimeter cells are added to the cluster.  To measure the isolation energy, the lepton energy in the isolation topo cluster is removed and the topo cluster is corrected for pileup and any lepton energy that was not subtracted away.  Final isolation cuts using the track- and calorimeter-based isolation variables are are classified as either \textit{fixed cut} or \textit{gradient}.  Fixed cut means the working point provides fixed efficiencies across the $\eta-\pt$ plane.  Gradient means the efficiencies are \pt-dependent, but still flat in $\eta$.  Isolation working points are provided for for three grades of isolation: \textit{Loose, Medium,} and \textit{Tight}, and can be based on track isolation, calorimeter isolation, or both.  The \textit{Tight} working points will provide the best rejection of backgrounds, but the lowest efficiencies.
 
Baseline electrons are seeded from energy deposits in the EM calorimeter and reconstructed with algorithms using EM calorimeter clusters that are matched to inner detector tracks.  Baseline electrons must pass a \pt{} threshold of 4.5 \GeV~and exclusively travel through the central detector region $|\eta | < 2.47$.  A longitudinal impact parameter requirement of $|z_0\sin\theta| < 0.5~\mathrm{mm}$ is also applied.  This analysis uses likelihood based identification criteria only.  Baseline electrons are required to satisfy \textit{VeryLooseLLH} identification while signal electrons must pass \textit{Tight} identification plus \textit{GradientLoose} isolation criteria.  Signal electrons also require transverse impact parameter significance $|d_0/\sigma(d_0)| < 5$.  Electron energy deposits in the LAr are generally narrow in $\eta$ and $\phi$ and mostly concentrated in the first two sampling layers.
\begin{figure}[h!]
 \centering
 \includegraphics[width=0.8\columnwidth]{/Users/sheenaschier/Documents/LaFiles/figures/thesis/eventselection/figaux_11.pdf}
 \caption{Signal lepton efficiencies for electrons and muons, averaged over all Higgsino and slepton samples. Efficiencies are shown for leptons within detector acceptance, and with lepton \pt within a factor of 3 of $\Delta~m(\tilde l\tilde\chi_1^0)$ for slepton samples or within a factor of 3 of $\Delta~m(\tilde \chi_2^0\tilde\chi_1^0)/2$ for Higgsino samples. Uncertainty bands represent the range of efficiencies observed across all signal samples for the given \pt bin.  The $\eta$ dependence is consistent with values reported in ATLAS combined performance papers.}
 \label{fig:sigeff}
 \end{figure}
 
Muon information primarily comes from tracks in the muon spectrometer that are often matched charged tracks in the inner detector.  Baseline muons are reconstructed with algorithms that combine tracks from the inner detector and muon spectrometer to form muon candidates.  They must pass a \pt{} threshold of 4 \GeV~and be in fiducial region $|\eta | < 2.5$. Like with electrons, muon likelihood identification is used, and the discriminating variables are extended to include information from tracks in the muon spectrometer.  Baseline muons are also expected to satisfy \textit{Medium} identification standards  and have a transverse impact parameter $|z_0\sin\theta| < 0.5~\mathrm{mm}$.  Signal muons must also satisfy \textit{FixedCutTightTrackOnly} isolation criteria and a transverse impact parameter significance of $|d_0/\sigma(d_0)| < 3$.

Lepton identification, isolation, impact parameter cuts, fiducial acceptance and \pt threshold all effect the lepton efficiencies and result in the efficiencies shown in Figure~\ref{fig:sigeff} that range from roughly $50\%$ for low-\pt muons and up to $90\%$ for higher \pt.  For electrons the efficiencies are roughly $20\%$ for low \pt electrons, and increase up to $\sim65\%$.  This is the average over signal samples that fall within some range, where the most compressed signal samples used to evaluate the low \pt leptons and so on.

Baseline jets are built from locally-calibrated three-dimensional topologically clustered calorimeter cells.  Topological clustering here is the same as described in the discussion of lepton isolation.  Jets are constructed using anti-$k_t$ clustering algorithms~\cite{antikt} with radius parameter R = 0.4.   Baseline jets must pass a \pt{} threshold of 20 \GeV and be in fiducial region $|\eta | < 4.5$.   Also, jets within $|\eta | < 2.5$ originating from $b$-hadrons are tagged with the 2-dimensional multivariate $b$-tagging algorithm MV2c10 with an 85\% working point.  Signal jets are further restricted to fiducial region $|\eta | < 2.8$, and pileup jets are removed using the jet vertex tagger (JVT) with \textit{Medium} working point efficiency applied to jets with $\pt>60\GeV$ and $|\eta|<2.4$. 

\begin{table}
\tiny
 \centering
  \begin{tabular}{l||c|c|c}
 \hline
\small Selection Criteria & \small \textbf{Electrons} & \small \textbf{Muons} & \small \textbf{Jets}  \\
 \hline
 \hline
\small \textbf{Baseline} &  & & \\ 
 \hline
\small Reco Algorithm &\small \textit{author 16 veto}  &&\\
\small Kinematic&\small $\pt{} > 4.5$ \GeV,  &\small $\pt{} > 4$ \GeV,  &\small $\pt{} > 20$ \GeV,\\
&\small $|\eta | < 2.47$&\small $|\eta | < 2.5$& $|\eta | < 4.5$\\
\small Impact Parameter &\small $|z_0\sin\theta|< 0.5$ mm &\small $|z_0\sin\theta|< 0.5$ mm &\\
& -- & -- &\\
\small Identification &\small \textit{VeryLooseLLH}  &\small \textit{Medium}  &                 \\
\small Isolation & --    & --  &   \\
\small Clustering & & &\small Anti-$k_t$ R = 0.4 EMTopo\\
\small Jet Vertex Tagging &&& -- \\
\small \textit{b}-tagging &&& -- \\
 \hline
 \hline
 \small \textbf{Signal} &  & \\ 
 \hline
 \small Reco Algorithm &\small \textit{author 16 veto}  &&\\
\small Kinematic&\small $\pt{} > 4.5 \GeV$, &\small $\pt{} > 4 \GeV$,  &\small $\pt{} > 30$ \GeV,\\
&\small $|\eta | < 2.47$&\small $|\eta | < 2.5$&\small $|\eta | < 2.8$\\
\small Impact Parameter &\small $|z_0\sin\theta|< 0.5~mm$,&\small $|z_0\sin\theta|< 0.5~mm$, &\\
&\small $|d_0/\sigma(d_0)|< 5$&\small $|d_0/\sigma(d_0)|< 3$&\\
\small Identification &\small \textit{Tight} &\small \textit{Medium}   &                 \\
\small Isolation &\small \textit{GradientLoose}     & \small \textit{FixedCutTightTrackOnly} &    \\
\small Clustering & & &\small Anti-$k_t$ R = 0.4 EMTopo\\
\small Jet Vertex Tagging &&&\small \textit{JVT Medium}\\
\small \textit{b}-tagging &&&\small $\pt{} > 20$, $|\eta | < 2.5$ \\
&&& \small MV2c10 FixedCutBeff 85\% \\
  \end{tabular}
  \caption{Summary of object definitions}
  \label{tab:objdef}
\end{table}

Well calibrated energy and momentum measurements of the directly observable objects is important for construction of the particles that traverse the detector without interacting.  These ``missing" particles carry away energy and momentum which is recovered by requiring momentum conservation in the plane transverse to the beam pipe.  The vector quantity missing transverse momentum $\vec p_{\mathrm{T}}$ is the negative vector sum of the transverse momentum of all the identified physics objects (electrons, muons, jets, photons) plus an additional soft term.  The scalar magnitude of the missing transverse momentum vector gives the missing transverse energy \met. The soft term is constructed from all the tracks not associated with any physics object, but are associated with the primary vertex.  Therefore, \met{} is adjusted for the best possible calibration of the jets and other identified physics objects and still independent of pileup in the soft term.  Pileup jets are removed with a jet vertexing technique that matches jets to primary vertices with track-vertex tagging.


\section{Special Treatment of Reconstructed Objects}
\label{sec:obj:treat}
Once objects are reconstructed and identified, special algorithms often need to be run before these objects can be used.  For this analysis, these final steps were the removal of overlapping objects and the isolation correction of closely-spaced leptons.

Overlap removal is performed to prevent double counting of physics objects by removing objects based in their separation $\Delta R$ in detector coordinates $\eta$ and $\phi$, given by:
\begin{equation}
\Delta R_{p_1p_2} = \sqrt{(\eta_{p_1}-\eta_{p_2})^2+(\phi_{p_1}-\phi_{p_2})^2}
\end{equation} 
First, jet-electron overlap removal is performed.  If $\Delta R_{\mathrm{jet, electron}}$ is less than 0.2 and the the jet is not tagged as a $b$-jet, the jet is removed and the electron is kept.  If the jet is identified as a $b$-jet, then the jet is kept and the electron object is removed since the electron is most likely from the semi-leptonic decay of a $b$-hadron.  If $\Delta R_{\mathrm{jet, electron}}$ is less than 0.4, we remove the electron and keep the jet.  Similarly, if the $\Delta R_{\mathrm{jet, muon}}$ is less than 0.4, we remove the muon and keep the jet unless the jet has less than three tracks; in which case the muon will be kept and the jet is discarded.  Lastly, we perform overlap removal on photons and other objects.  It is common that electron and muon objects will also be included in the photon container since they pass the LAr shower requirements, so overlapping photons and leptons will typically result in the photon object being removed from the photon container.  If $\Delta R_{\mathrm{photon, electron}}$ is less than 0.4 we remove the photon and keep the electron.  If $\Delta R_{\mathrm{photon, muon}}$ is less than 0.4, we remove the photon and keep the muon.  If $\Delta R_{\mathrm{photon, jet}}$ is less than 0.4, we keep the photon and remove the jet.  

 Soft leptons in a boosted system often have small angular separation, especially when they are products of a low-mass $Z^*$ decay.  These boosted leptons often lie within each other's isolation cones, leading to efficiency loss for very small mass splittings.  The top rows of Figures~ \ref{fig:EffRll_ISOCorr} and ~\ref{fig:EffMll_ISOCorr} illustrate the efficiency loss for nearby leptons within $\Delta R < 0.4$ and dilepton invariant mass ($m_{\ell\ell}$) $<5\GeV$ using an electroweakino signal sample with $m_{\tilde\chi_2^0}-m_{\tilde\chi_1^0}=10\GeV$.  This loss is corrected by using a dedicated tool that checks for baseline leptons that fail the isolation criteria due to another nearby lepton within its isolation cone and removes tracks associated with the nearby lepton from the track isolation sum.  If the nearby lepton is an electron, the topocluster $E_T$ is also removed from the calorimeter isolation sum.  The corrected isolation variables are then reanalyzed using the original isolation working point.  The bottom rows of Figures~ \ref{fig:EffRll_ISOCorr} and ~\ref{fig:EffMll_ISOCorr} exhibit the recovered dilepton efficiency in simulation after applying the isolation correction tool.  Figure~\ref{fig:nearbylepiso} shows the effect of this correction on low invariant mass dilepton pairs in data.  The data are chosen such that $\Delta\phi(\met, p_{t}^{j_1})<1.5$ to avoid the signal region, which selects $\Delta\phi(\met, p_{t}^{j_1})>2.0$, as explained in Chapter~\ref{ch:sr}.   
  \begin{figure}[tbp]
   % \centering
     \includegraphics[width=0.48\columnwidth]{/Users/sheenaschier/Documents/LaFiles/figures/thesis/eventselection/eff_EE_Rll_110_100_NoOS_NoISO.pdf}
       \includegraphics[width=0.48\columnwidth]{/Users/sheenaschier/Documents/LaFiles/figures/thesis/eventselection/eff_MM_Rll_110_100_GradLoose_NoOS_NoISO.pdf}\\
     \includegraphics[width=0.48\columnwidth]{/Users/sheenaschier/Documents/LaFiles/figures/thesis/eventselection/eff_EE_Rll_110_100_NoOS.pdf}
     \includegraphics[width=0.48\columnwidth]{/Users/sheenaschier/Documents/LaFiles/figures/thesis/eventselection/eff_MM_Rll_110_100_GradLoose_NoOS.pdf}\\
   \caption{Dilepton $\Delta$ R distribution before LepIsoCorrection (top) and after LepIsoCorrection (bottom) for the $ee$-channel (left) and $\mu\mu$-channel (right), using electroweakino signal samples with m($\tilde\chi_2^0,\tilde\chi_1^0)=(110, 100)\GeV$.}
   \label{fig:EffRll_ISOCorr}
 \end{figure}

  \begin{figure}[tbp]
   % \centering
     \includegraphics[width=0.48\columnwidth]{/Users/sheenaschier/Documents/LaFiles/figures/thesis/eventselection/eff_EE_Mll_110_100_NoOS_NoISO.pdf}
       \includegraphics[width=0.48\columnwidth]{/Users/sheenaschier/Documents/LaFiles/figures/thesis/eventselection/eff_MM_Mll_110_100_GradLoose_NoOS_NoISO.pdf}\\
     \includegraphics[width=0.48\columnwidth]{/Users/sheenaschier/Documents/LaFiles/figures/thesis/eventselection/eff_EE_Mll_110_100_NoOS.pdf}
     \includegraphics[width=0.48\columnwidth]{/Users/sheenaschier/Documents/LaFiles/figures/thesis/eventselection/eff_MM_Mll_110_100_GradLoose_NoOS.pdf}\\
   \caption{Dilepton invariant mass distribution before LepIsoCorrection (top) and after LepIsoCorrection (bottom) for the $ee$-channel (left) and $\mu\mu$-channel (right), using electroweakino signal samples with m($\tilde\chi_2^0\tilde\chi_1^0$) $(110, 100)\GeV$.}
   \label{fig:EffMll_ISOCorr}
 \end{figure}
 \begin{figure}[tbp]
 \centering
  \includegraphics[width=0.8\columnwidth,trim=1.2cm 0cm 1.9cm 0cm,clip]{/Users/sheenaschier/Documents/LaFiles/figures/thesis/nearbylepiso.pdf}
 %  \includegraphics[width=0.48\columnwidth]{/Users/sheenaschier/Documents/LaFiles/figures/thesis/nearbylepiso_signal.pdf}
  \caption{Impact of the \texttt{NearbyLepIsoCorrection} tool on the efficiency of low-mass dilepton pairs in data.  The data are shown in a region with $\Delta\phi(\met, p_{t}^{j1})<1.5$ to avoid the signal region.  Events are triggered with the inclusive-\met{} trigger.  The red trend shows events with two baseline leptons without applying any isolation; the green shows the impact of applying \texttt{GradientLoose} isolation; the blue shows the result of the \texttt{NearbyLepIsoCorrection} applied to the \texttt{GradientLoose} sample.  %(right) Impact of the correction on a Higgsino LSP signal sample with $\Delta m(\chi,\chi)=3\GeV$.
  }
 \label{fig:nearbylepiso}
 \end{figure}
 

 
 



%\iffalse
\chapter{Signal Region Optimization}
 \label{ch:sr}
This analysis relies on external predictions of signal and background processes in data to help interpret observations, and for observations to be meaningful, it is imperative to search for new physics where its presence is not excessively drowned out by SM backgrounds.  To achieve this, a signal enriched region in phase space, called a \textit{signal region} (SR), is defined through a series of selection cuts on kinematic variables targeting events where predicted signal yields display a significant excess over the estimated backgrounds, which are discussed in Chapter~\ref{ch:bkg}.   
 
 In the chapter, the discriminating variables that define the Higgsino and slepton signal regions are expounded first in Section~\ref{sec:sr:discvar}, then the signal regions are defined in Section~\ref{sec:sr:srdef}.  To exploit the Higgsino and sleptons models fully, they are treated by separate analyses in independent signal regions, but the compressed nature of these models makes many of their SR cuts overlap.  Section~\ref{sec:sr:srdef} is broken into sections, first detailing the common SR selection cuts in Section~\ref{sec:sr:commom}:, then the Higgsino SR specific cuts and the slepton SR specific cuts in Section~\ref{sec:sr:mll} and~\ref{sec:sr:mt2}. 
 
\section{Discriminating Variables}
\label{sec:sr:discvar}
This section will define all the discriminating variables used to define the signal regions, then the next section will detail how they are applied to the SRs, and what benefits or limitations they present.  These discriminating variables are presented in terms of three classifications, those that exploit the lepton information, those that exploit the topology of the jets and the \met{}, and those that exploit both.  

The variables that depend only on lepton information are: lepton flavor, lepton charge, the distance between a lepton pair ($\Delta R_{\ell\ell}$), and the invariant mass of a lepton pair ($m_{\ell\ell}$).  Lepton flavor refers to it being an electron or a muon, and the lepton charge is its positive or negative electric charge.  $\Delta R_{\ell\ell}$ is defined in terms of detector angles $\eta$ and $\phi$, as:
\begin{equation}
\Delta R_{\ell\ell} = \sqrt{(\eta_{\ell_1}-\eta_{\ell_2})^2+(\phi_{\ell_1}-\phi_{\ell_2})^2}
\end{equation} 
The invariant mass is taken from the energy-momentum 4-vector in Equation~\ref{eq:invarm}, and the invariant mass of two leptons is the magnitude of the summed lepton energy-momentum vectors, as in Equation~\ref{eq:mll}.
 \begin{equation}
m^2 = E^2-\mathbf{p}^2
\label{eq:invarm}
\end{equation} 
 \begin{equation}
m_{\ell\ell} = \sqrt{(E_{\ell_1}+E_{\ell_2})^2 - (\mathbf{p}_{\ell_1}+\mathbf{p}_{\ell_2})^2}
\label{eq:mll}
\end{equation} 

The variables that exploit the jet and \met topology are: \met{}, the \pt of the leading\footnote{In reference to particle objects, the term \textit{leading} always refers that type of object in an event with the highest measured \pt{}.  \textit{Subleading} always refers to the second highest \pt{} object in the event.} jet ($\pt(j_1)$), the number of $b$-tagged jets ($N_\mathrm{b-jets}$), the angular separation between missing transverse momentum and the leading jet ($|\Delta\phi(j_1, \pt^{miss})|$), and the minimum angular separation between missing transverse momentum and the nearest reconstructed jet ($min|\Delta\phi(jets, \pt^{miss})|$).  The angular separation between two objects in ATLAS is measured in terms of the azimuthal $\phi$, so $|\Delta\phi(j_1, \pt^{miss})|$ is simply the difference in the $\phi$ coordinates of the leading jet and \met{} in the interval [-$\pi$, $\pi$].  Similarly, to calculate the minimum separation between the \met and the reconstructed jets, $|\Delta\phi(j, \pt^{miss})|$ is measured for each jet and the minimum value is selected.

The variables that use combined information from the leptons, jets, and \met{} are: the transverse mass of the leading lepton and the missing transverse momentum ($m_\text{T}^{\ell_1}$), the ratio of \met{} over the scalar sum of the lepton transverse momenta ($\met/\HT^\text{lep}$), the di-tau invariant mass ($m_{\tau\tau}$), and the stransverse mass ($m_{T2}^{m_{\chi}}$).  The transverse mass of the combined leading lepton and missing transverse momentum is defined by the energy-momentum 4-vector using the transverse quantities:
\begin{equation}
\label{eq:mt}
m_\text{T}^{\ell_1} = \sqrt{2(E^{\ell_1}_TE^{miss}_T-\pt^{\ell_1}\pt^{miss})} 
\end{equation}
$m_{T2}^{m_{\chi}}$ is similar to $m_\text{T}^{\ell_1}$ in that it relates lepton transverse momentum and \met{}, but it is a bit more complicated.  To understand the $m_{T2}^{m_{\chi}}$ variable, one must consider a process like in Figure~\ref{fig:fn1} where a pp collision produces a slepton pair  which immediately decay to a visible lepton and and invisible LSP.  $m_{T2}^{m_{\chi}}$, detailed in Eq~\ref{eq:mt2}, essentially determines a bound on the masses of the invisible particles as a function of the \pt of the two leading leptons and the measured missing transverse momentum.  It is mathematically defined by the minimum value of $q_T$ for the maximum of the transverse mass of the leptons and invisible particles for some set value of $m_\chi$.  
\begin{equation}
\label{eq:mt2}
m^{m_\chi}_{T2}(\pt^{\ell_1}, \pt^{\ell_2}, \pt^{miss})  = \underaccent{\mathbf{q}_T}{\text{min}}\big(\text{max}[m_T(\pt^{\ell_1}, q_T; m_\chi), m_T(\pt^{\ell_2}, \pt^{miss}-q_T; m_\chi)]\big)
\end{equation}
Here, $q_T$ is the sum of the transverse momentum vectors of each of the invisible particles, as in Eq~\ref{eq:qt}.  The transverse mass of the leptons and invisible particles is shown explicitly in Eq~\ref{eq:mtchi}.
\begin{equation}
\label{eq:qt}
q_T = \pt^{\chi,1}+\pt^{\chi,2}
\end{equation}
\begin{equation}
 m_T\left(\mathbf{p}_T^{\ell}, \mathbf{q}_T, m_\chi\right)= \sqrt{m_\ell^2 + m_\chi^2 + 2\left(E_T^\ell E_T^q -\mathbf{p}_T\cdot \mathbf{q}_T\right)}
 \label{eq:mtchi}
 \end{equation}
   \begin{figure}
  \centering
  \input{/Users/sheenaschier/Documents/LaFiles/figures/thesis/ditau_schematic}
  \caption{Schematic illustrating the fully leptonic $(Z\to\tau\tau)$ + jets system motivating the construction of $m_{\tau\tau}$. }
  \label{fig:ditau_schematic}
  \end{figure}
 Lastly, the di-tau invariant mass, $m_{\tau\tau}\left(p_{\ell_1}, p_{\ell_2}, \mathbf{p}_\mathrm{T}^\mathrm{miss}\right) $ is used by this analysis to veto the $Z\rightarrow\tau\tau$ background.   The purpose of this variable is to reconstruct the di-tau invariant mass of the fully leptonic $Z\rightarrow\tau\tau$ process from the measurable quantities in the event, which are the 4-momenta of the two leptons and the missing transverse momentum.  A ($Z\rightarrow\tau\tau$) + jets event within the signal region relies on the $Z$ boson recoiling off the jet activity, boosting the decaying di-tau system oppositely along the jet axis.  A schematic of this process is displayed in Figure~\ref{fig:ditau_schematic}.  This kick from the jets causes the leptons and neutrinos to remain close to a single axis, so the 4-momentum of the invisible neutrino system $p_{\nu_i}$, for the $i_{th}$ $\tau$ in the event, can be well approximated by a simple rescaling of the lepton 4-momentum. The di-tau invariant mass is defined in Eq~\ref{eq:mtt}.
 \begin{equation}
 \label{eq:mtt}
 m^2_{\tau\tau}\left(p_{\ell_1}, p_{\ell_2}, \mathbf{p}_\mathrm{T}^\mathrm{miss}\right) \equiv 2p_{\ell_1}\cdot p_{\ell_2}(1+\xi_1)(1+\xi_2)
 \end{equation}
 where $\xi_1$ and $\xi_2$ are determined by solving Eq~\ref{eq:xi}, and the sign of $m^2_{\tau\tau}$ is given by Eq~\ref{eq:mtt2}.
  \begin{equation}
   \label{eq:xi}
  \mathbf{p}_\mathrm{T}^\mathrm{miss} = \xi_1\mathbf{p}_\mathrm{T}^\mathrm{\ell_1}+\xi_2\mathbf{p}_\mathrm{T}^\mathrm{\ell_2}
   \end{equation}
 \begin{equation}
 \label{eq:mtt2}
 m_{\tau\tau}\left(p_{\ell_1}, p_{\ell_2}, \mathbf{p}_\mathrm{T}^\mathrm{miss}\right) =
\begin{cases}
\hphantom{-}\sqrt{m_{\tau\tau}^2}~;               & m_{\tau\tau}^2 \geq 0,\\
 -\sqrt{\left| m_{\tau\tau}^2\right|}~; & m_{\tau\tau}^2 < 0.
\end{cases} 
 \end{equation} 
 %Z. Han, G. D. Kribs, A. Martin and A. Menon, Hunting quasidegenerate Higgsinos, Phys. Rev. D 89 (2014) 075007, arXiv: 1401.1235 [hep-ph].
 % H. Baer, A. Mustafayev and X. Tata, Monojet plus soft dilepton signal from light higgsino pair production at LHC14, Phys. Rev. D 90 (2014) 115007, arXiv: 1409.7058 [hep-ph].
 %A. Barr and J. Scoville, A boost for the EW SUSY hunt: monojet-like search for compressed sleptons at LHC14 with 100 fb?1, JHEP 04 (2015) 147, arXiv: 1501.02511 [hep-ph].
  
\section{Signal Region Definitions}
\label{sec:sr:srdef}
 There are two types of signal region used in this analysis: Higgsino SRs and slepton SRs.  Within each of the kinds of signal region, both inclusive and exclusive SR are defined, and these will be detailed in Sections~\ref{sec:sr:mt2} and~\ref{sec:sr:mt2}.  Among the selection cuts that define the Higgsino and slepton SRs are many that overlap, and these will be laid out in Section~\ref{sec:sr:commom}.

\subsection{Common Signal Regions}
\label{sec:sr:commom}
SR events are required to contain two signal leptons, and intermediate amount of \met, and at least one jet.  The leading lepton is required to have $\pt > 5~\GeV$ and the subleading lepton is required to have $\pt > 4.5~\GeV$, or $\pt > 4.5~\GeV$ for muons.  Furthermore, the two leptons are required to make a same-flavor-opposite-sign\footnote{\textit{Sign} is another term for positive or negative electric charge} (SFOS) pair.  For Higgsino signals, this prefers the dominant leptonic decay mode of the Higgsino, via an off-shell $Z^*$.  In slepton signals, light flavor sleptons always decay to two oppositely charged leptons of the same flavor.  Also, selecting OSSF pairs allows the SR to target the decays of this analysis and leaves different flavor or same signed lepton pairs to be exploited in the control and validation regions.  Collinear leptons from photon conversions are filtered out with a restriction on the minimum $\Delta R_{\ell\ell}$ between the leptons of $0.05$ and an invariant mass cut of $m_{\ell\ell} > 1 ~\GeV$.  

\met{} is an important variable when there are heavy invisible particle in the final state, and also because this analysis uses inclusive \met{} triggers.  A \met{} threshold of $200~\GeV$ is imposed to be fully efficient in the \met{} trigger, even though the optimal cut to achieve the best signal over background discrimination might be lower.  This limitation on the trigger is an unfortunate consequence of the increasing luminosity.  The \met is also correlated to $\pt(j_1)$.  Since the leptons are so light compared to the mass of the LSP, the boost from the hadronic recoil is mostly given to the \met{}. So, if the $\pt(j_1)$ threshold is too high, it will reduce the sensitivity in \met{}, but if the $\pt(j_1)$ threshold is too low, other subleading jets may contribute significantly to the recoil of the system.  The leading jet \pt threshold is set to $100~\GeV$.  The intermediate \met{} requirement sculpts the topology of the signal to prefer events where the direction of the \met and the direction of the leading jet are opposite each other in the transverse plane.  Because of the small mass-splittings between the electroweakinos or the sleptons and the LSP, the LSPs will typically only produce significant enough \met{} to pass the $\met{} > 200~\GeV$  cut when they are aligned opposite to the hadronic initial state radiation in the transverse plane.  A cut on $\Delta\phi(j, \pt^{miss}) > 2.0$ is established to take advantage of this topology and cut away backgrounds that are more agnostic to it.  

\met from jet mismeasurements tends to align the the $\pt^{miss}$ with some of the jets, leading to a small $\Delta\phi$ between them.  This mostly occurs in QCD and $Z$+jets events.  $min\Delta\phi(jets, \pt^{miss})$ considers the minimum angular separation between $\pt^{miss}$ and the nearest reconstructed jet, so this variable should have a minimum requirement to reduce the induced \met{}.  The cut is set at $min\Delta\phi(jets, \pt^{miss}) > 0.4$.  Top quark backgrounds are significantly enhanced in b-tagged jets while the Higgsino and slepton signals are not; therefore, a b-jet veto is set to discriminate against these processes.  Lastly, the region $m_{\tau\tau}^2$ = [0, 160] GeV is vetoed to reduce the Z($\rightarrow\tau\tau$)+jets background.  All of the common SR selection cuts are summarized in Table~\ref{tab:cSR}.
 \begin{table}[tbp]
 \centering
 \renewcommand{\arraystretch}{1.1}
 \begin{tabular}{ll}
 \hline
 Variable                                                & Requirement    \\
 \hline
  $N_\mathrm{leptons}$                                    & Exactly two signal leptons\\
 Lepton charge and flavor                               & $e^\pm e^\mp$ or $\mu^\pm \mu^\mp$\\
 %Author 16 Electrons (ambiguous conversions)             & Veto\\
 Leading electron (muon) $\pt^{\ell_1}$                  & $>5 (5)$ GeV             \\
 Subleading electron (muon) $\pt^{\ell_2}$               & $>4.5 (4)$ GeV             \\
  $m_{\ell\ell}$                                          &   [1, 3] or [3.2, 60] \GeV  \\
 $\Delta R_{\ell\ell}$                                   & $> 0.05$           \\
 \met                                                    & $>200$~GeV                 \\
 Leading jet $\pt(j_1)$                                  & $>100$ GeV              \\
 $|\Delta\phi(j_{1},\met)|$                                           & $>2.0$                     \\
 min$|\Delta\phi(all~jets,\met)|$                                       & $>0.4$                   \\
 $N_\mathrm{b-jet}^{20}$, 85\% WP                        & Exactly zero               \\
 $m_{\tau\tau}$                                          & $<0$ or $>160$ \GeV          \\
 \hline
 \end{tabular}
 \caption{Summary of common Higgsino and slepton SR cuts}
 \label{tab:cSR}
 \end{table}
 \FloatBarrier
 %%%%%%%%%%%%%%%%%%%%%%%%%%%%%%%%%%%%%%%%%%%%%%%%
 \subsection{Higgsino Signal Regions}
\label{sec:sr:mll}
For electroweakino signals, the leading lepton and the $\pt^{miss}$ are likely to have a small separation, and therefore a small $m_T^{\ell_1}$.  In background events with W bosons, the peak of the $m_T^{\ell_1}$ distribution is near the mass of the W, so cutting on $m_T(\pt^{\ell_1} < 70~\GeV$ can reduce the contribution from $t\bar{t}$, $WW/WZ$, and $W(\rightarrow\ell\nu)$+jets backgrounds.  The leptons in compressed electroweakino signals are also likely to have small separation, while most backgrounds do not.  For this reason, $\Delta R_{\ell\ell}$ tends to be a powerful discriminator for Higgsinos and a cut of $\Delta R_{\ell\ell} < 2.0$ is added to Higgsino SR selection.  Slepton SRs do not include this cut because the lepton topology is quite different.  Figure~\ref{fig:Rll_signals only} compares the $\Delta R_{\ell\ell}$ distributions of the Higgsino $\chi_2^0\chi_1^+$ and the slepton signals for $10~\GeV$ and $20~\GeV$ mass-splittings.  When no \met cut is applied, the Higgsino and slepton samples seem to show orthogonal response in $\Delta R_{\ell\ell}$.  The is because, unlike with Higgsino decays, the leptons come from separate sleptons, and without any boost to the system, which increases the \met, the sleptons are back-to-back.  Once \met is required, the sleptons become more collimated and $\Delta R_{\ell\ell}$ flattens, while for Higgsino samples, the shape in $\Delta R_{\ell\ell}$ becomes more pronounced. 
  \begin{figure}[tbp]
   % \centering
     \includegraphics[width=0.48\columnwidth]{/Users/sheenaschier/Documents/LaFiles/figures/thesis/higgsino_slep_signal_Rll_met0.pdf}
  %  \caption{No \met{} requirement (only truth filter).}
 %      \includegraphics[width=0.48\columnwidth]{/Users/sheenaschier/Documents/LaFiles/figures/thesis/higgsino_slep_signal_Rll_met100.pdf}\\
   % \caption{$\met{} > 100$ GeV.}
     \includegraphics[width=0.48\columnwidth]{/Users/sheenaschier/Documents/LaFiles/figures/thesis/higgsino_slep_signal_Rll_met200.pdf}\\
 %   \caption{$\met{} > 200$ GeV.}
 %    \includegraphics[width=0.48\columnwidth]{/Users/sheenaschier/Documents/LaFiles/figures/thesis/higgsino_slep_signal_Rll_met300.pdf}\\
%    \caption{$\met{} > 300$ GeV.}
   \caption{Comparison of Higgisno N2C1p (solid) and slepton (dashed) signals in the $R_{\ell\ell}$ variable for 10 GeV (dark) and 20 GeV (light) mass splittings. The \met{} here acts as a proxy for the boost of the system. Only a 2 signal lepton selection is applied.}
   \label{fig:Rll_signals only}
 \end{figure}

 For intermediate values of \met, SM diboson and $t\bar{t}$ background processes produce hard leptons, likewise diminishing the values of $\met/H_T^{lep}$.  In compressed electroweakino and slepton events, the \met{} is mostly from the boost of the hadronic recoil.  The recoiling jet affects the heavier invisible particle much more than it effects the lighter leptons; therefore, these signal events prefer larger values of $\met/H_T^{lep}$.  Figure~\ref{fig:METoverHTmll} shows the $\met/H_{T}^{lep}$ distribution for Higgsino samples after applying all the signal region cuts except $\met/H_{T}^{leptons}$ and $m_{ll}$
 \begin{figure}[tbp]
  \centering
  \includegraphics[width=0.7\columnwidth]{/Users/sheenaschier/Documents/LaFiles/figures/thesis/METoverHTLep_mll}
 \caption{Distributions of $\met/H_{T}^{lep}$ for the Higgsino selections, after applying all signal region cuts except those on the $\met/H_{T}^{lep}$ and $m_{ll}$.  The red solid line indicates the cut applied in the signal region; events in the region below the red line are rejected.}
 \label{fig:METoverHTmll}
 \end{figure}

Inclusive and exclusive Higgsino SRs are binned in $m_{ll}$. The dilepton mass $m_{\ell\ell}$ can both suppress backgrounds as well as exploit special features of the Higgsino model.  \textcolor{red}{Continue on about the kinematic endpoint..}  
 \begin{table}[]
 \tiny
\centering
\resizebox{\linewidth}{!}{
\begin{tabular}{llllllll}
\hline
Variable                  & Selection cut\\%\multicolumn{7}{l}{\textbf{\textcolor{red}{Selections optimised for Higgsinos}}}                                \\
\hline
$\met/\HT^\text{leptons}$ & \multicolumn{7}{l}{$> \text{Max}\left(5.0, 15 - 2 \cdot m_{\ell\ell}/\text{~GeV} \right)$}\\
$\Delta R_{\ell\ell}$     & \multicolumn{7}{l}{$<2.0$} \\
$m_\text{T}^{\ell_1}$     & \multicolumn{7}{l}{$<70$ GeV}                                                      \\
\hline
SRee-, SRmm-              & eMLLa   & eMLLb   & eMLLc    & eMLLd      & eMLLe      & eMLLf      & eMLLg     \\
$m_{\ell\ell}$ [GeV]      & $[1,3]$ & $[3.2,5]$ & $[5,10]$ & $[10, 20]$ & $[20, 30]$ & $[30, 40]$ & $[40,60]$ \\
\hline
SRSF-                     & iMLLa   & iMLLb   & iMLLc    & iMLLd      & iMLLe      & iMLLf      & iMLLg     \\
$m_{\ell\ell}$ [GeV]      & $<3$    & $<5$    & $<10$    & $<20$      & $<30$      & $<40$      & $<60$     \\
\hline
\end{tabular}
}
\caption{Higgsino specific SR cuts.}
\end{table}
\FloatBarrier
 
\subsection{Slepton Signal Regions}
\label{sec:sr:mt2}
Figure~\ref{fig:METoverHTmt2} shows the $\met/H_{T}^{lep}$ distribution for slepton samples after applying all the signal region cuts except $\met/H_{T}^{leptons}$ and $m_{T2}$.
 \begin{figure}[tbp]
  \centering
  \includegraphics[width=0.7\columnwidth]{/Users/sheenaschier/Documents/LaFiles/figures/thesis/METoverHTLep_mT2}
%\caption{Sleptons}
 \caption{Distributions of $\met/H_{T}^{lep}$ for the slepton selections, after applying all signal region cuts except those on the $\met/H_{T}^{leptons}$ and $m_{T2}$.  The red solid line indicates the cut applied in the signal region; events in the region below the red line are rejected. \textcolor{red}{Change to updated plot.}}
 \label{fig:METoverHTmt2}
 \end{figure}
 
 Inclusive and exclusive slepton SRs are binned $m^{m_\chi}_{T2}$.  Slepton signals a kinematic endpoint defined by the 'stransverse' mass $m^{m_\chi}_{T2}$, which is a function of the measures momentum of the leading two leptons $p_{\ell_1}$, $p_{\ell_2}$, the measured \pt, and the hypothesized invisible particle mass $m_\chi$.  \textcolor{red}{explain how the $m^{m_\chi}_{T2}$ is actually restricted in signal samples.. and more on all of this in section such and such}.    For the pair of semi-invisible particles in the slepton signal \textcolor{red}{is this the slepton pair that decay to leptons and neutralinos?}, $m^{m_\chi}_{T2}$ is always less than the parent slepton mass $m_{\tilde\ell}$ when the hypothesized $m_\chi$ mass is set to the neutralino mass in the underlying process.  This defines the lower kinematic endpoint in $m^{m_\chi}_{T2}$ for slepton signals.  Requiring $m^{m_\chi}_{T2} < m_{\tilde\ell}$, various mass scenarios can be probed in the slepton-neutrino mass plane.  Standard Model backgrounds not display this kind of feature since the invisible particles are massless neutrinos, therefor there is not such enhancement in background when making this requirement.  In fact, in the compressed region of the slepton-neutrino mass  plane, events populate a narrower region in $m_{T2}$, giving this variable more discriminating power.  

\begin{table}[]
 \tiny
\centering
\resizebox{\linewidth}{!}{
\begin{tabular}{llllllll}
\hline
Variable                  & \multicolumn{7}{l}{\textbf{\textcolor{red}{Selections optimised for sleptons}}}                           \\
\hline
$\met/\HT^\text{leptons}$ & \multicolumn{7}{l}{$> \text{Max}\left(3.0, 15 - 2 \cdot \left[m_\text{T2}^{100} / \text{~GeV}-100\right] \right)$}\\
\hline
SRee-, SRmm-              & eMT2a        & eMT2b       & eMT2c       & eMT2d        & eMT2e        & eMT2f      & \\
$m_\text{T2}^{100}$ [GeV] & $[100,102]$ & $[102,105]$ & $[105,110]$ & $[110, 120]$ & $[120, 130]$ & $\geq 130$ &\\
\hline
SRSF-                     & iMT2a       & iMT2b       & iMT2c       & iMT2d        & iMT2e       &  iMT2f      &\\
$m_\text{T2}^{100}$ [GeV] & $<102$      & $<105$      & $<110$      & $<120$       & $<130$      &  $\geq 100$ & \\
\hline
\end{tabular}
}
\caption{Slepton specific signal region cuts}
\label{tab:SRMLLMT2}
\end{table}

\iffalse
\section{Conclusion}
\label{sec:concl}
Here say some final summarizing remarks and reference these final cutflow plots.  \textcolor{red}{Say something about non-normalized cutflow with significance plot showing how the significance for signal improves as more cuts are added.}

\begin{figure}[h!]
\centering
\begin{subfigure}[b]{0.47\textwidth}
\includegraphics[width=\textwidth]{/Users/sheenaschier/Documents/LaFiles/figures/thesis/signal_regions/cutflow_bkg_SF.pdf}
\caption{Normalized cutflow, background-only}
\end{subfigure}
 \begin{subfigure}[b]{0.47\textwidth}
\includegraphics[width=\textwidth]{/Users/sheenaschier/Documents/LaFiles/figures/thesis/signal_regions/cutflow_norm_SF.pdf}
 \caption{Normalized cutflow, with signal}
\end{subfigure}
\begin{subfigure}[b]{0.58\textwidth}
\includegraphics[width=\textwidth]{/Users/sheenaschier/Documents/LaFiles/figures/thesis/signal_regions/cutflow_SF.pdf}
 \caption{Non-normalized cutflow with significance plot.}
\end{subfigure}

 \caption{Normalized Cutflow for background-only, signal-inclusive, and  non-normalized cutflow with significance plots. }
 \label{fig:cutflow_norm}
\end{figure}

\textcolor{red}{Show signal region plots?}
\fi




\chapter{Background Estimation}
%\label{sec:bkg}
Here just introduce the flow of the chapter and that the dominant background from detector induced lepton mismeasurements is discussed in a dedicated chapter.  The background estimation strategy, with a description of the control and validation regions is discussed in Section~\ref{sec:bkg:summary}.  Next, in Section~\ref{sec:bkg:tt} estimation of the irreducible ($t\bar{t}$, $Z\rightarrow\tau\tau$, $VV\rightarrow\ell\nu\ell\nu$) backgrounds is explained.  Lastly, in Section~\ref{sec:bkg:dy}, the backgrounds from Drell-Yan processes with instrumental \met is described.  The fake and non-prompt contribution is discussed is the next chapter.

\section{Summary of estimation strategy}
\label{sec:bkg:summary}
The majority, if not all, of the LHC collisions produces Standard Model processes, some of which look the same as Higgsino or slepton signal and sneak into the signal regions.  Table~\ref{tab:bkg:est} succinctly summarizes, in order of greatest contribution to least, the processes that contribute to the backgrounds, the type of background, and the method for estimating it. 

The dominant irreducible backgrounds come from $t\bar{t}$, $tW$, diboson, and $Z$+jets, where the $Z$-boson specifically decays to two $\tau$ leptons.  A top quark decays to a b-quark and a $W$-boson nearly $100\%$ of the time.  In the event that a b-jet fails the b-tagging algorithm, each $t\bar{t}$ event can essentially be seen as a diboson event with additional jets and some special topological features.  If both the $W$-bosons decay leptonically, you get two real leptons and \met from the neutrinos.  Similarly, insufficiently b-tagged $tW$ processes with leptonically decaying $W$-bosons also supply two real leptons, plus jets, and \met from neutrinos.  Even when one or both the $W$-bosons decay hadronically, the $tW$ and $t\bar{t}$ processes can still produce background events .  This means that one or two of the 'signal' leptons arise from jets faking leptons in the detector..  These contributions are accounted for in the data-driven fake estimate described in the next chapter.  

Diboson events are $WW$, $ZZ$, and $WZ$.  Fully leptonic WW production is the most prominent diboson background in the two lepton plus \met signal region.  The fully leptonic WW decays lead to two real leptons that likely have opposite charge, but are necessarily of the same flavor.  The real \met in the event comes from the neutrinos, and an additional hard jet must be present in the event.  Fully leptonic WZ events can also find their way into the signal region since there is certainly an oppositely signed same flavor lepton pair from the Z, and real \met from the netrino in the W decay, but the third lepton must fail identification for this be selected as a signal event.  Semi-leptonic ZZ and WZ processes can pass signal selection is one Z decays into a proper lepton pair and the quarks from the other vector boson induce enough \met to pass the \met trigger.  In the fully hadronic cases, there are four jets, two of which must be misidentified as leptons, leaving the other to induce a significant amount of \met.  These contributions are negligible.    

In $Z(\rightarrow\tau\tau)$+jets events, each leptonically decaying $\tau$ lepton produces one charged lepton and two neutrinos.  On the occasion that these leptons form an OSSF pair, this process well mimics signal events.  The top and $Z(\rightarrow\tau\tau)$+jets backgrounds are estimated with a semi-data-driven approach where the estimate is done in dedicated control regions designed to be enriched in the particular process.  The diboson backgrounds are evaluated with a combination of Monte Carlo and a validation region exploiting $\met/H_T$.

Control and validation regions are designed to be kinematically similar to each other and, most importantly, to the signal regions, but statistically orthogonal so to not share events.  Control regions are expected to have low signal contamination...  \textcolor{red}{make table?}

The Drell-Yan process contributes to SM backgrounds by \met induced by the mis-measurement of jet energy.. the rate is low and estimated using only Monte Carlo techniques.

Rare processes..  what are they?

Say something to segue into the next chapter all about the primary background for the analysis.

\begin{table}
%\begin{center}
\small
\begin{tabular}{lll}
\hline
\small Background Process  & \small Origin in Signal Region & \small Estimation Strategy \\
\hline 
\small Fakes ($W$+jets, $VV$(1$\ell$), $t\bar{t}$(1$\ell$) &\small Reducible, jet fakes $2^{nd}$ $\ell$ &\small Fake factor, same sign VR\\
\small $t\bar{t}$, $tW(2\ell)$  &\small Irreducible, b-jet fails ID  & \small CR using b-tagging\\
\small $Z\rightarrow(ee, \mu\mu)$+jets  &\small Instrumental \met  & \small Monte Carlo (MC)\\
\small $VV$ & \small Irreducible ($\ell\ell\ell$), missed $3^{rd}$ $\ell$ & \small MC, VR using \met/$H_T$ \\
\small $Z\rightarrow(\tau\tau)$+jets  &\small Irreducible ($\tau\tau\rightarrow\ell\nu\ell\nu$) & \small CR using $m_{\tau\tau}$\\
\small Low mass Drell-Yan  &\small Instrumental \met  & \small MC, data-driven cross check\\
\small Other rare & \small Irreducible leptonic decays & \small MC\\
\hline
\end{tabular}
\caption{Background estimation summary}
\label{tab:bkg:est}

\end{table}

\section{Irreducible Backgrounds}
\label{sec:bkg:tt}
This section describes the control regions constructed $t\bar{t}$ and $Z(\rightarrow\tau\tau)$+jets backgrounds.

\subsection{Top Control Region (CR-top)}
One of the most unique aspect of the top quark signature is the presence of a b-jet.  To enrich a dilepton sample in top quarks, at least one b-tagged jet is required in each event.  The control region is centered around this requirement.  The dilepton invariant mass is restricted to $m_{\ell\ell}<60~\GeV$ to stay kinematically consistent with the dilepton signal region.  \textcolor{red}{Say something about $\met/H_T$} A dilepton pair is required for CR-top, since the top and W decay widths to electrons versus muons is identical, different flavor lepton pairs are statistically the same and are accepted in CR-top selection. \textcolor{red}{..the last sentence is just horrible}. Figures~\ref{fig:CR-top-1} and~\ref{fig:CR-top-2} show the distributions of some of the variables used to define the Higgsino and slepton signal regions.  \textcolor{red}{Say more about these plots}.

\begin{figure}
    \centering
    \includegraphics[width=0.47\columnwidth]{/Users/sheenaschier/Documents/LaFiles/figures/thesis/backgrounds/CRtop/hist1d_met_Et_CR-top-AF}
    \includegraphics[width=0.47\columnwidth]{/Users/sheenaschier/Documents/LaFiles/figures/thesis/backgrounds/CRtop/hist1d_nJet30_CR-top-AF}
    \includegraphics[width=0.47\columnwidth]{/Users/sheenaschier/Documents/LaFiles/figures/thesis/backgrounds/CRtop/hist1d_nBJet20_MV2c10_CR-top-AF}
    \includegraphics[width=0.47\columnwidth]{/Users/sheenaschier/Documents/LaFiles/figures/thesis/backgrounds/CRtop/hist1d_METOverHTLep_CR-top-AF}
     \includegraphics[width=0.47\columnwidth]{/Users/sheenaschier/Documents/LaFiles/figures/thesis/backgrounds/CRtop/hist1d_DPhiJ1Met_CR-top-AF}
     \includegraphics[width=0.47\columnwidth]{/Users/sheenaschier/Documents/LaFiles/figures/thesis/backgrounds/CRtop/hist1d_minDPhiAllJetsMet_CR-top-AF}
        
    \caption{CR-top $ee+\mu\mu +e\mu + \mu e$ channel, pre-fit distributions.}
    \label{fig:CR-top-1}
\end{figure} 

\begin{figure}
    \centering
        \includegraphics[width=0.47\columnwidth]{/Users/sheenaschier/Documents/LaFiles/figures/thesis/backgrounds/CRtop/hist1d_lep1Pt_CR-top-AF}
        \includegraphics[width=0.47\columnwidth]{/Users/sheenaschier/Documents/LaFiles/figures/thesis/backgrounds/CRtop/hist1d_lep2Pt_CR-top-AF}
        \includegraphics[width=0.47\columnwidth]{/Users/sheenaschier/Documents/LaFiles/figures/thesis/backgrounds/CRtop/hist1d_MTauTau_CR-top-AF}
        \includegraphics[width=0.47\columnwidth]{/Users/sheenaschier/Documents/LaFiles/figures/thesis/backgrounds/CRtop/hist1d_mll_CR-top-AF}
        \includegraphics[width=0.47\columnwidth]{/Users/sheenaschier/Documents/LaFiles/figures/thesis/backgrounds/CRtop/hist1d_Rll_CR-top-AF}
        \includegraphics[width=0.47\columnwidth]{/Users/sheenaschier/Documents/LaFiles/figures/thesis/backgrounds/CRtop/hist1d_mt2leplsp_100_CR-top-AF}
        
    \caption{CR-top $ee+\mu\mu +e\mu + \mu e$ channel, pre-fit distributions.}
    \label{fig:CR-top-2}
\end{figure}

\subsection{Ditau Control Region (CR-tau)}
Getting a handle of the invariant mass of a ditau system is the clearest approach to constructing a dilepton sample enriched in $Z\rightarrow\tau\tau$ events.  The $m_{\tau\tau}$ variable, described in section~\ref{dunno}, is shown to do a good job blah blah.  Events in CR-tau are required to have an $m_{\tau\tau}$ between $60 \GeV$ and $120 \GeV$ as a way to envelope the $Z$ mass.  There are also upper and lower bounds on $\met/H_T$.  Say something about allowing different flavor lepton pairs because of the lepton flavor universality..  Figures~\ref{fig:CR-tau-1} and~\ref{fig:CR-tau-2} show distributions of the same variables used to define the Higgsino and slepton signal regions as show above to CR-top.  \textcolor{red}{Say more about these plots}


\begin{figure}
    \centering
        \includegraphics[width=0.48\columnwidth]{/Users/sheenaschier/Documents/LaFiles/figures/thesis/backgrounds/CRtau/hist1d_met_Et_CR-tau-AF}
        \includegraphics[width=0.48\columnwidth]{/Users/sheenaschier/Documents/LaFiles/figures/thesis/backgrounds/CRtau/hist1d_nJet30_CR-tau-AF}
        \includegraphics[width=0.48\columnwidth]{/Users/sheenaschier/Documents/LaFiles/figures/thesis/backgrounds/CRtau/hist1d_nBJet20_MV2c10_CR-tau-AF}
        \includegraphics[width=0.48\columnwidth]{/Users/sheenaschier/Documents/LaFiles/figures/thesis/backgrounds/CRtau/hist1d_METOverHTLep_CR-tau-AF}
        \includegraphics[width=0.48\columnwidth]{/Users/sheenaschier/Documents/LaFiles/figures/thesis/backgrounds/CRtau/hist1d_DPhiJ1Met_CR-tau-AF}
        \includegraphics[width=0.48\columnwidth]{/Users/sheenaschier/Documents/LaFiles/figures/thesis/backgrounds/CRtau/hist1d_minDPhiAllJetsMet_CR-tau-AF}
    \caption{CR-tau $ee+\mu\mu +e\mu + \mu e$ channel, pre-fit distributions.}
    \label{fig:CR-tau-1}
\end{figure} 

% set 2 vars ee
\begin{figure}
    \centering
        \includegraphics[width=0.48\columnwidth]{/Users/sheenaschier/Documents/LaFiles/figures/thesis/backgrounds/CRtau/hist1d_lep1Pt_CR-tau-AF}
        \includegraphics[width=0.48\columnwidth]{/Users/sheenaschier/Documents/LaFiles/figures/thesis/backgrounds/CRtau/hist1d_lep2Pt_CR-tau-AF}
        \includegraphics[width=0.48\columnwidth]{/Users/sheenaschier/Documents/LaFiles/figures/thesis/backgrounds/CRtau/hist1d_MTauTau_CR-tau-AF}
        \includegraphics[width=0.48\columnwidth]{/Users/sheenaschier/Documents/LaFiles/figures/thesis/backgrounds/CRtau/hist1d_mll_CR-tau-AF}
        \includegraphics[width=0.48\columnwidth]{/Users/sheenaschier/Documents/LaFiles/figures/thesis/backgrounds/CRtau/hist1d_Rll_CR-tau-AF}
        \includegraphics[width=0.48\columnwidth]{/Users/sheenaschier/Documents/LaFiles/figures/thesis/backgrounds/CRtau/hist1d_mt2leplsp_100_CR-tau-AF}
    \caption{CR-tau $ee+\mu\mu +e\mu + \mu e$ channel, pre-fit distributions.}
    \label{fig:CR-tau-2}
\end{figure} 



\subsection{Diboson Validation Region (VR-VV)}
B-jet veto and $\met/H_T<3.0$ requirement.  Remember, signal samples should populate high $H_T$, which mitigates signal contamination.  Figures~\ref{} and\ref{} show these distributions..

Refer to plots

\begin{figure}
    \centering
        \includegraphics[width=0.48\columnwidth]{/Users/sheenaschier/Documents/LaFiles/figures/thesis/backgrounds/VRVV/hist1d_met_Et_VR-VV-AF}
        \includegraphics[width=0.48\columnwidth]{/Users/sheenaschier/Documents/LaFiles/figures/thesis/backgrounds/VRVV/hist1d_nJet30_VR-VV-AF}
        \includegraphics[width=0.48\columnwidth]{/Users/sheenaschier/Documents/LaFiles/figures/thesis/backgrounds/VRVV/hist1d_nBJet20_MV2c10_VR-VV-AF}
        \includegraphics[width=0.48\columnwidth]{/Users/sheenaschier/Documents/LaFiles/figures/thesis/backgrounds/VRVV/hist1d_METOverHTLep_VR-VV-AF}
        \includegraphics[width=0.48\columnwidth]{/Users/sheenaschier/Documents/LaFiles/figures/thesis/backgrounds/VRVV/hist1d_DPhiJ1Met_VR-VV-AF}
        \includegraphics[width=0.48\columnwidth]{/Users/sheenaschier/Documents/LaFiles/figures/thesis/backgrounds/VRVV/hist1d_minDPhiAllJetsMet_VR-VV-AF}
    \caption{VR-VV $ee+\mu\mu +e\mu + \mu e$ channel, pre-fit distributions.}
    \label{fig:VR-VV-AF-set1vars}
\end{figure} 


% set 3 vars ee
\begin{figure}
    \centering
        \includegraphics[width=0.48\columnwidth]{/Users/sheenaschier/Documents/LaFiles/figures/thesis/backgrounds/VRVV/hist1d_lep1Pt_VR-VV-AF}
        \includegraphics[width=0.48\columnwidth]{/Users/sheenaschier/Documents/LaFiles/figures/thesis/backgrounds/VRVV/hist1d_lep2Pt_VR-VV-AF}
        \includegraphics[width=0.48\columnwidth]{/Users/sheenaschier/Documents/LaFiles/figures/thesis/backgrounds/VRVV/hist1d_MTauTau_VR-VV-AF}
        \includegraphics[width=0.48\columnwidth]{/Users/sheenaschier/Documents/LaFiles/figures/thesis/backgrounds/VRVV/hist1d_mll_VR-VV-AF}
        \includegraphics[width=0.48\columnwidth]{/Users/sheenaschier/Documents/LaFiles/figures/thesis/backgrounds/VRVV/hist1d_Rll_VR-VV-AF}
        \includegraphics[width=0.48\columnwidth]{/Users/sheenaschier/Documents/LaFiles/figures/thesis/backgrounds/VRVV/hist1d_mt2leplsp_100_VR-VV-AF}
    \caption{VR-VV $ee+\mu\mu +e\mu + \mu e$ channel, pre-fit distributions.}
    \label{fig:VR-VV-AF-set3vars}
\end{figure}
Say something about the value of the $m_{\ellell}$ cut imposed
\subsection{Different flavor validation regions}
The purpose of these VRs is to check the eventual extrapolation of the fitted Monte Carlo prediction of the irreducible backgrounds that are symmetric in $ee+\mu\mu$ and $e\mu+\mu e$.  Figures such and such \textcolor{red}{still need to add this last set of figures.}



\section{Drell-Yan Background}
\label{sec:bkg:dy}
Off-shell $z\rightarrow ll$ events.  Explain how these events get into the signal region.  Because of the \met trigger, contribution small but not negligible.

Refer to DY figures

Two strategies are employed to reduce this background, what are they?

After reducing this as much as possible, Monte Carlo estimates the remaining piece.

 \begin{figure}
 \centering
    \includegraphics[width=0.48\columnwidth]{/Users/sheenaschier/Documents/LaFiles/figures/thesis/backgrounds/dataCR_mll_MuMu_pre.pdf}
  % \caption{Di-muon.}
 \includegraphics[width=0.48\columnwidth]{/Users/sheenaschier/Documents/LaFiles/figures/thesis/backgrounds/dataCR_mll_ElEl_pre.pdf}
%  \caption{Di-electron.}
%  \includegraphics[width=0.6\columnwidth]{/Users/sheenaschier/Documents/LaFiles/figures/thesis/backgrounds/dataCR_mll_DF_pre.pdf}
%  \caption{Different flavour ($e\mu+\mu e$).}
  \caption{Data events passing inclusive \met{} triggers with opposite sign baseline leptons in the dilepton invariant mass $m_{\ell\ell}$ spectrum. The $\Delta\phi(j_1, \mathbf{p}_\mathrm{    T}^\mathrm{miss})$ variable is inverted to ensure this is orthogonal to the signal region.}
  \label{fig:mll_data}
 \end{figure}
\FloatBarrier


\chapter{Fake Factor Method}
\label{ch:fakefactor}
 For low pt dilepton signals, the primary reducible backgrounds are from fake leptons in W+jets events where one jets is misidentified as a lepton.  These backgrounds are estimated with a data-driven fake factor Monte Carlo simulation does not model the detector shortcomings that lead to these mismeasurements very well, so the best estimate of this background must comes from data.  The "fake factor" method is a data driven approach to modeling backgrounds from particle misidentification in the detector by estimating the lepton fake rate with a set of data kinematically enriched in events producing fake leptons.  The background estimate is validated in an orthogonal control region before it is estimated in the signal region. 

The rest of this chapter goes as follows:  Fake leptons backgrounds are introduced in Section~\ref{sec:FFintro}, then general overview of the method used to estimate the fake lepton background for this analysis is given in Section~\ref{sec:FFdesc}.  Next, the fake factor method applied to low \pt{} di-electron and di-muon events is explained in Section~\ref{sec:FFmethod}.  Finally, the validation of the fake background estimates is discussed in Section{sec:ssvr}, and the results are summarized in Section~\ref{sec:FFcon}.

%%%%%%%%%%%%%%%%%%%%
\section{Introduction}
\label{sec:FFintro}
\begin{itemize}
\item lepton identification and misidentification
\item Compare production cross-sections of signal and W+jets processes
\item Sources of electron and muon misidentification
\item How to model backgrounds from misidentification (can't use MC, must choose data driven method)
\item Concept of fake factor method
\item Primary fake background is W+jets (multi-jet is miniscule...  how do I qualify this?)
\item Rest of chapter describes FF method in the context of my analysis
\end{itemize}
Efficient lepton identification techniques make leptons powerful discriminators in ATLAS physics searches with large background rejection and heavily suppressed QCD multi-jets.  Jet suppression is very high in the range of lepton $\pt > 20\GeV$ but degrades at lower lepton $\pt$.  Misidentified electrons can be true but non-prompt electrons from photon conversions and heavy-flavor decays, where there is a real electron in the event that does not originate at the primary vertex like true, prompt electrons or they can be charged hadrons where the  hadronic jet activity in the detector fakes an electron.  
\begin{figure}[h!]
 \centering
 \includegraphics[width=0.49\columnwidth]{/Users/sheenaschier/Documents/LaFiles/figures/thesis/fakes/fig_01.pdf}
  \includegraphics[width=0.49\columnwidth]{/Users/sheenaschier/Documents/LaFiles/figures/thesis/fakes/fig_02.pdf}
 \caption{Electron identification efficiency}
 \label{fig:electronID}
 \end{figure}

**W+jets x-sec - \textcolor{red}{https://arxiv.org/pdf/1409.8639.pdf} and Higgsino/Slepton x-sections shown in Chapter 4.
 \FloatBarrier
 
 %%%%%%%%%%%%%%%%%%%%%%%%%
\section{Description of Fake Factor Method}
\label{sec:FFdesc}
Fake factor method is a data-driven technique for modeling fake rates in data.  The general approach is to measure the number of fake events in a control region that is kinematically equivalent to the signal region, then apply the measurement to the signal regions through a transfer factor.  The transfer factor, called the \textit{fake factor}, is measured in a kinematic region dominated by fake leptons.  Within the measurement region, two classes of lepton are defined.  \textit{ID} leptons, which are usually the same as the signal lepton, and \textit{anti-ID} leptons, which are required to fail certain signal lepton cuts.  An important feature of the anti-ID definition is that it mimics the fake lepton composition in the primary primary source; in this case, W($\rightarrow\ell\nu$)+jets.  Electrons and muons are treated separately.  Fake factor is the ratio of leptons passing analysis lepton identification criteria to the leptons passing anti-identification criteria, measured in a region of kinematic phase space contrived to be enriched in fake leptons.  This will be considered as the fake factor measurement region in this thesis.  The control region is meant to select events with misidentified leptons, is defined by signal region cuts but with one lepton chosen to satisfy a selection criteria that is more likely to include more misidentified particles than that used in the analysis signal region.  A control region designed to capture W+jets events where a jet is misidentified as a lepton would be the same as the signal region requiring two leptons, but only one lepton is defined as an analysis lepton, while the other has at least one orthogonal selection criteria cut that makes it easier to include jets in the container of lepton identified in this particular way.  Fake background contribution in the signal region is estimated by scaling the number of selected events in the control region by the fake factor.  Control region and anti-ID lepton definition might have contamination from sources that are not from the background of interest..

\begin{figure}
\centering
 \input{/Users/sheenaschier/Documents/LaFiles/figures/thesis/fakes/fakefactor_schematic.tex}
 \caption{Schematic illustrating the fake factor method to estimate the fake lepton contribution in the signal region.}
 \label{fig:fake_schematic}
 \end{figure}
 
The fake factors are computed from events  with $m_{\mathrm{T}}<40\GeV$, using the distributions in Fig.~\ref{fig:elec_FF_dists_pt}, as:
\begin{equation}
  F(\pt) = \frac{\mathrm{Numerator}_{\mathrm{data}} - \mathrm{Numerator}_{\mathrm{MC}}}{\mathrm{Denominator}_{\mathrm{data}} - \mathrm{Denominator}_{\mathrm{MC}}}
\end{equation}
  \FloatBarrier
  
  %%%%%%%%%%%%%%%%%%%%%%%%%%%%%%%%%%%
  \section{Fake Factor Method Applied to Low-$\pt$ Di-lepton Events}
  \label{sec:FFmethod}
As described in the previous section, the fake background contribution is estimated in a control region, which is scaled to the signal regions by a fake factor.  The fake factor is measured in a region of data that is selected to be enriched with fake lepton events.  In this region, two classes of lepton, ID and anti-ID, are defined, and the fake factor is equal to the ratio between the occurrence of these leptons in the measurement region.  The measurement regions and control regions, along with the ID and anti-ID leptons definitions used to measure the electron and muon fake factors, are detailed in this section.  \textcolor{red}{Mention same-sign validation region.  Also, that Monte Carlo samples are used to estimate and subtract of the prompt lepton contamination in the fake factor measurement region.}

Electron and muon fake factors are measured using 2015+2016 LHC pp data taken by the ATLAS detector with single electron and muon triggers.  The lepton trigger thresholds are chosen to accept the lowest \pt leptons possible and still maximize the event statistics.  The triggers used for this dataset are summarized in Table~\ref{tab:prescaledtrigs}.  In ATLAS, single electron and muon triggers with thresholds below 24 GeV are subject to prescales\footnote{The lowest unprescaled electron and muon trigger threshold evolved to 26 GeV by the end of 2016 data-taking.} because their true rates are too high for every event to be kept.  To resolve the different prescales applied to each trigger, they are unfolded to normalize the entire 2015+2016 dataset arbitrarily to $10\ipb$.  For this, and the rest of the discussion of fake factors, the electron and muon samples are treated separately.
\begin{table}[tbp]
  \centering
  \begin{tabular}{lll}
    \hline
    
    \hline
   Trigger Threshold                            &\multicolumn{2}{c}{Prescaled Luminosity [\ipb]}\\
                                        &2015           &2016\\
    \hline
       
   \hline
   Single Electron Trigger  \\
   \hline 
    5 GeV            &0.1               &0.1    \\
    10 GeV     &0.5               &0.8    \\
    15 GeV  &5.5               &9    \\
    20 GeV          &10                &17    \\
      \hline
            
    \hline
      Single Muon Trigger \\
      \hline 
    4 GeV                    &0.5               &0.5    \\
    10 GeV                   &2.3               &2.5    \\
   14 GeV                   &25                &14    \\
    18 GeV                   &26                &48    \\
    \hline
    
    \hline
  \end{tabular}
  \caption{Pre-scaled single-lepton triggers from 2015+2016 used to compute lepton fake factors.}
  \label{tab:prescaledtrigs}
\end{table}

The fake estimate control regions are constructed using 2015+2016 ATLAS data triggered by the lowest unprescaled inclusive \met{} triggers, as described in Chapter~\ref{sec:data}.  Control region events are selected with all the same cuts as the signal region, but instead of selecting two signal leptons to form the SFOS pair, one signal lepton and one anti-ID lepton is selected to make SFOS pair. 
  \FloatBarrier
  %%%%%%%%%%%%%%%%%%%
  \iffalse
  \subsection{Fake Lepton Composition}
  \textcolor{red}{Might take this section out.}
 Monte Carlo studies of fake and non-prompt lepton composition is done separately for events with opposite sign lepton pairs and events with same-sign lepton pairs.  In the MC samples, there is a variable MCTruthClassifier that determines lepton categories based on their source.  Real prompt leptons fall into two categories: \textit{isolated} and $\ell\rightarrow \gamma \rightarrow \ell$, which refers to truth matched leptons that arise from a Bremsstrahlung to photon conversion process. Fake and non-prompt leptons occupy the remaining categories: \textit{non-isolated}, which are mostly from heavy flavor decays, \textit{photons}, which are either photons faking leptons or actualy leptons from photon conversions, \textit{hadron}, which are from light flavor decays, and \textit{unknown, unknown electron, or unknown muon}, which are primarily from pile-up.\\
 
(\textbf{Describe Figure~\ref{fig:elMC} and Figure~\ref{fig:muMC}}). Sources of fake leptons in the di-muon and di-electron signal regions mostly come from heavy flavor decays, and fake leptons in the di-muon and di-electron control regions are primarily from light flavor decays.  One important result from this study is the similarity of fake lepton contribution between the opposite sign lepton pain and same sign lepton pair events.  This gives confidence that the same-sign validation region can be successfully used to validate our fake background predictions in the data without accidentally unblinding our signal region and biasing the results.  

  
\begin{figure}[htb]
        \centering
        \includegraphics[width=.48\textwidth]{/Users/sheenaschier/Documents/LaFiles/figures/thesis/fakes/fakeLeptonComposition/626_cdsComments_mu_SR2_lep2Pt.pdf}
       \includegraphics[width=.48\textwidth]{/Users/sheenaschier/Documents/LaFiles/figures/thesis/fakes/fakeLeptonComposition/626_cds_noIso_mu_SR2_lep2Pt.pdf}
       % \includegraphics[width=.48\textwidth]{/Users/sheenaschier/Documents/LaFiles/figures/thesis/fakes/fakeLeptonComposition/626_cds_noIso_mu_QCR2_lep2Pt.pdf}
      \includegraphics[width=.48\textwidth]{/Users/sheenaschier/Documents/LaFiles/figures/thesis/fakes/fakeLeptonComposition/626_cds_ss_wIso_mu_SR2_lep2Pt.pdf}
      \includegraphics[width=.48\textwidth]{/Users/sheenaschier/Documents/LaFiles/figures/thesis/fakes/fakeLeptonComposition/626_cds_ss_mu_SR2_lep2Pt.pdf}
        %\includegraphics[width=.48\textwidth]{/Users/sheenaschier/Documents/LaFiles/figures/thesis/fakes/fakeLeptonComposition/626_cds_ss_mu_QCR2_lep2Pt.pdf}
        \caption{Fake lepton composition as a function of subleading lepton $p_{T}$, with and without prompt (``Isolated'' plus ``lep$\to$gamma$\to$lep'') leptons, for opposite sign muon pairs in the signal region.  Top left: SR no iso, top right: CR no iso, bottom left: ssSR iso, bottom right: ssCR iso.}
        \label{fig:muMC}
\end{figure}

 
\begin{figure}[htb]
        \centering
        \includegraphics[width=.48\textwidth]{/Users/sheenaschier/Documents/LaFiles/figures/thesis/fakes/fakeLeptonComposition/725_cdsComments_el_SR2_lep2Pt.pdf}
        \includegraphics[width=.48\textwidth]{/Users/sheenaschier/Documents/LaFiles/figures/thesis/fakes/fakeLeptonComposition/725_cds_noIso_el_SR2_lep2Pt.pdf}
         %\includegraphics[width=.48\textwidth]{/Users/sheenaschier/Documents/LaFiles/figures/thesis/fakes/fakeLeptonComposition/725_cds_noIso_el_QCR2_lep2Pt.pdf}
        \includegraphics[width=.48\textwidth]{/Users/sheenaschier/Documents/LaFiles/figures/thesis/fakes/fakeLeptonComposition/725_cds_ss_wIso_el_SR2_lep2Pt.pdf}
       %\includegraphics[width=.48\textwidth]{/Users/sheenaschier/Documents/LaFiles/figures/thesis/fakes/fakeLeptonComposition/725_cds_ss_el_QCR2_lep2Pt.pdf}
          \includegraphics[width=.48\textwidth]{/Users/sheenaschier/Documents/LaFiles/figures/thesis/fakes/fakeLeptonComposition/725_cds_ss_el_SR2_lep2Pt.pdf}
        \caption{Fake lepton composition in opposite sign signal and control region as a function subleading lepton $p_{T}$, with and without prompt (``Isolated'' plus ``lep$\to$gamma$\to$lep'') leptons, for opposite sign electron pairs in the signal region.  Top left: SR no iso, top right: CR no iso, bottom left: ssSR iso, bottom right: ssCR iso.} 
        \label{fig:elMC}
\end{figure}

\fi
 \FloatBarrier
 
%%%%%%%%%%%%%%%%%%%%%%%
 \subsection{ID \& Anti-ID Lepton Definitions}
For both electron and muon fake factors, ID leptons are defined by the same signal lepton criteria as for the lepton pairs in the signal regions.  Anti-ID lepton definitions are chosen so that this category is mostly populated with fakes and depleted in real prompt leptons.  This fake lepton enhancement is achieved by inverting the cuts used to suppress lepton misidentification.  Having an anti-ID definition that is close to the ID definition reduces the systematic uncertainties on the fake background prediction.  Adversely, tighter anti-ID cuts will decrease the acceptance of fakes, which increases the statistical uncertainty on the fake background prediction.  This section will detail the anti-ID lepton selections. 

%%%%%%%%
\subsubsection{Electron Definitions}

ID electrons are constructed with the same definition as signal electrons, summarized in Table~\ref{tab:objdef}.  These are baseline electrons that also pass \textit{TightLLH} identification, \textit{GradientLoose} isolation, and $|d_0/\sigma(d_0)|<5.0$ requirements.  Anti-ID electrons start as baseline electrons, but are required to pass a slightly tighter PID, \textit{LooseAndBLayerLLH}.  Additionally, anti-ID electrons are required to fail at least one of the signal electron criteria.  This means anti-ID electrons must fail \textit{TightLLH} identification, or \textit{GradientLoose} isolation, or $|d_0/\sigma(d_0)|<5.0$, or some combination of these. %Studies motivating the definition of the anti-ID electrons were performed and are documented in this section. 
All ID and anti-ID electrons are required to pass the $|z_0\sin\theta| < 0.5$~mm requirement to reduce the impact of pileup.  The ID and anti-ID electron definitions are summarized in Table~\ref{tab:AllElDefs}. 
\begin{table}[!htb]
\begin{center}
\begin{tabular}{c|c}
\hline
Signal Electron Definition  & Anti-ID Electron Definition \\
\hline \hline
\multicolumn{2}{c}{$\pt > 4.5\GeV$}      \\
\multicolumn{2}{c}{$\abseta < 2.47$ }     \\
\multicolumn{2}{c}{$|z_0\sin\theta| < 0.5$~mm} \\
\multicolumn{2}{c}{Electron \textit{author} $!= 16$}\\
Pass \textit{Tight} Identification & Pass \textit{LooseAbdBLayer} Identification\\
%Pass \textit{Tight} Identification}  &  Pass \textit{LooseAndBLayer} Identification\\
      &             \textbf{(}Fail \textit{Tight} Identification \textbf{\textit{or}} \\
Pass \textit{GradientLoose} Isolation  & Fail \textit{GradientLoose} Isolation \textbf{\textit{or}} \\   
$|d_0/\sigma(d_0)| < 5$  &   $|d_0/\sigma(d_0)| > 5$\textbf{)} \\
\hline
\end{tabular}
\caption{Summary of electron definitions.}
\label{tab:AllElDefs}
\end{center}
\end{table}
The fractional composition of anti-ID electrons in the fake factor measurement region, according to the set of failed signal electron criteria, is shown in Figure~\ref{fig:elDeco}. Here, the $m_{T}$ distribution is plotted over the entire $m_{T}$ spectrum, while the $\met$, \pt{} and $\eta$ distributions are all shown for $m_{T} <40\GeV$.  The significant of the $m_{T}$ cut is explained in Section~\ref{sec:FFel}.  From Fig~\ref{fig:elDeco}, we learn that the anti-ID electrons are 40-50$\%$ electrons that fail both the \textit{TightLLH} identification and \textit{GradientLoose} isolation, 25$\%$ electrons that only fail identification, 25$\%$ electrons that only fail isolation, and a tiny fraction of electrons that failed $|d_0/\sigma(d_0)|<5.0$. 

\begin{figure}[htb]
        \centering
         \includegraphics[width=.49\textwidth]{/Users/sheenaschier/Documents/LaFiles/figures/thesis/fakes/FF_electron/AID_deco_AntiIDelPt}
        \includegraphics[width=.49\textwidth]{/Users/sheenaschier/Documents/LaFiles/figures/thesis/fakes/FF_electron/AID_deco_AntiIDelEta}
        \includegraphics[width=.49\textwidth]{/Users/sheenaschier/Documents/LaFiles/figures/thesis/fakes/FF_electron/AID_deco_Mt}
        \includegraphics[width=.49\textwidth]{/Users/sheenaschier/Documents/LaFiles/figures/thesis/fakes/FF_electron/AID_deco_MET}
        \caption{Fake electron composition as a function of electron \pt{} (top left), electron $\eta$ (top right), $m_{T}$, (bottom left) and $\met$ (bottom right). All distributions corresponds to events with $m_{T} < 40\GeV$, excluding the $m_{T}$ distribution. }
        \label{fig:elDeco}
\end{figure}
In choosing the best anti-ID definition to use, there is a trade-off between systematic and statistical uncertainties.  A dedicated study of different anti-ID electron definitions to determine which best models the source of fake electron backgrounds and relatively minimizes the statistical uncertainties in the fake background estimate was performed.  It was observed that requiring a tighter electron identification working-point enhances fraction of heavy flavor decays.  Requiring tracks to have a hit in the b-layer reduces fraction of fakes from conversions.  A Loose or Medium isolation requirement narrows the source of fakes towards heavy and light hadronic decays.  Lastly, requiring a large $d_0/\sigma_{d_0}$ can increase the fraction of heavy flavor decays and conversions.  Unfortunately, the Medium isolation and the large $d_0/\sigma_{d_0}$ requirements starkly decrease the number of electrons that pass the anti-ID requirements.
 \begin{figure}
 \centering
 \iffalse
 \begin{subfigure}[b]{0.47\textwidth}
    \includegraphics[width=\textwidth]{/Users/sheenaschier/Documents/LaFiles/figures/thesis/fakes/antiIDStudies/AllMC_ee_SR_lep2Pt.pdf}
 %   \caption{Signal lepton;\\ 9.99 MC W+jet events.}
    \end{subfigure}
     \begin{subfigure}[b]{0.47\textwidth}
  \includegraphics[width=\textwidth]{/Users/sheenaschier/Documents/LaFiles/figures/thesis/fakes/antiIDStudies/AllMC_ee_VeryLoose_FailSignal_lep2Pt.pdf}
% \caption{ VeryLoose \& !signal;\\ 433.02 MC W+jet events.}
 \end{subfigure}
 \fi
  \begin{subfigure}[b]{0.47\textwidth}
     \includegraphics[width=\textwidth]{/Users/sheenaschier/Documents/LaFiles/figures/thesis/fakes/antiIDStudies/AllMC_ee_VeryLooseBL_FailSignal_lep2Pt.pdf}
  %    \caption{VeryLoose \& PassBL \& !signal;\\ 313.38 MC W+jet events.}
 \end{subfigure}
   \begin{subfigure}[b]{0.47\textwidth}
     \includegraphics[width=\textwidth]{/Users/sheenaschier/Documents/LaFiles/figures/thesis/fakes/antiIDStudies/AllMC_ee_LooseBL_FailSignal_lep2Pt.pdf}
%      \caption{Loose \& PassBL \&\& !signal;\\ 168.85 MC W+jet events.}
 \end{subfigure}       
    \begin{subfigure}[b]{0.46\textwidth}
     \includegraphics[width=\textwidth]{/Users/sheenaschier/Documents/LaFiles/figures/thesis/fakes/antiIDStudies/AllMC_ee_Medium_FailSignal_lep2Pt.pdf}
%      \caption{Loose \&\& Medium \& !signal;\\ 75.28 MC W+jet events.}
 \end{subfigure}
    \begin{subfigure}[b]{0.46\textwidth}
     \includegraphics[width=\textwidth]{/Users/sheenaschier/Documents/LaFiles/figures/thesis/fakes/antiIDStudies/AllMC_ee_LooseBL_D0SigGt_OR_Medium_lep2Pt.pdf}
  %    \caption{Medium$||$(Loose \& PassBL \& $d_0/\sigma_{d_0}> 1.5$) \& !signal; 86.28 MC W+jet events.}
 \end{subfigure} 
 
    \caption{Fake lepton composition as a function of the subleading lepton \pt. \textcolor{red}{Talk to Mike about showing these plots.} }
 \label{fig:LeptonIDComposition}
\end{figure}


  \FloatBarrier
   %%%%%%%%%% 
\subsubsection{Muon Definitions}

ID muons are defined with the same selection criteria as signal muons, summarized in Table~\ref{tab:objdef}. These are baseline muons that also satisfy the \textit{FixedCutTightTrackOnly} isolation requirements and the impact paramter significance requirement $|d_0/\sigma(d_0)|<3.0$.  Anti-ID muons are also baseline muons, but instead of requiring they pass the isolation and $d_0$ significance requirements of the ID muons, they instead must fail the \textit{FixedCutTightTrackOnly} isolation or $|d_0/\sigma(d_0)|<3.0$ criteria\footnote{Failing both the isolation and the $d_0$ significance cut still satisfies the anti-ID definition.}. Both the ID and anti-ID muons are required to pass the $|z_0\sin\theta| < 0.5$~mm requirement to reduce the impact of pileup.  One notable difference with respect to the signal muon requirements is that the muon-jet overlap removal is relaxed when performing the fake factor measurement.  This enhances the statistics used for deriving the fake factors, and is motivated by the observation that the muon-jet overlap removal is primarily designed to reduce the number of heavy flavor decays which are mistakenly being classified as prompt muons.  A summary of the ID and anti-ID muon definitions are summarized in Table~\ref{tab:AllMuDefs}
\begin{table}[!htb]
\begin{center}
\begin{tabular}{c|c}
\hline
Signal Muon Definition  & Anti-ID Muon Definition \\
\hline \hline
\multicolumn{2}{c}{$\pt > 4\GeV$}      \\
\multicolumn{2}{c}{$\abseta < 2.5$ }     \\
\multicolumn{2}{c}{$|z_0\sin\theta| < 0.5$~mm} \\
\multicolumn{2}{c}{Pass \textit{Medium} Identification}     \\
$|d_0/\sigma(d_0)| < 3$  &   \textbf{(}$|d_0/\sigma(d_0)| > 3$ \textbf{\textit{or}}\\
Pass \textit{FixedCutTightTrackOnly} Isolation  & Fail \textit{FixedCutTightTrackOnly} Isolation\textbf{)} \\   \\
\hline
\end{tabular}
\caption{Summary of muon definitions.}
\label{tab:AllMuDefs}
\end{center}
\end{table}

The anti-ID muons decomposition, according to which set of ID criteria failed, is shown in Fig~\ref{fig:muDeco}. The $m_{T}$ distribution is plotted over the entire $m_{T}$ range, while the $\met$, \pt{} and $\eta$ distributions are all shown for $m_{T} <40\GeV$, corresponding to the fake enriched region where the fake factors are measured.  Note that these distributions are separated into categories: events with exactly zero $b$-jets, events with one or more $b$-jets.  In studying the fake factor dependence on different kinematic variables, which is discussed later in Section~\ref{sec:ff}, b-jet multiplicity was found to have a large variation.  In events with exactly zero b-jets, the anti-ID muon composition is approximately 50-65$\%$ muons that fail only the \textit{FixedCutTightTrackOnly} isolation, and 20-40$\%$ muons that fail both isolation and $d_0$ significance at low \pt.  In events with one or more b-jets, the fraction of anti-ID muons from failing only isolation is reduced, but still the majority, and the fraction from failing both isolation and $d_0$ significance is a bit higher.     
\begin{figure}
        \centering
        \includegraphics[width=.4\textwidth]{/Users/sheenaschier/Documents/LaFiles/figures/thesis/fakes/FF_muon/AID_deco_Mt_b0}
                \includegraphics[width=.4\textwidth]{/Users/sheenaschier/Documents/LaFiles/figures/thesis/fakes/FF_muon/AID_deco_Mt_b1}
        \includegraphics[width=.4\textwidth]{/Users/sheenaschier/Documents/LaFiles/figures/thesis/fakes/FF_muon/AID_deco_MET_b0}
        \includegraphics[width=.4\textwidth]{/Users/sheenaschier/Documents/LaFiles/figures/thesis/fakes/FF_muon/AID_deco_MET_b1}\\
                \includegraphics[width=.4\textwidth]{/Users/sheenaschier/Documents/LaFiles/figures/thesis/fakes/FF_muon/AID_deco_AntiIDmuPt_b0}
                        \includegraphics[width=.4\textwidth]{/Users/sheenaschier/Documents/LaFiles/figures/thesis/fakes/FF_muon/AID_deco_AntiIDmuPt_b1}
        \includegraphics[width=.4\textwidth]{/Users/sheenaschier/Documents/LaFiles/figures/thesis/fakes/FF_muon/AID_deco_AntiIDmuEta_b0}
        \includegraphics[width=.4\textwidth]{/Users/sheenaschier/Documents/LaFiles/figures/thesis/fakes/FF_muon/AID_deco_AntiIDmuEta_b1}\\
        \caption{Anti-ID muon composition in events with exactly zero $b$-jets(left) and one or more $b$-jets(right) as a function of $m_{T}$, $\met$, muon \pt{}, and muon $\eta$. All but the $m_{T}$ distribution corresponds to events with $m_{T} < 40\GeV$.}
        \label{fig:muDeco}
\end{figure}


  \FloatBarrier
  
 \subsection{Fake Factor Measurement}
   \textcolor{red}{Generalize the process of calculating fake factors...}  We have these data samples, and we select events that have at least one ID or anti-ID lepton within fiducial acceptance of the detector.  We plot all the kinematics of the ID and anti-ID lepton events.  The fake factors are measured in the region $m_{T} < 40\GeV$ because this region is dominated by fake leptons.  This is shown in the $m_{T}$ plots in data overlaid with the stacked Monte Carlo.  The prompt lepton contamination in the measurement region is subtracted off, but first it is normalized to the data in the $\met{} > 200~\GeV$ region that should be dominated by real prompt leptons.
  \subsubsection{Electron Fake Factors}
  \label{sec:FFel}

\begin{table}[tbp]
  \centering
  \begin{tabular}{|c|c|}
    \hline
    el trigger  & \pt{} range [\GeV]\\
    \hline
    HLT\_e5\_lvhloose & 5--11  \\
    HLT\_e10\_lvhloose\_L1EM7 & 11--18  \\
    HLT\_e15\_lvhloose\_L1EM13VH & 18--23  \\
    HLT\_e20\_lvhloose & $>$ 23  \\
    \hline
  \end{tabular}
  \caption{Single-Electron triggers used for fake factor computation and their corresponding \pt{} range.}
  \label{tab:elec_trigger_range}
\end{table}


Electron fake factors show the largest dependance on electron \pt{}, but also display a dependence on the leading jet \pt{}, which is evident in Fig.~\ref{fig:elec_FF_hist_noCut} that shows electron fake factors as a function of electron \pt{} and leading jet \pt{} separately. Given this trend, and the fact that all signal regions used in this analysis require a hard jet with \pt{} greater than 100\GeV, we design the fake factor measurement region to also require a hard jet of \pt{} greater than 100\GeV.  Fake factors as a function of other kinematic variables are also studied as a cross-check and for understanding systematic uncertainties.



\begin{figure}[tbp]
  \centering
  \includegraphics[width=0.48\columnwidth]{/Users/sheenaschier/Documents/LaFiles/figures/thesis/fakes/FF_electron/ID_CR_MET}
  \includegraphics[width=0.48\columnwidth]{/Users/sheenaschier/Documents/LaFiles/figures/thesis/fakes/FF_electron/ID_CR_Mt}\\
  \includegraphics[width=0.48\columnwidth]{/Users/sheenaschier/Documents/LaFiles/figures/thesis/fakes/FF_electron/AID_CR_MET}
  \includegraphics[width=0.48\columnwidth]{/Users/sheenaschier/Documents/LaFiles/figures/thesis/fakes/FF_electron/AID_CR_Mt}
  \caption{The \met{} (left) and $m_{T}$ (right) distributions for numerator (top) and denominator (bottom) electrons in the pre-scaled single-lepton-trigger sample.  MC has been scaled to the data in the $\met > 200\GeV$ region.}
  \label{fig:elec_FF_dists_1}
\end{figure}

\begin{figure}[tbp]
  \centering
  \includegraphics[width=0.48\columnwidth]{/Users/sheenaschier/Documents/LaFiles/figures/thesis/fakes/FF_electron/ID_SR_IDelPt}
  \includegraphics[width=0.48\columnwidth]{/Users/sheenaschier/Documents/LaFiles/figures/thesis/fakes/FF_electron/AID_SR_AntiIDelPt}\\
  \caption{Electron \pt{} for numerator (left) and denominator (right) objects in the pre-scaled single-lepton-trigger sample for events with $m_{T} < 40\GeV$.  MC has been scaled to the data in the $\met > 200\GeV$ region.}
  \label{fig:elec_FF_dists_pt}
\end{figure}

% actual fake factors
% Trigger distributions in lepton pt
\begin{figure}[tbp]
  \centering
  \includegraphics[width=0.48\columnwidth]{/Users/sheenaschier/Documents/LaFiles/figures/thesis/fakes/FF_electron/electronTriggers}
  \includegraphics[width=0.48\columnwidth]{/Users/sheenaschier/Documents/LaFiles/figures/thesis/fakes/FF_electron/AIDelectronTriggers}\\
  \caption{The numerator electron (left) and denominator electron (right) \pt{} distributions for pre-scaled single-lepton-trigger, normalized to 1~\ipb{}. Blue curve: HLT\_e5\_lvhloose, red curve: HLT\_e10\_lvhloose\_L1EM7, purple curve: HLT\_e15\_lvhloose\_L1EM13, green curve: HLT\_e20\_lvhloose.}
  \label{fig:triggers}
\end{figure}


\begin{figure}[tbp]
  \centering
  \includegraphics[width=0.48\columnwidth]{/Users/sheenaschier/Documents/LaFiles/figures/thesis/fakes/FF_electron/FakeFactor_el_pt_noCut}
  \includegraphics[width=0.48\columnwidth]{/Users/sheenaschier/Documents/LaFiles/figures/thesis/fakes/FF_electron/FakeFactor_el_j1pt_noCut}\\
  \caption{Electron fake factors \textit{before} requiring a hard jet of $\pt{} > 100\GeV$, computed from single-electron prescaled triggers as a function of electron \pt{} (left) and leading jet \pt{} (right). Fake factors for electron $\pt{}~ 4.5-5\GeV$ are taken to be the same as electron $\pt{}~5-6\GeV$.  A red line denotes the average electron fake factor over all electron \pt{} of 0.267. }
  \label{fig:elec_FF_hist_noCut}
\end{figure}


Final fake factors computed as a function of electron \pt{} are shown in Fig.~\ref{fig:elec_FF_hist}a.  In addition, fake factors as functions of other variables are also inspected to check for significant trends:
\begin{itemize}
\item the dependence of the fake factors on $|\eta|$ is shown in Fig.~\ref{fig:elec_FF_hist}b,
\item fake factors as a function of leading jet \pt{} and  $\Delta\phi_{jet-\met}$ are shown in Fig.~\ref{fig:elec_FF_hadronic},
\item fake factors as a function of jet multiplicity and $b$-jet multiplicity are shown in Fig.~\ref{fig:elec_FF_njet},
\item fake factors as a function of pile up variables, such as average interaction per bunch crossing and number of primary vertices, are also shown in Fig.~\ref{fig:elec_FF_pileup}.
\end{itemize}
The relative uncertianties on the final electron fake factors versus electron \pt{} are shown in Fig.~\ref{fig:elec_FF_rel_uncert}.

%The relative statistical uncertaintiess are shown in Fig.~\ref{fig:elec_FF_2D} and  will be incorporated into the total systematic uncertainty on the electron fake factors.
\begin{figure}[tbp]
  \centering
  \includegraphics[width=0.48\columnwidth]{/Users/sheenaschier/Documents/LaFiles/figures/thesis/fakes/FF_electron/FakeFactor_el_pt}
  \includegraphics[width=0.48\columnwidth]{/Users/sheenaschier/Documents/LaFiles/figures/thesis/fakes/FF_electron/FakeFactor_el_eta}
  \caption{Electron fake factors computed from single-electron prescaled triggers as a function of electron \pt{} (left) and electron $\eta$ (right) in the kinematic region with leading jet$ \pt{}>100\GeV$  Fake factors for electron $\pt{}~ 4.5-5\GeV$ are taken to be the same as electron $\pt{}~5-6\GeV$.  A red line denotes the average electron fake factor over all electron \pt{} of 0.211. }
  \label{fig:elec_FF_hist}
\end{figure}

\begin{figure}[tbp]
  \centering
  \includegraphics[width=0.48\columnwidth]{/Users/sheenaschier/Documents/LaFiles/figures/thesis/fakes/FF_electron/FakeFactor_el_j1pt}
  \includegraphics[width=0.48\columnwidth]{/Users/sheenaschier/Documents/LaFiles/figures/thesis/fakes/FF_electron/FakeFactor_el_dphij}\\
  \caption{Electron fake factors computed from single-electron prescaled triggers as a function of leading jet \pt{} (left) and $\Delta\phi_{jet-\met}$ (right). A red line denotes the average electron fake factor over all electron \pt{} of 0.211.}
  \label{fig:elec_FF_hadronic}
\end{figure}

\begin{figure}[tbp]
  \centering
  \includegraphics[width=0.48\columnwidth]{/Users/sheenaschier/Documents/LaFiles/figures/thesis/fakes/FF_electron/FakeFactor_el_njet}
  \includegraphics[width=0.48\columnwidth]{/Users/sheenaschier/Documents/LaFiles/figures/thesis/fakes/FF_electron/FakeFactor_el_nbjet}\\
  \caption{Electron fake factors computed from single-electron prescaled triggers as a function of the jet multiplicity (left) and the $b$-jet multiplicity (right). A red line denotes the average electron fake factor over all electron \pt{} of 0.211.}
  \label{fig:elec_FF_njet}
\end{figure}


\begin{figure}[tbp]
  \centering
  \includegraphics[width=0.48\columnwidth]{/Users/sheenaschier/Documents/LaFiles/figures/thesis/fakes/FF_electron/FakeFactor_el_mu}
  \includegraphics[width=0.48\columnwidth]{/Users/sheenaschier/Documents/LaFiles/figures/thesis/fakes/FF_electron/FakeFactor_el_npv}\\
  \caption{Electron fake factors computed from single-electron prescaled triggers as a function of the average interaction per bunch crossing (left) and the number of primary vertices (right). A red line denotes the average electron fake factor over all electron \pt{} of 0.211.}
  \label{fig:elec_FF_pileup}
\end{figure}

\begin{figure}[tbp]
  \centering
  \includegraphics[width=0.48\columnwidth]{/Users/sheenaschier/Documents/LaFiles/figures/thesis/fakes/FF_electron/FakeFactor_el_pt_uncert}\\
  \caption{Relative uncertainties on electron fake factors binned electron \pt{}.}
  \label{fig:elec_FF_rel_uncert}
\end{figure}
 \FloatBarrier
%%%%%%%%%%%%%%%%%%%%
 \subsubsection{Muon Fake Factors}
 Both data and MC contributions to the numerator and denominator samples in the single-muon trigger sample are normalized to 10~\ipb, to remove the effects of the prescales in the data.  The MC is then re-scaled to the data in events with $\met{}>200$\GeV, a kinematic region expected to pure in prompt leptons.  For events with exactly 0 $b$-jets, the MC re-scaling factor for numerator muons is $1.01 \pm 0.13$, for denominator muons it is $1.20\pm 0.29$. For events with one or more $b$-jets, the MC re-scaling factor for numerator muons is $1.24 \pm 0.20$, for denominator muons it is $7.34\pm 5.00$. If instead, the MC is re-scaled to match the data for events with $m_{T} > 100$\GeV, a region that should also be pure in prompt leptons, the re-scaling factors for events with exactly 0 $b$-jets are $2.37 \pm 0.10$ for numerator muons and $11.68 \pm 2.28$ for denominator muons; events with one or more $b$-jets have re-scale factors $1.60 \pm 0.06$ for numerator muons and $10.41 \pm 6.34$ for denominator muons. The re-scaling factors vary significantly between the two methods but the fake factors themselves exhibit small changes between the two methods and can be used as a systematic uncertainty.

Distributions of \met{} and $m_{T}$ for numerator and denominator muons for events with exactly zero $b$-jets are shown in Fig.~\ref{fig:muon_FF_dists_b0}, and for events with one or more $b$-jets in Fig.~\ref{fig:muon_FF_dists_b1}.  Muon \pt{} distributions for events with exactly zero $b$-jet are shown in Fig.~\ref{fig:muon_FF_dists_pt_b0}, and for events with one or more $b$-jets in Fig.~\ref{fig:muon_FF_dists_pt_b1}.

% mT, MET, and lepton pT plots for ID, anti-ID
\begin{figure}[tbp]
  \centering
  \includegraphics[width=0.48\columnwidth]{/Users/sheenaschier/Documents/LaFiles/figures/thesis/fakes/FF_muon/IDb0_CR_MET}
  \includegraphics[width=0.48\columnwidth]{/Users/sheenaschier/Documents/LaFiles/figures/thesis/fakes/FF_muon/IDb0_CR_Mt}\\
  \includegraphics[width=0.48\columnwidth]{/Users/sheenaschier/Documents/LaFiles/figures/thesis/fakes/FF_muon/AIDb0_CR_MET}
  \includegraphics[width=0.48\columnwidth]{/Users/sheenaschier/Documents/LaFiles/figures/thesis/fakes/FF_muon/AIDb0_CR_Mt}
  \caption{The \met{} (left) and  $m_{T}$ (right) distributions for numerator (top) and denominator (bottom) muons in the prescaled single-lepton-trigger sample for events with exactly zero $b$-jets.  MC has been scaled to the data in the $\met > 200\GeV$ region.}
  \label{fig:muon_FF_dists_b0}
\end{figure}

\begin{figure}[tbp]
  \centering
  \includegraphics[width=0.48\columnwidth]{/Users/sheenaschier/Documents/LaFiles/figures/thesis/fakes/FF_muon/IDb1_CR_MET}
  \includegraphics[width=0.48\columnwidth]{/Users/sheenaschier/Documents/LaFiles/figures/thesis/fakes/FF_muon/IDb1_CR_Mt}\\
  \includegraphics[width=0.48\columnwidth]{/Users/sheenaschier/Documents/LaFiles/figures/thesis/fakes/FF_muon/AIDb1_CR_MET}
  \includegraphics[width=0.48\columnwidth]{/Users/sheenaschier/Documents/LaFiles/figures/thesis/fakes/FF_muon/AIDb1_CR_Mt}
  \caption{The \met{} (left) and $m_{T}$ (right) distributions for numerator (top) and denominator (bottom) muons in the prescaled single-lepton-trigger sample for events with one or more $b$-jets.  MC has been scaled to the data in the $\met > 200\GeV$ region.}
  \label{fig:muon_FF_dists_b1}
\end{figure}

\begin{figure}[tbp]
  \centering
  \includegraphics[width=0.48\textwidth]{/Users/sheenaschier/Documents/LaFiles/figures/thesis/fakes/FF_muon/IDb0_SR_IDmuPt}
  \includegraphics[width=0.48\textwidth]{/Users/sheenaschier/Documents/LaFiles/figures/thesis/fakes/FF_muon/AIDb0_SR_AntiIDmuPt}
  \caption{Muon \pt{} for numerator (left) and denominator (right) objects in the prescaled single-muon trigger sample for events with $m_{T} < 40\GeV$.  MC has been scaled to the data in the $m_{T} > 100\GeV$ region. Distributions from~\cite{Boerner:2231917}.}
  \label{fig:muon_FF_dists_pt_b0}
\end{figure}

\begin{figure}[tbp]
  \centering
  \includegraphics[width=0.48\textwidth]{/Users/sheenaschier/Documents/LaFiles/figures/thesis/fakes/FF_muon/IDb1_SR_IDmuPt}
  \includegraphics[width=0.48\textwidth]{/Users/sheenaschier/Documents/LaFiles/figures/thesis/fakes/FF_muon/AIDb1_SR_AntiIDmuPt}
  \caption{Muon \pt{} for numerator (left) and denominator (right) objects in the prescaled single-muon trigger sample for events with $m_{T}< 40\GeV$.  MC has been scaled to the data in the $m_{T} > 100\GeV$ region. Distributions from~\cite{Boerner:2231917}.}
  \label{fig:muon_FF_dists_pt_b1}
\end{figure}


% actual fake factors
The fake factors are computed using events with $m_{\mathrm{T}}<40\GeV$, using the distribution in Figs.~\ref{fig:muon_FF_dists_pt_b0} and \ref{fig:muon_FF_dists_pt_b1}, as
\begin{equation}
  F(\pt) = \frac{\mathrm{Numerator}_{\mathrm{data}} - \mathrm{Numerator}_{\mathrm{MC}}}{\mathrm{Denominator}_{\mathrm{data}} - \mathrm{Denominator}_{\mathrm{MC}}}
\end{equation}
where the fake factor $F$ is computed in discrete \pt{} bins with different single-muon triggers applied. The specific trigger applied to each range in lepton \pt{} was chosen to reduce the effect of the trigger turn on and maintain good statistics. Muon \pt{} distributions for the prescaled triggers shown in Fig.~\ref{fig:mu_triggers} are arbitrarily normalized to 1~\ipb.  HLT\_mu4 trigger is required for muon \pt{} $4 - 11\GeV$, HLT\_mu10 is required for muon \pt{} $11- 15\GeV$, HLT\_mu14 is required for muon \pt{} $15-20\GeV$, and HLT\_mu18 is required for muon \pt{} $>20\GeV$. A table of these triggers and corresponding \pt{} range is shown in Table~\ref{tab:muon_trigger_range}  %The final fake factors are shown in Table~\ref{fig:muon_FF_values}.

% Trigger distributions in lepton pt
\begin{figure}[tbp]
  \centering
  \includegraphics[width=0.48\columnwidth]{/Users/sheenaschier/Documents/LaFiles/figures/thesis/fakes/FF_muon/IDmuonTriggers}
  \includegraphics[width=0.48\columnwidth]{/Users/sheenaschier/Documents/LaFiles/figures/thesis/fakes/FF_muon/AntiIDmuonTriggers}\\
  \caption{The numerator muon (left) and denominator denominator (right) \pt{} distributions for prescaled single-muon triggers, normalized to 1~\ipb{}. Blue curve: HLT\_mu4, red curve: HLT\_mu10, purple curve: HLT\_mu14, green curve: HLT\_mu18.}
  \label{fig:mu_triggers}
\end{figure}
\begin{table}[tbp]
  \centering
  \begin{tabular}{|c|c|}
    \hline
    el trigger  & \pt{} range [\GeV]\\
    \hline
    HLT\_mu4 &4 --11  \\
    HLT\_mu10 & 11--15  \\
    HLT\_mu14 & 18--20  \\
    HLT\_mu18 & $>$ 20  \\
    \hline
  \end{tabular}
  \caption{Single-muon triggers used for fake factor computation and their corresponding \pt{} range.}
  \label{tab:muon_trigger_range}
\end{table}

Muon fake factors depend strongly on muon \pt, but also display a systematic dependence on the leading jet \pt{}.  Unlike the electron fake factors, there is also a separate dependence on $b$-jet multiplicity.  Fig.~\ref{fig:muon_FF_hist_noCut} shows the muon fake factors as functions of muon \pt{}, leading jet \pt{}, and $b$-jet multiplicity before any hard jet requirement.  Similar to the electron fake factor calculation, the fake factor measurement region requires a hard jet of \pt{} greater than $100\GeV$, but unlike the electron fake factors, the muon fake factros are also separated into two $b$-jet multiplicity bins: exactly zero $b$-jets, and one or more $b$-jets.  The bin with exactly zero $b$-jets is used to estimate the fake contribution in the signal region, and the bin with one or more $b$-jets is used to estimate the fake contribution in the $t\bar{t}$ control region.

\begin{figure}[tbp]
  \centering
  \includegraphics[width=0.48\columnwidth]{/Users/sheenaschier/Documents/LaFiles/figures/thesis/fakes/FF_muon/FakeFactor_mu_pt}
  \includegraphics[width=0.48\columnwidth]{/Users/sheenaschier/Documents/LaFiles/figures/thesis/fakes/FF_muon/FakeFactor_mu_j1pt}\\
  \includegraphics[width=0.48\columnwidth]{/Users/sheenaschier/Documents/LaFiles/figures/thesis/fakes/FF_muon/FakeFactor_mu_nbjet}\\
  \caption{Muon fake factors \textit{before} requiring a hard jet of $\pt{}> 100\GeV$, computed from single-muon prescaled triggers as a function of muon \pt{} (top-left), as a function of leading jep \pt{} (top-right), and as a function of $b$-jet multiplicity (bottom). A red line denotes the average muon fake factor over all muon \pt{}.}
  \label{fig:muon_FF_hist_noCut}
\end{figure}

The final fake factors are shown in Fig.~\ref{fig:muon_FF_hist} as a functions of muon \pt{} for each of the $b$-jet multiplicity bins.  In addition to the final fake factors binned in \pt, fake factors binned in other variables are also inspected to check for significant trends:
\begin{itemize}
\item Fake factors as a function of muon $\eta$ are shown in Fig.~\ref{fig:muon_FF_hist_eta},
\item Fake factors as a function of $\Delta\phi_{jet1-met}$ are shown in Fig.~\ref{fig:muon_FF_dphij1},
\item Fake factors as a function of jet multiplicity are shown in Fig.~\ref{fig:muon_FF_njet},
%\item Fake factors as a function of $b$-jet multiplicity are shown in Fig.~\ref{fig:muon_FF_nbjet},
\item Fake factors as a function of average interactions per bunch crossing are shown in Fig.~\ref{fig:muon_FF_mu},
\item Fake factors as a function of the number of primary vertices are shown in Fig.~\ref{fig:muon_FF_npv}.
\end{itemize}
The relative uncertianties on the muons fake factors versus muon \pt{} for the separate $b$-jet multiplicity bins are show in Fig.~\ref{fig:muon_FF_rel_uncert}.

\begin{figure}[tbp]
  \centering
  \includegraphics[width=0.48\columnwidth]{/Users/sheenaschier/Documents/LaFiles/figures/thesis/fakes/FF_muon/FakeFactor_mu_ptb0}
  \includegraphics[width=0.48\columnwidth]{/Users/sheenaschier/Documents/LaFiles/figures/thesis/fakes/FF_muon/FakeFactor_mu_ptb1}\\
  \caption{Muon fake factors computed from single-muon prescaled triggers as a function of muon \pt{} in events with exactly zero $b$-jets (left) and one or more $b$-jets (right). A red line denotes the average muon fake factor over all muon \pt{}.}
  \label{fig:muon_FF_hist}
\end{figure}

\begin{figure}[tbp]
  \centering
  \includegraphics[width=0.48\columnwidth]{/Users/sheenaschier/Documents/LaFiles/figures/thesis/fakes/FF_muon/FakeFactor_mu_etab0}
  \includegraphics[width=0.48\columnwidth]{/Users/sheenaschier/Documents/LaFiles/figures/thesis/fakes/FF_muon/FakeFactor_mu_etab1}\\
  \caption{Muon fake factors computed from single-muon prescaled triggers as a function of muon $\eta$ in events with exactly zero $b$-jets (left) and one or more $b$-jets (right). A red line denotes the average muon fake factor over all muon \pt{}.}
  \label{fig:muon_FF_hist_eta}
\end{figure}

\begin{figure}[tbp]
  \centering
  \includegraphics[width=0.48\columnwidth]{/Users/sheenaschier/Documents/LaFiles/figures/thesis/fakes/FF_muon/FakeFactor_mu_dphijb0}
  \includegraphics[width=0.48\columnwidth]{/Users/sheenaschier/Documents/LaFiles/figures/thesis/fakes/FF_muon/FakeFactor_mu_dphijb1}
  \caption{Muon fake factors computed from single-muon prescaled triggers as a function of $\Delta\phi_{jet-\met}$ in events with exactly zero $b$-jets (left) and one or more $b$-jets (right).  A red line denotes the average muon fake factor over all muon \pt{}}
  \label{fig:muon_FF_dphij1}
\end{figure}

\begin{figure}[tbp]
  \centering
  \includegraphics[width=0.48\columnwidth]{/Users/sheenaschier/Documents/LaFiles/figures/thesis/fakes/FF_muon/FakeFactor_mu_njetb0}
  \includegraphics[width=0.48\columnwidth]{/Users/sheenaschier/Documents/LaFiles/figures/thesis/fakes/FF_muon/FakeFactor_mu_njetb1}\\
  \caption{Muon fake factors computed from single-muon prescaled triggers as a function of the jet multiplicity in events with exactly zero $b$-jets (left) and one or more $b$-jets (right).  A red line denotes the average muon fake factor over all muon \pt{}}
  \label{fig:muon_FF_njet}
\end{figure}

\begin{figure}[tbp]
  \centering
  \includegraphics[width=0.48\columnwidth]{/Users/sheenaschier/Documents/LaFiles/figures/thesis/fakes/FF_muon/FakeFactor_mu_mub0}
  \includegraphics[width=0.48\columnwidth]{/Users/sheenaschier/Documents/LaFiles/figures/thesis/fakes/FF_muon/FakeFactor_mu_mub1}\\
  \caption{Muon fake factors computed from single-muon prescaled triggers as a function of the average number of interactions per bunch crossing in events with exactly zero $b$-jets (left) and one or more $b$-jets (right).  A red line denotes the average muon fake factor over all muon \pt{}}
  \label{fig:muon_FF_mu}
\end{figure}

\begin{figure}[tbp]
  \centering
  \includegraphics[width=0.48\columnwidth]{/Users/sheenaschier/Documents/LaFiles/figures/thesis/fakes/FF_muon/FakeFactor_mu_npvb0}
  \includegraphics[width=0.48\columnwidth]{/Users/sheenaschier/Documents/LaFiles/figures/thesis/fakes/FF_muon/FakeFactor_mu_npvb1}\\
  \caption{Muon fake factors computed from single-muon prescaled triggers as a function of the number of primary vertices in events with exactly zero $b$-jets (left) and one or more $b$-jets (right).  A red line denotes the average muon fake factor over all muon \pt{}}
  \label{fig:muon_FF_npv}
\end{figure}

\begin{figure}[tbp]
  \centering
  \includegraphics[width=0.48\columnwidth]{/Users/sheenaschier/Documents/LaFiles/figures/thesis/fakes/FF_muon/FakeFactor_mu_ptb0_uncert}
  \includegraphics[width=0.48\columnwidth]{/Users/sheenaschier/Documents/LaFiles/figures/thesis/fakes/FF_muon/FakeFactor_mu_ptb1_uncert}\\
  \caption{Relative uncertianties on muon fake factors versus muon \pt{} in zero $b$-jets bin (left) and one or more $b$-jets bin (right).}
  \label{fig:muon_FF_rel_uncert}
\end{figure}

 \FloatBarrier

\section{Same-Sign Validation Region (SS-VR)}
\label{sec:ssvr}
Explain the validation of the fake background estimate in the SS-VR
 
 \section{Conclusion}
 \label{sec:FFcon}
 This chapter went over a lot of material and I think somehow I have to reiterate the important points here..

\part{Analysis and Results}
\chapter{Systematic Uncertianties}
\label{sec:syst}
Systematic uncertainties are split into two categories: experimental and theoretical.  The major sources of experimental uncertainties are the modeling of particle reconstruction in detector simulation, luminosity and pileup measurements, and systematic effects from data-driven estimates.  The main theoretical uncertainties emerge from the modeling of Standard Model background processes.  Simulation of these processes relies on cross-section measurements, parton distribution functions, and renormalization and factorization scale assumptions. Systematic uncertainties propagate to the final expected yields of signal to background, and limit the resolution of predictions. 

This chapter is organized as follows: experimental uncertainties are described in Section~\ref{sec:sys:exp}, where first CP Group uncertainties on measurements of pile-up re-weighting, luminosity, jets, electrons, muons, and missing transverse energy are summarized in Section~\ref{sec:sys:expCP}, and next fake factor uncertainties are described in Section~\ref{sec:sys:expFF}.  Finally, theoretical uncertainties on SM background modeling are dissected in Section~\ref{sec:sys:thy}.

\section{Experimental Uncertainties}
\label{sec:sys:exp}
This chapter will cover uncertainties from CP group recommendations and fake factor measurements.  
\subsection{CP Group Uncertainties}
\label{sec:sys:expCP}
Combined Performance (CP) groups are dedicated teams in ATLAS that work to optimize the characteristic measurements of certain classes of particle.  These groups make recommendations to analysis teams about pile-up re-weighting, luminosity measurements, and which jet, electron, muon, and missing transverse energy definitions to use.  The uncertainties associated with these objects and measurements are discussed in this section.

Multiple pile-up interactions need to be modeled well in Monte Carlo so that the simulated detector response and particle reconstruction conditions match the actual data.  The distribution of the average number of interactions per bunch crossing applied to Monte Caro events, the $\mu$ profile, is based on relevant assumptions and does not always agree with the $\mu$ profile observed in data.  To resolve these disagreements, the $\mu$ profile for Monte Carlo is reweighted to better match the shape in data.  This is typically called pile-up reweighting.  Studies of the data/MC agreement for the number of primary vertices versus $\mu$ suggest an additional rescaling of the $\mu$ distribution in data of $1/1.16$.  A systematic uncertainty for the pile-up reweighting scheme is assigned by varying the scaling factor assigned to data between 1.00 and 1.21 and assessing the change in event yields.  An uncertainty on the luminosity measurement is also examined.  For the 2015+2016 combined datasets, the luminosity uncertainty is observed as $3.2\%$.

Uncertainties on the jet energy scale and jet energy resolution are measured using five parameters varied up and down for the energy uncertainty estimate, and one parameter varied up and down for the uncertainty on the resolution.  A separate uncertainty is assigned to account for the differences in the jet-vertex tagging and b-jet tagging efficiencies between Monte Carlo and data. Uncertainties on the electron energy and momentum scale and resolution are also considered, along with uncertainties on the electron and muon scale factors applied to Monte Carlo events that ensure the simulated reconstruction, identification, isolation, and track-to-vertex association efficiencies match the data.  Furthermore, uncertainties on the missing transverse energy and momentum arise from the propagation of error in the transverse momentum measurements of hard physics objects.  Additional uncertainties on the \met propagate from the scale and resolution of the track-based soft term, described in Chapter~\ref{sec:obj:reco}.  The dominant CP group systematic is from the jet energy scale and resolution.

\subsection{Fake Factor Uncertainties}
\label{sec:sys:expFF}
Fake and non-prompt lepton backgrounds are estimated with a data-driven fake factor method, as described in Chapter~\ref{ch:fakefactor}.  Uncertainties arise from several sources, but are mainly from: kinematic dependencies, non-closure in the same-sign validation region, statistical uncertainties on the applied fake factors, and prompt lepton subtraction using Monte Carlo.

The primary fake factor uncertainty comes from kinematic dependancies on variables that are not included in the fake factor binning.  Fake factors are measured as a function of electron \pt for the electrons, and as a function of muon \pt and $N_{b-jet}$ for the muons.  These choices are motivated by the strong correlation of the fake factors and these variables, but other, smaller kinematic dependencies are present.  The fake factor vulnerabilities are not large enough to consider binning them in every variable, so they are accounted for as a systematic. Figure~\ref{fig:elec_FF_all} presents electron fake factors, and Figures~\ref{fig:muon_FF_hist_eta} -~\ref{fig:muon_FF_npv} present muon fake factors binned in alternative variables.  We consider the largest, statistically meaningful variation of the fake factors binned in the alternative relevant variables and subtract it from the average fake factor for the electron and muon samples separately.  The resulting uncertainty is $25\%$ for each, both driven by the variation in lepton $\eta$.

The relationship between the fake lepton estimate and the data in VR-SS is another source of systematic uncertainty.  This is quantified by comparing data in a version of the VR-SS that does not require an $\met/H_T$ cut in the envelope containing the systematic variations described above.  The root mean square of the variations is compared with the data and the quadrature difference is interpreted as the closure systematic.  This uncertainty is determined to be $38\%$ for electrons with $\pt < 7\GeV$, $97\%$ for muons with \pt 7-10 GeV, and $0\%$ everywhere else. %This $0\%$ is assigned because the fake lepton estimate and the data agree within their uncertainties in VR-SS for the other pT bins considered

Statistical uncertainties on the fake factors are due to the limited size of the samples used to derive them.  These samples use pre-scaled single lepton triggers to select events in data, which are further scrutinized based on the identification, isolation, and impact parameter of the reconstructed leptons to be determined tas either an "ID" or "anti-ID" lepton event.  It is possible that there are overlapping events in these two categories, but it is a rare occurrence since less than $10\%$ of the events have more than one lepton, and both the "ID" and the "anti-ID" leptons would need to fall in the \pt range associated with highest lepton \pt trigger that fired.  Figures~\ref{fig:elec_FF_rel_uncert} and~\ref{fig:muon_FF_rel_uncert} show the relative systematic uncertainties on the electron and muon fake factors per lepton \pt bin.   For electrons, statistical uncertainties range from about $32\%$ in the lowest \pt bin to about $58\%$ in the highest \pt bin.  For muons, the uncertainties on fake factors used to estimate fake backgrounds in the signal regions vary between $12\%$ in the lowest \pt bin to about $32\%$ in the highest \pt bin, and uncertainties on fake factors used to estimate fake backgrounds in the $t\bar{t}$ control region vary between $16\%$ and $38\%$.

Fake factors are measured in regions of data enriched with fake leptons, but prompt lepton contamination is still present.  In the measurement region $m_T<40~\GeV$, prompt lepton events are subtracted from the \pt distributions using SM Monte Carlo that have been rescaled to match data in the high \met region.  To calculate the systematic uncertainty on this method of prompt subtraction, the change in the binned fake factors is studied as three key parameters are varied.  The \met region, where the scale factor for the prompt subtraction is computed, is varied up and down by $20~\GeV$ from the nominal $\met>200~\GeV$ selection, the region where the fake factors are measured is varied up and down by $10~\GeV$ from the nominal $m_T<40~\GeV$ selection, and the scale factor that is applied to the subtracted Monte Carlo is varied up and down by $20\%$.  Uncertainty contributions in the prompt subtraction are assed further by recomputing the Monte Carlo scale factor in the region $m_T>100~\GeV$ and assessing the change in the fake factors.  All together, the resulting uncertainties on both electron and muon scale factors are less that $10\%$, but for one exception in the muon \pt bin above $20~\GeV$, where the uncertainty is $19\%$.  The overall contribution from prompt subtraction is minute compared to the other sources.

\section{Theoretical Uncertainties}
\label{sec:sys:thy}
Theoretical uncertainties from signal and background simulation arise from the uncertainties on the underlying parameters in the Monte Carlo generation.


\subsection{Uncertainty on Simulated Signal Events}
Statistical uncertainties on Higgsino and slepton simulated signal events dominantly arise from the next-to-leading order calculations of the hadronic initial state radiation (ISR), factorization and renormalization scale (FSR), and the underlying event.  ISR/FSR/EU are all around $20\%$.  PDF uncertainties on signal acceptances are also estimated to be around $10\%$.  Uncertainties on signal cross-section are around $5\%$.

\subsection{Uncertainty on Simulated Background Events}
Diboson, $Z(\rightarrow\tau\tau)$+jets, and $t\bar{t}$ are the dominant background processes estimated with Monte Carlo simulation.  There are three main sources of uncertainty: choice of QCD renormalization and factorization scales $\mu_R$ and $\mu_F$, choice of strong coupling constant $\alpha_s$, and choice of PDF set.  To calculate the uncertainties, each of these is varied symmetrically around some parameter, or, in the case of the PDF uncertainty, varied by PDF set.  The effect of the variations on the predicted yield from each of the dominant background processes is evaluated in the signal, control, and validation regions. $\mu_R$ and $\mu_F$ are deviated up and down by a factor of 2 and $\alpha_s$ is varied within its uncertainty of 0.001, and the range of impact on the expected yields are evaluated as the uncertainties.  PDF uncertainties are obtained from the envelope of symmetrized variations within acceptance of the MMHT2014, CT14, NNPDF PDF sets.  Figures~\ref{fig:theoryUncsVV}, \ref{fig:theoryUncsZtt}, and \ref{fig:theoryUncsttbar} show the assortment of event yields in the Higgsino and slepton SRs for the diboson, $Z(\rightarrow\tau\tau)$+jets, and $t\bar{t}$ predictions.  The final uncertainty in each region is calculated as the quadrature sum of all the individual contributions.  

 \begin{figure}
  \centering
  \includegraphics[width=0.4\columnwidth]{/Users/sheenaschier/Documents/LaFiles/figures/thesis/systematics/scaleVars_diboson2L_mll_SR_hg_SFDF_shape.pdf}
 %\caption{$\mu_{F}$ and $\mu_{R}$ uncertainties on the $m_{\ell\ell}$ distribution in the Higgsino signal region.}
  \includegraphics[width=0.4\columnwidth]{/Users/sheenaschier/Documents/LaFiles/figures/thesis/systematics/scaleVars_diboson2L_mt2leplsp_100_SR_sl_SFDF_shape.pdf}
% \caption{$\mu_{F}$ and $\mu_{R}$ uncertainties on the $m_{\text{T}2}$ distribution in the slepton signal region.}
 \includegraphics[width=0.4\columnwidth]{/Users/sheenaschier/Documents/LaFiles/figures/thesis/systematics/alphaVars_diboson2L_mll_SR_hg_SFDF_shape.pdf}
 % \caption{$\alpha_{s}$ uncertainties on the $m_{\ell\ell}$ distribution in the Higgsino signal region.}
 \includegraphics[width=0.4\columnwidth]{/Users/sheenaschier/Documents/LaFiles/figures/thesis/systematics/alphaVars_diboson2L_mt2leplsp_100_SR_sl_SFDF_shape.pdf}
% \caption{$\alpha_{s}$ uncertainties on the $m_{\text{T}2}$ distribution in the slepton signal region.}
  \includegraphics[width=0.4\columnwidth]{/Users/sheenaschier/Documents/LaFiles/figures/thesis/systematics/PDFVars_diboson2L_mll_SR_hg_SFDF_shape_allPDFs.pdf}
 % \caption{PDF uncertainties on the $m_{\ell\ell}$ distribution in the Higgsino signal region.}
  \includegraphics[width=0.4\columnwidth]{/Users/sheenaschier/Documents/LaFiles/figures/thesis/systematics/PDFVars_diboson2L_mt2leplsp_100_SR_sl_SFDF_shape_allPDFs.pdf}
%\caption{PDF uncertainties on the $m_{\text{T}2}$ distribution in the slepton signal region.}
 \caption{QCD scale, $\alpha_{s}$ and PDF uncertainties on the shape and normalization of the diboson background in the Higgsino (left) and slepton (right) signal regions, but with no lepton flavor requirement.}
\label{fig:theoryUncsVV}
 \end{figure}
 
  \begin{figure}
  \centering
   \includegraphics[width=0.4\columnwidth]{/Users/sheenaschier/Documents/LaFiles/figures/thesis/systematics/scaleVars_Zttjets_mll_SR_hg_SFDF_shape.pdf}
 %\caption{$\mu_{F}$ and $\mu_{R}$ uncertainties on the $m_{\ell\ell}$ distribution in the Higgsino signal region.}
  \includegraphics[width=0.4\columnwidth]{/Users/sheenaschier/Documents/LaFiles/figures/thesis/systematics/scaleVars_Zttjets_mt2leplsp_100_SR_sl_SFDF_shape.pdf}
% \caption{$\mu_{F}$ and $\mu_{R}$ uncertainties on the $m_{\text{T}2}$ distribution in the slepton signal region.}
 \includegraphics[width=0.4\columnwidth]{/Users/sheenaschier/Documents/LaFiles/figures/thesis/systematics/alphaVars_Zttjets_mll_SR_hg_SFDF_shape.pdf}
 % \caption{$\alpha_{s}$ uncertainties on the $m_{\ell\ell}$ distribution in the Higgsino signal region.}
 \includegraphics[width=0.4\columnwidth]{/Users/sheenaschier/Documents/LaFiles/figures/thesis/systematics/alphaVars_Zttjets_mt2leplsp_100_SR_sl_SFDF_shape.pdf}
% \caption{$\alpha_{s}$ uncertainties on the $m_{\text{T}2}$ distribution in the slepton signal region.}
  \includegraphics[width=0.4\columnwidth]{/Users/sheenaschier/Documents/LaFiles/figures/thesis/systematics/PDFVars_Zttjets_mll_SR_hg_SFDF_shape_allPDFs.pdf}
 % \caption{PDF uncertainties on the $m_{\ell\ell}$ distribution in the Higgsino signal region.}
  \includegraphics[width=0.4\columnwidth]{/Users/sheenaschier/Documents/LaFiles/figures/thesis/systematics/PDFVars_Zttjets_mt2leplsp_100_SR_sl_SFDF_shape_allPDFs.pdf}
%\caption{PDF uncertainties on the $m_{\text{T}2}$ distribution in the slepton signal region.}
\caption{QCD scale, $\alpha_{s}$ and PDF uncertainties on the shape and normalization of the $Z\to\tau\tau$ background in the Higgsino (left) and slepton (right) signal regions, but with no lepton flavor requirement.}
\label{fig:theoryUncsZtt}
 \end{figure}
 
  \begin{figure}
  \centering 
   \includegraphics[width=0.4\columnwidth]{/Users/sheenaschier/Documents/LaFiles/figures/thesis/systematics/scaleVars_alt_ttbar_PowPy8_dilep_hdamp258p75_mll_SR_hg_SFDF_shape.pdf}
 %\caption{$\mu_{F}$ and $\mu_{R}$ uncertainties on the $m_{\ell\ell}$ distribution in the Higgsino signal region.}
  \includegraphics[width=0.4\columnwidth]{/Users/sheenaschier/Documents/LaFiles/figures/thesis/systematics/scaleVars_alt_ttbar_PowPy8_dilep_hdamp258p75_mt2leplsp_100_SR_sl_SFDF_shape.pdf}
% \caption{$\mu_{F}$ and $\mu_{R}$ uncertainties on the $m_{\text{T}2}$ distribution in the slepton signal region.}
  \includegraphics[width=0.4\columnwidth]{/Users/sheenaschier/Documents/LaFiles/figures/thesis/systematics/PDFVars_alt_ttbar_PowPy8_dilep_hdamp258p75_mll_SR_hg_SFDF_shape_allPDFs.pdf}
 % \caption{PDF uncertainties on the $m_{\ell\ell}$ distribution in the Higgsino signal region.}
  \includegraphics[width=0.4\columnwidth]{/Users/sheenaschier/Documents/LaFiles/figures/thesis/systematics/PDFVars_alt_ttbar_PowPy8_dilep_hdamp258p75_mt2leplsp_100_SR_sl_SFDF_shape_allPDFs.pdf}
%\caption{PDF uncertainties on the $m_{\text{T}2}$ distribution in the slepton signal region.}
\caption{QCD scale and PDF uncertainties on the shape and normalization of the $t\bar{t}$ background in the Higgsino (left) and slepton (right) signal regions, but with no lepton flavor requirement.}
\label{fig:theoryUncsttbar}
 \end{figure}

\chapter{Statistical Analysis}
Explain all the things I don't yet understand about likelihood functions and such....
\chapter{Results}
\label{ch:results}
\section{Background Only Fit}
In the background only fit, only the CRs are used to constrain the fit parameters by maximizing the likelihood function assuming there are no signal events in the CRs.  In this way, the SM background predictions are independent of the signal regions.  The factors $\mu_{top}$ and $\mu_{\tau\tau}$, used to normalize of the combined $t$, $tW$, and  $t\bar{t}$ samples and the  $Z(\rightarrow\tau\tau)$+jets samples, are obtained in a simultaneous fit to data in CR-top and CR-tau.  For exclusion, two simultaneous shape fits are performed across $ee$ and $\mu\mu$ channels, one in the $m_{\ell\ell}$ variable, and the other in the $m_{T2}^{100}$ variable.   The normalization parameters $\mu_{top}$ and $\mu_{\tau\tau}$ for the background only fit are $\mu_{top} = 0.72\pm0.13$ and $\mu_{\tau\tau} = 1.02\pm0.09$, where the uncertainty is the combination of the statistical and systematic contributions.

Data and background prediction are shown for the diboson, same-sign, and different-flavor validation regions are shown in Figure~\ref{fig:pull_plot_summary_yields}.  The accuracy of the background prediction is tested in each of the validation regions and is consistently within 1.5 standard deviations of the observed data yields.  Figure~\ref{fig:postfitplots} shows distributions of the data and expected backgrounds for a selection of VRs and kinematic variables, including the $m_{\ell\ell}$ distribution in VR-VV and the $m_{T2}$ distribution in VR-SS.  Similar levels of agreement are observed in other kinematic distributions for VR-SS and VR-VV.  Data and background predictions are compatible within uncertainties.  Figure~\ref{fig:SRpostfitplots} shows kinematic distributions of data and expected backgrounds in the inclusive Higgsino and slepton signal regions.  No significant excesses above expected backgrounds are observed.


\begin{figure}
 \centering
\includegraphics[width=0.9\columnwidth]{/Users/sheenaschier/Documents/LaFiles/figures/thesis/results/histpull_HiggsinoFit_doCRonly_VRs.pdf}
   \caption{Summary of Monte Carlo yields in control, validation and signal regions in a background-only fit using data only in the two CRs to constrain the fit.}
  \label{fig:pull_plot_summary_yields}
 \end{figure}

 \begin{figure}%[h!]
  \begin{center}
  \includegraphics[width=0.49\textwidth]{/Users/sheenaschier/Documents/LaFiles/figures/thesis/results/Higgsino_bkg_VRDF_iMT2f_VRDF_iMT2f_lep2Pt.pdf}
  \includegraphics[width=0.49\textwidth]{/Users/sheenaschier/Documents/LaFiles/figures/thesis/results/Higgsino_bkg_VR_VV_VR_VV_mt2leplsp_100.pdf}
   \includegraphics[width=0.49\textwidth]{/Users/sheenaschier/Documents/LaFiles/figures/thesis/results/Higgsino_bkg_VR_SS_AF_VR_SS_AF_lep2Pt.pdf}
   \includegraphics[width=0.49\textwidth]{/Users/sheenaschier/Documents/LaFiles/figures/thesis/results/Higgsino_bkg_VR_SS_AF_VR_SS_AF_mll.pdf}
   \end{center}
 \caption{Kinematic distributions of data and expected backgrounds after the background-only fit.  Top left plot shows the sub-leading lepton \pt distribution in the different-flavor validation region VRDF-$m_\text{T2}^{100}$; the top right plot shows the $m_\text{T2}^{100}$ distribution in the diboson validation region VR-VV (top right); the sub-leading lepton \pt  distribution in the bottom right plot and the $m_{\ell\ell}$ distribution in the bottom left are shown in the same-sign validation region VR-SS inclusive of lepton flavor.  Background processes containing fewer than two prompt leptons are categorized as `Fake/nonprompt'.  The category `Others' contains rare backgrounds from triboson, Higgs boson, and multi-top processes.  The last bin includes overflow.}
 \label{fig:postfitplots}
 \end{figure}
 
 \begin{figure}%[tp]
  \begin{center}
   \includegraphics[width=0.49\textwidth]{/Users/sheenaschier/Documents/LaFiles/figures/thesis/results/Higgsino_bkg_SRSF_iMLLg_SRSF_iMLLg_METOverHTLep_METOverHTLep.pdf}
   \includegraphics[width=0.49\textwidth]{/Users/sheenaschier/Documents/LaFiles/figures/thesis/results/Higgsino_bkg_SRSF_iMLLg_SRSF_iMLLg_mll.pdf}
   \includegraphics[width=0.49\textwidth]{/Users/sheenaschier/Documents/LaFiles/figures/thesis/results/Higgsino_bkg_SRSF_iMT2f_SRSF_iMT2f_METOverHTLep_METOverHTLep.pdf}
   \includegraphics[width=0.49\textwidth]{/Users/sheenaschier/Documents/LaFiles/figures/thesis/results/Higgsino_bkg_SRSF_iMT2f_SRSF_iMT2f_mt2leplsp_100.pdf}
   \end{center}
%   \caption{Kinematic distributions after the background-only fit}
 \caption{Kinematic distributions after the background-only fit showing the data as well as the expected background in the most inclusive electroweakino SR$\ell\ell$-$m_{\ell\ell}$~$[1, 60]$ (top) and slepton $m_\text{T2}^{100}$~$[100, \infty]$ (bottom) signal regions. The arrow in the $\met/H_{T}^{lep}$ variables indicates the minimum value of the requirement imposed in the final SR selection. The $m_{\ell\ell}$ and $m_\text{T2}^{100}$ distributions (right) have all the SR requirements applied. Background processes containing fewer than two prompt leptons are categorized as `Fake/nonprompt'. The category `Others' contains rare backgrounds from triboson, Higgs boson, and multi-top processes.  The last bin includes overflow. The dashed lines represent benchmark signal samples corresponding to the Higgsino $\widetilde{H}$ and slepton $\tilde\ell$ simplified models. Orange arrows in the Data/SM panel indicate values that are beyond the y-axis range.}
  \label{fig:SRpostfitplots}
 \end{figure}


\section{Model Independent Upper Limits on New Physics}
 {\renewcommand{\arraystretch}{1.3}
\input{/Users/sheenaschier/Documents/LaFiles/figures/thesis/results/Merged_SimpleDiscoveryUpperLimits_unrounded.tex}
Model independent limits are useful so that, for any signal model of interest, one can evaluate the number of events predicted in a signal region and check if the model is excluded by current measurements.  For this, single-binned inclusive SRs are used, since binning in the SRs requires some model-based assumptions about the distribution of the signal over these bins.  Table~\ref{table.results.exclxsec.pval.upperlimit.SRSF_iMLLa} present the observed and expected event yields, the upper limits on the number of observed and expected signal events, and the visible cross-section for new physics in each of the inclusive Higgsino SR$\ell\ell$-$m_{\ell\ell}$ and slepton $m_\text{T2}^{100}$ signal regions.  An upper limit on the number of observed ($S^{95}_{obs}$) and expected ($S^{95}_{exp}$) signal events in each SR at $95\%$ CL is procured in the same way as the background only fit, but now using CRs and SRs and with the observed number of events in a signal region given as inputs to the fit.  The observed ($N_{\mathrm{obs}}$) and predicted ($N_{\mathrm{exp}}$) event yields are used to set the upper limits by including one inclusive signal region at a time in a simultaneous fit with the CRs.  The profile-likelihood hypothesis test performed to get the upper limits uses the background estimates obtained from the background only test in the CRs and SRs, and both the expected and observed upper limits use the same background estimates.  
An upper limit on the visible cross-section for new physics in a given SR, $\langle\epsilon\mathrm{\sigma}\rangle_\text{obs}^{95}$ [fb], is equal to product of the signal region acceptance, the reconstruction efficiency, and the production cross-section.  The discovery p-value, p(s=0) in the right most column of the table, represents the significance of an excess of events in a signal region by considering the probability that the backgrounds in a SR are more signal-like than observed. 

\section{Model Dependent Sensitivity with Shape Fit}
{\renewcommand{\arraystretch}{1.3}
\input{/Users/sheenaschier/Documents/LaFiles/figures/thesis/results/MyYieldsTable_exclSRs.tex}
Here we assume the Higgsino and slepton signals give rise to the $m_{\ell\ell}$ and $M_{T2}$ distributions in our signal regions.  This consideration provides better constraining power for these models over the model independent upper limits of the 'Discovery' fit.  Like in the model independent case, the fit is performed on the CRs and SRs simultaneously, but different from the model independent case, the multi-binned exclusive SRs and considered.  Background and signal samples are included in both the CR and SR fits to account for any signal contamination in the CRs.  

Table~\ref{tab:results:exclusiveSRYields} summarizes the observed event yields in the exclusive electroweakinio signal regions, and Table~\ref{tab:results:exclusiveSRYields2} summarizes the observed event yields in the exclusive slepton signal regions after the fit is performed using an exclusion fit configuration where the signal strength parameter is set to zero.  Extending the background only fit to include the signal regions further constrains the background contributions in the absence of any signal, therefore these predicted yields differ slightly compared to those obtained with the background only fit.  Figure~\ref{fig:pull_plot_summary_yields:exclSRs} demonstrates the harmony between the fitted and observed yields in these signal regions.  No significant contrast between the fitted background estimates and the observed event yields are observed in any of the exclusive signal regions.

\begin{figure}
\centering
 \includegraphics[width=\textwidth]{/Users/sheenaschier/Documents/LaFiles/figures/thesis/results/histpull_HiggsinoFit_doCRplusSRMLLandMT2_exclSR.pdf}
 \caption{Comparison of observed and expected event yields after the
exclusion fit.
Background processes containing fewer than two prompt leptons are categorized as `Fake/nonprompt'.
The category `Others' contains rare backgrounds from triboson, Higgs boson, and multi-top processes. Uncertainties in the background estimates include
     both the statistical and systematic uncertainties, where $\sigma_\text{tot}$ denotes the total uncertainty.}
     \label{fig:pull_plot_summary_yields:exclSRs}
 \end{figure}
\FloatBarrier







 

\chapter{Interpretations}
\label{ch:interpretations}

In absence of any significant excesses over backgrounds, the results are interpreted as constraints on the SUSY models presented in Chapter~\ref{ch:thy} using the exclusive, multi-binned Higgsino and slepton signal regions.  The background only fit is extended to allow for a signal model with a corresponding signal strength parameter in a simultaneous fit of all CRs and relevant SRs, this is referred to as the exclusion fit.  In the previous chapter, background-level estimates obtained from a background-only fit in the CRs only were presented.  When electroweakino simplified models are assumed, the results are interpreted in the 14 exclusive Higgsino signal regions, binned in $m_{\ell\ell}$ and split evenly between the $ee$ and $\mu\mu$ channels.  By statistically combining these signal regions, the signal shape of the $m_{\ell\ell}$ spectrum can be exploited to improve the sensitivity. When slepton simplified models are assumed, the results are interpreted in 12 slepton signal regions, binned in $m_{T2^{100}}$ with 6 SRs the $ee$-channel and 6 in the $\mu\mu$ channel are used for the fit.

\section{Compressed Higgsino}
Hypothesis tests are performed to set limits on simplified model scenarios using the $CL_s$ prescription.  Figure~\ref{fig:exclusion_contour_higgsino} shows the $95\%$ confidence interval limits set on the Higgsino simplified model projected onto the plane defined by the mass difference between the lightest and next-to-lightest neutralino as a function of the next-to-lightest neutralino mass.  These limits are based on an exclusion fit that exploits the shape of the dilepton invariant mass spectrum from the exclusive electroweakino signal regions and exclude next-to-lightest neutralino masses up to $130~\GeV$ for mass splittings between $5$ and $10~\GeV$.  For mass splittings down to $3~\GeV$ next-to-lightest neutralino masses are excluded up to $100~\GeV$. 

 \begin{figure}
 \centering
 \includegraphics[width=0.95\columnwidth]{/Users/sheenaschier/Documents/LaFiles/figures/thesis/results/exclusion_contour_higgsino.pdf}
  \caption{
 Expected 95\% CL exclusion sensitivity (blue dashed line) with $\pm 1 \sigma_\text{exp}$ (yellow band) from experimental systematics
   and observed limits (red solid) with $\pm 1 \sigma_\text{theory}$ (dotted red) from signal cross section uncertainties.
A shape fit of Higgsino signals to the $m_{\ell\ell}$ spectrum is used to derive
 the limit is displayed in the $m(\tilde{\chi}^0_2) - m(\tilde{\chi}^0_1)$ vs $m(\tilde{\chi}^0_2)$ plane.
 The chargino $\tilde{\chi}^\pm_1$ mass is assumed to be half way between the two lightest neutralinos.
  The grey region denotes the lower chargino mass limit from LEP~\cite{LEPlimits}.}
   \label{fig:exclusion_contour_higgsino}
 \end{figure}
% \FloatBarrier
 
 \section{Compressed Wino}
 The $95\%$ confidence level intervals for the wino-bino simplified model are shown in Figure~\ref{fig:exclusion_contour_wino}.  Just like in the Higgsino exclusion plot, these limits are based on an exclusion fit that exploits the shape of the dilepton invariant mass spectrum from the exclusive electroweakino signal regions.  Exclusion limits are projected onto the mass difference $\Delta m(\tilde{\chi}^0_2, \tilde{\chi}^0_1)$ plane as a function of the $\tilde{\chi}^0_2$ mass.  For wino-bino simplified models, next-to-lightest neutralino masses are excluded up to $170~\GeV$ for mass splittings above $10~\GeV$, and excluded up to $100~\GeV$ for mass splittings down to $2.5~\GeV$. 
   \begin{figure}
 \centering
 \includegraphics[width=0.95\columnwidth]{/Users/sheenaschier/Documents/LaFiles/figures/thesis/results/exclusion_contour_wino.pdf}
  \caption{Expected 95\% CL exclusion sensitivity (blue dashed line) with $\pm1\sigma$ exp (yellow band) from experimental systematic uncertainties and observed limits (red solid line) with pm1?theory (dotted red line) from signal cross-section uncertainties for simplified models direct wino production. 
  A shape fit of wino signals to the $m_{\ell\ell}$ spectrum is used to derive
 the limit is displayed in the $m(\tilde{\chi}^0_2) - m(\tilde{\chi}^0_1)$ vs $m(\tilde{\chi}^0_2)$ plane.
 The chargino $\tilde{\chi}^\pm_1$ mass is assumed equal to the $m(\tilde{\chi}^0_2)$ mass.
  The grey region denotes the lower chargino mass limit from LEP~\cite{LEPlimits}, and the blue region in the lower plot indicates the limit from the 2$\ell$+3$\ell$ combination of ATLAS Run 1.} 
     \label{fig:exclusion_contour_wino}
 \end{figure}
 
  \section{Compressed Slepton}
 Figure~\ref{fig:exclusion_contour_slepton} shows the $95\%$ confidence interval limits set on the slepton simplified model projected onto the plane defined by the mass difference between the slepton and lightest neutralino as a function of the slepton mass.  These limits are based on an exclusion fit that exploits the shape of the $m_{T2}$ spectrum from the exclusive slepton signal regions and exclude slepton masses up to $180~\GeV$ for mass splittings down to $5~\GeV$.  For mass splittings down to $1~\GeV$ slepton masses are excluded up to $70~\GeV$.  In slepton simplified models, a fourfold degeneracy is assumed between the left and right-handed selectrons and smuons: $\tilde{e}_R=\tilde{e}_L=\tilde{\mu}_R=\tilde{\mu}_L$.
  \begin{figure}
 \centering
 \includegraphics[width=0.95\columnwidth]{/Users/sheenaschier/Documents/LaFiles/figures/thesis/results/exclusion_contour_slepton.pdf}
  \caption{
Expected 95\% CL exclusion sensitivity (blue dashed line) with $\pm 1 \sigma_\text{exp}$ (yellow band) from experimental systematics
and observed limits (red solid) with $\pm 1 \sigma_\text{theory}$ (dotted red) from signal cross section uncertainties.
A shape fit of slepton signals to the $m_\text{T2}^{100}$ spectrum is used to derive
the limit projected into the $m(\tilde{\ell}) - m(\tilde{\chi}^0_1)$ vs $m(\tilde{\ell})$ plane.
The slepton $\tilde{\ell}$ refers to a 4-fold mass degenerate system of left- and right-handed selectron and smuon.
The grey region denotes a conservative right-handed smuon $\tilde{\mu}_R$ mass limit from LEP~\cite{LEPlimits},
while the blue region is the 4-fold mass degenerate slepton limit from ATLAS Run 1~\cite{SUSY-2013-11}.}
   \label{fig:exclusion_contour_slepton}
 \end{figure}
 % \FloatBarrier
% \subsection{NUHM2}
 

\chapter{Conclusion}
%\label{sec:conclusion}
A search for supersymmetry in scenarios with compressed mass spectra was performed using ATLAS data collected in 2015 and 2016 at $\sqrt(s)$ 13 TeV, corresponding to $36.1 fb^{-1}$.  We searched for directly produced electroweakinos and sleptons in events containing two soft, oppositely signed and same flavored leptons and including missing transverse momentum energy recoiling against initial state hadronic radiation.  The directly produced electroweakinos and sleptons subsequently decay to their Standard Model partners and the lightest SUSY particle which is nearly degenerate in mass.  No significant excess in data over Standard Model background was found; therefore, results were consistent with Standard Model prediction. 


%\fi

\nocite{*}
\bibliographystyle{plain}
\bibliography{uctest.bib}
%\printbibliography
\iffalse
\appendix
\chapter{Appendix A}
Auxiliary fake factor materials

\section{Fake lepton composition study}

\begin{figure}[htb]
        \centering
        \includegraphics[width=.4\textwidth]{/Users/sheenaschier/Documents/LaFiles/figures/thesis/fakes/fakeLeptonComposition/725_cds_noIso_el_SR2_lep2Pt.pdf}
        \includegraphics[width=.4\textwidth]{/Users/sheenaschier/Documents/LaFiles/figures/thesis/fakes/fakeLeptonComposition/725_cdsComments_el_QCR2_lep2Pt.pdf}
        \includegraphics[width=.4\textwidth]{/Users/sheenaschier/Documents/LaFiles/figures/thesis/fakes/fakeLeptonComposition/725_cds_ss_el_SR2_lep2Pt.pdf}
                \includegraphics[width=.4\textwidth]{/Users/sheenaschier/Documents/LaFiles/figures/thesis/fakes/fakeLeptonComposition/725_cds_ss_wIso_el_QCR2_lep2Pt.pdf}
        \caption{Fake lepton composition in opposite sign signal and control region as a function of leading and subleading lepton $p_{T}$, with and without prompt (``Isolated'' plus ``lep$\to$gamma$\to$lep'') leptons, for opposite sign electron pairs in the signal region.}
        \label{fig:el}
\end{figure}

\begin{figure}[htb]
        \centering
        \includegraphics[width=.45\textwidth]{/Users/sheenaschier/Documents/LaFiles/figures/thesis/fakes/fakeLeptonComposition/725_cdsComments_el_SR1_lep1Pt.pdf}
        \includegraphics[width=.45\textwidth]{/Users/sheenaschier/Documents/LaFiles/figures/thesis/fakes/fakeLeptonComposition/725_cds_noIso_el_SR1_lep1Pt.pdf}
        \includegraphics[width=.45\textwidth]{/Users/sheenaschier/Documents/LaFiles/figures/thesis/fakes/fakeLeptonComposition/725_cdsComments_el_QCR1_lep1Pt.pdf}
          \includegraphics[width=.45\textwidth]{/Users/sheenaschier/Documents/LaFiles/figures/thesis/fakes/fakeLeptonComposition/725_cds_noIso_el_QCR1_lep1Pt.pdf}
        \caption{Fake lepton composition as a function of leading and subleading lepton $p_{T}$, with and without prompt (``Isolated'' plus ``lep$\to$gamma$\to$lep'') leptons, for opposite sign electron pairs in the signal region.}
        \label{fig:elSR}
\end{figure}

  
\begin{figure}[htb]
        \centering
        \includegraphics[width=.45\textwidth]{/Users/sheenaschier/Documents/LaFiles/figures/thesis/fakes/fakeLeptonComposition/725_cds_ss_wIso_el_SR1_lep1Pt.pdf}
        \includegraphics[width=.45\textwidth]{/Users/sheenaschier/Documents/LaFiles/figures/thesis/fakes/fakeLeptonComposition/725_cds_ss_el_SR1_lep1Pt.pdf}
                \includegraphics[width=.45\textwidth]{/Users/sheenaschier/Documents/LaFiles/figures/thesis/fakes/fakeLeptonComposition/725_cds_ss_wIso_el_QCR1_lep1Pt.pdf}
                       \includegraphics[width=.45\textwidth]{/Users/sheenaschier/Documents/LaFiles/figures/thesis/fakes/fakeLeptonComposition/725_cds_ss_el_QCR1_lep1Pt.pdf}
        \caption{Fake lepton composition as a function of leading and subleading lepton $p_{T}$, with and without prompt (``Isolated'' plus ``lep$\to$gamma$\to$lep'') leptons, for same sign electron pairs in the signal region.}
        \label{fig:elSSSR}
\end{figure}




\begin{figure}[htb]
        \centering
        \includegraphics[width=.45\textwidth]{/Users/sheenaschier/Documents/LaFiles/figures/thesis/fakes/fakeLeptonComposition/626_cdsComments_mu_SR1_lep1Pt.pdf}
        \includegraphics[width=.45\textwidth]{/Users/sheenaschier/Documents/LaFiles/figures/thesis/fakes/fakeLeptonComposition/626_cdsComments_mu_SR2_lep2Pt.pdf}
        \includegraphics[width=.45\textwidth]{/Users/sheenaschier/Documents/LaFiles/figures/thesis/fakes/fakeLeptonComposition/626_cds_noIso_mu_SR1_lep1Pt.pdf}
        \includegraphics[width=.45\textwidth]{/Users/sheenaschier/Documents/LaFiles/figures/thesis/fakes/fakeLeptonComposition/626_cds_noIso_mu_SR2_lep2Pt.pdf}
        \caption{Fake lepton composition as a function of leading and subleading lepton $p_{T}$, with and without prompt (``Isolated'' plus ``lep$\to$gamma$\to$lep'') leptons, for opposite sign muon pairs in the signal region.}
        \label{fig:muSR}
\end{figure}

\begin{figure}[htb]
        \centering
        \includegraphics[width=.45\textwidth]{/Users/sheenaschier/Documents/LaFiles/figures/thesis/fakes/fakeLeptonComposition/626_cdsComments_mu_QCR1_lep1Pt.pdf}
        \includegraphics[width=.45\textwidth]{/Users/sheenaschier/Documents/LaFiles/figures/thesis/fakes/fakeLeptonComposition/626_cdsComments_mu_QCR2_lep2Pt.pdf}
        \includegraphics[width=.45\textwidth]{/Users/sheenaschier/Documents/LaFiles/figures/thesis/fakes/fakeLeptonComposition/626_cds_noIso_mu_QCR1_lep1Pt.pdf}
        \includegraphics[width=.45\textwidth]{/Users/sheenaschier/Documents/LaFiles/figures/thesis/fakes/fakeLeptonComposition/626_cds_noIso_mu_QCR2_lep2Pt.pdf}
        \caption{Fake lepton composition as a function of leading and subleading lepton $p_{T}$, with and without prompt (``Isolated'' plus ``lep$\to$gamma$\to$lep'') leptons, for opposite sign muon pairs in the fake lepton control region.}
        \label{fig:muCR}
\end{figure}
\begin{figure}[htb]
        \centering
        \includegraphics[width=.45\textwidth]{/Users/sheenaschier/Documents/LaFiles/figures/thesis/fakes/fakeLeptonComposition/626_cds_ss_wIso_mu_SR1_lep1Pt.pdf}
        \includegraphics[width=.45\textwidth]{/Users/sheenaschier/Documents/LaFiles/figures/thesis/fakes/fakeLeptonComposition/626_cds_ss_wIso_mu_SR2_lep2Pt.pdf}
        \includegraphics[width=.45\textwidth]{/Users/sheenaschier/Documents/LaFiles/figures/thesis/fakes/fakeLeptonComposition/626_cds_ss_mu_SR1_lep1Pt.pdf}
        \includegraphics[width=.45\textwidth]{/Users/sheenaschier/Documents/LaFiles/figures/thesis/fakes/fakeLeptonComposition/626_cds_ss_mu_SR2_lep2Pt.pdf}
        \caption{Fake lepton composition as a function of leading and subleading lepton $p_{T}$, with and without prompt (``Isolated'' plus ``lep$\to$gamma$\to$lep'') leptons, for same sign muon pairs in the signal region.}
        \label{fig:muSSSR}
\end{figure}
\begin{figure}[htb]
        \centering
        \includegraphics[width=.45\textwidth]{/Users/sheenaschier/Documents/LaFiles/figures/thesis/fakes/fakeLeptonComposition/626_cds_ss_wIso_mu_QCR1_lep1Pt.pdf}
        \includegraphics[width=.45\textwidth]{/Users/sheenaschier/Documents/LaFiles/figures/thesis/fakes/fakeLeptonComposition/626_cds_ss_wIso_mu_QCR2_lep2Pt.pdf}
        \includegraphics[width=.45\textwidth]{/Users/sheenaschier/Documents/LaFiles/figures/thesis/fakes/fakeLeptonComposition/626_cds_ss_mu_QCR1_lep1Pt.pdf}
        \includegraphics[width=.45\textwidth]{/Users/sheenaschier/Documents/LaFiles/figures/thesis/fakes/fakeLeptonComposition/626_cds_ss_mu_QCR2_lep2Pt.pdf}
        \caption{Fake lepton composition as a function of leading and subleading lepton $p_{T}$, with and without prompt (``Isolated'' plus ``lep$\to$gamma$\to$lep'') leptons, for same sign muon pairs in the fake lepton control region.}
        \label{fig:muSSCR}

\end{figure}

\chapter{Appendix B}
Auxiliary CR and VR material

\section{Control Region Plots}


 \input{/Users/sheenaschier/Documents/LaFiles/figures/thesis/control_plots}
\include{apndx-C}
Ancillary material should be put in appendices, which appear after the
bibliography. 
\fi
\end{document}
