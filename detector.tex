\chapter{ATLAS Experiment}
%\label{sec:detector}
The Large Hadron Collider (LHC) is a 27 km long circular proton accelerator with proton beams in moving in opposite directions around the ring at speeds near 99.99\% the speed of light.  The beams travel around the ring in separate vacuum beam pipes and are accelerated and directed around the ring using gigantic semi-conducting magnets.  To reach LHC energies the proton beams are accelerated in smaller accelerator structures gradually increasing in size until they are injected into the LHC, which is still the largest and most powerful accelerator in the world.  The beams are made to collide at 4 different interaction points at which there are 4 different detector experiments: ALICE, LHC-B, CMS, and ATLAS.  

%%%%%%%%%%%%%%%%%%%%%%
\section{Tracking}
\subsection{Pixel Detector}
Inner most pixelated tracker.
\subsection{Semi-Conductor Tracker}
Middle silicon strip tracker.
\subsection{Transition Radiation Tracker}
Outer most straw tube tracker.

%%%%%%%%%%%%%%%%%%%%%%
\section{Calorimetry}
\subsection{Electromagnetic Calorimeter}
Measures energy of electromagnetic objects.
\subsection{Hadronic Calorimeter}
Measures energy of hadronic objects

%%%%%%%%%%%%%%%%%%%%%%
\section{Muon System}
An outer tracker dedicated entirely to tracking muons.

%%%%%%%%%%%%%%%%%%%%%%
\section{DAQ and Trigger}
Complex computing system to acquire and store data.

