\chapter{The LHC and The ATLAS Experiment}
\label{ch:detector}
This chapter gives an overview of the LHC and the ATLAS detector used for this physics analysis.  First, the LHC in introduced in Section~\ref{sec:LHC}, then a review of the ATLAS detector in Section \ref{sec:ATLAS}.  This section is broken into smaller pieces that detail the ATLAS subdetectors and trigger system. %data trigger and acquisition.
\section{The Large Hadron Collider Machine}
\label{sec:LHC}

The Large Hadron Collider (LHC) is a circular proton accelerator and collider at CERN, operating in the 26.7 km long tunnel that was originally built for the CERN LEP machine.  In the tunnel, there are two separate vacuum beam pipes with counter-rotating proton beams that are accelerated to the TeV energy scale by a gigantic semi-conducting magnet system.  To reach LHC energies, the proton beams first move through a stream of smaller accelerator structures that increase the kinetic energy of the beam at each step, until the beam is finally injected into the LHC, which is still, at the completion of this thesis, the largest and most powerful accelerator in the world.  There are two transfer tunnels, each about 2.5 km long, that join the LHC to the CERN accelerator complex, now acting as the injector for the LHC.  The LHC tunnel is broken into octets with eight straight sides and eight curves.  This is not an LHC design, but rather an artifact of LEP.   That being said, each octet is considered as a reference point around the ring; for instance, octet 1 is "point 1", octet 2 is "point 2" and so on.  The beams collide at four interaction points located approximately 100m underground, and surrounding each interaction point is a physics detector apparatus to collect data from the proton collisions.  The four different detector experiments are ALICE, LHC-B, CMS, and ATLAS.  Figure~\ref{fig:lhc} depicts the tunnel octets and the beam injection and dump points.  It also shows the placement of the four detectors; ATLAS is located at point 1.
  \begin{figure}[tbp]
    \centering
% http://cdsweb.cern.ch/record/1095926
 \includegraphics[width=0.7\columnwidth]{/Users/sheenaschier/Documents/LaFiles/figures/thesis/detector/lhc-schematic.jpg}
    \caption{Schematic of the LHC layout}
   \label{fig:lhc}
 \end{figure}
 
The primary objective of the LHC is find the Higgs boson, which was discovered by both ATLAS and CMS in 2012, and to expose Beyond Standard Model (BSM) physics.  To attempt these goals, the accelerator was designed to supply proton collisions with enough center of mass energy to produce a Higgs with mass above $100~\GeV$ and to unlock possible new physics interactions at the $100~\GeV - 1~\TeV$ scale.  The initial aim was a proton-proton center of mass energy of 14 TeV, but due to instabilities in the magnet system at such high energy, only 13 TeV has successfully been achieved.  Many BSM theories predict new particle interactions with weak-scale cross-sections or lower, creating the need enough luminosity to measure these low probability events.  The machine luminosity depends only on beam parameters, as expressed in Eq~\ref{eq:lumi}.

\begin{equation}
L=\frac{N_b^2n_bf_{rev}\gamma_r}{4\pi\epsilon_n\beta\star}F
\label{eq:lumi}
\end{equation}
In the numerator of Eq~\ref{eq:lumi}, $N_b$ is the number of particles per bunch, $n_b$ is the number of bunches per beam, $f_{rev}$ is the revolution frequency, and $\gamma_r$ is the relativistic gamma factor of the highly relativistic beam particles traveling near the speed of light.  In the denominator of Eq~\ref{eq:lumi}, $\epsilon_n$ is the normalized transverse beam emittance and $\beta\star$ in the beta function at the collision point.  $F$ is the geometric luminosity reduction factor due to the beams crossing at an angle at the interaction points rather than directly head-on:
\begin{equation}
F=(1+(\frac{\Theta_c\sigma_z}{2\sigma^{\star}})^2)^{-1/2}
\label{eq:reduction}
\end{equation}
$\Theta_c$ is the full crossing angle at the interaction point, $sigma_z$ is the RMS bunch length, and $\sigma^{\star}$ is the transverse RMS beam size at the interaction point.  ATLAS, one of the high luminosity experiments at the LHC, achieved a peak luminosity in 2016 above $L=10^{34}cm^2s^1$.

%\textcolor{red}{Say more about the achieved luminosity and all that}
 
%TODO: Paragraph about magnet system	

The general design for detectors at the LHC is informed by the benchmark physics goals and the experimental environment and constraints.  The high energy and luminosity demands make radiation-hard sensor elements and read-out electronics a necessity.  Large numbers of interactions per bunch crossing, called pileup, creates the need for highly granular detectors to resolve the separate events in space.  To search for new physics, a detector needs to be as general as possible, meaning it tries to see everything.  This requires a high acceptance in pseudorapidity with coverage over nearly the full azimuthal angle of the detector, high track reconstruction efficiency and good resolution on charged-particle momentum measurements.  Fairly precise electromagnetic calorimetry is also needed for efficient electron and photon identification.  Now that we understand these demands, we turn to a description of the ATLAS detector 

 
\section{The ATLAS Experiment}
\label{sec:ATLAS}
The ATLAS experiment is a general purpose detector apparatus that almost completely covers the entire solid angle around one of the LHC beam collision points.  ATLAS recorded its first LHC $pp$ collisions in 2009 at center of mass energy $7~TeV$, and has since recorded events at several different center of mass energies, including the most extensive energy reach in the history at $13~TeV$.  %(http://inspirehep.net/record/1240374/files/CHARGED%202012_011.pdf). 
ATLAS achieves central coverage in the symmetric cylindrical barrel, and forward-backward detecting capabilities in the end-caps.  The complete detector system is 44m long, 25m in diameter, and weighs 4000 tons.  The ATLAS detector, shown in Figure~\ref{fig:ATLAS}, is comprised of several sub-detector systems, each calibrated and optimized for a different observational purpose.  The sub-detectors, listed in order from the beam pipe outward, are: the inner tracking detector (ID), the electromagnetic calorimeter (eCAL), the hadronic calorimeter (hCAL), and the muon spectrometer (MS).  Together, these sub-detectors measure the energy and momentum of a variety of particles and reconstruct the dynamics of each recorded event.  

ATLAS uses a right-handed coordinate system with the center of the detector as the origin.  The z-axis runs through the center of the barrel along the beam pipe, and the y-axis points upward through the barrel from the origin.  The x-axis points outward from the origin, perpendicular to both the y- and z-axes.  Cylindrical coordinates ($r,\phi$) map out the transverse plane, where $r$ is the radius in the plane, and $\phi$ is the azimuthal angle around the z-axis.  The pseudorapidity $\eta$, given by Eq~\ref{eq:eta} is a transformation of the polar angle that is commonly used in particle detector experiments.  At $\theta=\pi/2$, $\eta=0$; at $\theta=\pi/18$, $\eta=2.88$; as $\theta$ approaches zero, $\eta$ approaches infinity.
\begin{equation}
\eta=-ln[tan(\theta/2)]
\label{eq:eta}
\end{equation}
The combination of the detector systems provide charged particle measurements and efficient lepton and photon measurements out to $|\eta| < 2.4$.  Jets and MET are reconstructed using the full set of information out to $|\eta| < 4.9$.    \begin{figure}[tbp]
  \centering
 \includegraphics[width=0.6\columnwidth]{/Users/sheenaschier/Documents/LaFiles/figures/thesis/detector/ATLAS.pdf}
    \caption{Cut-away view of the complete ALTAS Detector}
   \label{fig:ATLAS}
 \end{figure}
  %Transverse energy and momenta are defined as $\pt{}=psin\theta$ and $E_T=Esin\theta$.}  
%%%%%%%%%%%%%%%%%%%%%%
\subsection{Inner Tracking Detector}
\label{sec:ID}
  \begin{figure}[tbp]
 \includegraphics[width=0.49\columnwidth]{/Users/sheenaschier/Documents/LaFiles/figures/thesis/detector/IDlayout.pdf}
  \includegraphics[width=0.49\columnwidth]{/Users/sheenaschier/Documents/LaFiles/figures/thesis/detector/IDtotal.pdf}
    \caption{Layout of the ALTAS Inner Detector}
   \label{fig:ID}
 \end{figure}
The ATLAS Inner Detector (ID), show in Figure~\ref{fig:ID}, provides position measurements of charged particles passing through the fiducial region $|\eta|~<~2.5$ by combining information from three separate tracking systems; the Pixel detector, the Semi-Conductor Tracker (SCT), and the Transition Radiation Tracker (TRT).  The ID is made of a central cylindrical barrel that covers the region $|\eta|~<~1.5$, and two end-caps that complete the ID range $1.5~<~|\eta|~<~2.5$ . The ID is surrounded by a superconducting solenoid that encases the entire ID in a 2 Tesla magnetic field.  The 2 T magnetic field bends the charged particles traveling through the tracker and the induced curvature is driven by the momentum of the particle.  \textcolor{red}{Add mathematical description of hoe the momentum is calculated from the curvature and the momentum?}
\iffalse
  \begin{figure}[tbp]
 \includegraphics[width=0.48\columnwidth]{/Users/sheenaschier/Documents/LaFiles/figures/thesis/detector/ID.pdf}
  \includegraphics[width=0.48\columnwidth]{/Users/sheenaschier/Documents/LaFiles/figures/thesis/detector/IDendcap.pdf}
    \caption{Layout of the ALTAS Inner Detector}
   \label{fig:IDscematic}
 \end{figure}
\fi

The Pixel detector is the inner most pixelated tracker and has the highest granularity sensors in the ID.  There are three pixel layers in the central barrel and three layers in each end cap, providing up to three space-points per track.  The Pixel detector has approximately 80.4 million readout channels bonded to pixel sensors segmented in the $R-\phi$ and $z$ directions.  The Pixel sensors have dimensions $50\mu m \times 400\mu m$ in $R-\phi \times z$, and provide an intrinsic resolution of $10\mu m$ in $R-\phi$ and $115 \mu m$ along $z$.  Besides mitigating the effects of pileup, the high granularity of the Pixel detector also helps discriminate prompt from non-prompt leptons in cases where a heavy mesonic decay, producing a non-prompt lepton, happens after entering the ID. The rich granularity helps resolve secondary vertices formed by the charged decay products.  

The Semi-Conductor Tracker (SCT) is a silicon micro-strip tracker just outside of the the Pixel detector, with an overall radial extension of $255~mm < R < 549~mm$ in the barrel and $251~mm < R < 610~mm$ in the end-caps.  It has eight paired strip layers that provide four space points per track.  In the barrel (end-cap), one set of strips is aligned parallel (perpendicular) the beam axis and is daisy chained to a second set of strips, each misaligned with the its partner by a 40 $mrad$ stereo angle.  The strip pitch is $80~\mu m$.  The resulting intrinsic resolution in both the barrel and the end-caps is $17~\mu m$ in  $R-\phi$, and in the barrel (end-caps) it is $580~\mu m$ in $z$ ($R$).  There are approximately 6.3 million readout channels.

The Transition Radiation Tracker (TRT) is the outer most detector in the ID.  It is comprised of straw tubes filled with diluted Xenon gas, some of which ionizes as charged particles pass through.   The outer shell of each straw is held at a negative potential while an anode wire running down the center of the tube is held at ground.  As some of the gas ionizes during the charged particle passage, an avalanche of ionization electrons form on the wire, amplifying the signal an of order $10^4$   

The TRT provides an average of 36 track position measurements, making it responsible for the large majority of track hits.   Between the barrel and the end-caps, as described in Figure~\ref{fig:ID}(a), the TRT can track charged particles through the region $|\eta| < 2.0$.  All straw tubes in the TRT are $4~mm$ in diameter but vary in length between the barrels and the end-caps.  In the barrel, the straw tubes are 144 cm (37cm) long and positioned parallel to the beam axis; in the end-caps, the tubes are 37 cm long and arranged transverse to the beam axis in the radial direction.  There are approximately 351,000 readout channels.  \textcolor{red}{Say something about how the TRT also helps identify electrons over pions.  Talk to Mike about the low-energy electron case and why particle ID breaks depreciates.}

\FloatBarrier
%%%%%%%%%%%%%%%%%%%%%%
\subsection{Calorimeters}
  \begin{figure}[tbp]
  \centering
 \includegraphics[width=0.6\columnwidth]{/Users/sheenaschier/Documents/LaFiles/figures/thesis/detector/cal.pdf}
    \caption{Picture of the ALTAS Calorimeters}
   \label{fig:cal}
 \end{figure}
 Just outside of the ID and solenoid magnet is the ATLAS calorimeter system.  The electromagnetic and hadronic calorimeters, extending to $|\eta|~<~4.9$, measure the energy of electromagnetic and hadronic objects as it dissipates inside the calorimeter material.  These calorimeters are samplers, meaning they only directly measure a fraction of the absorbed energy, and from this, infer the shape and strength of the full shower.

 %To gain the extra coverage over the ID in the forward regions $3.2~<~|\eta|~<~4.9$, LAr calorimeters are used for both electromagnetic and hadronic measurements.  In the region  $|\eta|~<~2.5$, also secured by the ID, the EM calorimeter is finely segmented for precise measurement of electrons and photons. 

%\subsubsection{Electromagnetic Calorimeter}
The eCal measures the energy of electrons and photons by inducing electromagnetic showering inside the eCal layers through continuous photon conversions and Bremsstrahlung, spreading out among calorimeter cells until all the energy of the incident particle has been absorbed.  The eCal is composed of electrodes submerged in liquid argon (LAr) that induce the electromagnetic shower, layered in an accordion shape with lead absorber plates in between. It is divided into a central barrel with $|\eta|~<~1.475$ and two end-caps enclosing each side of the barrel.  The end-cap regions have an inner wheel corresponding to $1.375~<~|\eta|~<~2.5$, and an outer wheel for $2.5~<~|\eta|~<~3.2$.  

The eCal is split into three layers.  The first layer is the most finely segmented in $\eta$ to aid the discrimination between true photons and neutral pions that have decayed to a pair of pions.  Both objects are trackless in flight and undetectable until they interact with the eCAL.  Two closely-spaced photons from a boosted neutral pion decay are difficult to resolve as separate photons without the extremely fine grain of this first layer.  The fine grain also helps improve the resolution of the shower position, shape and direction.  The second layer is more granular and is also the thickest layer where the majority of the electromagnetic showering occurs, and the third and most granular layer samples from the tail of the shower.  The eCAL is preceded by a pre-sampler at $|\eta| < 1.8$ to correct for upstream energy losses.  


%\subsubsection{Hadronic Calorimeter}
When hadrons pass through the eCal, they deposit only a small amount of energy as they travel to the hCal, where they will deposit the rest.  The barrel of the hCal spans the region $|\eta| < 1.7$ and sits just outside the EM calorimeter, extending radially from 2.28 m to 4.25 m.  Shown in Fig~\ref{fig:cal}, it is made of iron-scintillator tile and steel absorbers and separated into three sections, the central barrel and two extended barrels. The LAr hadronic end-caps cover the $\eta$ range $1.7~<~|\eta|~<~3.2$.  Forward LAr detectors measure both electromagnetic and hadronic showers and extend out to $\eta<4.9$.

%%%%%%%%%%%%%%%%%%%%%%
\subsection{Muon System}
%https://arxiv.org/pdf/1603.05598.pdf
The muon spectrometer, a tracking detector dedicated entirely to tracking muons, is the outer most sub-detector in ATLAS.  It is designed to track muons in the pseudorapidity region $|\eta|~<~2.7$ with a central barrel covering $|\eta|~<~1.05$ and two end-caps at $1.05~<~|\eta|~<~2.7$.  A network of three large super-conducting toroidal magnets, each with eight coils, supplies a magnetic to the muon spectrometer with am integral bending power in the barrel of around 2.5 Tm and up to 6Tm in the end caps.  Resistive plate chambers in the central region $|\eta|~<~1.05$ and end gap chambers in the forward-backward region $1.05~<~|\eta|~<~2.7$ impart triggering capabilities to the MS as well as position measurements in $\eta$ and $\phi$ with a spacial resolution of 5-10mm. Monitored drift tube chambers provide precision tracking out to $|\eta| < 2.7$ where each chamber provides 6-8 hits in $\eta$ along the muon flight path. 

%%%%%%%%%%%%%%%%%%%%%%
\iffalse
\subsection{Trigger and DAQ}
\textcolor{blue}{Timing and  trigger control logic, the TDAQ system is complex computing system to acquire and store data.  It is partitioned into sub-systems that are typically affiliated with with sub-detectors that have the same logic components and building blocks.}  \fi
\subsection{Trigger System}
%https://cds.cern.ch/record/2133909/files/ATL-DAQ-PROC-2016-003.pdf
Originally a three-level trigger system in Run-1, the trigger was restructured into a two-level system with only a hardware level-1 (L1) trigger and a software-based high-level (HL) trigger.  Each trigger level makes decisions about which events to store and which events to throw away forever.
The L1 trigger searches for electrons, muons, photons, jets, hadronically decaying $\tau$-leptons, and missing transverse momentum.  In each event, the L1 trigger defines Regions-of-Interest (ROIs), which are detector regions where interesting activity is identified, then stores the geographical ($\eta$, $\phi$) coordinates, the basic characteristics of the detector response in that region, and the set of criteria that triggered the L1.  This information is subsequently passed to the HL trigger to perform a more refined event selection.

\textcolor{red}{Next paragraph go into rates and amounts of data..}

\iffalse
\subsubsection{Readout and Data Acquisition}

Electronic elements called the Readout Drivers (RODs) gather information the detector-specific front-end data streams to multiplex highly concentrated data.  The front-end electronics are these combined sub-systems: the front-end analogue-to-digital converter, the L1 buffer that stores the information the L1 trigger will process, the de-randomising buffer that holds data accepted by the L1 trigger before passing it to the HLT, and the buses that move the front-end data stream to the HLT.  The L1 and the de-randomising buffers are important for decreasing the dead-time of the sensors and the L1 trigger.  The L1 buffer stores the information in buckets pass the information along a chain before passing to the L1 trigger to process.  This sequence produces a large enough latency in the detector output account for the intrinsic L1 trigger latency \textcolor{red}{what is the exact latency?}. The de-randomizing buffer allows the L1 store it decisions and process the next event without waiting for the HLT to finishing processing the previous event.
\fi

