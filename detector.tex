\chapter{The LHC and The ATLAS Experiment}
\label{ch:detector}
This chapter gives an overview of the LHC and the ATLAS detector used for this physics analysis.  First, the LHC is introduced in Section~\ref{sec:LHC}, then a review of the ATLAS detector is given in Section \ref{sec:ATLAS}.  This section is broken into smaller pieces that detail the ATLAS subdetectors and trigger system. %data trigger and acquisition.
\section{The Large Hadron Collider Machine}
\label{sec:LHC}

The Large Hadron Collider (LHC) is a circular proton accelerator and collider at CERN~\cite{Evans:2008zzb}, operating in the $26.7~\mathrm{km}$-long tunnel that was originally built for the Large Electron-Positron Collider (LEP).  In the tunnel, there are two separate vacuum beam pipes with counter-rotating proton beams that are accelerated to the TeV energy scale by a gigantic super-conducting magnet system.  To reach LHC energies, the proton beams first move through a stream of smaller accelerator structures that increase the kinetic energy of the beam at each step, until the beam is finally injected into the LHC, which is still the largest and most powerful accelerator in the world.  There are two transfer tunnels, each about $2.5~\mathrm{km}$ long, that join the LHC to the SPS, now acting as the injector for the LHC.  The LHC tunnel is broken into octants with eight straight sides and eight curves.  This is not an LHC design, but rather an artifact of LEP \cite{lep}.   That being said, each octants is considered as a reference point around the ring; for instance, octant 1 is centered around ``point 1", octant 2 is centered around "point 2", and so on.  The beams collide at four interaction points located approximately $100~\mathrm{m}$ underground, and surrounding each interaction point is a physics detector apparatus to collect data from the proton collisions.  The four different detector experiments are ALICE, LHC-B, CMS, and ATLAS \cite{Jean-Luc:841555}.  Figure~\ref{fig:lhc} depicts the tunnel octants and the beam injection and dump points.  It also shows the placement of the four detectors; ATLAS is located at point 1.
  \begin{figure}[tbp]
    \centering
% http://cdsweb.cern.ch/record/1095926
 \includegraphics[width=0.7\columnwidth]{/Users/sheenaschier/Documents/LaFiles/figures/thesis/detector/lhc-schematic.jpg}
    \caption{Schematic of the LHC layout~\cite{octant}}
   \label{fig:lhc}
 \end{figure}
 
The primary objective of the LHC is produce the Higgs boson, which was discovered by both ATLAS and CMS in 2012, and to expose Beyond Standard Model (BSM) physics.  To attempt these goals, the accelerator was designed to supply proton collisions with enough center-of-mass energy to produce a Higgs with mass above $100\GeV$ and to unlock possible new physics interactions at the $100\GeV$ - multi-TeV scale.  The initial aim was a proton-proton center-of-mass energy of $14 \TeV$, but due to instabilities in the magnet system at such high energy, only $13\TeV$ has successfully been achieved.  Many BSM theories predict new particle interactions with weak-scale cross-sections or lower, creating the need for enough luminosity to measure these low probability events.  The machine luminosity (L) depends only on beam parameters, as expressed in Eq~\ref{eq:lumi}.

\begin{equation}
L=\frac{N_p^2f}{4\epsilon\beta^{\star}}F
\label{eq:lumi}
\end{equation}
In the numerator of Eq~\ref{eq:lumi}, $N_p$ is the number of particles per bunch and $f$ is the bunch crossing frequency.  In the denominator of Eq~\ref{eq:lumi}, $\epsilon$ is the transverse beam emittance and $\beta^{\star}$ in the amplitude function at the collision point.  The geometric luminosity reduction factor $F$ is due to the beams crossing at an angle at the interaction points rather than directly head-on, and is about equal to 1~\cite{cid}.  Luminosity is generally in units of $\mathrm{cm}^{-2}\mathrm{s}^{-1}$, and these units are better understood by rewriting $\epsilon\beta^{\star}$ of Equation~\ref{eq:lumi} in terms of bunch cross-section $\sigma$:
\begin{equation}
\epsilon\beta^{\star}=4\pi\sigma^2\\
\label{eq:eb}
\end{equation} 
\begin{equation}
L=\frac{N_p^2f}{4\pi\sigma^2}
\label{eq:lumi2}
\end{equation} 
 ATLAS, one of the high luminosity experiments at the LHC, recorded a peak luminosity in 2016 above $L=10^{34}\mathrm{cm}^{-2}\mathrm{s}^{-1}$, for which the values corresponding to Equation~\ref{eq:lumi2} are shown in Table~\ref{tab:lumi}.
\begin{table}[!htp]
\centering
\small
\begin{tabular}{ll}
Parameter  & Value  \\
\hline \hline
$N_p$ & $1.15\times 10^{11}~\mathrm{protons}$ \\ 
%$n_b$ & $2808$ bunches  \\  
$f$ & $40\times10^{6}\mathrm{s}^{-1}$ \\  
%\hline
$\sigma$& $16\times10^{-4}~\mathrm{cm}^2$   \\
%$\beta^*$ & $0.55~\mathrm{m}$ \\
\hline 
\end{tabular}
\caption{Luminosity parameters in Equation~\ref{eq:lumi2}, corresponding to ATLAS peak luminosity $L\sim10^{34}\mathrm{cm}^{-2}\mathrm{s}^{-1}$}
\label{tab:lumi}
\end{table} 

\section{The ATLAS Experiment}
\label{sec:ATLAS}
The general design for detectors at the LHC is informed by the benchmark physics goals and the experimental environment and constraints \cite{tdr}.  The high energy and luminosity demands make radiation-hard sensor elements and read-out electronics a necessity.  Large numbers of interactions per bunch crossing, called pileup, create the need for highly granular detectors to resolve the separate events in space.  To search for new physics, a detector needs to be as general as possible, meaning it tries to see everything.  This requires a high acceptance in pseudorapidity with coverage over nearly the full azimuthal angle of the detector, high track reconstruction efficiency and good resolution on charged-particle momentum measurements.  Fairly precise electromagnetic calorimetry is also needed for efficient electron and photon identification.  Now that we understand these demands, we turn to a description of the ATLAS detector. 

The ATLAS experiment is a general purpose detector apparatus~\cite{Aad:2008zzm} that almost completely covers the entire solid angle around one of the LHC beam collision points.  ATLAS recorded its first LHC $pp$ collisions in 2009 at center-of-mass energy $7\TeV$, and has since recorded events at several different center-of-mass energies, including the most extensive energy reach in history at $13\TeV$.  %(http://inspirehep.net/record/1240374/files/CHARGED%202012_011.pdf). 
ATLAS achieves central coverage in the symmetric cylindrical barrel, and forward-backward detecting capabilities in the end-caps.  The complete detector system is $44~\mathrm{m}$ long, $25~\mathrm{m}$ in diameter, and weighs 4000 tons.  The ATLAS detector, shown in Figure~\ref{fig:ATLAS}, is comprised of several sub-detector systems, each calibrated and optimized for a different observational purpose.  The sub-detectors, listed in order from the beam pipe outward, are: the inner tracking detector, the electromagnetic calorimeter, the hadronic calorimeter, and the muon spectrometer.  Together, these sub-detectors measure the energy and momentum of a variety of particles and reconstruct the dynamics of each recorded event.  

ATLAS uses a right-handed coordinate system with the center of the detector as the origin.  The $z$-axis runs through the center of the barrel along the beam pipe, and the $y$-axis points upward through the barrel from the origin.  The $x$-axis points outward from the origin, perpendicular to both the $y$- and $z$-axes.  Cylindrical coordinates ($r,\phi$) map out the transverse plane, where $r$ is the radius in the plane, and $\phi$ is the azimuthal angle around the $z$-axis.  The pseudorapidity $\eta$, given by Eq~\ref{eq:eta}, is a transformation of the polar angle $\theta$ that is commonly used in particle detector experiments.  At $\theta=\pi/2$, $\eta=0$; at $\theta=\pi/18$, $\eta=2.88$; as $\theta$ approaches zero, $\eta$ approaches infinity.
\begin{equation}
\eta=-\ln[\tan(\theta/2)]
\label{eq:eta}
\end{equation}
The combination of the detector systems provide charged particle measurements and efficient lepton and photon measurements out to $|\eta| < 2.4$.  Missing transverse momentum ($\pt^{\mathrm{miss}}$) is the negative vector sum of the transverse momentum of all the visible objects in the detector, and is often referred to by the same nomenclature as the scalar magnitude, missing transverse energy (\met{}). Jets and \met{} are reconstructed using the full set of information out to $|\eta| < 4.9$.    \begin{figure}[tbp]
  \centering
 \includegraphics[width=0.8\columnwidth]{/Users/sheenaschier/Documents/LaFiles/figures/thesis/detector/ATLAS.pdf}
    \caption{Cut-away view of the complete ATLAS Detector~\cite{Nayak:2012np}}
   \label{fig:ATLAS}
 \end{figure}
  %Transverse energy and momenta are defined as $\pt{}=psin\theta$ and $E_T=Esin\theta$.}  
%%%%%%%%%%%%%%%%%%%%%%
\subsection{Inner Tracking Detector}
   \begin{figure}[tbp]
    \centering
  \includegraphics[width=0.8\columnwidth]{/Users/sheenaschier/Documents/LaFiles/figures/thesis/detector/IDtotal.pdf}
    \caption{Layout of the ALTAS Inner Detector}
   \label{fig:ID}
 \end{figure}
\label{sec:ID}
  \begin{figure}[tbp]
   \centering
 \includegraphics[width=0.8\columnwidth]{/Users/sheenaschier/Documents/LaFiles/figures/thesis/detector/IDlayout.pdf}
    \caption{Layout of the ALTAS Inner Detector}
   \label{fig:ID2}
 \end{figure}
The ATLAS Inner Detector (ID), shown in Figure~\ref{fig:ID}, provides position measurements of charged particles passing through the fiducial region $|\eta|~<~2.5$ by combining information from three separate tracking systems; the Pixel detector, the Semi-Conductor Tracker (SCT), and the Transition Radiation Tracker (TRT).  The ID is made of a central cylindrical barrel that covers the region $|\eta|~<~1.5$, and two end-caps that complete the ID range $1.5~<~|\eta|~<~2.5$. The layout of the separate tracking layers in $|\eta|$ is illustrated in Figure~\ref{fig:ID2}.  The ID is surrounded by a superconducting solenoid that encases the entire ID in a 2 T magnetic field.  The magnetic field bends the charged particles traveling through the tracker and the induced curvature depends on the momentum of the particle. % \textcolor{red}{Add mathematical description of hoe the momentum is calculated from the curvature and the momentum?}
\iffalse
  \begin{figure}[tbp]
 \includegraphics[width=0.48\columnwidth]{/Users/sheenaschier/Documents/LaFiles/figures/thesis/detector/ID.pdf}
  \includegraphics[width=0.48\columnwidth]{/Users/sheenaschier/Documents/LaFiles/figures/thesis/detector/IDendcap.pdf}
    \caption{Layout of the ATLAS Inner Detector}
   \label{fig:IDscematic}
 \end{figure}
\fi

The Pixel detector is the inner most pixelated tracker and has the finest granularity sensors in the ID.  There are four pixel layers in the central barrel and the end caps, providing up to four space-points per track.  The inner-most layer, called the Insertable B-Layer (IBL), was added during the ATLAS Run-2 upgrade~\cite{Takubo:2014qsa}.  Planar IBL sensors cover the central region of the barrel, and 3D sensors cover the outer regions.  The Pixel detector has approximately 92 million readout channels bonded to pixel sensors segmented in the $r-\phi$ and $z$ directions.  The first three layers of Pixel sensors have dimensions $50~\mu \mathrm{m} \times 400~\mu \mathrm{m}$ in $r-\phi \times z$, and provide an intrinsic resolution of $10~\mu \mathrm{m}$ in $r-\phi$ and $115~\mu \mathrm{m}$ along $z$.  The IBL has pixel dimensions $50~\mu \mathrm{m} \times 250~\mu \mathrm{m}$ with intrinsic resolutions $9~\mu \mathrm{m}$ and $60~\mu \mathrm{m}$ in the azimuthal and $z$ directions.  One benefit of the fine granularity of the Pixel detector is the discrimination between prompt and non-prompt leptons.  The added layer closer to the beam pipe helps recover late decays from heavy hadrons and $\tau$-leptons, and the rich granularity helps resolve secondary vertices formed by the charged decay products. 

The Semi-Conductor Tracker (SCT) is a silicon micro-strip tracker just outside of the the Pixel detector, with an overall radial extension of $255~\mathrm{mm} < r < 549~\mathrm{mm}$ in the barrel and $251~\mathrm{mm} < r < 610~\mathrm{mm}$ in the end-caps.  It has eight paired strip layers that provide four space points per track.  In the barrel (end-cap), one set of strips is aligned parallel (perpendicular) to the beam axis and is daisy chained to a second set of strips, each misaligned with the its partner by a 40~$\mathrm{m}rad$ stereo angle\cite{Abdesselam:2006wt}.  The strip pitch is $80~\mu \mathrm{m}$.  The resulting intrinsic resolution in both the barrel and the end-caps is $17~\mu \mathrm{m}$ in  $r-\phi$, and in the barrel (end-caps) it is $580~\mu \mathrm{m}$ in $z$ ($r$).  There are approximately 6.3 million readout channels.  The Pixel and SCT layers are are subject to the adverse conditions of event pileup from the large number of interactions at each bunch crossing.  Pileup is predominantly produced by the soft scattering of hadrons, which blurs the spacial reconstruction of the interaction point of a hard-scattering collision.  The interaction point, called the \textit{primary vertex}, is reconstructed from tracks in the Pixel and SCT layers, and is a critical reference point for events with tracks.  This is described more in Chapter~\ref{ch:obj}.

The Transition Radiation Tracker (TRT) is the outermost detector in the ID.  It is comprised of straw tubes filled with diluted xenon gas~\cite{Mitsou:2003rp}, some of which ionizes as charged particles pass through.   The outer shell of each straw is held at a negative potential while an anode wire running down the center of the tube is held at ground.  As some of the gas ionizes during the charged particle passage, an avalanche of ionization electrons forms on the wire, amplifying the signal by an order of $10^4$. Each straw tube in the TRT is $4~\mathrm{mm}$ in diameter but can vary in length between the barrels and the end-caps.  In the barrel, the straw tubes are $144~\mathrm{cm}$ long and positioned parallel to the beam axis; in the end-caps, the tubes are $37~\mathrm{cm}$ long and arranged transverse to the beam axis in the radial direction.  %There are approximately $351,000$ readout channels.
In both the barrel and the end-caps, the readout electronics have two discriminating thresholds, a low threshold at $300\eV$ and a high threshold at $6\keV$.  The high threshold is used to determine the presence of transition radiation photons from the electrons traversing the xenon gas.  This gives the TRT special discrimination power between electrons and charged pions with energy in the range $1\GeV - 100\GeV$.  Scattering effects of low-\pt electrons in the ID strains electron/pion discrimination and degrades electron identification efficiency~\cite{MINDUR2017257}.  

  \begin{figure}[tbp]
  \centering
 \includegraphics[width=0.8\columnwidth]{/Users/sheenaschier/Documents/LaFiles/figures/thesis/detector/IDrad.png}
    \caption{ATLAS simulation of material in the Pixel and SCT detectors in terms of the differential radiation length projected on the $r-z$ plane~\cite{Aaboud:2017pjd}.}
   \label{fig:IDmat}
 \end{figure}
 The material in the ID is on average $2.3$ radiation lengths at $\eta=0$ and increases with pseudorapidity in the barrel.  A radiation length ($X_0$) is the distance over which an electron's energy is reduced by a factor $\frac{1}{e}$ due to bremsstrahlung and $\frac{7}{9}$ of the mean free path $\lambda$ needed for photon pair production by a high energy photon.  Figure~\ref{fig:IDmat} shows the simulation of material in the Pixel and SCT detectors in differential radiation lengths $\Delta N_{X_0}/\Delta r~[\mathrm{mm}^{-1}]$.
\FloatBarrier

%%%%%%%%%%%%%%%%%%%%%%
\subsection{Calorimeters}
  \begin{figure}[tbp]
  \centering
 \includegraphics[width=0.8\columnwidth]{/Users/sheenaschier/Documents/LaFiles/figures/thesis/detector/cal.pdf}
    \caption{Picture of the ATLAS Calorimeters}
   \label{fig:cal}
 \end{figure}
 Just outside of the ID and the solenoid magnet is the ATLAS calorimeter system.  The electromagnetic and hadronic calorimeters, extending to $|\eta|~<~4.9$, measure the energy of electromagnetic and hadronic objects as it dissipates inside the calorimeter material.  These calorimeters are samplers, meaning they only directly measure a fraction of the absorbed energy, and from this, infer the shape and strength of the full shower.

 %To gain the extra coverage over the ID in the forward regions $3.2~<~|\eta|~<~4.9$, LAr calorimeters are used for both electromagnetic and hadronic measurements.  In the region  $|\eta|~<~2.5$, also secured by the ID, the EM calorimeter is finely segmented for precise measurement of electrons and photons. 

%\subsubsection{Electromagnetic Calorimeter}
The electromagnetic calorimeter (LAr)\footnote{\textit{LAr} stands for Liquid Argon.} measures the energy of electrons and photons by inducing electromagnetic showering inside the LAr layers through continuous photon conversions and Bremsstrahlung, spreading out among calorimeter cells until all the energy of the incident particle has been absorbed.  The LAr is composed of electrodes submerged in liquid argon that induce the electromagnetic shower, layered in an accordion shape with lead absorber plates in between. It is divided into a central barrel with $|\eta|~<~1.475$ and two end-caps enclosing each side of the barrel.  The end-cap regions have an inner wheel corresponding to $1.375~<~|\eta|~<~2.5$, and an outer wheel for $2.5~<~|\eta|~<~3.2$.  The total thickness is $> 24~X_0$ in the barrel and $>26~X_0 $ in the end caps~\cite{wilkens:J160}.  

The LAr is split into three layers.  The first layer is the most finely segmented in $\eta$ to aid the discrimination between true photons and neutral pions that have decayed to a pair of photons.  Both objects are trackless in flight and undetectable until they interact with the LAr.  Two closely-spaced photons from a boosted neutral pion decay are difficult to resolve as separate photons without the extremely fine grain of this first layer.  The fine grain also helps improve the resolution of the shower position, shape and direction.  The second layer is more coarsely grained and is also the thickest layer where the majority of the electromagnetic showering occurs.  The third layer has the largest granularity layer and it samples from the tail of the shower.  The LAr is preceded by a pre-sampler at $|\eta| < 1.8$ to correct for upstream energy losses. 


%\subsubsection{Hadronic Calorimeter}
The hadronic calorimeters, shown in Fig~\ref{fig:cal}, capture and measure the energy of jets, hadrons, and hadronically decaying $\tau$-leptons to $|\eta|<4.9$.  The barrel region $|\eta| < 1.7$ is made of iron-scintillator tile and steel absorbers and sits just outside the LAr, extending radially from $2.28~\mathrm{m}$ to $4.25~\mathrm{m}$.  Outside the barrel, in the region $1.5<|\eta|<3.2$ are the LAr hadronic end-cap calorimeters, and in the range $3.1<|\eta|<4.9$ are the LAr forward calorimeters that measure both electromagnetic and hadronic showers~\cite{henriqu:tile}.  The thickness of the TileCal and the hadronic LAr is about 10 interaction lengths, with an added 1 $\lambda$ of outside material to prevent punch-through into the muon system.  The nuclear interaction length ($\lambda_n$) gives the mean free path over which a strongly-interacting particle loses energy by a factor $\frac{1}{e}$.  There is also about $1.2~\lambda_n$ of material in the LAr before the TileCal.   
%%%%%%%%%%%%%%%%%%%%%%
\subsection{Muon System}
%https://arxiv.org/pdf/1603.05598.pdf
The muon spectrometer, a tracking detector dedicated entirely to tracking muons, is the outermost sub-detector in ATLAS.  It is designed to track muons in the pseudorapidity region $|\eta|~<~2.7$ with a central barrel covering $|\eta|~<~1.05$ and two end-caps at $1.05~<~|\eta|~<~2.7$.  A network of three large super-conducting toroidal magnets, each with eight coils, supplies a magnetic field to the muon spectrometer with am integral bending power in the barrel of around 2.5 T-m and up to 6 T-m in the end caps.  Resistive plate chambers in the central region $|\eta|~<~1.05$ and end gap chambers in the forward-backward region $1.05~<~|\eta|~<~2.7$ impart triggering capabilities to the MS as well as position measurements in $\eta$ and $\phi$ with a spacial resolution of $5$-$10~\mathrm{mm}$. Monitored drift tube chambers provide precision tracking out to $|\eta| < 2.7$ where each chamber provides $6$-$8$ hits in $\eta$ along the muon flight path.  %\textcolor{red}{Talk about muon momentum resolution and how low in pt we can construct things.  How much momentum is needed for a muon to reach the muon system? Hint: muons lose about $3\GeV$ in TileCal.}

%%%%%%%%%%%%%%%%%%%%%%
\iffalse
\subsection{Trigger and DAQ}
\textcolor{blue}{Timing and  trigger control logic, the TDAQ system is complex computing system to acquire and store data.  It is partitioned into sub-systems that are typically affiliated with with sub-detectors that have the same logic components and building blocks.}  \fi
\subsection{Trigger System}
%https://cds.cern.ch/record/2133909/files/ATL-DAQ-PROC-2016-003.pdf
Originally a three-level trigger system in Run-1, the trigger was restructured in Run-2 into a two-level system with only a hardware level-1 (L1) trigger and a software-based high-level (HL) trigger.  The LHC collision rate is about $40~\mathrm{MHz}$.  The L1 trigger reduces this to $\sim100~\mathrm{kHz}$, and the HLT further decreases the event rates to $\sim1~\mathrm{kHz}$.  In each event, the L1 trigger identifies Regions-of-Interest (ROIs), which are detector regions where interesting activity is identified. The geographical ($\eta$, $\phi$) coordinates, the basic characteristics of the detector response in that region, and the set of criteria that triggered the L1 are passed to the HLT for further discrimination.  RoI candidates are muons, electromagnetic clusters, jets, or taus.  Also, comprehensive sums of missing transverse energy and total energy are assembled.  HLT decisions are more sophisticated and can trigger on physics objects such as muons, electrons, photons, jets, b-jets, missing transverse energy, taus and b-hadrons.  

%Currently the "mht-algorithm" for reconstructing \met{} in the HLT shows the best performance~\cite{1748-0221-12-03-C03024}, and this is the type of \met trigger used for the compressed electroweak searches in this thesis.  \textcolor{red}{Talk specifically about the inclusive MET triggers and how "x $\rightarrow$ y rates.  Say that we get $100\%$ efficiency in signal regions for offline selection of MET greater than $200$.  Can I find a trigger schematic??}

\iffalse
\subsubsection{Readout and Data Acquisition}

Electronic elements called the Readout Drivers (RODs) gather information the detector-specific front-end data streams to multiplex highly concentrated data.  The front-end electronics are these combined sub-systems: the front-end analogue-to-digital converter, the L1 buffer that stores the information the L1 trigger will process, the de-randomising buffer that holds data accepted by the L1 trigger before passing it to the HLT, and the buses that move the front-end data stream to the HLT.  The L1 and the de-randomising buffers are important for decreasing the dead-time of the sensors and the L1 trigger.  The L1 buffer stores the information in buckets pass the information along a chain before passing to the L1 trigger to process.  This sequence produces a large enough latency in the detector output account for the intrinsic L1 trigger latency \textcolor{red}{what is the exact latency?}. The de-randomizing buffer allows the L1 store it decisions and process the next event without waiting for the HLT to finishing processing the previous event.
\fi


