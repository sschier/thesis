\chapter{Physics Object Reconstruction and Identification}
Every ATLAS analysis begins with identified and reconstructed physics objects from ATLAS data that represent the existence and characteristics of the observed particles traversing the detector volume.
 ATLAS was specifically designed to optimize particle identification and the measurements of their momentum and energy.
Verbosely say that object definitions are based on recommendations by the Combined Performance groups and applied by SUSYTools. 
 There are two categories; \textit{signal} and \textit{baseline} objects

\section{Primary Vertices}
Primary vertices are identified using inner detector tracks that satisfy a set of requirements.  To be considered in the construction of a primary vertex, a track must have $ \pt{} > 400~GeV$, $|\eta| < 2.5$, between 9 ($|\eta| \leq 1.65$) and 11 ($|\eta| > 1.65$) silicon hits, at least 1 hit in the IBL or B-Layer, a maximum of one shared pixel hit or two shared SCT hits, no holes in the pixel layers, and no more that one hole in the SCT layers.  Any primary vertex must have at least two associated tracks for reconstruction.  The track criteria is summarized in Table().

\section{Electrons}
Baseline electrons are triggered from energy deposits in the electromagnetic calorimeter and reconstructed with algorithms using the electromagnetic calorimeter clusters that are matched to inner detector tracks.  Baseline electrons must pass \pt{} threshold of 4.5 ~\GeV and be in central detector region $|\eta | < 2.47$.  A transverse impact parameter requirement of $|z_0sin\theta| < 0.5~mm$ is also applied.  Electrons are distinguished from other particles using identification criteria that rely on the shower shape in the electromagnetic calorimeter, tracking quantities, and the health of the track to EM calorimeter cluster matching.  Identification criteria varies from loose to tight and evolves with increasing strictness in criteria cuts and can be based either on independent requirements or on the single requirement set on the output of the likelihood function based on all the discriminating quantities listed above.  This analysis uses likelihood based identification criteria only.  Baseline electrons are required to satisfy \textit{VeryLooseLLH} identification while signal electrons must pass  \textit{Tight} identification plus \textit{GradientLoose} isolation criteria.  Signal electrons also require a transverse impact parameter $|d_0/\sigma(d_0)| < 5$. Furthermore, electrons with author 16 are vetoed.  

\section{Muons}
Muon information primarily comes from charged tracks in the inner detector and tracks in the muon spectrometer.  The calorimeters essentially don't contribute any useful information about the muons that pass through since energy measurements in the Ecal or Hcal relay on a particle's complete energy deposition in the calorimeter layers.   Baseline muons are reconstructed with algorithms that combine tracks from the inner detector and muon spectrometer to form muon candidates.  They must pass \pt{} threshold of 4 ~\GeV and be in fiducial region $|\eta | < 2.5$. Baseline muons are also expected to satisfy \textit{Medium} identification standards (DEFINE THIS) and have a transverse impact parameter $|z_0sin\theta| < 0.5~mm$.  Signal muons must also satisfy \textit{FixedCutTightTrackOnly} isolation criteria and a transverse impact parameter requirement of $|d_0/\sigma(d_0)| < 3$.

\section{Jets}
Baseline jets are reconstructed using locally-calibrated three-dimensional topological clusters built from calorimeter cells.  Topo-clustering starts by determining calorimeter cells with energy significance $4\sigma$ about the quadrature sum of electronic and pileup noise.  Neighboring jets with energy significance $2\sigma$ about noise are iteratively added, forming seed cluster, then a ring of direct neighbor cells are added to the final topo-clusters.  Jets are constructed using anti-$K_t$ algorithm with radius parameter R = 0.4 in this case. Baseline jets must pass \pt{} threshold of 20 ~\GeV and be in fiducial region $|\eta | < 4.5$.  Also, jets within $|\eta | < 2.5$ originating from b-hadrons are identified with the \textit{MV2c10} algorithm (define this!) with an 85\% working point.  Signal jets are further restricted to fiducial region $|\eta | < 2.8$.



\section{Missing Transverse Momentum}
Missing Transverse momentum (MET) is the negative vector sum of the transverse momentum of all the identified physics objects (electrons, muons, jets, photons) plus an additional soft term.  The soft term is constructed from all the tracks not associated with any physics object, but are associated with the primary vertex.  Therefore, the met is adjusted for the best possible calibration of the jets and other identified physics objects and still pileup independent in the soft term. 

\section{Overlap Removal}
Overlap removal is performed to prevent double counting of physics objects. First, jet-electron overlap removal, next jet-electron and jet-muon overlap removal using remaining jets, and finaly overlap removal with photons and other objects.

\section{Isolation Correction for Closely-Spaced Lepton}
 Soft leptons in a boosted system can have small angular separation, especially is products of a low-mass $Z^*$ decay.  These closely-spaced leptons can lies within each others isolation cones, leading to a loss of efficiency for very small mass-splittings.  *Refer to Figures~ \ref{fig:EffRll_ISOCorr} and ~\ref{fig:EffMll_ISOCorr} for additional motivation and explaination  This loss is corrected for using a dedicated tool that checks baseline leptons that fail the isolation criteria for other nearby leptons that are within it's isolation cone.  Tracks associated with the nearby lepton are removed from the track isolation sum.  If the nearby lepton is an electron, the topocluster $E_T$ is removed from the calorimeter isolation sum.  The corrected isolation variables are then reanalyzed using the original isolation working point.  Figure~\ref{fig:nearbylepiso} shows the effect of this correction on low invariant mass dilepton pairs in both data and Monte Carlo samples.


  \begin{figure}[tbp]
   % \centering
     \includegraphics[width=0.48\columnwidth]{/Users/sheenaschier/Documents/LaFiles/figures/thesis/eventselection/eff_EE_Rll_110_100_NoOS_NoISO.pdf}
       \includegraphics[width=0.48\columnwidth]{/Users/sheenaschier/Documents/LaFiles/figures/thesis/eventselection/eff_MM_Rll_110_100_GradLoose_NoOS_NoISO.pdf}\\
     \includegraphics[width=0.48\columnwidth]{/Users/sheenaschier/Documents/LaFiles/figures/thesis/eventselection/eff_EE_Rll_110_100_NoOS.pdf}
     \includegraphics[width=0.48\columnwidth]{/Users/sheenaschier/Documents/LaFiles/figures/thesis/eventselection/eff_MM_Rll_110_100_GradLoose_NoOS.pdf}\\
   \caption{Dilepton $\Delta$ R distribution before LepIsoCorrection (top) and after LepIsoCorrection (bottom) for the $ee$-channel (left) and $\mu\mu$-channel (right).}
   \label{fig:EffRll_ISOCorr}
 \end{figure}

  \begin{figure}[tbp]
   % \centering
     \includegraphics[width=0.48\columnwidth]{/Users/sheenaschier/Documents/LaFiles/figures/thesis/eventselection/eff_EE_Mll_110_100_NoOS_NoISO.pdf}
       \includegraphics[width=0.48\columnwidth]{/Users/sheenaschier/Documents/LaFiles/figures/thesis/eventselection/eff_MM_Mll_110_100_GradLoose_NoOS_NoISO.pdf}\\
     \includegraphics[width=0.48\columnwidth]{/Users/sheenaschier/Documents/LaFiles/figures/thesis/eventselection/eff_EE_Mll_110_100_NoOS.pdf}
     \includegraphics[width=0.48\columnwidth]{/Users/sheenaschier/Documents/LaFiles/figures/thesis/eventselection/eff_MM_Mll_110_100_GradLoose_NoOS.pdf}\\
   \caption{Dilepton invarient mass distribution before LepIsoCorrection (top) and after LepIsoCorrection (bottom) for the $ee$-channel (left) and $\mu\mu$-channel (right).}
   \label{fig:EffMll_ISOCorr}
 \end{figure}

 \begin{figure}[tbp]
  \includegraphics[width=0.48\columnwidth,trim=1.2cm 0cm 1.9cm 0cm,clip]{/Users/sheenaschier/Documents/LaFiles/figures/thesis/nearbylepiso.pdf}
   \includegraphics[width=0.48\columnwidth]{/Users/sheenaschier/Documents/LaFiles/figures/thesis/nearbylepiso_signal.pdf}
  \caption{(left) Impact of the \texttt{NearbyLepIsoCorrection} tool on the efficiency of low-mass dilepton pairs in data.  The data are shown in a region with $\Delta\phi(\met, p_{t}^{j1})<1.5$ to avoid the signal region.  Events are triggered with the inclusive-\met{} trigger.  The red trend shows events with two baseline leptons without applying any isolation; the green shows the impact of applying \texttt{GradientLoose} isolation; the blue shows the result of the \texttt{NearbyLepIsoCorrection} applied to the \texttt{GradientLoose} sample.  (right) Impact of the correction on a Higgsino LSP signal sample with $\Delta m(\chi,\chi)=3~\GeV$.}
 \label{fig:nearbylepiso}
 \end{figure}
 
 \section{Truth Matching}
 While detector effects that result in the misidentification of physics objects are not well modeled in simulation, misidentification still occurs during reconstruction.  Reducible fake lepton backgrounds are estimated with a data-driven method and therefore are already accounted for.  Background estimated done in Monte Carlo use only truth matched leptons to prevent overlap in the MC and data-driven estimates. 
 
 


