\chapter{Systematic Uncertianties}
\label{sec:syst}
Systematic uncertainties are split into two categories: experimental and theoretical.  The major sources of experimental uncertainties are the modeling of particle reconstruction in detector simulation, luminosity and pileup measurements, and systematic effects from data-driven estimates.  The main theoretical uncertainties emerge from the modeling of Standard Model background processes.  Simulation of these processes relies on cross-section measurements, parton distribution functions, and renormalization and factorization scale assumptions. Systematic uncertainties propagate to the final expected yields of signal to background, and limit the resolution of predictions. 

This chapter is organized as follows: experimental uncertainties are described in Section~\ref{sec:sys:exp}, where first CP Group uncertainties on measurements of pile-up re-weighting, luminosity, jets, electrons, muons, and missing transverse energy are summarized in Section~\ref{sec:sys:expCP}, and next fake factor uncertainties are described in Section~\ref{sec:sys:expFF}.  Finally, theoretical uncertainties on SM background modeling are dissected in Section~\ref{sec:sys:thy}.

\section{Experimental Uncertainties}
\label{sec:sys:exp}
This chapter will cover uncertainties from CP group recommendations and fake factor measurements.  
\subsection{CP Group Uncertainties}
\label{sec:sys:expCP}
Combined Performance (CP) groups are dedicated teams in ATLAS that work to optimize the characteristic measurements of certain classes of particle.  These groups make recommendations to analysis teams about pile-up re-weighting, luminosity measurements, and which jet, electron, muon, and missing transverse energy definitions to use.  The uncertainties associated with these objects and measurements are discussed in this section.

Multiple pile-up interactions need to be modeled well in Monte Carlo so that the simulated detector response and particle reconstruction conditions match the actual data.  The distribution of the average number of interactions per bunch crossing applied to Monte Caro events, the $\mu$ profile, is based on relevant assumptions and does not always agree with the $\mu$ profile observed in data.  To resolve these disagreements, the $\mu$ profile for Monte Carlo is reweighted to better match the shape in data.  This is typically called pile-up reweighting.  Studies of the data/MC agreement for the number of primary vertices versus $\mu$ suggest an additional rescaling of the $\mu$ distribution in data of $1/1.16$.  A systematic uncertainty for the pile-up reweighting scheme is assigned by varying the scaling factor assigned to data between 1.00 and 1.21 and assessing the change in event yields.  An uncertainty on the luminosity measurement is also examined.  For the 2015+2016 combined datasets, the luminosity uncertainty is observed as $3.2\%$.

Uncertainties on the jet energy scale and jet energy resolution are measured using five parameters varied up and down for the energy uncertainty estimate, and one parameter varied up and down for the uncertainty on the resolution.  A separate uncertainty is assigned to account for the differences in the jet-vertex tagging and b-jet tagging efficiencies between Monte Carlo and data. Uncertainties on the electron energy and momentum scale and resolution are also considered, along with uncertainties on the electron and muon scale factors applied to Monte Carlo events that ensure the simulated reconstruction, identification, isolation, and track-to-vertex association efficiencies match the data.  Furthermore, uncertainties on the missing transverse energy and momentum arise from the propagation of error in the transverse momentum measurements of hard physics objects.  Additional uncertainties on the \met propagate from the scale and resolution of the track-based soft term, described in Chapter~\ref{sec:obj:reco}.  The dominant CP group systematic is from the jet energy scale and resolution.

\subsection{Fake Factor Uncertainties}
\label{sec:sys:expFF}
Fake and non-prompt lepton backgrounds are estimated with a data-driven fake factor method, as described in Chapter~\ref{ch:fakefactor}.  Uncertainties arise from several sources, but are mainly from: kinematic dependencies, non-closure in the same-sign validation region, statistical uncertainties on the applied fake factors, and prompt lepton subtraction using Monte Carlo.

The primary fake factor uncertainty comes from kinematic dependancies on variables that are not included in the fake factor binning.  Fake factors are measured as a function of electron \pt for the electrons, and as a function of muon \pt and $N_{b-jet}$ for the muons.  These choices are motivated by the strong correlation of the fake factors and these variables, but other, smaller kinematic dependencies are present.  The fake factor vulnerabilities are not large enough to consider binning them in every variable, so they are accounted for as a systematic. Figure~\ref{fig:elec_FF_all} presents electron fake factors, and Figures~\ref{fig:muon_FF_hist_eta} -~\ref{fig:muon_FF_npv} present muon fake factors binned in alternative variables.  We consider the largest, statistically meaningful variation of the fake factors binned in the alternative relevant variables and subtract it from the average fake factor for the electron and muon samples separately.  The resulting uncertainty is $25\%$ for each, both driven by the variation in lepton $\eta$.

The relationship between the fake lepton estimate and the data in VR-SS is another source of systematic uncertainty.  This is quantified by comparing data in a version of the VR-SS that does not require an $\met/H_T$ cut in the envelope containing the systematic variations described above.  The root mean square of the variations is compared with the data and the quadrature difference is interpreted as the closure systematic.  This uncertainty is determined to be $38\%$ for electrons with $\pt < 7\GeV$, $97\%$ for muons with \pt 7-10 GeV, and $0\%$ everywhere else. %This $0\%$ is assigned because the fake lepton estimate and the data agree within their uncertainties in VR-SS for the other pT bins considered

Statistical uncertainties on the fake factors are due to the limited size of the samples used to derive them.  These samples use pre-scaled single lepton triggers to select events in data, which are further scrutinized based on the identification, isolation, and impact parameter of the reconstructed leptons to be determined tas either an "ID" or "anti-ID" lepton event.  It is possible that there are overlapping events in these two categories, but it is a rare occurrence since less than $10\%$ of the events have more than one lepton, and both the "ID" and the "anti-ID" leptons would need to fall in the \pt range associated with highest lepton \pt trigger that fired.  Figures~\ref{fig:elec_FF_rel_uncert} and~\ref{fig:muon_FF_rel_uncert} show the relative systematic uncertainties on the electron and muon fake factors per lepton \pt bin.   For electrons, statistical uncertainties range from about $32\%$ in the lowest \pt bin to about $58\%$ in the highest \pt bin.  For muons, the uncertainties on fake factors used to estimate fake backgrounds in the signal regions vary between $12\%$ in the lowest \pt bin to about $32\%$ in the highest \pt bin, and uncertainties on fake factors used to estimate fake backgrounds in the $t\bar{t}$ control region vary between $16\%$ and $38\%$.

Fake factors are measured in regions of data enriched with fake leptons, but prompt lepton contamination is still present.  In the measurement region $m_T<40~\GeV$, prompt lepton events are subtracted from the \pt distributions using SM Monte Carlo that have been rescaled to match data in the high \met region.  To calculate the systematic uncertainty on this method of prompt subtraction, the change in the binned fake factors is studied as three key parameters are varied.  The \met region, where the scale factor for the prompt subtraction is computed, is varied up and down by $20~\GeV$ from the nominal $\met>200~\GeV$ selection, the region where the fake factors are measured is varied up and down by $10~\GeV$ from the nominal $m_T<40~\GeV$ selection, and the scale factor that is applied to the subtracted Monte Carlo is varied up and down by $20\%$.  Uncertainty contributions in the prompt subtraction are assed further by recomputing the Monte Carlo scale factor in the region $m_T>100~\GeV$ and assessing the change in the fake factors.  All together, the resulting uncertainties on both electron and muon scale factors are less that $10\%$, but for one exception in the muon \pt bin above $20~\GeV$, where the uncertainty is $19\%$.  The overall contribution from prompt subtraction is minute compared to the other sources.

\section{Theoretical Uncertainties}
\label{sec:sys:thy}
Theoretical uncertainties from signal and background simulation arise from the uncertainties on the underlying parameters in the Monte Carlo generation.


\subsection{Uncertainty on Simulated Signal Events}
Statistical uncertainties on Higgsino and slepton simulated signal events dominantly arise from the next-to-leading order calculations of the hadronic initial state radiation (ISR), factorization and renormalization scale (FSR), and the underlying event.  ISR/FSR/EU are all around $20\%$.  PDF uncertainties on signal acceptances are also estimated to be around $10\%$.  Uncertainties on signal cross-section are around $5\%$.

\subsection{Uncertainty on Simulated Background Events}
Diboson, $Z(\rightarrow\tau\tau)$+jets, and $t\bar{t}$ are the dominant background processes estimated with Monte Carlo simulation.  There are three main sources of uncertainty: choice of QCD renormalization and factorization scales $\mu_R$ and $\mu_F$, choice of strong coupling constant $\alpha_s$, and choice of PDF set.  To calculate the uncertainties, each of these is varied symmetrically around some parameter, or, in the case of the PDF uncertainty, varied by PDF set.  The effect of the variations on the predicted yield from each of the dominant background processes is evaluated in the signal, control, and validation regions. $\mu_R$ and $\mu_F$ are deviated up and down by a factor of 2 and $\alpha_s$ is varied within its uncertainty of 0.001, and the range of impact on the expected yields are evaluated as the uncertainties.  PDF uncertainties are obtained from the envelope of symmetrized variations within acceptance of the MMHT2014, CT14, NNPDF PDF sets.  Figures~\ref{fig:theoryUncsVV}, \ref{fig:theoryUncsZtt}, and \ref{fig:theoryUncsttbar} show the assortment of event yields in the Higgsino and slepton SRs for the diboson, $Z(\rightarrow\tau\tau)$+jets, and $t\bar{t}$ predictions.  The final uncertainty in each region is calculated as the quadrature sum of all the individual contributions.  

 \begin{figure}
  \centering
  \includegraphics[width=0.4\columnwidth]{/Users/sheenaschier/Documents/LaFiles/figures/thesis/systematics/scaleVars_diboson2L_mll_SR_hg_SFDF_shape.pdf}
 %\caption{$\mu_{F}$ and $\mu_{R}$ uncertainties on the $m_{\ell\ell}$ distribution in the Higgsino signal region.}
  \includegraphics[width=0.4\columnwidth]{/Users/sheenaschier/Documents/LaFiles/figures/thesis/systematics/scaleVars_diboson2L_mt2leplsp_100_SR_sl_SFDF_shape.pdf}
% \caption{$\mu_{F}$ and $\mu_{R}$ uncertainties on the $m_{\text{T}2}$ distribution in the slepton signal region.}
 \includegraphics[width=0.4\columnwidth]{/Users/sheenaschier/Documents/LaFiles/figures/thesis/systematics/alphaVars_diboson2L_mll_SR_hg_SFDF_shape.pdf}
 % \caption{$\alpha_{s}$ uncertainties on the $m_{\ell\ell}$ distribution in the Higgsino signal region.}
 \includegraphics[width=0.4\columnwidth]{/Users/sheenaschier/Documents/LaFiles/figures/thesis/systematics/alphaVars_diboson2L_mt2leplsp_100_SR_sl_SFDF_shape.pdf}
% \caption{$\alpha_{s}$ uncertainties on the $m_{\text{T}2}$ distribution in the slepton signal region.}
  \includegraphics[width=0.4\columnwidth]{/Users/sheenaschier/Documents/LaFiles/figures/thesis/systematics/PDFVars_diboson2L_mll_SR_hg_SFDF_shape_allPDFs.pdf}
 % \caption{PDF uncertainties on the $m_{\ell\ell}$ distribution in the Higgsino signal region.}
  \includegraphics[width=0.4\columnwidth]{/Users/sheenaschier/Documents/LaFiles/figures/thesis/systematics/PDFVars_diboson2L_mt2leplsp_100_SR_sl_SFDF_shape_allPDFs.pdf}
%\caption{PDF uncertainties on the $m_{\text{T}2}$ distribution in the slepton signal region.}
 \caption{QCD scale, $\alpha_{s}$ and PDF uncertainties on the shape and normalization of the diboson background in the Higgsino (left) and slepton (right) signal regions, but with no lepton flavor requirement.}
\label{fig:theoryUncsVV}
 \end{figure}
 
  \begin{figure}
  \centering
   \includegraphics[width=0.4\columnwidth]{/Users/sheenaschier/Documents/LaFiles/figures/thesis/systematics/scaleVars_Zttjets_mll_SR_hg_SFDF_shape.pdf}
 %\caption{$\mu_{F}$ and $\mu_{R}$ uncertainties on the $m_{\ell\ell}$ distribution in the Higgsino signal region.}
  \includegraphics[width=0.4\columnwidth]{/Users/sheenaschier/Documents/LaFiles/figures/thesis/systematics/scaleVars_Zttjets_mt2leplsp_100_SR_sl_SFDF_shape.pdf}
% \caption{$\mu_{F}$ and $\mu_{R}$ uncertainties on the $m_{\text{T}2}$ distribution in the slepton signal region.}
 \includegraphics[width=0.4\columnwidth]{/Users/sheenaschier/Documents/LaFiles/figures/thesis/systematics/alphaVars_Zttjets_mll_SR_hg_SFDF_shape.pdf}
 % \caption{$\alpha_{s}$ uncertainties on the $m_{\ell\ell}$ distribution in the Higgsino signal region.}
 \includegraphics[width=0.4\columnwidth]{/Users/sheenaschier/Documents/LaFiles/figures/thesis/systematics/alphaVars_Zttjets_mt2leplsp_100_SR_sl_SFDF_shape.pdf}
% \caption{$\alpha_{s}$ uncertainties on the $m_{\text{T}2}$ distribution in the slepton signal region.}
  \includegraphics[width=0.4\columnwidth]{/Users/sheenaschier/Documents/LaFiles/figures/thesis/systematics/PDFVars_Zttjets_mll_SR_hg_SFDF_shape_allPDFs.pdf}
 % \caption{PDF uncertainties on the $m_{\ell\ell}$ distribution in the Higgsino signal region.}
  \includegraphics[width=0.4\columnwidth]{/Users/sheenaschier/Documents/LaFiles/figures/thesis/systematics/PDFVars_Zttjets_mt2leplsp_100_SR_sl_SFDF_shape_allPDFs.pdf}
%\caption{PDF uncertainties on the $m_{\text{T}2}$ distribution in the slepton signal region.}
\caption{QCD scale, $\alpha_{s}$ and PDF uncertainties on the shape and normalization of the $Z\to\tau\tau$ background in the Higgsino (left) and slepton (right) signal regions, but with no lepton flavor requirement.}
\label{fig:theoryUncsZtt}
 \end{figure}
 
  \begin{figure}
  \centering 
   \includegraphics[width=0.4\columnwidth]{/Users/sheenaschier/Documents/LaFiles/figures/thesis/systematics/scaleVars_alt_ttbar_PowPy8_dilep_hdamp258p75_mll_SR_hg_SFDF_shape.pdf}
 %\caption{$\mu_{F}$ and $\mu_{R}$ uncertainties on the $m_{\ell\ell}$ distribution in the Higgsino signal region.}
  \includegraphics[width=0.4\columnwidth]{/Users/sheenaschier/Documents/LaFiles/figures/thesis/systematics/scaleVars_alt_ttbar_PowPy8_dilep_hdamp258p75_mt2leplsp_100_SR_sl_SFDF_shape.pdf}
% \caption{$\mu_{F}$ and $\mu_{R}$ uncertainties on the $m_{\text{T}2}$ distribution in the slepton signal region.}
  \includegraphics[width=0.4\columnwidth]{/Users/sheenaschier/Documents/LaFiles/figures/thesis/systematics/PDFVars_alt_ttbar_PowPy8_dilep_hdamp258p75_mll_SR_hg_SFDF_shape_allPDFs.pdf}
 % \caption{PDF uncertainties on the $m_{\ell\ell}$ distribution in the Higgsino signal region.}
  \includegraphics[width=0.4\columnwidth]{/Users/sheenaschier/Documents/LaFiles/figures/thesis/systematics/PDFVars_alt_ttbar_PowPy8_dilep_hdamp258p75_mt2leplsp_100_SR_sl_SFDF_shape_allPDFs.pdf}
%\caption{PDF uncertainties on the $m_{\text{T}2}$ distribution in the slepton signal region.}
\caption{QCD scale and PDF uncertainties on the shape and normalization of the $t\bar{t}$ background in the Higgsino (left) and slepton (right) signal regions, but with no lepton flavor requirement.}
\label{fig:theoryUncsttbar}
 \end{figure}
