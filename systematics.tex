\chapter{Systematic Uncertianties}
\label{sec:syst}
Systematic uncertainties can be split into two categories: experimental and theoretical.  Sources of experimental uncertainties are particle reconstruction modeling in detector simulation, luminosity and pileup measurements, and systematic effects from data-driven estimates.  Theoretical uncertainties emerge from modeling of Standard Model background processes.  Simulation of these processes relies on cross-section measurements, parton distribution functions employment, and renormalization and factorization scale assumptions. These systematic uncertainties propagate to the final expected yields of signal to background, and limit the resolution of your predictions. \textcolor{red}{Describe how the chapter will flow.}

\section{Experimental Uncertainties}
\label{sec:sys:texp}
This chapter, the CP group uncertainties will first be overviewed, and next fake factor uncertainties.  
\subsection{CP Group Uncertainties}
The list is: Pile-up re-weighting, jets, electrons and muons, missing transverse energy, luminosity

Multiple pile-up interactions need to be modeled well in Monte Carlo so that the simulated detector response and particle reconstruction conditions match the actual data.  The distribution of the average number of interactions per bunch crossing applied to Monte Caro events, the $\mu$ profile is based on relevant assumptions and does not always agree with the $\mu$ profile observed in data.  To resolve these disagreements, the $\mu$ profile for Monte Carlo is reweighted to better match the profile in data.  This is typically called pile-up reweighting.  Studies of the data/MC agreement for the number of primary vertices versus $\mu$ suggest an additional rescaling of the $\mu$ distribution in data of $1/1.16$.  A systematic uncertainty for the pile-up reweighting scheme is assigned by varying the scaling factor assigned to data between 1.00 and 1.21 and assessing the change in event yields.  An uncertainty on the luminosity measurement is also examined.  For the 2015+2016 combined datasets, this is observed as $3.2\%$.

Uncertainties on the jet energy scale and jet energy resolution are measured.  Five parameters were varied up and down for the energy uncertainty estimate, and one parameter was varied for the uncertainty on the resolution.  Also must assign an uncertainty to account for the differences in the jet-vertex tagging and b-jet tagging efficiencies between Monte Carlo and data. Uncertainties on the electron energy and momentum scale and resolution must also be considered, as well as the uncertainties on the electron and muon scale factors applied to Monte Carlo events to ensure the simulated reconstruction, identification, isolation and track-to-vertex association efficiencies match the data.  Furthermore, uncertainties on the missing transverse arise from the propagation of error in the transverse momentum measurements of hard physics objects.  Additional uncertainties on the \met propagate from the scale and resolution of the track-based soft term, described in Chapter~\ref{sec:obj:reco}.

Dominant systematic is from the jet energy scale and resolution. \textcolor{red}{Give a quantitative summary if you can}.

\subsection{Fake Factor Uncertainties}
Fake and non-prompt lepton backgrounds are estimated with a data-driven fake factor method, as described in Chapter~\ref{ch:fakefactor}, and uncertainties arise from several sources.  The list is: statistical uncertainties on the applied fake factors, prompt subtraction, kinematic dependencies, non-closure in the same-sign validation region.

Statistical uncertainties on the fake factors themselves are due to the limited size of the samples used to derive the fake factors.  These samples use pre-scaled single lepton triggers to select events in data, which are further scrutinized based on the identification, isolation, and impact parameter of the reconstructed leptons and determined to be either an "ID" or "anti-ID" lepton event.  It is possible that there are overlapping events in these two categories, but it is a rare occurrence since less the $10\%$ of the events have more than one lepton, and both the "ID" and the "anti-ID" lepton would need to fall in the same \pt range accepted depending in which trigger was fired.  \textcolor{red}{Refer to the plots that show the extent of the statistical error on the fake factors and try to give some quantitative summary}.

Fake factors are measured then applied in regions of data that are enriched in events with fakes, and depleted in events with true, prompt leptons, but there are still contribution from prompt leptons in these regions.  In the fake factor measurement region, the prompt contribution is subtracted from the distributions in which the fake factors are measured using Standard Model Monte Carlo events that have been rescaled to match anti-ID lepton events in data in the high \met region.  To calculate the systematic uncertainty on this method of prompt subtraction, the change in the binned fake factors is studied as three key parameters are varied.  The \met region, where the scale factor for the prompt subtraction is computed, is varied up and down $20~\GeV$ from the nominal $\met>200~\GeV$ selection, the region where the fake factors are measured is varied up and down by $10~\GeV$ from the nominal $m_T<40~\GeV$ selection, and the scale factor that is applied to the subtracted Monte Carlo is varied up and down by $20\%$.  Uncertainty contributions in the prompt subtraction is assed further by recomputing the scale factor and prompt subtraction using the hight $m_T$ region, $m_T>100~\GeV$ (\textcolor{red}{check this, just guessing here}), of anti-ID lepton events.  All together, the resulting uncertainties on both electron and muon scale factors are less that $10\%$, but for one exception of the muon \pt bin above $10~\GeV$, where the uncertainty is $19\%$.  The overall contribution from prompt subtraction is minuscule compared to the other sources.

The fake factors are measured as a function of electron \pt for the electrons, and as a function of muon \pt and $N_{b-jet}$ for the muons.  This choice was motivated by the strong kinematic dependencies of the fake factors on these variables, but other, smaller kinematic dependencies are present.  These dependencies are not large enough to consider binning the fake factors in every variable, so they are accounted for as a systematic.  \textcolor{red}{Refer to the plots all the fake factors in all the different variables}.  We consider the largest statistically meaningful variation of the fake factors binned in the alternative relevant variables and subtract it from the average fake for the electron and muon samples separately, and the resulting uncertainty is $25\%$ for both, driven by the variation in lepton $\eta$.

\section{Theoretical Uncertainties}
Theoretical uncertainties are different for signal and background simulation and arise from the uncertainties on the underlying parameters in the Monte Carlo generation.
\label{sec:sys:thy}

\subsection{Uncertainty on Simulated Signal Events}
The initial state radiation in signal processes require next-to-leading order calculations that give rise to systematic uncertainty. \textcolor{red}{This is ISR, the is also FSR and UE (underlying event)}. ISR/FSR/EU are all around $20\%$.  PDF uncertainties on signal acceptances are also estimated to be around $10\%$.  Uncertainties on signal cross-section are around $5\%$.

\subsection{Uncertainty on Simulated Background Events}
Diboson, $Z(\rightarrow\tau\tau)$+jets, and $t\bar{t}$ are dominant backgrounds, and uncertainties on these predictions are estimated using the LHE3 weights \textcolor{red}{what are these?} available in the derivations.  There are three main sources of uncertainty: choice of QCD renormalization and factorization scales, $\mu_R$ and $\mu_F$, choice of strong coupling constant, $\alpha_s$, and choice of PDF set.  $\mu_R$ and $\mu_F$ are varied up and down by a factor of 2, $\alpha_s$ is varied within its uncertainty \textcolor{red}{(which is?)}, variations within acceptance with the respect to MMHT2014, CT14, NNPDF PDF sets is symmetrized, and the envelope is taken to be the uncertainty.  The impact is evaluated directly on the predicted yield from each of the dominant background processes in the signal, control, validation regions, or bins within where they contribute the most (SR-MLL, SR-MT2, CR-top $t\bar{t}$ only, CR-tau, VRDF-MLL, VRDF-MT2, VR-VV.  For each region/bin, the final uncertainty is equal to the quadrature addition of all the individual contributions.

 \begin{figure}
  \centering
  \includegraphics[width=0.4\columnwidth]{/Users/sheenaschier/Documents/LaFiles/figures/thesis/systematics/scaleVars_diboson2L_mll_SR_hg_SFDF_shape.pdf}
 %\caption{$\mu_{F}$ and $\mu_{R}$ uncertainties on the $m_{\ell\ell}$ distribution in the Higgsino signal region.}
  \includegraphics[width=0.4\columnwidth]{/Users/sheenaschier/Documents/LaFiles/figures/thesis/systematics/scaleVars_diboson2L_mt2leplsp_100_SR_sl_SFDF_shape.pdf}
% \caption{$\mu_{F}$ and $\mu_{R}$ uncertainties on the $m_{\text{T}2}$ distribution in the slepton signal region.}
 \includegraphics[width=0.4\columnwidth]{/Users/sheenaschier/Documents/LaFiles/figures/thesis/systematics/alphaVars_diboson2L_mll_SR_hg_SFDF_shape.pdf}
 % \caption{$\alpha_{s}$ uncertainties on the $m_{\ell\ell}$ distribution in the Higgsino signal region.}
 \includegraphics[width=0.4\columnwidth]{/Users/sheenaschier/Documents/LaFiles/figures/thesis/systematics/alphaVars_diboson2L_mt2leplsp_100_SR_sl_SFDF_shape.pdf}
% \caption{$\alpha_{s}$ uncertainties on the $m_{\text{T}2}$ distribution in the slepton signal region.}
  \includegraphics[width=0.4\columnwidth]{/Users/sheenaschier/Documents/LaFiles/figures/thesis/systematics/PDFVars_diboson2L_mll_SR_hg_SFDF_shape_allPDFs.pdf}
 % \caption{PDF uncertainties on the $m_{\ell\ell}$ distribution in the Higgsino signal region.}
  \includegraphics[width=0.4\columnwidth]{/Users/sheenaschier/Documents/LaFiles/figures/thesis/systematics/PDFVars_diboson2L_mt2leplsp_100_SR_sl_SFDF_shape_allPDFs.pdf}
%\caption{PDF uncertainties on the $m_{\text{T}2}$ distribution in the slepton signal region.}
 \caption{QCD scale, $\alpha_{s}$ and PDF uncertainties on the shape and normalization of the diboson background in the Higgsino and slepton signal regions (with no lepton flavor requirement).}
\label{fig:theoryUncsVV}
 \end{figure}
 
  \begin{figure}
  \centering
   \includegraphics[width=0.4\columnwidth]{/Users/sheenaschier/Documents/LaFiles/figures/thesis/systematics/scaleVars_Zttjets_mll_SR_hg_SFDF_shape.pdf}
 %\caption{$\mu_{F}$ and $\mu_{R}$ uncertainties on the $m_{\ell\ell}$ distribution in the Higgsino signal region.}
  \includegraphics[width=0.4\columnwidth]{/Users/sheenaschier/Documents/LaFiles/figures/thesis/systematics/scaleVars_Zttjets_mt2leplsp_100_SR_sl_SFDF_shape.pdf}
% \caption{$\mu_{F}$ and $\mu_{R}$ uncertainties on the $m_{\text{T}2}$ distribution in the slepton signal region.}
 \includegraphics[width=0.4\columnwidth]{/Users/sheenaschier/Documents/LaFiles/figures/thesis/systematics/alphaVars_Zttjets_mll_SR_hg_SFDF_shape.pdf}
 % \caption{$\alpha_{s}$ uncertainties on the $m_{\ell\ell}$ distribution in the Higgsino signal region.}
 \includegraphics[width=0.4\columnwidth]{/Users/sheenaschier/Documents/LaFiles/figures/thesis/systematics/alphaVars_Zttjets_mt2leplsp_100_SR_sl_SFDF_shape.pdf}
% \caption{$\alpha_{s}$ uncertainties on the $m_{\text{T}2}$ distribution in the slepton signal region.}
  \includegraphics[width=0.4\columnwidth]{/Users/sheenaschier/Documents/LaFiles/figures/thesis/systematics/PDFVars_Zttjets_mll_SR_hg_SFDF_shape_allPDFs.pdf}
 % \caption{PDF uncertainties on the $m_{\ell\ell}$ distribution in the Higgsino signal region.}
  \includegraphics[width=0.4\columnwidth]{/Users/sheenaschier/Documents/LaFiles/figures/thesis/systematics/PDFVars_Zttjets_mt2leplsp_100_SR_sl_SFDF_shape_allPDFs.pdf}
%\caption{PDF uncertainties on the $m_{\text{T}2}$ distribution in the slepton signal region.}
\caption{QCD scale, $\alpha_{s}$ and PDF uncertainties on the shape and normalization of the $Z\to\tau\tau$ background in the Higgsino and slepton signal regions (with no lepton flavor requirement).}
\label{fig:theoryUncsZtt}
 \end{figure}
 
  \begin{figure}
  \centering 
   \includegraphics[width=0.4\columnwidth]{/Users/sheenaschier/Documents/LaFiles/figures/thesis/systematics/scaleVars_alt_ttbar_PowPy8_dilep_hdamp258p75_mll_SR_hg_SFDF_shape.pdf}
 %\caption{$\mu_{F}$ and $\mu_{R}$ uncertainties on the $m_{\ell\ell}$ distribution in the Higgsino signal region.}
  \includegraphics[width=0.4\columnwidth]{/Users/sheenaschier/Documents/LaFiles/figures/thesis/systematics/scaleVars_alt_ttbar_PowPy8_dilep_hdamp258p75_mt2leplsp_100_SR_sl_SFDF_shape.pdf}
% \caption{$\mu_{F}$ and $\mu_{R}$ uncertainties on the $m_{\text{T}2}$ distribution in the slepton signal region.}
  \includegraphics[width=0.4\columnwidth]{/Users/sheenaschier/Documents/LaFiles/figures/thesis/systematics/PDFVars_alt_ttbar_PowPy8_dilep_hdamp258p75_mll_SR_hg_SFDF_shape_allPDFs.pdf}
 % \caption{PDF uncertainties on the $m_{\ell\ell}$ distribution in the Higgsino signal region.}
  \includegraphics[width=0.4\columnwidth]{/Users/sheenaschier/Documents/LaFiles/figures/thesis/systematics/PDFVars_alt_ttbar_PowPy8_dilep_hdamp258p75_mt2leplsp_100_SR_sl_SFDF_shape_allPDFs.pdf}
%\caption{PDF uncertainties on the $m_{\text{T}2}$ distribution in the slepton signal region.}
\caption{QCD scale and PDF uncertainties on the shape and normalization of the $t\bar{t}$ background in the Higgsino and slepton signal regions (with no lepton flavour requirement).}
\label{fig:theoryUncsttbar}
 \end{figure}
 
   \begin{figure}
  \centering 
   \includegraphics[width=0.4\columnwidth]{/Users/sheenaschier/Documents/LaFiles/figures/thesis/systematics/ttbarComp_mll_SR_hg_SFDF_allBjet_v2.pdf}
 %\caption{$m_{\ell\ell}$ distribution for $t\bar{t}$ in the Higgsino signal region.}
  \includegraphics[width=0.4\columnwidth]{/Users/sheenaschier/Documents/LaFiles/figures/thesis/systematics/ttbarComp_mt2leplsp_100_SR_sl_SFDF_allBjet_v2.pdf}
% \caption{$m_{\text{T}2}$ distribution for $t\bar{t}$ in the slepton signal region.}
  \includegraphics[width=0.4\columnwidth]{/Users/sheenaschier/Documents/LaFiles/figures/thesis/systematics/VVcomp_mll_SR_hg_SFDF_VVsafe.pdf}
 % \caption{$m_{\ell\ell}$ distribution for $VV$ in the Higgsino signal region.}
  \includegraphics[width=0.4\columnwidth]{/Users/sheenaschier/Documents/LaFiles/figures/thesis/systematics/VVcomp_mt2leplsp_100_SR_sl_SFDF_VVsafe.pdf}
%\caption{$m_{\text{T}2}$ distribution for $VV$ in the slepton signal region.}
\caption{Comparison of the $m_{\ell\ell}$ (left) and $m_{\text{T}2}$ (right) shapes predicted by different $t\bar{t}$ (top) and $VV$ (bottom) MC generators, in the Higgsino and slepton signal regions. All distributions are normalized to the same number of entries. The gray band displayed in the ratio pad under each distribution represents the modeling uncertainty assigned to each background in each of the bins.}
\label{fig:genComparisonBkgModelling}
 \end{figure}
\FloatBarrier
