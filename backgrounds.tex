\chapter{Background Estimation}
\label{ch:bkg}
Signal regions are specifically designed to be enriched in a signal process of interest, but SM backgrounds can still be present and need to be understood.  This chapter will detail how the SM backgrounds are evaluated for this analysis.  The general background estimation strategy is first discussed in Section~\ref{sec:bkg:summary}.  Next, Section~\ref{sec:bkg:irreduce} focuses on the estimation of irreducible ($t\bar{t}$, $Z(\rightarrow\tau\tau$) + jets, $VV\rightarrow\ell\nu\ell\nu$) backgrounds, and the last Section~\ref{sec:bkg:dy} describes the estimate of backgrounds from instrumental \met. The fake and non-prompt lepton background estimate is discussed is in Chapter~\ref{ch:fakefactor}.

\section{Summary of estimation strategy}
\label{sec:bkg:summary}
The vast majority, and possibly all, LHC pp collisions interact only through Standard Model processes, some of which lead to final states that look the same as Higgsino or slepton signals and pass the signal region cuts.  We classify two types of backgrounds, irreducible and reducible. Irreducible backgrounds are Standard Model processes that produce the same particle final state (two soft-leptons, \met, and jets) as our BSM final state.  In this case, Monte Carlo simulation is robust enough to model these background processes so their rates can be estimated in the data.  Reducible backgrounds arise from Standard Model events that in theory should not produce the same final state as signal events; and yet, because of mismeasurements inside the detector, can still pass signal selection cuts.  Table~\ref{tab:bkg:est} succinctly summarizes, from greatest to least contribution, the processes that constitute the SM background in the SRs and the methods used to estimate them. 

For a two soft-lepton, \met{}, and hard jet analysis, the dominant irreducible backgrounds come from diboson, $t\bar{t}$, $tW$, and $Z(\rightarrow\tau\tau)$+jets processes.  Diboson events are $WW$, $ZZ$, and $WZ$.   These backgrounds are evaluated with Monte Carlo and tested in a validation region that exploits $\met/H_T$.  Fully leptonic $WW$ production is the most prominent diboson background in the two lepton plus \met signal region.  The fully leptonic $WW$ decays lead to two real leptons that likely have opposite charge, but are not necessarily of the same flavor.  The real \met in the event comes from the neutrinos, and an additional hard jet must be present.  Fully leptonic $WZ$ events can also make their way into the SRs since there is certainly an oppositely signed same flavor lepton pair from the $Z$, and real \met from the neutrino in the $W$ decay, but since the SRs require exactly two leptons, the third lepton must either fall outside of acceptance or fail \pt, identification, or isolation cuts for this kind of event be selected.  Semi-leptonic $ZZ$ and $WZ$ processes can pass SR selection if one $Z$ decays into a proper lepton pair and the quarks from the other vector boson induce enough \met from mismeasured jet energies to pass the \met trigger.  Lastly, in the fully hadronic cases, there are four jets, two of which must be misidentified as leptons, leaving the others to induce a significant amount of \met.  Fully hadronic contributions are negligible.    

The top and $Z(\rightarrow\tau\tau)$+jets backgrounds are estimated with a semi-data-driven approach where the estimate is done in dedicated control regions enriched in the particular process.  A top quark decays to a b-quark and a $W$-boson nearly $100\%$ of the time.  In the event that a b-jet fails the b-tagging algorithm, each $t\bar{t}$ event can look like a diboson event with additional jets and some special topological features.  If both the $W$-bosons decay leptonically, you get two real leptons and \met from the neutrinos.  Similarly, insufficiently b-tagged $tW$ events with leptonically decaying $W$-bosons also supply two real leptons, jets, and real \met from neutrinos.  Even when one or both of the $W$-bosons decay hadronically, the $tW$ and $t\bar{t}$ processes can still produce backgrounds in the SRs.  This happens when one or two of the signal leptons arise from jets faking leptons in the detector.  These contributions are accounted for in the data-driven fake estimates.  In $Z(\rightarrow\tau\tau)$+jets, each leptonically decaying $\tau$-lepton produces one charged lepton and two neutrinos.  Together with the hadronic ISR radiation, that produces a final state with two leptons, real \met from the four neutrinos, and additional jets.  On the occasion that these leptons form an SFOS pair, this process will mimic signal events.   

Drell-Yan events, when a quark and an antiquark annihilate into and lepton/anti-lepton pair through the virtual exchange of a $\gamma^*/Z$, contribute to the SM backgrounds in Higgsino and slepton samples when enough \met is generated through jet energy mismeasurements.  Requiring \met{} above 200 GeV in the SRs causes the rate of this process to be low; therefore, Monte Carlo techniques are sufficient for estimating Drell-Yan backgrounds.  Other contributions estimated with pure Monte Carlo techniques are rare processes from: Higgs, triboson, and muti-top\footnote{Multi-top refers to the production or more that two top quarks in an event} production.  While detector effects that result in the misidentification of physics objects are not well modeled in simulation, misidentification still occurs during reconstruction.  Reducible fake lepton backgrounds are estimated with a data-driven method and therefore are already accounted for.  Background estimates done with Monte Carlo use only truth matched leptons to prevent overlap in the MC and data-driven estimates.
\begin{table}
%\begin{center}
\small
\begin{tabular}{lll}
\hline
\small Background Process  & \small Origin in Signal Region & \small Estimation Strategy \\
\hline 
\small Fakes ($W$+jets, $VV$(1$\ell$), $t\bar{t}$(1$\ell$) &\small Reducible, jet fakes $2^{nd}$ $\ell$ &\small Fake factor, same sign VR\\
\small $t\bar{t}$, $tW(2\ell)$  &\small Irreducible, b-jet fails ID  & \small CR using b-tagging\\
\small $Z\rightarrow(\tau\tau)$+jets  &\small Irreducible ($\tau\tau\rightarrow\ell\nu\ell\nu$) & \small CR using $m_{\tau\tau}$\\
\small $VV$ & \small Irreducible ($\ell\ell\ell$), missed $3^{rd}$ $\ell$ & \small MC, VR using \met/$H_T$ \\
\small $Z\rightarrow(ee, \mu\mu)$+jets  &\small Instrumental \met  & \small Monte Carlo (MC)\\
\small Low mass Drell-Yan  &\small Instrumental \met  & \small MC, data-driven cross check\\
\small Other rare & \small Irreducible leptonic decays & \small MC\\
\hline
\end{tabular}
\caption{Background estimation summary}
\label{tab:bkg:est}

\end{table}

\section{Irreducible Backgrounds}
\label{sec:bkg:irreduce}
This section describes the semi-data-driven techniques used to evaluate irreducible $t\bar{t}$, $tW$, $Z(\rightarrow\tau\tau)$+jets, and diboson backgrounds.  The approach to the $t\bar{t}$, $tW$, and $Z(\rightarrow\tau\tau)$+jets estimates requires defining new kinematic regions, called \textit{control regions} (CRs), that are enriched in these backgrounds.  Monte Carlo simulated events for these backgrounds in the SRs are then normalized in a simultaneous fit with their corresponding CR to constrain their contribution to the SR.  This fitting procedure is described in Chapter~\ref{ch:statanal}.  For this method to be valid, a CR must select events that are orthogonal to the SR to eliminate statistical correlations with the SRs, and yet are kinematically similar enough for a meaningful extrapolation.  To validate the extrapolation of Monte Carlo events in the SR to data events in the CR, a third region is defined called a \textit{validation region} (VR) that lies kinematically between the CR and SR and is orthogonal to both.

\subsection{Top Control Region (CR-top)}
The control region designed for the $t\bar{t}$ and $tW$ estimates (CR-top) is defined in this section.  One of the most unique aspect of the top-quark signature is the presence of $b$-jets.  To enrich a dilepton sample in top quarks, at least one b-tagged jet is required in each event.  CR-top is centered around this requirement.  The dilepton invariant mass is restricted to $m_{\ell\ell}<60~\GeV$ to stay kinematically consistent with the dilepton SRs.  Also, $\met/H_T$ is constrained to the region $[4,8]$ to reduce contamination from signal events with a fake b-tagged jet.  All leptons in $t\bar{t}$ and $tW$ decays are from leptonically decaying $W$s, and the electron and muon branching fractions are identical; so to increase statistics, different-flavor lepton pairs are also accepted in the CR-top selection.  Other than the selection criteria just described, CR-top includes all the common preselection cuts in Table~\ref{tab:cSR}. Figures~\ref{fig:CR-top-1} and~\ref{fig:CR-top-2} show pre-fit distributions in CR-top.  The purity of $t\bar{t}$ and $tW$ events is $72\%$ with a signal contamination of less than $3\%$

\begin{figure}
    \centering
    \includegraphics[width=0.47\columnwidth]{/Users/sheenaschier/Documents/LaFiles/figures/thesis/backgrounds/CRtop/hist1d_met_Et_CR-top-AF}
    \includegraphics[width=0.47\columnwidth]{/Users/sheenaschier/Documents/LaFiles/figures/thesis/backgrounds/CRtop/hist1d_nJet30_CR-top-AF}
    \includegraphics[width=0.47\columnwidth]{/Users/sheenaschier/Documents/LaFiles/figures/thesis/backgrounds/CRtop/hist1d_nBJet20_MV2c10_CR-top-AF}
    \includegraphics[width=0.47\columnwidth]{/Users/sheenaschier/Documents/LaFiles/figures/thesis/backgrounds/CRtop/hist1d_METOverHTLep_CR-top-AF}
     \includegraphics[width=0.47\columnwidth]{/Users/sheenaschier/Documents/LaFiles/figures/thesis/backgrounds/CRtop/hist1d_DPhiJ1Met_CR-top-AF}
     \includegraphics[width=0.47\columnwidth]{/Users/sheenaschier/Documents/LaFiles/figures/thesis/backgrounds/CRtop/hist1d_minDPhiAllJetsMet_CR-top-AF}
        
    \caption{CR-top $ee+\mu\mu +e\mu + \mu e$ channel, pre-fit distributions.}
    \label{fig:CR-top-1}
\end{figure} 

\begin{figure}
    \centering
        \includegraphics[width=0.47\columnwidth]{/Users/sheenaschier/Documents/LaFiles/figures/thesis/backgrounds/CRtop/hist1d_lep1Pt_CR-top-AF}
        \includegraphics[width=0.47\columnwidth]{/Users/sheenaschier/Documents/LaFiles/figures/thesis/backgrounds/CRtop/hist1d_lep2Pt_CR-top-AF}
        \includegraphics[width=0.47\columnwidth]{/Users/sheenaschier/Documents/LaFiles/figures/thesis/backgrounds/CRtop/hist1d_MTauTau_CR-top-AF}
        \includegraphics[width=0.47\columnwidth]{/Users/sheenaschier/Documents/LaFiles/figures/thesis/backgrounds/CRtop/hist1d_mll_CR-top-AF}
        \includegraphics[width=0.47\columnwidth]{/Users/sheenaschier/Documents/LaFiles/figures/thesis/backgrounds/CRtop/hist1d_Rll_CR-top-AF}
        \includegraphics[width=0.47\columnwidth]{/Users/sheenaschier/Documents/LaFiles/figures/thesis/backgrounds/CRtop/hist1d_mt2leplsp_100_CR-top-AF}
        
    \caption{CR-top $ee+\mu\mu +e\mu + \mu e$ channel, pre-fit distributions.}
    \label{fig:CR-top-2}
\end{figure}
%\FloatBarrier
\subsection{Ditau Control Region (CR-tau)}
The control region used to estimate $Z\rightarrow\tau\tau$ + jets backgrounds (CR-tau) is defined in this section.  Getting a handle of the invariant mass of a ditau system is the clearest approach to constructing a dilepton sample enriched in $Z\rightarrow\tau\tau$ + jets events, and the $m_{\tau\tau}$ variable, described in section~\ref{sec:sr:discvar}, is a good proxy for this.  Events in CR-tau are required to have an $m_{\tau\tau}$ between $60 \GeV$ and $120 \GeV$ as a way to envelope the $Z$ mass.  There are also upper and lower bounds on $\met/H_T$.  Just like with CR-top, lepton universality makes different-flavor lepton pairs probabilistically equivalent to same-flavor pairs, so they are also accepted in the CR-tau selection.  Figures~\ref{fig:CR-tau-1} and~\ref{fig:CR-tau-2} show pre-fit distributions of the some of the variables used to define the Higgsino and slepton signal regions as show above for CR-top.  The purity of $Z\rightarrow\tau\tau$ + jets events in CR-tau is $80\%$ with a signal contamination of less than $3\%$
\begin{figure}
    \centering
        \includegraphics[width=0.48\columnwidth]{/Users/sheenaschier/Documents/LaFiles/figures/thesis/backgrounds/CRtau/hist1d_met_Et_CR-tau-AF}
        \includegraphics[width=0.48\columnwidth]{/Users/sheenaschier/Documents/LaFiles/figures/thesis/backgrounds/CRtau/hist1d_nJet30_CR-tau-AF}
        \includegraphics[width=0.48\columnwidth]{/Users/sheenaschier/Documents/LaFiles/figures/thesis/backgrounds/CRtau/hist1d_nBJet20_MV2c10_CR-tau-AF}
        \includegraphics[width=0.48\columnwidth]{/Users/sheenaschier/Documents/LaFiles/figures/thesis/backgrounds/CRtau/hist1d_METOverHTLep_CR-tau-AF}
        \includegraphics[width=0.48\columnwidth]{/Users/sheenaschier/Documents/LaFiles/figures/thesis/backgrounds/CRtau/hist1d_DPhiJ1Met_CR-tau-AF}
        \includegraphics[width=0.48\columnwidth]{/Users/sheenaschier/Documents/LaFiles/figures/thesis/backgrounds/CRtau/hist1d_minDPhiAllJetsMet_CR-tau-AF}
    \caption{CR-tau $ee+\mu\mu +e\mu + \mu e$ channel, pre-fit distributions.}
    \label{fig:CR-tau-1}
\end{figure} 

% set 2 vars ee
\begin{figure}
    \centering
        \includegraphics[width=0.48\columnwidth]{/Users/sheenaschier/Documents/LaFiles/figures/thesis/backgrounds/CRtau/hist1d_lep1Pt_CR-tau-AF}
        \includegraphics[width=0.48\columnwidth]{/Users/sheenaschier/Documents/LaFiles/figures/thesis/backgrounds/CRtau/hist1d_lep2Pt_CR-tau-AF}
        \includegraphics[width=0.48\columnwidth]{/Users/sheenaschier/Documents/LaFiles/figures/thesis/backgrounds/CRtau/hist1d_MTauTau_CR-tau-AF}
        \includegraphics[width=0.48\columnwidth]{/Users/sheenaschier/Documents/LaFiles/figures/thesis/backgrounds/CRtau/hist1d_mll_CR-tau-AF}
        \includegraphics[width=0.48\columnwidth]{/Users/sheenaschier/Documents/LaFiles/figures/thesis/backgrounds/CRtau/hist1d_Rll_CR-tau-AF}
        \includegraphics[width=0.48\columnwidth]{/Users/sheenaschier/Documents/LaFiles/figures/thesis/backgrounds/CRtau/hist1d_mt2leplsp_100_CR-tau-AF}
    \caption{CR-tau $ee+\mu\mu +e\mu + \mu e$ channel, pre-fit distributions.}
    \label{fig:CR-tau-2}
\end{figure} 

\FloatBarrier

\subsection{Diboson Validation Region (VR-VV)}
Constructing a diboson CR that is free of signal contamination and pure enough to diboson event to confidently normalize Monte Carlo simulation to data is not easy.  Therefore, diboson backgrounds are estimated with Monte Carlo samples.  The background estimate and the associated uncertainties are validated in a dedicated kinematic region called the \textit{diboson validation region} (VR-VV).  This region uses all the same selection cuts as the common signal region in Table~\ref{tab:cSR} with an additional $\met/H_T<3.0$ requirement to reduce signal events that typically will populate high $H_T$.  Figures~\ref{fig:VR-VV-AF-set1vars} and~\ref{fig:VR-VV-AF-set2vars} show VR-VV distributions of some of the variables used to define the Higgsino and slepton signal regions.  The VR-VV composition is approximately $40\%$ diboson events, $25\%$ fake/non-prompt leptons events, $23\%$ $t\bar{t}$ and $tW$ events, about $5\%$ $Z\rightarrow\tau\tau$ + jets events, and not more than $8\%$ signal events.

\begin{figure}
    \centering
        \includegraphics[width=0.48\columnwidth]{/Users/sheenaschier/Documents/LaFiles/figures/thesis/backgrounds/VRVV/hist1d_met_Et_VR-VV-AF}
        \includegraphics[width=0.48\columnwidth]{/Users/sheenaschier/Documents/LaFiles/figures/thesis/backgrounds/VRVV/hist1d_nJet30_VR-VV-AF}
        \includegraphics[width=0.48\columnwidth]{/Users/sheenaschier/Documents/LaFiles/figures/thesis/backgrounds/VRVV/hist1d_nBJet20_MV2c10_VR-VV-AF}
        \includegraphics[width=0.48\columnwidth]{/Users/sheenaschier/Documents/LaFiles/figures/thesis/backgrounds/VRVV/hist1d_METOverHTLep_VR-VV-AF}
        \includegraphics[width=0.48\columnwidth]{/Users/sheenaschier/Documents/LaFiles/figures/thesis/backgrounds/VRVV/hist1d_DPhiJ1Met_VR-VV-AF}
        \includegraphics[width=0.48\columnwidth]{/Users/sheenaschier/Documents/LaFiles/figures/thesis/backgrounds/VRVV/hist1d_minDPhiAllJetsMet_VR-VV-AF}
    \caption{VR-VV $ee+\mu\mu +e\mu + \mu e$ channel, pre-fit distributions.}
    \label{fig:VR-VV-AF-set1vars}
\end{figure} 


% set 3 vars ee
\begin{figure}
    \centering
        \includegraphics[width=0.48\columnwidth]{/Users/sheenaschier/Documents/LaFiles/figures/thesis/backgrounds/VRVV/hist1d_lep1Pt_VR-VV-AF}
        \includegraphics[width=0.48\columnwidth]{/Users/sheenaschier/Documents/LaFiles/figures/thesis/backgrounds/VRVV/hist1d_lep2Pt_VR-VV-AF}
        \includegraphics[width=0.48\columnwidth]{/Users/sheenaschier/Documents/LaFiles/figures/thesis/backgrounds/VRVV/hist1d_MTauTau_VR-VV-AF}
        \includegraphics[width=0.48\columnwidth]{/Users/sheenaschier/Documents/LaFiles/figures/thesis/backgrounds/VRVV/hist1d_mll_VR-VV-AF}
        \includegraphics[width=0.48\columnwidth]{/Users/sheenaschier/Documents/LaFiles/figures/thesis/backgrounds/VRVV/hist1d_Rll_VR-VV-AF}
        \includegraphics[width=0.48\columnwidth]{/Users/sheenaschier/Documents/LaFiles/figures/thesis/backgrounds/VRVV/hist1d_mt2leplsp_100_VR-VV-AF}
    \caption{VR-VV $ee+\mu\mu +e\mu + \mu e$ channel, pre-fit distributions.}
    \label{fig:VR-VV-AF-set2vars}
\end{figure}

\subsection{Different Flavor Validation Regions (VR-DF)}
Additional validation regions are established to check the extrapolation of the fitted Monte Carlo predictions of the irreducible top and ditau backgrounds in the inclusive and exclusive Higgsino and slepton SR, defined in Chapter~\ref{ch:sr}.  These VRs take advantage of the flavor symmetry of the $t\bar{t}$, $tW$, $Z(\rightarrow\tau\tau)$+jets processes by selecting only events with different-flavor same-sign lepton pairs in concert with all the other signal region cuts.  This includes the inclusive and exclusive binning in $m_{\ell\ell}$ and $m_{\mathrm{T}2}^{100}$ that define the Higgsino and slepton signal regions.

\section{Drell-Yan Background}
\label{sec:bkg:dy}
Drell-Yan (DY) occurs when two quarks annihilate to produce a lepton pair through the exchange of a virtual $Z^*/\gamma$.  The invariant mass of the two leptons from this process shows a smooth off-resonance distribution marked by on-resonance peaks at the $J/\psi$, $\Upsilon$, and $Z$ masses near $3$, $10$, and $100\GeV$.  Figure~\ref{fig:mll_dy} shows the strong presence of DY in data events with two same-sign electrons or muons that pass the inclusive \met triggers.  These events can have sizable jet activity, but there is no source of real \met{}, so pass the \met{} trigger and pass signal region selection, a large amount of instrumental \met{} from calorimeter jet mismeasurements must be present.  A cut on $m_{\ell\ell}$ between $3$ and $3.2\GeV$ removes contributions in the $J/\psi$ peak.  Secondly, the signal region cut $min|\Delta\phi(\mathrm{all jets}, \pt^\mathrm{miss})|>0.4$ removes events where the might is very aligned with a single jet, which reduces the occasion of high instrumental \met from a single mismeasured jet in the detector.  DY background contribution significantly, making it small enough to estimate reliably with Monte Carlo. 

 \begin{figure}
 \centering
    \includegraphics[width=0.8\columnwidth]{/Users/sheenaschier/Documents/LaFiles/figures/thesis/backgrounds/dataCR_mll_MuMu_pre.pdf}
  % \caption{Di-muon.}
 \includegraphics[width=0.8\columnwidth]{/Users/sheenaschier/Documents/LaFiles/figures/thesis/backgrounds/dataCR_mll_ElEl_pre.pdf}
%  \caption{Di-electron.}
  \caption{Data events passing inclusive \met{} triggers with opposite sign baseline leptons in the dilepton invariant mass $m_{\ell\ell}$ spectrum. The $\Delta\phi(j_1, \mathbf{p}_\mathrm{    T}^\mathrm{miss})$ variable is inverted to ensure this is orthogonal to the signal region.}
  \label{fig:mll_dy}
 \end{figure}
\FloatBarrier



