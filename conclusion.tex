\chapter{Conclusion}
\label{ch:conclusion}
A search for supersymmetry in scenarios with compressed mass spectra was performed using ATLAS data, collected in 2015 and 2016 at $\sqrt(s)$ 13 TeV, corresponding to $36.1 fb^{-1}$. % \textcolor{red}{Motivate the searches for compressed electroweakinos and sleptons.}  
We searched for directly produced electroweakinos and sleptons in events containing two soft, opposite-sign same-flavor leptons with an intermediate amount of missing transverse energy and a hard jet.  Signal event characteristics are studied with Higgsino and slepton simplified models.  The directly produced electroweakinos and sleptons subsequently decay to a lightest SUSY particle which is nearly degenerate in mass and their Standard Model partners.  The energy of the visible leptons is related to the mass-splitting between the neutral electroweakinos $\tilde\chi_2^0$ and $\tilde\chi_1^0$ or between the sleptons $\tilde\ell_{L,R}$ and the lightest neutral electroweakino $\tilde\chi_1^0$.  The relationship between lepton momentum and the mass-splittings provides discriminating variables unique to the electroweakino and slepton decays.  Electroweakino signals are sensitive to the invariant mass of the dilepton system, $m_{\ell\ell}$, and slepton signals are sensitive to the stransverse mass of the \met and leptons, $m_{T2}^{100}$.  Inclusive and exclusive signal regions are binned in $m_{\ell\ell}$ for the searches targeting electroweakino production, and in $m_{T2}^{100}$ for the search targeting sleptons.
 
The dominant backgrounds to signal event with soft leptons and \met are from jets faking leptons in the detector.  These are estimated with a data-driven fake factor technique and tested in a same-sign validation region that includes $ee+\mu e$ events in the electron channel, and $\mu\mu+e\mu$ events in the muon channel.  Irreducible backgrounds from $t\bar t$, $tW$, and $Z(\rightarrow\tau\tau)$+jets processes were estimated with Monte Carlo and normalized in data-driven control regions.  Irreducible diboson backgrounds were estimated with Monte Carlo and tested in a dedicated diboson validation region.  Low mass Drell Yan, Higgs, triboson, and multi-top backgrounds were estimated with Monte Carlo only.

Background only fits were performed on CR-top and CR-tau to obtain background normalization parameters $\mu_{top} =0.72\pm0.13$ and $\mu_{\tau\tau}=1.02\pm0.09$, respectively.  The accuracy of the background prediction was tested in each of the validation regions and is consistently within $1.5~\sigma$ of the observed data.  Model independent upper limits were set at $95\%$ CL on the observed and expected upper limits on the number of signal events in the inclusive SRs were set with simultaneous fits in each SR and the CRs, assuming the background only hypothesis.  No significant excess in data over Standard Model background was found; therefore, results were consistent with Standard Model prediction. 

For model dependent interpretations, shape fits in $m_{\ell\ell}$ and $m_{T2}^{100}$ were performed.  These are full simultaneous fits over the exclusive, multi-binned SRs and the CRs including both signal and backgrounds predictions.  In the absence of significant excesses in data over background, result were interpreted as constraints on SUSY electroweakino and slepton models.  Higgsino models are excluded for next-to-lightest neutralino masses up to 130 GeV for mass splittings between 5 and 10 GeV. For mass splittings down to 3 GeV next-to-lightest neutralino masses are excluded up to 100 GeV.  For wino-bino simplified models, next-to-lightest neutralino masses are excluded up to 170 GeV for mass splittings above 10 GeV, and excluded up to 100 GeV for mass splittings down to 2.5 GeV.  For slepton simplified models, slepton masses are excluded up to 180 GeV for mass splittings down to 5 GeV. For mass splittings down to 1 GeV slepton masses are excluded up to 70 GeV.

Figure~\ref{fig:final} summarizes the current limits on compressed electroweak SUSY set by LEP in 2001, the ATLAS Run 2 disappearing track analysis, and the two-soft-lepton analysis that is the subject of this thesis.  Future versions of the two-soft-lepton analysis will try to extend the reach in the $\Delta m(\tilde\chi_1^\pm\tilde\chi_1^0)$, $m(\tilde\chi_1^\pm)$ phase space, with different techniques designed to extend the limits in different directions.  To reach farther in $m(\tilde\chi_1^\pm)$ or $m(\tilde\chi_2^0)$ will require more data to overcome the falling cross-sections as the chargino and neutralino masses grow.  In 2017 alone, ATLAS doubled the amount of data it took in 2015 and 2016 combined, and data-taking for 2018 is currently underway.  Extending the search to target mass-splittings above $10\GeV$ will require a new optimization of the electroweakino and slepton signal regions.  For mass-splittings below $10\GeV$, $m_{\ell\ell}$ and $m_{T2}^{100}$ are powerful discriminators for compressed signals, but once the mass-splittings rise, the kinematic end-point still exists, but the distributions begin to flatten, washing-out some of the signal shape.  Re-optimising the signal regions with the use of recursive jigsaw variables~\cite{Jackson:2016mfb} that exploit boosted, compressed systems when the \met associated with the invisible LSPs gets most of its energy from the kick of an ISR jet, may help to recover signal over backgrounds for larger mass-splittings.  Lastly, the limiting factor on the minimum mass-splittings available to this search is the minimum \pt at which leptons are reconstructed.   A similar version of this analysis, searching for a single identified lepton and an isolated track, is a possible way to get around the \pt limit for reconstructed leptons.  Each of these modifications are being studied, and a new versions of this search using the full Run 2 data set are under construction.
 \begin{figure}%[h!]
  \begin{center}
  \includegraphics[width=0.9\textwidth]{/Users/sheenaschier/Documents/LaFiles/figures/SUSY_approval/APS/money.png}
   \end{center}
 \caption{Summary plot for compressed electroweak searches, including combined LEP limits in grey.  The ATLAS Run 2 disappearing track analysis, which targets mass-splittings $\mathcal{O}1\GeV$, is shown in orange.  The two-soft-lepton analysis described in this thesis is shown in light blue.}
 \label{fig:final}
 \end{figure}
 



