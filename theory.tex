\chapter{The Standard Model of Particle Physics and Additional Theories}
The universe seems like a very complex place made out of many types of material that interacts by various complex mechanisms.  Although this will always remain true, particle physics has constructed a theory that incorporates all fundamental particles and explains their existence and interactions in simplicity through the field equations that describe the fundamental forces in the universe.  This theory is called the Standard Model of Particle Physics and, apart from the absence of gravity which is far too weak to be described by particle interactions, is fundamentally complete.


\section{Forces and Particles}
\label{sec:theory1}
The Standard Model describes three of the four known fundamental forces or our universe; electromagnetic force, the strong force, and the weak force.  It leaves out the gravitational force only because the energy scale at which gravity does its business is so many orders of magnitude below the other forces that there are intrinsic incompatibilities in their description of particle interactions.  According to experiment, there are only a hand full of fundamental particles, among which can be separated into two distinct categories: fermions and bosons.  These two types of particles play completely different roles in the state and phenomena of the universe. \cite{tully}

You and I and the entire world we experience is comprised of fermions, spin $\frac{1}{2}$ Dirac particles, that can be further categorized as leptons and quarks depending on their intrinsic propensity to interact with a given fundamental force field or not.  The lepton family consists of three types of electron ($e, \mu, \tau$) and their associated neutrino partners ($\nu_e, \nu_\mu, \nu_\tau$).

\section{Mathematical Formalism of the Standard Model}
\label{sec:theory2}
Field equations and fermion and boson interactions.

\section{Gauge Symmetries and Spontaneous Symmetry Breaking}
\label{sec:theory3}
$U(1)_{Y} \times SU(2)_{L} \times SU(3)_{S}$

\section{Higgs Mechanism and Gauge Boson Masses}
\label{sec:theory4}
Explain local SU(2) gauge symmetry breaking, the production of the Higgs boson and how this allows for massive weakly interacting gauge bosons.

\section{Shortcomings of the Standard Model}
\label{sec:theory5}
\subsection{Darkmatter}
One alarming problem with the Standard Model is its incapability the explain dark matter.

\section{Supersymmetry}
\label{sec:theory6}
Extension of the Poincare Group which leads to boson-fermion symmetry.

\section{Phenomenology of Direct Production of Higgsinos and Sleptons in Compressed Scenarios}
Here talk about the discriminating variables for the higgsino and slepton signal regions (mll and mt2).