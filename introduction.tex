\chapter{Introduction}

Since the world's first particle accelerator went online in the 1930's at the Cavendish Laboratory, colliding protons against a fixed lithium target, particle collisions have been providing physicists with portals into the subatomic realm where quantum physics is the supreme ruler.  Progressively, accelerators have become more and more powerful, and the depth at which physicists can peer into the atom, into the structure of neutrons and protons, and eventually into interactions of the most fundamental, has hastened.  Today, we stand at the frontier of high energy physics experiments, with the Standard Model of Particle Physics in hand, a theory that could appear as a complete map of fundamental particles and interactions, to guide us through the sea of quantum possibilities, while astronomical observations, for one, give us the distinct sense that we are holding only a small slice of the key.  

Another historic event in the 1930's was the first hint of dark matter in astronomical observation.  J.H. Oort, the namesake of the Oort Cloud, measured the velocities of stars using their Doppler shifts.  Surprisingly, the galactic mass that binds stars in their gravitational orbits should not be strong enough to overcome their velocities, and the stars should escape.  It wasn't until 40 years later, that Vera Rubin, with her studies or galactic rotations, actually opened the eyes of science to the alarming possibility of a new kind of matter, so different from the electrically interacting matter that makes our universe observable, that we call it 'dark'.  Dark matter is now observed to be so abundant in the universe, we believe there is five times more of it than the matter that constitutes all the stars, planets, gas clouds, and anything else made of Standard Model particles.  But this extraordinary matter might not be completely dark.  It might couple to the Standard Model extremely weakly, and when it does, we hope to be there to witness.

The European Organization for Nuclear Research, or Conseil Europ\'een pour la Recherche Nucl\`eaire (CERN), began as an official scientific union between 12 European countries in 1954, when engineers started digging the first hole near Geneva, Switzerland, which marked the beginning of a new era of particle collisions.  Experiments at CERN have been heroes in electroweak physics, with discoveries of the $W^\pm$ and $Z^0$ bosons in 1983 at the Super Proton Synchrotron (SPS), and the discovery of what so far looks sufficiently like the Standard Model Higgs boson in 2012 at the Large Hadron Collider (LHC).  Currently, the LHC is the largest and most powerful accelerator on Earth, colliding protons with a center of mass energy of $13~\TeV$.  With this machine, we step into the energy scales of the early universe before thermal freeze-out, when the universe became too cold for dark matter production or any other interaction at the dark matter physics-scale.  There is a chance the LHC will produce dark matter particles if they interact with the electroweak force.  There is also the chance to produce a plethora of other particles that do not account for dark matter, but are motivated by predictive new physics models. 

Some of these models predict a new symmetry, and with this new symmetry, a new particle paired to each Standard Model constituent.  A generic extension of the SM that introduces a new weak multiplet would naturally have a similar structure as the $W$ and $Z$ bosons with nearly degenerate masses.  If the lightest neutral weak particle is stable it may explain the abundance of dark matter in the universe.  On the occasion that the candidate dark matter particle would produce an overabundance of dark matter, this can be mitigated by the coannihilation of dark matter with some other similar-mass state, as a way to dilute the DM abundance in the early universe.  This requires semi-compressed spectra between the lightest neutral weak particle and the coannihilation states.  If any of these particles exist, the LHC may capable of producing them, but if their mass spectra are compressed, meaning the masses are within a few $\GeV$ of each other, the signatures of these events will be hard to resolve in a detector.  It takes dedicated teams and a lot of strategy to do physics at the edge of detector limits. 

This thesis presents a search for new compressed electroweak physics marked by "soft', low-momentum leptons and a sufficient amount of energy deduced to have left the detector unseen.  The analysis is broken into three parts.  Part 1 will engage the theories that give context to this search and will also describe the LHC and the particle detector used for the experiment.  Part 2 will describe all the work done in performing the analysis, and Part 3 will overview the uncertainties, results, and interpretations.

Enjoy.







