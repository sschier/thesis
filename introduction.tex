\chapter{Introduction}

Since the 1930's, when the world's first particle accelerator went online at the Cavendish Laboratory in Cambridge, England, colliding protons against a fixed lithium target, high energy collisions have been proving physicists with portals into the subatomic realm where quantum physics is the supreme ruler.  Progressively, particle accelerators have become more and more powerful, and the depth at which physicists can peer into the atom, into the structure of particles, and eventually into interactions of the most fundamental, has hastened.  Today, we stand at the energy frontier of particle experiments with a complete map of fundamental particles and interactions in hand to guide us through the sea of quantum possibilities, while astronomical observations, for one, give us the distinct sense that we are holding only a small slice of the truth.  

Currently, the Large Hadron Collider (LHC) is the largest and most powerful accelerator on Earth, colliding protons with a center of mass energy of $13~\TeV$.  With this machine, we step into the realm of Big Bang physics, where all possibilities 




