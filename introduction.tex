\chapter{Introduction}

Particle accelerators have been around since such and such time and have been the source of the most precise physics measurements and lead to deeper understanding of the most fundamental forces and particles in nature.  Discovery of top quarks at this collider and the W and Z bosons.  Da da da LEP $e^+e^-$ machine blah blah.  The Large Hadron Collider pushing into unseen energy realms has the imminent task of finding the Higgs boson, but that is not all.  There must be more, and it was believed for decades new physics was right around the corner from LEP(and others).  

This analysis relies on external predictions of the background and signal pieces in the data to help interpret the observations.  We define a region of phase space through a series of selection cuts on kinematic variables targeting a region in phase space where there is a significant excess in the signal events over the predicted background events.  This type of signal enriched region in phase space is called a \textit{signal region}, or SR.  While signal regions are enriched in the process of interest, backgrounds are still present, and to estimate the background contamination in the SR, a semi-data-driven approach is employed.  This requires regions enriched with certain backgrounds, for instance top or $Z\rightarrow\tau\tau$, that are free of signal contamination.  These regions are called \textit{control regions}.  Lastly, to validate the model used to predict the background contribution to the signal regions in data, another region in kinematic phase space, called the \textit{validation region} or VR, is defined in a way that maximizes the region's statistical significance while minimizing its signal contamination.  Kinematically positioned between the CR and the SR, it should also help mediate the assumptions made in the CR to SR extrapolation.  The extrapolation happens in variables chosen to separate the control regions and bins within them.

Keeping the SRs and CR statistically independent means they can be described by different probability density functions and can be combined into a simultaneous fit. Channels are multi-binned distributions of $m_{\ell\ell}$ and $m_{T2}$.  A channel object can represent a CR, SR, or VR.  Samples are components of RooFit probability density functions that are decorated by HistFitter meta-data and correspond to a specific physics process.  These samples can be defined in a specific channel or simultaneously over multiple channels.  Systematic uncertainties are taken into account for each samples by providing HistFitter with a distribution representing the best possible available prediction.  \textcolor{blue}{For  each  model  component,  a  nominal  distribution  representing  the  best  available  prediction  is
typically provided to the physics analysis as a histogram owned by a Sample object.  These components typically have systematic uncertainties whose impact gets quantified in dedicated studies. This is often modeled as variations of one standard deviation around the nominal prediction, provided to the physics analysis as sets of two additional histograms.  The systematic uncertainties are parameterized in the PDF as Gaussian distributed nuisance parameters.}
