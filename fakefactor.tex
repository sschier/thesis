\chapter{Fake Factor Method}
\label{ch:fakefactor}
 For low pt dilepton signals, the primary reducible backgrounds are from fake leptons in W+jets events where one jets is misidentified as a lepton.  These backgrounds are estimated with a data-driven fake factor Monte Carlo simulation does not model the detector shortcomings that lead to these mismeasurements very well, so the best estimate of this background must comes from data.  The "fake factor" method is a data driven approach to modeling backgrounds from particle misidentification in the detector by estimating the lepton fake rate with a set of data kinematically enriched in events producing fake leptons.  The background estimate is validated in an orthogonal control region before it is estimated in the signal region. 

The rest of this chapter goes as follows:  Fake leptons backgrounds are introduced in Section~\ref{sec:FFintro}, then general overview of the method used to estimate the fake lepton background for this analysis is given in Section~\ref{sec:FFdesc}.  Next, the fake factor method applied to low \pt{} di-electron and di-muon events is explained in Section~\ref{sec:FFmethod}.  Finally, the validation of the fake background estimates is discussed in Section{sec:ssvr}, and the results are summarized in Section~\ref{sec:FFcon}.

%%%%%%%%%%%%%%%%%%%%
\section{Introduction}
\label{sec:FFintro}
\begin{itemize}
\item lepton identification and misidentification
\item Compare production cross-sections of signal and W+jets processes
\item Sources of electron and muon misidentification
\item How to model backgrounds from misidentification (can't use MC, must choose data driven method)
\item Concept of fake factor method
\item Primary fake background is W+jets (multi-jet is miniscule...  how do I qualify this?)
\item Rest of chapter describes FF method in the context of my analysis
\end{itemize}
Efficient lepton identification techniques make leptons powerful discriminators in ATLAS physics searches with large background rejection and heavily suppressed QCD multi-jets.  Jet suppression is very high in the range of lepton $\pt > 20\GeV$ but degrades at lower lepton $\pt$.  Misidentified electrons can be true but non-prompt electrons from photon conversions and heavy-flavor decays, where there is a real electron in the event that does not originate at the primary vertex like true, prompt electrons or they can be charged hadrons where the  hadronic jet activity in the detector fakes an electron.  
\begin{figure}[h!]
 \centering
 \includegraphics[width=0.49\columnwidth]{/Users/sheenaschier/Documents/LaFiles/figures/thesis/fakes/fig_01.pdf}
  \includegraphics[width=0.49\columnwidth]{/Users/sheenaschier/Documents/LaFiles/figures/thesis/fakes/fig_02.pdf}
 \caption{Electron identification efficiency}
 \label{fig:electronID}
 \end{figure}

**W+jets x-sec - \textcolor{red}{https://arxiv.org/pdf/1409.8639.pdf} and Higgsino/Slepton x-sections shown in Chapter 4.
 \FloatBarrier
 
 %%%%%%%%%%%%%%%%%%%%%%%%%
\section{Description of Fake Factor Method}
\label{sec:FFdesc}
Fake factor method is a data-driven technique for modeling fake rates in data.  The general approach is to measure the number of fake events in a control region that is kinematically equivalent to the signal region, then apply the measurement to the signal regions through a transfer factor.  The transfer factor, called the \textit{fake factor}, is measured in a kinematic region dominated by fake leptons.  Within the measurement region, two classes of lepton are defined.  \textit{ID} leptons, which are usually the same as the signal lepton, and \textit{anti-ID} leptons, which are required to fail certain signal lepton cuts.  An important feature of the anti-ID definition is that it mimics the fake lepton composition in the primary primary source; in this case, W($\rightarrow\ell\nu$)+jets.  Electrons and muons are treated separately.  Fake factor is the ratio of leptons passing analysis lepton identification criteria to the leptons passing anti-identification criteria, measured in a region of kinematic phase space contrived to be enriched in fake leptons.  This will be considered as the fake factor measurement region in this thesis.  The control region is meant to select events with misidentified leptons, is defined by signal region cuts but with one lepton chosen to satisfy a selection criteria that is more likely to include more misidentified particles than that used in the analysis signal region.  A control region designed to capture W+jets events where a jet is misidentified as a lepton would be the same as the signal region requiring two leptons, but only one lepton is defined as an analysis lepton, while the other has at least one orthogonal selection criteria cut that makes it easier to include jets in the container of lepton identified in this particular way.  Fake background contribution in the signal region is estimated by scaling the number of selected events in the control region by the fake factor.  Control region and anti-ID lepton definition might have contamination from sources that are not from the background of interest..

\begin{figure}
\centering
 \input{/Users/sheenaschier/Documents/LaFiles/figures/thesis/fakes/fakefactor_schematic.tex}
 \caption{Schematic illustrating the fake factor method to estimate the fake lepton contribution in the signal region.}
 \label{fig:fake_schematic}
 \end{figure}
 
The fake factors are computed from events  with $m_{\mathrm{T}}<40\GeV$, using the distributions in Fig.~\ref{fig:elec_FF_dists_pt}, as:
\begin{equation}
  F(\pt) = \frac{\mathrm{Numerator}_{\mathrm{data}} - \mathrm{Numerator}_{\mathrm{MC}}}{\mathrm{Denominator}_{\mathrm{data}} - \mathrm{Denominator}_{\mathrm{MC}}}
\end{equation}
  \FloatBarrier
  
  %%%%%%%%%%%%%%%%%%%%%%%%%%%%%%%%%%%
  \section{Fake Factor Method Applied to Low-$\pt$ Di-lepton Events}
  \label{sec:FFmethod}
As described in the previous section, the fake background contribution is estimated in a control region, which is scaled to the signal regions by a fake factor.  The fake factor is measured in a region of data that is selected to be enriched with fake lepton events.  In this region, two classes of lepton, ID and anti-ID, are defined, and the fake factor is equal to the ratio between the occurrence of these leptons in the measurement region.  The measurement regions and control regions, along with the ID and anti-ID leptons definitions used to measure the electron and muon fake factors, are detailed in this section.  \textcolor{red}{Mention same-sign validation region.  Also, that Monte Carlo samples are used to estimate and subtract of the prompt lepton contamination in the fake factor measurement region.}

Electron and muon fake factors are measured using 2015+2016 LHC pp data taken by the ATLAS detector with single electron and muon triggers, called the \textit{fake factor measurement sample} or \textit{FF sample}.  The lepton trigger thresholds are chosen to accept the lowest \pt leptons possible and still maximize the event statistics.  The triggers used for this dataset are summarized in Table~\ref{tab:prescaledtrigs}.  In ATLAS, single electron and muon triggers with thresholds below 24 GeV are subject to prescales\footnote{The lowest unprescaled electron and muon trigger threshold evolved to 26 GeV by the end of 2016 data-taking.} because their true rates are too high for every event to be kept.  To resolve the different prescales applied to each trigger, they are unfolded to normalize the entire 2015+2016 dataset arbitrarily to $10\ipb$.  For this, and the rest of the discussion of fake factors, the electron and muon samples are treated separately.  FF samples are subject to an offline preselection of events with at least two baseline jets according to the jet object definitions summarized in Table~\ref{tab:objdef}.
\begin{table}[tbp]
  \centering
  \begin{tabular}{lll}
    \hline
    
    \hline
   Trigger Threshold                            &\multicolumn{2}{c}{Prescaled Luminosity [\ipb]}\\
                                        &2015           &2016\\
    \hline
       
   \hline
   Single Electron Trigger  \\
   \hline 
    5 GeV            &0.1               &0.1    \\
    10 GeV     &0.5               &0.8    \\
    15 GeV  &5.5               &9    \\
    20 GeV          &10                &17    \\
      \hline
            
    \hline
      Single Muon Trigger \\
      \hline 
    4 GeV                    &0.5               &0.5    \\
    10 GeV                   &2.3               &2.5    \\
   14 GeV                   &25                &14    \\
    18 GeV                   &26                &48    \\
    \hline
    
    \hline
  \end{tabular}
  \caption{Pre-scaled single-lepton triggers from 2015+2016 used to compute lepton fake factors.}
  \label{tab:prescaledtrigs}
\end{table}

The fake estimate control regions are constructed using 2015+2016 ATLAS data triggered by the lowest unprescaled inclusive \met{} triggers, as described in Chapter~\ref{sec:data}.  Control region events are selected with all the same cuts as the signal region, but instead of selecting two signal leptons to form the SFOS pair, one signal lepton and one anti-ID lepton are selected to make the SFOS lepton pair. 
  \FloatBarrier
  %%%%%%%%%%%%%%%%%%%
  \iffalse
  \subsection{Fake Lepton Composition}
  \textcolor{red}{Might take this section out.}
 Monte Carlo studies of fake and non-prompt lepton composition is done separately for events with opposite sign lepton pairs and events with same-sign lepton pairs.  In the MC samples, there is a variable MCTruthClassifier that determines lepton categories based on their source.  Real prompt leptons fall into two categories: \textit{isolated} and $\ell\rightarrow \gamma \rightarrow \ell$, which refers to truth matched leptons that arise from a Bremsstrahlung to photon conversion process. Fake and non-prompt leptons occupy the remaining categories: \textit{non-isolated}, which are mostly from heavy flavor decays, \textit{photons}, which are either photons faking leptons or actualy leptons from photon conversions, \textit{hadron}, which are from light flavor decays, and \textit{unknown, unknown electron, or unknown muon}, which are primarily from pile-up.\\
 
(\textbf{Describe Figure~\ref{fig:elMC} and Figure~\ref{fig:muMC}}). Sources of fake leptons in the di-muon and di-electron signal regions mostly come from heavy flavor decays, and fake leptons in the di-muon and di-electron control regions are primarily from light flavor decays.  One important result from this study is the similarity of fake lepton contribution between the opposite sign lepton pain and same sign lepton pair events.  This gives confidence that the same-sign validation region can be successfully used to validate our fake background predictions in the data without accidentally unblinding our signal region and biasing the results.  

  
\begin{figure}[htb]
        \centering
        \includegraphics[width=.48\textwidth]{/Users/sheenaschier/Documents/LaFiles/figures/thesis/fakes/fakeLeptonComposition/626_cdsComments_mu_SR2_lep2Pt.pdf}
       \includegraphics[width=.48\textwidth]{/Users/sheenaschier/Documents/LaFiles/figures/thesis/fakes/fakeLeptonComposition/626_cds_noIso_mu_SR2_lep2Pt.pdf}
       % \includegraphics[width=.48\textwidth]{/Users/sheenaschier/Documents/LaFiles/figures/thesis/fakes/fakeLeptonComposition/626_cds_noIso_mu_QCR2_lep2Pt.pdf}
      \includegraphics[width=.48\textwidth]{/Users/sheenaschier/Documents/LaFiles/figures/thesis/fakes/fakeLeptonComposition/626_cds_ss_wIso_mu_SR2_lep2Pt.pdf}
      \includegraphics[width=.48\textwidth]{/Users/sheenaschier/Documents/LaFiles/figures/thesis/fakes/fakeLeptonComposition/626_cds_ss_mu_SR2_lep2Pt.pdf}
        %\includegraphics[width=.48\textwidth]{/Users/sheenaschier/Documents/LaFiles/figures/thesis/fakes/fakeLeptonComposition/626_cds_ss_mu_QCR2_lep2Pt.pdf}
        \caption{Fake lepton composition as a function of subleading lepton $p_{T}$, with and without prompt (``Isolated'' plus ``lep$\to$gamma$\to$lep'') leptons, for opposite sign muon pairs in the signal region.  Top left: SR no iso, top right: CR no iso, bottom left: ssSR iso, bottom right: ssCR iso.}
        \label{fig:muMC}
\end{figure}

 
\begin{figure}[htb]
        \centering
        \includegraphics[width=.48\textwidth]{/Users/sheenaschier/Documents/LaFiles/figures/thesis/fakes/fakeLeptonComposition/725_cdsComments_el_SR2_lep2Pt.pdf}
        \includegraphics[width=.48\textwidth]{/Users/sheenaschier/Documents/LaFiles/figures/thesis/fakes/fakeLeptonComposition/725_cds_noIso_el_SR2_lep2Pt.pdf}
         %\includegraphics[width=.48\textwidth]{/Users/sheenaschier/Documents/LaFiles/figures/thesis/fakes/fakeLeptonComposition/725_cds_noIso_el_QCR2_lep2Pt.pdf}
        \includegraphics[width=.48\textwidth]{/Users/sheenaschier/Documents/LaFiles/figures/thesis/fakes/fakeLeptonComposition/725_cds_ss_wIso_el_SR2_lep2Pt.pdf}
       %\includegraphics[width=.48\textwidth]{/Users/sheenaschier/Documents/LaFiles/figures/thesis/fakes/fakeLeptonComposition/725_cds_ss_el_QCR2_lep2Pt.pdf}
          \includegraphics[width=.48\textwidth]{/Users/sheenaschier/Documents/LaFiles/figures/thesis/fakes/fakeLeptonComposition/725_cds_ss_el_SR2_lep2Pt.pdf}
        \caption{Fake lepton composition in opposite sign signal and control region as a function subleading lepton $p_{T}$, with and without prompt (``Isolated'' plus ``lep$\to$gamma$\to$lep'') leptons, for opposite sign electron pairs in the signal region.  Top left: SR no iso, top right: CR no iso, bottom left: ssSR iso, bottom right: ssCR iso.} 
        \label{fig:elMC}
\end{figure}

\fi
 \FloatBarrier
 
%%%%%%%%%%%%%%%%%%%%%%%
 \subsection{ID \& Anti-ID Lepton Definitions}
For both electron and muon fake factors, ID leptons are defined by the same signal lepton criteria as for the lepton pairs in the signal regions.  Anti-ID lepton definitions are chosen so that this category is mostly populated with fakes and depleted in real prompt leptons.  This fake lepton enhancement is achieved by inverting the cuts used to suppress lepton misidentification.  Having an anti-ID definition that is close to the ID definition reduces the systematic uncertainties on the fake background prediction.  Adversely, tighter anti-ID cuts will decrease the acceptance of fakes, which increases the statistical uncertainty on the fake background prediction.  This section will detail the anti-ID lepton selections. 

%%%%%%%%
\subsubsection{Electron Definitions}

ID electrons are constructed with the same definition as signal electrons, summarized in Table~\ref{tab:objdef}.  These are baseline electrons that also pass \textit{TightLLH} identification, \textit{GradientLoose} isolation, and $|d_0/\sigma(d_0)|<5.0$ requirements.  Anti-ID electrons start as baseline electrons, but are required to pass a slightly tighter PID, \textit{LooseAndBLayerLLH}.  Additionally, anti-ID electrons are required to fail at least one of the signal electron criteria.  This means anti-ID electrons must fail \textit{TightLLH} identification, or \textit{GradientLoose} isolation, or $|d_0/\sigma(d_0)|<5.0$, or some combination of these. %Studies motivating the definition of the anti-ID electrons were performed and are documented in this section. 
All ID and anti-ID electrons are required to pass the $|z_0\sin\theta| < 0.5$~mm requirement to reduce the impact of pileup.  The ID and anti-ID electron definitions are summarized in Table~\ref{tab:AllElDefs}. 
\begin{table}[!htb]
\begin{center}
\begin{tabular}{c|c}
\hline
Signal Electron Definition  & Anti-ID Electron Definition \\
\hline \hline
\multicolumn{2}{c}{$\pt > 4.5\GeV$}      \\
\multicolumn{2}{c}{$\abseta < 2.47$ }     \\
\multicolumn{2}{c}{$|z_0\sin\theta| < 0.5$~mm} \\
\multicolumn{2}{c}{Electron \textit{author} $!= 16$}\\
Pass \textit{Tight} Identification & Pass \textit{LooseAbdBLayer} Identification\\
%Pass \textit{Tight} Identification}  &  Pass \textit{LooseAndBLayer} Identification\\
      &             \textbf{(}Fail \textit{Tight} Identification \textbf{\textit{or}} \\
Pass \textit{GradientLoose} Isolation  & Fail \textit{GradientLoose} Isolation \textbf{\textit{or}} \\   
$|d_0/\sigma(d_0)| < 5$  &   $|d_0/\sigma(d_0)| > 5$\textbf{)} \\
\hline
\end{tabular}
\caption{Summary of electron definitions.}
\label{tab:AllElDefs}
\end{center}
\end{table}
The fractional composition of anti-ID electrons in the fake factor measurement region, according to the set of failed signal electron criteria, is shown in Figure~\ref{fig:elDeco}. Here, the $\mathrm{m_T}$ distribution is plotted over the entire $\mathrm{m_T}$ spectrum, while the $\met$, \pt{} and $\eta$ distributions are all shown for $\mathrm{m_T} <40\GeV$.  The significant of the $\mathrm{m_T}$ cut is explained in Section~\ref{sec:FFel}.  From Fig~\ref{fig:elDeco}, we learn that the anti-ID electrons are 40-50$\%$ electrons that fail both the \textit{TightLLH} identification and \textit{GradientLoose} isolation, 25$\%$ electrons that only fail identification, 25$\%$ electrons that only fail isolation, and a tiny fraction of electrons that failed $|d_0/\sigma(d_0)|<5.0$. 
\begin{figure}[htb]
        \centering
         \includegraphics[width=.49\textwidth]{/Users/sheenaschier/Documents/LaFiles/figures/thesis/fakes/FF_electron/AID_deco_AntiIDelPt}
        \includegraphics[width=.49\textwidth]{/Users/sheenaschier/Documents/LaFiles/figures/thesis/fakes/FF_electron/AID_deco_AntiIDelEta}
        \includegraphics[width=.49\textwidth]{/Users/sheenaschier/Documents/LaFiles/figures/thesis/fakes/FF_electron/AID_deco_Mt}
        \includegraphics[width=.49\textwidth]{/Users/sheenaschier/Documents/LaFiles/figures/thesis/fakes/FF_electron/AID_deco_MET}
        \caption{Fake electron composition as a function of electron \pt{} (top left), electron $\eta$ (top right), $\mathrm{m_T}$, (bottom left) and $\met$ (bottom right). All distributions corresponds to events in the $\mathrm{m_T}$ measurement region, except the $\mathrm{m_T}$ distribution itself. }
\label{fig:elDeco}
\end{figure}

In choosing the best anti-ID definition to use, there is a trade-off between systematic and statistical uncertainties.  A dedicated study of different anti-ID electron definitions to determine which best models the source of fake electron backgrounds and relatively minimizes the statistical uncertainties in the fake background estimate was performed.  It was observed that requiring a tighter electron identification working-point enhances fraction of heavy flavor decays.  Requiring tracks to have a hit in the b-layer reduces fraction of fakes from conversions.  A Loose or Medium isolation requirement narrows the source of fakes towards heavy and light hadronic decays.  Lastly, requiring a large $d_0/\sigma_{d_0}$ can increase the fraction of heavy flavor decays and conversions.  Unfortunately, the Medium isolation and the large $d_0/\sigma_{d_0}$ requirements starkly decrease the number of electrons that pass the anti-ID requirements.
 \begin{figure}
 \centering
 \iffalse
 \begin{subfigure}[b]{0.47\textwidth}
    \includegraphics[width=\textwidth]{/Users/sheenaschier/Documents/LaFiles/figures/thesis/fakes/antiIDStudies/AllMC_ee_SR_lep2Pt.pdf}
 %   \caption{Signal lepton;\\ 9.99 MC W+jet events.}
    \end{subfigure}
     \begin{subfigure}[b]{0.47\textwidth}
  \includegraphics[width=\textwidth]{/Users/sheenaschier/Documents/LaFiles/figures/thesis/fakes/antiIDStudies/AllMC_ee_VeryLoose_FailSignal_lep2Pt.pdf}
% \caption{ VeryLoose \& !signal;\\ 433.02 MC W+jet events.}
 \end{subfigure}
 \fi
  \begin{subfigure}[b]{0.47\textwidth}
     \includegraphics[width=\textwidth]{/Users/sheenaschier/Documents/LaFiles/figures/thesis/fakes/antiIDStudies/AllMC_ee_VeryLooseBL_FailSignal_lep2Pt.pdf}
  %    \caption{VeryLoose \& PassBL \& !signal;\\ 313.38 MC W+jet events.}
 \end{subfigure}
   \begin{subfigure}[b]{0.47\textwidth}
     \includegraphics[width=\textwidth]{/Users/sheenaschier/Documents/LaFiles/figures/thesis/fakes/antiIDStudies/AllMC_ee_LooseBL_FailSignal_lep2Pt.pdf}
%      \caption{Loose \& PassBL \&\& !signal;\\ 168.85 MC W+jet events.}
 \end{subfigure}       
    \begin{subfigure}[b]{0.46\textwidth}
     \includegraphics[width=\textwidth]{/Users/sheenaschier/Documents/LaFiles/figures/thesis/fakes/antiIDStudies/AllMC_ee_Medium_FailSignal_lep2Pt.pdf}
%      \caption{Loose \&\& Medium \& !signal;\\ 75.28 MC W+jet events.}
 \end{subfigure}
    \begin{subfigure}[b]{0.46\textwidth}
     \includegraphics[width=\textwidth]{/Users/sheenaschier/Documents/LaFiles/figures/thesis/fakes/antiIDStudies/AllMC_ee_LooseBL_D0SigGt_OR_Medium_lep2Pt.pdf}
  %    \caption{Medium$||$(Loose \& PassBL \& $d_0/\sigma_{d_0}> 1.5$) \& !signal; 86.28 MC W+jet events.}
 \end{subfigure} 
 
    \caption{Fake lepton composition as a function of the subleading lepton \pt. \textcolor{red}{Talk to Mike about showing these plots.} }
 \label{fig:LeptonIDComposition}
\end{figure}


  \FloatBarrier
   %%%%%%%%%% 
\subsubsection{Muon Definitions}

ID muons are defined with the same selection criteria as signal muons, summarized in Table~\ref{tab:objdef}. These are baseline muons that also satisfy the \textit{FixedCutTightTrackOnly} isolation requirements and the impact paramter significance requirement $|d_0/\sigma(d_0)|<3.0$.  Anti-ID muons are also baseline muons, but instead of requiring they pass the isolation and $d_0$ significance requirements of the ID muons, they instead must fail the \textit{FixedCutTightTrackOnly} isolation or $|d_0/\sigma(d_0)|<3.0$ criteria\footnote{Failing both the isolation and the $d_0$ significance cut still satisfies the anti-ID definition.}. Both the ID and anti-ID muons are required to pass the $|z_0\sin\theta| < 0.5$~mm requirement to reduce the impact of pileup.  One notable difference with respect to the signal muon requirements is that the muon-jet overlap removal is relaxed when performing the fake factor measurement.  This enhances the statistics used for deriving the fake factors, and is motivated by the observation that the muon-jet overlap removal is primarily designed to reduce the number of heavy flavor decays which are mistakenly being classified as prompt muons.  A summary of the ID and anti-ID muon definitions are summarized in Table~\ref{tab:AllMuDefs}
\begin{table}[!htb]
\begin{center}
\begin{tabular}{c|c}
\hline
Signal Muon Definition  & Anti-ID Muon Definition \\
\hline \hline
\multicolumn{2}{c}{$\pt > 4\GeV$}      \\
\multicolumn{2}{c}{$\abseta < 2.5$ }     \\
\multicolumn{2}{c}{$|z_0\sin\theta| < 0.5$~mm} \\
\multicolumn{2}{c}{Pass \textit{Medium} Identification}     \\
$|d_0/\sigma(d_0)| < 3$  &   \textbf{(}$|d_0/\sigma(d_0)| > 3$ \textbf{\textit{or}}\\
Pass \textit{FixedCutTightTrackOnly} Isolation  & Fail \textit{FixedCutTightTrackOnly} Isolation\textbf{)} \\   \\
\hline
\end{tabular}
\caption{Summary of muon definitions.}
\label{tab:AllMuDefs}
\end{center}
\end{table}

The anti-ID muons decomposition, according to which set of ID criteria failed, is shown in Fig~\ref{fig:muDeco}. The $\mathrm{m_T}$ distribution is plotted over the entire $\mathrm{m_T}$ range, while the $\met$, \pt{} and $\eta$ distributions are all shown for $\mathrm{m_T} <40\GeV$, corresponding to the fake enriched region where the fake factors are measured.  Note that these distributions are separated into categories: events with exactly zero $b$-jets, events with one or more $b$-jets.  In studying the fake factor dependence on different kinematic variables, which is discussed later in Section~\ref{sec:ff}, b-jet multiplicity was found to have a large variation.  In events with exactly zero b-jets, the anti-ID muon composition is approximately 50-65$\%$ muons that fail only the \textit{FixedCutTightTrackOnly} isolation, and 20-40$\%$ muons that fail both isolation and $d_0$ significance at low \pt.  In events with one or more b-jets, the fraction of anti-ID muons that fail only isolation is reduced, but still the majority, and the fraction that fail both isolation and $d_0$ significance is a bit higher.     
\begin{figure}
        \centering
        \includegraphics[width=.4\textwidth]{/Users/sheenaschier/Documents/LaFiles/figures/thesis/fakes/FF_muon/AID_deco_Mt_b0}
                \includegraphics[width=.4\textwidth]{/Users/sheenaschier/Documents/LaFiles/figures/thesis/fakes/FF_muon/AID_deco_Mt_b1}
        \includegraphics[width=.4\textwidth]{/Users/sheenaschier/Documents/LaFiles/figures/thesis/fakes/FF_muon/AID_deco_MET_b0}
        \includegraphics[width=.4\textwidth]{/Users/sheenaschier/Documents/LaFiles/figures/thesis/fakes/FF_muon/AID_deco_MET_b1}\\
                \includegraphics[width=.4\textwidth]{/Users/sheenaschier/Documents/LaFiles/figures/thesis/fakes/FF_muon/AID_deco_AntiIDmuPt_b0}
                        \includegraphics[width=.4\textwidth]{/Users/sheenaschier/Documents/LaFiles/figures/thesis/fakes/FF_muon/AID_deco_AntiIDmuPt_b1}
        \includegraphics[width=.4\textwidth]{/Users/sheenaschier/Documents/LaFiles/figures/thesis/fakes/FF_muon/AID_deco_AntiIDmuEta_b0}
        \includegraphics[width=.4\textwidth]{/Users/sheenaschier/Documents/LaFiles/figures/thesis/fakes/FF_muon/AID_deco_AntiIDmuEta_b1}\\
        \caption{Anti-ID muon composition in events with exactly zero $b$-jets(left) and one or more $b$-jets(right) as a function of $\mathrm{m_T}$, $\met$, muon \pt{}, and muon $\eta$. All but the $\mathrm{m_T}$ distribution corresponds to events with $\mathrm{m_T} < 40\GeV$.}
        \label{fig:muDeco}
\end{figure}


  \FloatBarrier
  
 \subsection{Fake Factor Measurement}
   \textcolor{red}{Generalize the process of calculating fake factors...}  Data samples, are used to select events that have at least one ID or anti-ID lepton within fiducial acceptance of the detector.  Monte Carlo samples of SM W+jest, Z+jets, t$\bar t$, single-top, and diboson processes are used to represent the contribution from prompt leptons in the fake factor measurement region.  We plot all the kinematics of the ID and anti-ID lepton events.  The fake factors are measured in the region $\mathrm{m_T} < 40\GeV$ because this region is dominated by fake leptons.  This is shown in the $\mathrm{m_T}$ plots in data overlaid with the stacked Monte Carlo.  The prompt lepton contamination in the measurement region is subtracted off, but first it is normalized to the data in the $\met{} > 200~\GeV$ region that should be dominated by real prompt leptons.  The fake factors are expected to depend almost exclusively on lepton \pt. 
  \subsubsection{Electron Fake Factors}
  \label{sec:FFel}
Events in the electron FF samples generally contain just one lepton, and through ID and anti-ID electron selection, these events get separated into ID and anti-ID electron samples.  Besides the two baseline jets requirement, the only selection requirement for the ID and anti-ID samples is that the electron \pt fall within some \pt range set by the highest single-electron trigger that fired.  Restricting the electrons in this way alleviates the effect of having overlapping trigger prescales to unfold.  The most efficient \pt range associated with each trigger is determined from the electron distributions for each trigger, displayed in Figure~\ref{fig:FFeltrig}.  
\begin{figure}[tbp]
  \centering
  \includegraphics[width=0.48\columnwidth]{/Users/sheenaschier/Documents/LaFiles/figures/thesis/fakes/FF_electron/electronTriggers}
  \includegraphics[width=0.48\columnwidth]{/Users/sheenaschier/Documents/LaFiles/figures/thesis/fakes/FF_electron/AIDelectronTriggers}\\
  \caption{The ID electron (left) and anti-ID electron (right) \pt{} distributions for pre-scaled single-lepton-trigger, normalized to 1~\ipb{}. Blue curve: 5 GeV trigger threshold, red curve: 10 GeV threshold, magenta curve: 15 GeV threshold, green curve: 20 GeV threshold.}
  \label{fig:FFeltrig}
\end{figure}
The 5 GeV electron trigger in blue is used to select electrons with \pt 5-11 GeV, the 10 GeV electron trigger in red selects electrons with \pt 11-18 GeV, the 15 GeV electron trigger in magenta selects electrons with \pt 18-23 GeV, and the 20 GeV electron trigger is used to select electrons with \pt above 23 GeV.  The \pt range corresponding to each single-electron trigger is shown in Table~\ref{tab:elec_trigger_range}.  
\begin{table}[tbp]
  \centering
  \begin{tabular}{|c|c|c|}
    \hline
    Trigger Name  & Trigger Threshold & $e^\pm$ \pt{} range [\GeV]\\
    \hline
    HLT\_e5\_lvhloose & 5 GeV & 5--11  \\
    HLT\_e10\_lvhloose\_L1EM7 & 10 GeV & 11--18  \\
    HLT\_e15\_lvhloose\_L1EM13VH & 15 GeV & 18--23  \\
    HLT\_e20\_lvhloose & 20 GeV & $>$ 23  \\
    \hline
  \end{tabular}
  \caption{Single-Electron triggers and their corresponding \pt{} range.}
  \label{tab:elec_trigger_range}
\end{table}

To calculate fake factors, two kinematic regions are established: one region dominated by fake leptons and used to measure the fake factors, and another region dominated by real leptons and used to normalize the total Monte Carlo yield to the data.  In Figure~\ref{fig:elec_FF_met}, the \met distributions in data for ID and anti-ID electrons display a unique shape at low \met compared to Monte Carlo, but at high \met the data and Monte Carlo distributions follow the same trend.  This is because there are many more fake lepton events occupying the low \met region and Monte Carlo can not simulate this in the data very well.  Oppositely, real leptons mostly occupy the high \met, and up to some scale-factor, this is well modeled by simulation.  Therefore, Monte Carlo is normalized to the data in region $\met > 200~\GeV$ with a separate scale-factor for the ID electrons and for the anti-ID events.  These scale factors are presented in Table~\ref{tab:ff_ele_sf}.  If instead, the MC is re-scaled to match the data for events with $\mathrm{m_T} > 100$\GeV, a region that should also be pure in prompt leptons, the scale factors are $2.39 \pm 0.10$ for ID electrons and $10.69\pm 0.81$ for anti-ID muons.  The re-scaling factors variations in scale factors results in small changes in the scale factors.  The difference between the two methods is used as a systematic uncertainty.
\begin{figure}[tbp]
  \centering
  \includegraphics[width=0.48\columnwidth]{/Users/sheenaschier/Documents/LaFiles/figures/thesis/fakes/FF_electron/ID_CR_MET}
  \includegraphics[width=0.48\columnwidth]{/Users/sheenaschier/Documents/LaFiles/figures/thesis/fakes/FF_electron/AID_CR_MET}
  \caption{The \met{} distributions for ID (left) and anti-ID (right) electrons in FF sample.  MC is rescaled to match data in the $\met > 200\GeV$ region.}
  \label{fig:elec_FF_met}
\end{figure}
\begin{table}[tbp]
  \centering
  \begin{tabular}{|c|c|c|}
    \hline
    ID Electron  & Anti-ID Electron \\ 
    Scale Factor & Scale Factor\\
    \hline
    $1.42\pm0.39$ &  $5.07\pm3.82$  \\
\hline

    \hline
  \end{tabular}
  \caption{ID and anti-ID normalization scale factors calculated in $\met > 200\GeV$..}
  \label{tab:ff_ele_sf}
\end{table} 

Figure~\ref{fig:elec_FF_mt} shows the $\mathrm{m_T}$ distributions for ID and anti-ID electrons in data and Monte Carlo.  Just like with the \met, the data points at low $\mathrm{m_T}$ display a different shape than simulation, but in high $\mathrm{m_T}$, data and Monte Carlo progress in more or less the same way.  In the case where a hadronic jets fakes a lepton, the difference in the jet and electron electron energy scales affects the calibration of the object.  The leptons will typically be measured as a lower energy object than if it were correctly identified as a jet, and this loss of energy reappears as \met aligned with the lepton.  This alignment results in a diminished $\mathrm{m_T}$ calculation; therefore, fake lepton events are understood to occupy low values of $\mathrm{m_T}$.  The fake factor measurement region is defined as $\mathrm{m_T} < 40\GeV$.
\begin{figure}[tbp]
  \centering
  \includegraphics[width=0.48\columnwidth]{/Users/sheenaschier/Documents/LaFiles/figures/thesis/fakes/FF_electron/ID_CR_Mt}
  \includegraphics[width=0.48\columnwidth]{/Users/sheenaschier/Documents/LaFiles/figures/thesis/fakes/FF_electron/AID_CR_Mt}
  \caption{$\mathrm{m_{T}}$ distributions for ID (left) and anti-ID (right) electrons in the FF sample.  MC has been scaled to the data in the $\met > 200\GeV$ region.}
  \label{fig:elec_FF_mt}
\end{figure}

Electron fake factors are assumed to depend almost exclusively on lepton \pt.  Figure~\ref{fig:elec_FF_dists_pt} shows the ID and anti-ID electron \pt distributions in the measurement region after the Monte Carlo normalization factors are applied.  The discontinuities in the data curves between 10 GeV and 20 GeV are a relic of the trigger prescales, and further shifting of the associated \pt bins does not soften this effect any more without significant losses in statistics.  The fake factors are calculated as the ratio of ID electron events to anti-ID electron events bin by bin after the Monte Carlo 'prompt lepton' events have been subtracted out of each \pt distribution.  While electron fake factors show the largest dependance on electron \pt{}, they also display a dependence on the leading jet \pt{}.  Fig.~\ref{fig:elec_FF_hist_noCut} shows electron fake factors as a function of electron \pt{} and leading jet \pt{} separately. Given this trend, and the fact that all signal regions used in this analysis require a hard jet with \pt{} greater than 100\GeV, the fake factor measurement region is augmented to also require a jet with $\pt{} < 100\GeV$.  
 \begin{figure}[tbp]
  \centering
  \includegraphics[width=0.48\columnwidth]{/Users/sheenaschier/Documents/LaFiles/figures/thesis/fakes/FF_electron/ID_SR_IDelPt}
  \includegraphics[width=0.48\columnwidth]{/Users/sheenaschier/Documents/LaFiles/figures/thesis/fakes/FF_electron/AID_SR_AntiIDelPt}\\
  \caption{\pt{} distributions of ID (left) and anti-ID (right) electrons in FF sample for events with $\mathrm{m_T} < 40\GeV$.  MC has been rescaled to match data for $\met > 200\GeV$.}
  \label{fig:elec_FF_dists_pt}
\end{figure}
\begin{figure}[tbp]
  \centering
  \includegraphics[width=0.48\columnwidth]{/Users/sheenaschier/Documents/LaFiles/figures/thesis/fakes/FF_electron/FakeFactor_el_pt_noCut}
  \includegraphics[width=0.48\columnwidth]{/Users/sheenaschier/Documents/LaFiles/figures/thesis/fakes/FF_electron/FakeFactor_el_j1pt_noCut}\\
  \caption{Electron fake factors \textit{before} requiring a jet with $\pt{} > 100\GeV$, as a function of electron \pt{} (left) and leading jet \pt{} (right).  The average electron fake factor over all \pt{} is 0.267. }
  \label{fig:elec_FF_hist_noCut}
\end{figure}

Final fake factors computed as a function of electron \pt{} are shown in Fig.~\ref{fig:elec_FF_hist}a.  In addition, fake factors are computed in terms of other kinematic variables to check any unforeseen fake factor dependence in one of these variables.  Small correlations compared to the electron \pt are folded into the systematic uncertainties.  
Fake factors binned in $|\eta|$, $\Delta\phi_{jet-\met}$, jet multiplicity, $b$-jet multiplicity, average interaction per bunch crossing $\mu$, and number of primary vertices nPV are shown in Figure~\ref{fig:elec_FF_all}.  Relative uncertainties on the final electron fake factors versus electron \pt{} are shown in Fig.~\ref{fig:elec_FF_rel_uncert}.
\begin{figure}[tbp]
  \centering
  \includegraphics[width=0.8\columnwidth]{/Users/sheenaschier/Documents/LaFiles/figures/thesis/fakes/FF_electron/FakeFactor_el_pt}
  \caption{Electron fake factors as a function of electron \pt{} in the measurement region $\mathrm{m_T} < 40\GeV$ and leading jet$ \pt{}>100\GeV$.  Fake factors for electron $\pt{}~ 4.5-5\GeV$ are taken to be the same as electron $\pt{}~5-6\GeV$.  A red line marks the average electron fake factor over all electron \pt{}; 0.211. }
  \label{fig:elec_FF_hist}
\end{figure}

\begin{figure}[tbp]
  \centering
  \includegraphics[width=0.48\columnwidth]{/Users/sheenaschier/Documents/LaFiles/figures/thesis/fakes/FF_electron/FakeFactor_el_eta}
    \includegraphics[width=0.48\columnwidth]{/Users/sheenaschier/Documents/LaFiles/figures/thesis/fakes/FF_electron/FakeFactor_el_dphij}
      \includegraphics[width=0.48\columnwidth]{/Users/sheenaschier/Documents/LaFiles/figures/thesis/fakes/FF_electron/FakeFactor_el_njet}
  \includegraphics[width=0.48\columnwidth]{/Users/sheenaschier/Documents/LaFiles/figures/thesis/fakes/FF_electron/FakeFactor_el_nbjet}\\
    \includegraphics[width=0.48\columnwidth]{/Users/sheenaschier/Documents/LaFiles/figures/thesis/fakes/FF_electron/FakeFactor_el_mu}
  \includegraphics[width=0.48\columnwidth]{/Users/sheenaschier/Documents/LaFiles/figures/thesis/fakes/FF_electron/FakeFactor_el_npv}\\
  \caption{Electron fake factors binned in alternative kinematic variable in the measurement region $\mathrm{m_T} < 40\GeV$ and leading jet$ \pt{}>100\GeV$.  A red line marks the average electron fake factor over all electron \pt{}; 0.211. }
  \label{fig:elec_FF_all}
\end{figure}
\iffalse
\begin{figure}[tbp]
  \centering
  \includegraphics[width=0.48\columnwidth]{/Users/sheenaschier/Documents/LaFiles/figures/thesis/fakes/FF_electron/FakeFactor_el_pt}
  \includegraphics[width=0.48\columnwidth]{/Users/sheenaschier/Documents/LaFiles/figures/thesis/fakes/FF_electron/FakeFactor_el_eta}
  \caption{Electron fake factors as a function of electron \pt{} (left) and electron $\eta$ (right) in the kinematic region with leading jet$ \pt{}>100\GeV$  Fake factors for electron $\pt{}~ 4.5-5\GeV$ are taken to be the same as electron $\pt{}~5-6\GeV$.  A red line marks the average electron fake factor over all electron \pt{}; 0.211. }
  \label{fig:elec_FF_hist}
\end{figure}

\begin{figure}[tbp]
  \centering
  \includegraphics[width=0.48\columnwidth]{/Users/sheenaschier/Documents/LaFiles/figures/thesis/fakes/FF_electron/FakeFactor_el_j1pt}
  \includegraphics[width=0.48\columnwidth]{/Users/sheenaschier/Documents/LaFiles/figures/thesis/fakes/FF_electron/FakeFactor_el_dphij}\\
  \caption{Electron fake factors as a function of leading jet \pt{} (left) and $\Delta\phi_{jet-\met}$ (right). A red line denotes the average electron fake factor over all electron \pt{} of 0.211.}
  \label{fig:elec_FF_hadronic}
\end{figure}

\begin{figure}[tbp]
  \centering
  \includegraphics[width=0.48\columnwidth]{/Users/sheenaschier/Documents/LaFiles/figures/thesis/fakes/FF_electron/FakeFactor_el_njet}
  \includegraphics[width=0.48\columnwidth]{/Users/sheenaschier/Documents/LaFiles/figures/thesis/fakes/FF_electron/FakeFactor_el_nbjet}\\
  \caption{Electron fake factors as a function of the jet multiplicity (left) and the $b$-jet multiplicity (right). A red line denotes the average electron fake factor over all electron \pt{} of 0.211.}
  \label{fig:elec_FF_njet}
\end{figure}


\begin{figure}[tbp]
  \centering
  \includegraphics[width=0.48\columnwidth]{/Users/sheenaschier/Documents/LaFiles/figures/thesis/fakes/FF_electron/FakeFactor_el_mu}
  \includegraphics[width=0.48\columnwidth]{/Users/sheenaschier/Documents/LaFiles/figures/thesis/fakes/FF_electron/FakeFactor_el_npv}\\
  \caption{Electron fake factors as a function of the average interaction per bunch crossing (left) and the number of primary vertices (right). A red line denotes the average electron fake factor over all electron \pt{} of 0.211.}
  \label{fig:elec_FF_pileup}
\end{figure}
\fi
\begin{figure}[tbp]
  \centering
  \includegraphics[width=0.48\columnwidth]{/Users/sheenaschier/Documents/LaFiles/figures/thesis/fakes/FF_electron/FakeFactor_el_pt_uncert}\\
  \caption{Relative uncertainties on electron fake factors binned in electron \pt{}.}
  \label{fig:elec_FF_rel_uncert}
\end{figure}
 \FloatBarrier
 
 
 
%%%%%%%%%%%%%%%%%%%%
 \subsubsection{Muon Fake Factors}
 The muon fake factors are calculated in nearly the same way as for electrons.  Muon FF samples with a two baseline jet requirement are used and events are selected according to muon definition.  ID muon samples give the numerator component of the fake factor, and anti-ID samples give the denominator component.  Both data and MC contributions to the ID and anti-ID samples in the single-muon trigger sample are normalized to 10~\ipb, to remove the effects of the prescales in the data.  Both the ID and anti-ID samples have the requirement that the muon \pt lie within the \pt range associated with highest single muon trigger that fired.  Just like in the electron case, associating each trigger with an exclusive \pt range reduces the complexity of using multiple pre-scaled triggers.  The ID and anti-ID muon \pt distributions for each trigger are presented in Figure~\ref{fig:mu_triggers}. 
\begin{figure}[tbp]
  \centering
  \includegraphics[width=0.48\columnwidth]{/Users/sheenaschier/Documents/LaFiles/figures/thesis/fakes/FF_muon/IDmuonTriggers}
  \includegraphics[width=0.48\columnwidth]{/Users/sheenaschier/Documents/LaFiles/figures/thesis/fakes/FF_muon/AntiIDmuonTriggers}\\
  \caption{The ID (left) and anti-ID (right) muon \pt{} distributions for pre-scaled single-muon triggers, normalized to 1~\ipb{}. Blue curve: 4 GeV trigger threshold, red curve: 10 GeV threshold, magenta curve: 14 GeV threshold, green curve: 18 GeV trigger threshold.}
  \label{fig:mu_triggers}
\end{figure}
The 4 GeV muon trigger in blue is used to select muons with \pt 4-11 GeV, the 10 GeV muon trigger in red selects muons with \pt 11-15 GeV, the 14 GeV muon trigger in magenta selects muons with \pt 15-20 GeV, and the 20 GeV electron trigger is used to select electrons with \pt above 23 GeV.  The \pt range corresponding to each single-muon trigger is displayed in Table~\ref{tab:muon_trigger_range}.
\begin{table}[tbp]
  \centering
  \begin{tabular}{|c|c|c|}
    \hline
        Trigger Name  & Trigger Threshold   & $\mu$ \pt{} range [\GeV]\\
    \hline
    HLT\_mu4 & 4 GeV &4 --11  \\
    HLT\_mu10 & 10 GeV &11--15  \\
    HLT\_mu14 & 14 GeV &15--20  \\
    HLT\_mu18 & 18 GeV & $>$ 20  \\
    \hline
  \end{tabular}
  \caption{Single-muon triggers used for fake factor computation and their corresponding \pt{} range.}
  \label{tab:muon_trigger_range}
\end{table}

 For both ID and anti-ID muon samples, the Monte Carlo events are re-scaled to match the data in events with $\met{}>200\GeV$, a kinematic region expected to be pure in prompt leptons.  The \met distributions after applying the rescale factors are displayed in Figure~\ref{fig:ff_muon_met}.  The region $\met < 100\GeV$ shows a distinct difference in the shapes for data compared to simulation.  This signifies the overwhelming presence of fake muons that are poorly modeled with Monte Carlo.  Events with $\met{}>200\GeV$ show nice agreement between data and Monte Carlo, as they did in the electron samples.    

\begin{figure}[tbp]
  \centering
  \includegraphics[width=0.48\columnwidth]{/Users/sheenaschier/Documents/LaFiles/figures/thesis/fakes/FF_muon/IDb0_CR_MET}
  \includegraphics[width=0.48\columnwidth]{/Users/sheenaschier/Documents/LaFiles/figures/thesis/fakes/FF_muon/AIDb0_CR_MET}
    \includegraphics[width=0.48\columnwidth]{/Users/sheenaschier/Documents/LaFiles/figures/thesis/fakes/FF_muon/IDb1_CR_MET}
      \includegraphics[width=0.48\columnwidth]{/Users/sheenaschier/Documents/LaFiles/figures/thesis/fakes/FF_muon/AIDb1_CR_MET}
  \caption{The \met{} distributions for ID (left) and anti-ID (right) muons in muon FF samples for events with exactly zero $b$-jets (top), and events with at least one $b$-jet (bottom).  MC is scaled to match the data in the region $\met > 200\GeV$.}
  \label{fig:ff_muon_met}
\end{figure}
\begin{table}[tbp]
  \centering
  \begin{tabular}{|c|c|c|}
    \hline
  &  ID Muon  & Anti-ID Muon \\ 
  &  Scale Factor & Scale Factor\\
    \hline
0 $b$-jets   & $1.01\pm0.13$ &  $1.20\pm0.29$  \\
 $>$ 0 $b$-jets     & $1.24\pm0.20$ &  $7.34\pm5.00$  \\
\hline

    \hline
  \end{tabular}
  \caption{ID and anti-ID muon scale factors calculated in $\met > 200\GeV$ separated by muon definition and }
  \label{tab:ff_muon_sf}
\end{table} 
The scale factors corresponding to each \met distribution in Figure~\ref{fig:ff_muon_met} are summarized in Table~\ref{tab:ff_muon_sf}.  If instead, the MC is re-scaled to match the data for events with $\mathrm{m_T} > 100$\GeV, a region that should also be pure in prompt leptons, the re-scaling factors for events with exactly 0 $b$-jets are $2.37 \pm 0.10$ for ID muons and $11.68 \pm 2.28$ for anti-ID muons; events with one or more $b$-jets have re-scale factors $1.60 \pm 0.06$ for ID muons and $10.41 \pm 6.34$ for anti-ID muons. The re-scaling factors vary significantly between the two methods but the fake factors themselves exhibit small changes between the two methods and can be used as a systematic uncertainty.

\begin{figure}[tbp]
  \centering
  \includegraphics[width=0.48\columnwidth]{/Users/sheenaschier/Documents/LaFiles/figures/thesis/fakes/FF_muon/IDb0_CR_Mt}
    \includegraphics[width=0.48\columnwidth]{/Users/sheenaschier/Documents/LaFiles/figures/thesis/fakes/FF_muon/AIDb0_CR_Mt}
  \includegraphics[width=0.48\columnwidth]{/Users/sheenaschier/Documents/LaFiles/figures/thesis/fakes/FF_muon/IDb1_CR_Mt}
  \includegraphics[width=0.48\columnwidth]{/Users/sheenaschier/Documents/LaFiles/figures/thesis/fakes/FF_muon/AIDb1_CR_Mt}
  \caption{The $\mathrm{m_T}$ distributions for ID (left) and anti-ID (right) muons in muon FF samples for events with exactly zero $b$-jets (top), and events with one or more $b$-jets (bottom).  MC is scaled to the data in the region $\met > 200\GeV$.}
  \label{fig:ff_muon_mt}
\end{figure}
Figure~\ref{fig:ff_muon_mt} shows the $\mathrm{m_T}$ distributions for ID and anti-ID muons in data and Monte Carlo events.  In the region $\mathrm{m_T}> 100\GeV$, the cumulative Monte Carlo trend matches the shape of the data, but in the region $\mathrm{m_T}<40\GeV$, the data is greatly more populated with fake muons.  The explanation is the same as described in the previous section and is mostly due to the instrumental \met that often accompanies mismeasured jets. 

\begin{figure}[tbp]
  \centering
  \includegraphics[width=0.48\textwidth]{/Users/sheenaschier/Documents/LaFiles/figures/thesis/fakes/FF_muon/IDb0_SR_IDmuPt}
  \includegraphics[width=0.48\textwidth]{/Users/sheenaschier/Documents/LaFiles/figures/thesis/fakes/FF_muon/AIDb0_SR_AntiIDmuPt}
    \includegraphics[width=0.48\textwidth]{/Users/sheenaschier/Documents/LaFiles/figures/thesis/fakes/FF_muon/IDb1_SR_IDmuPt}
  \includegraphics[width=0.48\textwidth]{/Users/sheenaschier/Documents/LaFiles/figures/thesis/fakes/FF_muon/AIDb1_SR_AntiIDmuPt}
          \caption{ID (left) and anti-ID (right) muon \pt{} in the fake factor measurement region $\mathrm{m_T} < 40\GeV$ for events with exactly zero $b$-jets (top), and events with one or more $b$-jets (bottom).  MC has been rescaled to the data in the region $\met > 200\GeV$.}
  \label{fig:muon_FF_dists_pt}
\end{figure}

\iffalse
\begin{figure}[tbp]
  \centering
  \includegraphics[width=0.48\textwidth]{/Users/sheenaschier/Documents/LaFiles/figures/thesis/fakes/FF_muon/IDb1_SR_IDmuPt}
  \includegraphics[width=0.48\textwidth]{/Users/sheenaschier/Documents/LaFiles/figures/thesis/fakes/FF_muon/AIDb1_SR_AntiIDmuPt}
  \caption{Muon \pt{} for numerator (left) and denominator (right) objects in the prescaled single-muon trigger sample for events with $\mathrm{m_T}< 40\GeV$.  MC has been scaled to the data in the $\mathrm{m_T} > 100\GeV$ region. Distributions from~\cite{Boerner:2231917}.}
  \label{fig:muon_FF_dists_pt_b1}
\end{figure}
\fi
Muon fake factors are computed as the ratio of ID electron events to anti-ID electron events bin by bin after subtracting the Monte Carlo ?prompt muons? in the region $m_{\mathrm{T}}<40\GeV$.  Like with the electrons, fake factors are initially assumed to depend exclusively on muon \pt, and are calculated bin by bin using the distribution in Figure~\ref{fig:muon_FF_dists_pt}.  But this assumption does not always work.  Muon fake factors display a particular dependence on the presence of $b$-jets, which is visible in Figure~\ref{fig:muon_FF_hist_noCut}.  The fake factors also show a similar variation in leading jet \pt as did the electrons fake factors.  \begin{figure}[tbp]
  \centering
  \includegraphics[width=0.48\columnwidth]{/Users/sheenaschier/Documents/LaFiles/figures/thesis/fakes/FF_muon/FakeFactor_mu_pt}
  \includegraphics[width=0.48\columnwidth]{/Users/sheenaschier/Documents/LaFiles/figures/thesis/fakes/FF_muon/FakeFactor_mu_j1pt}\\
  \includegraphics[width=0.48\columnwidth]{/Users/sheenaschier/Documents/LaFiles/figures/thesis/fakes/FF_muon/FakeFactor_mu_nbjet}\\
  \caption{Muon fake factors \textit{before} requiring a hard jet of $\pt{}> 100\GeV$, computed from muon FF samples as a function of muon \pt{} (top-left), as a function of leading jet \pt{} (top-right), and as a function of $b$-jet multiplicity (bottom). A red line marks the average muon fake factor over all muon \pt{}.}
  \label{fig:muon_FF_hist_noCut}
\end{figure}


For the final muon fake factor calculation, the measurement region is modified to require a jet with \pt{} greater than $100\GeV$ and the fake factors are binned in muon \pt and in number of $b$-jets.  The bin with exactly zero $b$-jets is used to estimate the fake contribution in the signal region, and the bin with one or more $b$-jets is used to estimate the fake contribution in the $t\bar{t}$ control region.  The final fake factors are shown in Fig.~\ref{fig:muon_FF_hist} as a functions of muon \pt{} for each of the $b$-jet multiplicity bins.  In addition to the final fake factors binned in \pt, fake factors binned in other variables are also inspected to check for significant trends:
\begin{itemize}
\item Fake factors as a function of muon $\eta$ are shown in Fig.~\ref{fig:muon_FF_hist_eta},
\item Fake factors as a function of $\Delta\phi_{jet1-met}$ are shown in Fig.~\ref{fig:muon_FF_dphij1},
\item Fake factors as a function of jet multiplicity are shown in Fig.~\ref{fig:muon_FF_njet},
\item Fake factors as a function of average interactions per bunch crossing are shown in Fig.~\ref{fig:muon_FF_mu},
\item Fake factors as a function of the number of primary vertices are shown in Fig.~\ref{fig:muon_FF_npv}.
\end{itemize}
The relative uncertianties on the muons fake factors versus muon \pt{} for the separate $b$-jet multiplicity bins are show in Fig.~\ref{fig:muon_FF_rel_uncert}.
\begin{figure}[tbp]
  \centering
  \includegraphics[width=0.48\columnwidth]{/Users/sheenaschier/Documents/LaFiles/figures/thesis/fakes/FF_muon/FakeFactor_mu_ptb0}
  \includegraphics[width=0.48\columnwidth]{/Users/sheenaschier/Documents/LaFiles/figures/thesis/fakes/FF_muon/FakeFactor_mu_ptb1}\\
  \caption{Muon fake factors as a function of muon \pt{} in events with exactly zero $b$-jets (left) and one or more $b$-jets (right). A red line denotes the average muon fake factor over all muon \pt{}.}
  \label{fig:muon_FF_hist}
\end{figure}

\begin{figure}[tbp]
  \centering
  \includegraphics[width=0.48\columnwidth]{/Users/sheenaschier/Documents/LaFiles/figures/thesis/fakes/FF_muon/FakeFactor_mu_etab0}
  \includegraphics[width=0.48\columnwidth]{/Users/sheenaschier/Documents/LaFiles/figures/thesis/fakes/FF_muon/FakeFactor_mu_etab1}\\
  \caption{Muon fake factors as a function of muon $\eta$ in events with exactly zero $b$-jets (left) and one or more $b$-jets (right).}
  \label{fig:muon_FF_hist_eta}
\end{figure}

\begin{figure}[tbp]
  \centering
  \includegraphics[width=0.48\columnwidth]{/Users/sheenaschier/Documents/LaFiles/figures/thesis/fakes/FF_muon/FakeFactor_mu_dphijb0}
  \includegraphics[width=0.48\columnwidth]{/Users/sheenaschier/Documents/LaFiles/figures/thesis/fakes/FF_muon/FakeFactor_mu_dphijb1}
  \caption{Muon fake factors as a function of $\Delta\phi_{jet-\met}$ in events with exactly zero $b$-jets (left) and one or more $b$-jets (right).}
  \label{fig:muon_FF_dphij1}
\end{figure}

\begin{figure}[tbp]
  \centering
  \includegraphics[width=0.48\columnwidth]{/Users/sheenaschier/Documents/LaFiles/figures/thesis/fakes/FF_muon/FakeFactor_mu_njetb0}
  \includegraphics[width=0.48\columnwidth]{/Users/sheenaschier/Documents/LaFiles/figures/thesis/fakes/FF_muon/FakeFactor_mu_njetb1}\\
  \caption{Muon fake factors as a function of the jet multiplicity in events with exactly zero $b$-jets (left) and one or more $b$-jets (right).}
  \label{fig:muon_FF_njet}
\end{figure}

\begin{figure}[tbp]
  \centering
  \includegraphics[width=0.48\columnwidth]{/Users/sheenaschier/Documents/LaFiles/figures/thesis/fakes/FF_muon/FakeFactor_mu_mub0}
  \includegraphics[width=0.48\columnwidth]{/Users/sheenaschier/Documents/LaFiles/figures/thesis/fakes/FF_muon/FakeFactor_mu_mub1}\\
  \caption{Muon fake factors as a function of the average number of interactions per bunch crossing in events with exactly zero $b$-jets (left) and one or more $b$-jets (right).}
  \label{fig:muon_FF_mu}
\end{figure}

\begin{figure}[tbp]
  \centering
  \includegraphics[width=0.48\columnwidth]{/Users/sheenaschier/Documents/LaFiles/figures/thesis/fakes/FF_muon/FakeFactor_mu_npvb0}
  \includegraphics[width=0.48\columnwidth]{/Users/sheenaschier/Documents/LaFiles/figures/thesis/fakes/FF_muon/FakeFactor_mu_npvb1}\\
  \caption{Muon fake factors as a function of the number of primary vertices in events with exactly zero $b$-jets (left) and one or more $b$-jets (right).}
  \label{fig:muon_FF_npv}
\end{figure}

\begin{figure}[tbp]
  \centering
  \includegraphics[width=0.48\columnwidth]{/Users/sheenaschier/Documents/LaFiles/figures/thesis/fakes/FF_muon/FakeFactor_mu_ptb0_uncert}
  \includegraphics[width=0.48\columnwidth]{/Users/sheenaschier/Documents/LaFiles/figures/thesis/fakes/FF_muon/FakeFactor_mu_ptb1_uncert}\\
  \caption{Relative uncertianties on muon fake factors versus muon \pt{} in zero $b$-jets bin (left) and one or more $b$-jets bin (right).}
  \label{fig:muon_FF_rel_uncert}
\end{figure}

 \FloatBarrier

\section{Same-Sign Validation Regions (SS-VR)}
\label{sec:ssvr}
The same sign validation regions (VR-SS) are defined by the same selection criteria used to define the signal regions in Chapter~\ref{} except that instead of same-flavor opposite-sign lepton pairs, same-flavor same-sign (SFSS) and different-flavor same-sign (DFSS) lepton pairs are selected.  W($\rightarrow\ell\nu$)+jets events are understood to be flavor agnostic since the jet faking a lepton does not depend on the flavor of the W-decay.  To motivate the use of same-sign events to construct fake factor validation regions, Monte Carlo W+jets samples are used to compare the composition of lepton fakes between SFOS events and SFSS events.  Figure~\ref{fig:muMC} illustrates this comparison with same-sign muons in events in the signal region with and without isolation applied.  Among the opposite sign and same sign distributions, the same general composition is observed.  
\begin{figure}[htb]
        \centering
        \includegraphics[width=.48\textwidth]{/Users/sheenaschier/Documents/LaFiles/figures/thesis/fakes/fakeLeptonComposition/626_cdsComments_mu_SR2_lep2Pt.pdf}
       \includegraphics[width=.48\textwidth]{/Users/sheenaschier/Documents/LaFiles/figures/thesis/fakes/fakeLeptonComposition/626_cds_noIso_mu_SR2_lep2Pt.pdf}
      \includegraphics[width=.48\textwidth]{/Users/sheenaschier/Documents/LaFiles/figures/thesis/fakes/fakeLeptonComposition/626_cds_ss_wIso_mu_SR2_lep2Pt.pdf}
      \includegraphics[width=.48\textwidth]{/Users/sheenaschier/Documents/LaFiles/figures/thesis/fakes/fakeLeptonComposition/626_cds_ss_mu_SR2_lep2Pt.pdf}
        \caption{Fake lepton composition as a function of subleading lepton $p_{T}$, with and without prompt (``Isolated'' plus ``lep$\to$gamma$\to$lep'') leptons, for opposite sign muon pairs in the signal region.  Top left: SR iso, top right: SR no iso, bottom left: ssSR iso, bottom right: ssSR no iso.}
        \label{fig:muMC}
\end{figure}

\begin{figure}[htb]
        \centering
      \includegraphics[width=.8\textwidth]{/Users/sheenaschier/Documents/LaFiles/figures/thesis/fakes/VRSS-ee}\\
      \includegraphics[width=.8\textwidth]{/Users/sheenaschier/Documents/LaFiles/figures/thesis/fakes/VRSS-mm}
        \caption{Distributions after the background-only fit for the same-sign validation regions, where the subleading lepton is either the electron $ee+\mu e$ (top) or muon $\mu\mu+e\mu$ (bottom). The category `Others' contains rare backgrounds from triboson, Higgs boson, and the multi-top processes. The last bin includes overflow.}
        \label{fig:VRSS-el}
\end{figure}

 
 
 \FloatBarrier
 \section{Conclusion}
 \label{sec:FFcon}
 This chapter went over a lot of material and I think somehow I have to reiterate the important points here..

Electron and muon fake factors are calculated with data and Monte Carlo using single lepton triggers.  Electron fake factors are binned in electron \pt in events requiring a jet wit $\pt > 100\GeV$.  Muon fake factors 
