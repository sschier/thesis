\chapter{Interpretations}
\label{ch:interpretations}

In absence of any significant excesses over backgrounds, the results are interpreted as constraints on the SUSY models presented in Chapter~\ref{ch:thy} using the exclusive, multi-binned Higgsino and slepton signal regions.  The background only fit is extended to allow for a signal model with a corresponding signal strength parameter in a simultaneous fit of all CRs and relevant SRs, this is referred to as the exclusion fit.  In the previous chapter, background-level estimates obtained from a background-only fit in the CRs only were presented.  When electroweakino simplified models are assumed, the results are interpreted in the 14 exclusive Higgsino signal regions, binned in $m_{\ell\ell}$ and split evenly between the $ee$ and $\mu\mu$ channels.  By statistically combining these signal regions, the signal shape of the $m_{\ell\ell}$ spectrum can be exploited to improve the sensitivity. When slepton simplified models are assumed, the results are interpreted in 12 slepton signal regions, binned in $m_{T2^{100}}$ with 6 SRs the $ee$-channel and 6 in the $\mu\mu$-channel.

\section{Compressed Higgsino}
Hypothesis tests are performed to set limits on simplified model scenarios using the $CL_s$ prescription.  Figure~\ref{fig:exclusion_contour_higgsino} shows the $95\%$ confidence interval limits set on the Higgsino simplified model projected onto the plane defined by the mass difference between the lightest and next-to-lightest neutralino as a function of the next-to-lightest neutralino mass.  These limits are based on an exclusion fit that exploits the shape of the dilepton invariant mass spectrum from the exclusive electroweakino signal regions and exclude next-to-lightest neutralino masses up to $130~\GeV$ for mass splittings between $5$ and $10~\GeV$.  For mass splittings down to $3~\GeV$ next-to-lightest neutralino masses are excluded up to $100~\GeV$. 

 \begin{figure}
 \centering
 \includegraphics[width=0.95\columnwidth]{/Users/sheenaschier/Documents/LaFiles/figures/thesis/results/exclusion_contour_higgsino.pdf}
  \caption{
 Expected 95\% CL exclusion sensitivity (blue dashed line) with $\pm 1 \sigma_\text{exp}$ (yellow band) from experimental systematics
   and observed limits (red solid) with $\pm 1 \sigma_\text{theory}$ (dotted red) from signal cross section uncertainties.
A shape fit of Higgsino signals to the $m_{\ell\ell}$ spectrum is used to derive
 the limit is displayed in the $m(\tilde{\chi}^0_2) - m(\tilde{\chi}^0_1)$ vs $m(\tilde{\chi}^0_2)$ plane.
 The chargino $\tilde{\chi}^\pm_1$ mass is assumed to be half way between the two lightest neutralinos.
  The grey region denotes the lower chargino mass limit from LEP~\cite{LEPlimits}.}
   \label{fig:exclusion_contour_higgsino}
 \end{figure}
% \FloatBarrier
 
 \section{Compressed Wino}
 The $95\%$ confidence level intervals for the wino-bino simplified model are shown in Figure~\ref{fig:exclusion_contour_wino}.  Just like in the Higgsino exclusion plot, these limits are based on an exclusion fit that exploits the shape of the dilepton invariant mass spectrum from the exclusive electroweakino signal regions.  Exclusion limits are projected onto the mass difference $\Delta m(\tilde{\chi}^0_2, \tilde{\chi}^0_1)$ plane as a function of the $\tilde{\chi}^0_2$ mass.  For wino-bino simplified models, next-to-lightest neutralino masses are excluded up to $170~\GeV$ for mass splittings above $10~\GeV$, and excluded up to $100~\GeV$ for mass splittings down to $2.5~\GeV$. 
   \begin{figure}
 \centering
 \includegraphics[width=0.95\columnwidth]{/Users/sheenaschier/Documents/LaFiles/figures/thesis/results/exclusion_contour_wino.pdf}
  \caption{Expected 95\% CL exclusion sensitivity (blue dashed line) with $\pm1\sigma$ exp (yellow band) from experimental systematic uncertainties and observed limits (red solid line) with pm1?theory (dotted red line) from signal cross-section uncertainties for simplified models direct wino production. 
  A shape fit of wino signals to the $m_{\ell\ell}$ spectrum is used to derive
 the limit is displayed in the $m(\tilde{\chi}^0_2) - m(\tilde{\chi}^0_1)$ vs $m(\tilde{\chi}^0_2)$ plane.
 The chargino $\tilde{\chi}^\pm_1$ mass is assumed equal to the $m(\tilde{\chi}^0_2)$ mass.
  The grey region denotes the lower chargino mass limit from LEP~\cite{LEPlimits}, and the blue region in the lower plot indicates the limit from the 2$\ell$+3$\ell$ combination of ATLAS Run 1.} 
     \label{fig:exclusion_contour_wino}
 \end{figure}
 
  \section{Compressed Slepton}
 Figure~\ref{fig:exclusion_contour_slepton} shows the $95\%$ confidence interval limits set on the slepton simplified model projected onto the plane defined by the mass difference between the slepton and lightest neutralino as a function of the slepton mass.  These limits are based on an exclusion fit that exploits the shape of the $m_{T2}$ spectrum from the exclusive slepton signal regions and exclude slepton masses up to $180~\GeV$ for mass splittings down to $5~\GeV$.  For mass splittings down to $1~\GeV$ slepton masses are excluded up to $70~\GeV$.  In slepton simplified models, a fourfold degeneracy is assumed between the left and right-handed selectrons and smuons: $\tilde{e}_R=\tilde{e}_L=\tilde{\mu}_R=\tilde{\mu}_L$.
  \begin{figure}
 \centering
 \includegraphics[width=0.95\columnwidth]{/Users/sheenaschier/Documents/LaFiles/figures/thesis/results/exclusion_contour_slepton.pdf}
  \caption{
Expected 95\% CL exclusion sensitivity (blue dashed line) with $\pm 1 \sigma_\text{exp}$ (yellow band) from experimental systematics
and observed limits (red solid) with $\pm 1 \sigma_\text{theory}$ (dotted red) from signal cross section uncertainties.
A shape fit of slepton signals to the $m_\text{T2}^{100}$ spectrum is used to derive
the limit projected into the $m(\tilde{\ell}) - m(\tilde{\chi}^0_1)$ vs $m(\tilde{\ell})$ plane.
The slepton $\tilde{\ell}$ refers to a 4-fold mass degenerate system of left- and right-handed selectron and smuon.
The grey region denotes a conservative right-handed smuon $\tilde{\mu}_R$ mass limit from LEP~\cite{LEPlimits},
while the blue region is the 4-fold mass degenerate slepton limit from ATLAS Run 1~\cite{SUSY-2013-11}.}
   \label{fig:exclusion_contour_slepton}
 \end{figure}
 % \FloatBarrier
% \subsection{NUHM2}
 
