\chapter{Interpretations}
\label{ch:interpretations}

In absence of any significant excesses over backgrounds, the results are interpreted as constraints on the SUSY models presented in Chapter~\ref{ch:thy} using the exclusive, multi-binned Higgsino and slepton signal regions.  \textcolor{blue}{The background only fit is extended to allow for a signal model with a corresponding signal strength parameter in a simultaneous fit of all CRs and relevant SRs, this is referred to as the exclusion fit}.  \textcolor{red}{I can say it better than this}.  In the previous chapter, background-level estimates obtained from a background-only fit in the CRs only were presented.  When electroweakino simplified models are assumed, the results are interpreted in the 14 exclusive Higgsino signal regions, binned in $m_{\ell\ell}$ and split evenly between the $ee$ and $\mu\mu$ channels.  \textcolor{blue}{By statistically combining these signal regions, the signal shape of the $m_{\ell\ell}$ spectrum can be exploited to improve the sensitivity.} When slepton simplified models are assumed, the results are interpreted in 12 slepton signal regions, binned in $m_{T2^{100}}$ with 6 SRs the $ee$-channel and 6 in the $\mu\mu$ channel are used for the fit.

Table~\ref{tab:results:exclusiveSRYields} summarizes the observed event yields in the exclusive electroweakinio signal regions, and Table~\ref{tab:results:exclusiveSRYields2} summarizes the observed event yields in the exclusive slepton signal regions after the fit is performed using an exclusion fit configuration where the signal strength parameter is set to zero.  Extending the background only fit to include the signal regions further constrains the background contributions in the absence of any signal, therefore these predicted yields differ slightly compared to those obtained with the background only fit.

{\renewcommand{\arraystretch}{1.3}
\input{/Users/sheenaschier/Documents/LaFiles/figures/thesis/results/MyYieldsTable_exclSRs.tex}
\FloatBarrier

\subsection{Compressed Higgsino}
 \begin{figure}
 \centering
 \includegraphics[width=0.6\columnwidth]{/Users/sheenaschier/Documents/LaFiles/figures/thesis/results/exclusion_contour_higgsino.pdf}
  \caption{
 Expected 95\% CL exclusion sensitivity (blue dashed line) with $\pm 1 \sigma_\text{exp}$ (yellow band) from experimental systematics
   and observed limits (red solid) with $\pm 1 \sigma_\text{theory}$ (dotted red) from signal cross section uncertainties.
A shape fit of Higgsino signals to the $m_{\ell\ell}$ spectrum is used to derive
 the limit is displayed in the $m(\tilde{\chi}^0_2) - m(\tilde{\chi}^0_1)$ vs $m(\tilde{\chi}^0_2)$ plane.
 The chargino $\tilde{\chi}^\pm_1$ mass is assumed to be half way between the two lightest neutralinos.
  The grey region denotes the lower chargino mass limit from LEP~\cite{LEPlimits}.}
   \label{fig:exclusion_contour_higgsino}
 \end{figure}
 \FloatBarrier
 
 \subsection{Compressed Slepton}
  \begin{figure}
 \centering
 \includegraphics[width=0.6\columnwidth]{/Users/sheenaschier/Documents/LaFiles/figures/thesis/results/exclusion_contour_slepton.pdf}
  \caption{
Expected 95\% CL exclusion sensitivity (blue dashed line) with $\pm 1 \sigma_\text{exp}$ (yellow band) from experimental systematics
and observed limits (red solid) with $\pm 1 \sigma_\text{theory}$ (dotted red) from signal cross section uncertainties.
A shape fit of slepton signals to the $m_\text{T2}^{100}$ spectrum is used to derive
the limit projected into the $m(\tilde{\ell}) - m(\tilde{\chi}^0_1)$ vs $m(\tilde{\ell})$ plane.
The slepton $\tilde{\ell}$ refers to a 4-fold mass degenerate system of left- and right-handed selectron and smuon.
The grey region denotes a conservative right-handed smuon $\tilde{\mu}_R$ mass limit from LEP~\cite{LEPlimits},
while the blue region is the 4-fold mass degenerate slepton limit from ATLAS Run 1~\cite{SUSY-2013-11}.}
   \label{fig:exclusion_contour_slepton}
 \end{figure}
 
 \subsection{Compressed Wino}
   \begin{figure}
 \centering
 \includegraphics[width=0.6\columnwidth]{/Users/sheenaschier/Documents/LaFiles/figures/thesis/results/exclusion_contour_slepton.pdf}
  \caption{Expected 95\% CL exclusion sensitivity (blue dashed line) with $\pm1\sigma$ exp (yellow band) from experimental systematic uncertainties and observed limits (red solid line) with pm1?theory (dotted red line) from signal cross-section uncertainties for simplified models of direct Higgsino (top) and wino (bottom) production.} %A fit of signals to the m?? spectrum is used to derive the limit, which is projected into the $\delta$m(?20, ?10) vs. m(?20) plane. For Higgsino production, the chargino ?1pm mass is assumed to be halfway between the two lightest neutralino masses, while m(?20) = m(?1pm) is assumed for the wino--bino model. The gray regions denote the lower chargino mass limit from LEP. The blue region in the lower plot indicates the limit from the 2$\ell$+3$\ell$ combination of ATLAS Run 1.}
     \label{fig:exclusion_contour_slepton}
 \end{figure}
 \subsection{NUHM2}
 
