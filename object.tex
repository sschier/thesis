\chapter{Physics Object Reconstruction and Identification}
\label{ch:obj}
The term \textit{reconstruction} describes the process of interpreting signal output from the detector and using that information to make measurements associated with actual physics objects.  The ATLAS detector and its reconstruction algorithms are designed for efficient particle identification and precise energy and momentum measurement.  Reliable tracking and vertexing are the building blocks for efficient reconstruction and identification of most objects.  In this chapter, the assembly of tracks and vertices will first be described in Section~\ref{sec:obj:bb}.  Next, reconstruction are identification variables are defined for directly and indirectly observable objects in Section~\ref{sec:obj:reco}.  In ATLAS, these objects are; electrons, muons, jets, photons, and missing energy and momentum.  Lastly, Section~\ref{sec:obj:treat} describes the techniques of overlap removal and isolation correction of closely-spaced leptons as subsequent treatment of reconstructed objects before analysis.  

\section{The Building Blocks}
\label{sec:obj:bb}
%1704.07983.pdf - 1
%Salzburger_2015_J._Phys.__Conf._Ser._664_072042.pdf - 2

%luminosities $2*10^{33}cm^{-2}s^{-1}$ in 2008 -> $10^{34}cm^{-2}s^{-1}$
Track reconstruction, also called \textit{tracking}, provides the important information needed for primary and secondary vertex reconstruction, charged particle reconstruction, jet flavor tagging, and photon conversions; therefore, track reconstruction algorithms must be swift, concise, and perform with high efficiency, low fake rates and with proper resolution on tracking parameters.  In 2015, at the start of Run 2, the LHC extended the center of mass energy in proton-proton collisions to 13 TeV, and over the duration of Run 2, ramped-up the instantaneous luminosity pushing the average interactions per bunch crossing ($\mu$) to above 40 by the end of Run 2.  This extension of center of mass energy and instantaneous luminosity enhances the outlook of discovery while simultaneously slowing down track reconstruction and degrading its efficiency.  Events with jet showers in the TeV range and $\tau$ leptons and B-hadrons that traverse multiple ID layers before decaying, occur at rates high enough to be considered in optimizing track and cluster reconstruction in Run 2 \cite{aad}.  In the core of boosted hadronic jets and $\tau$ lepton decays, particles in flight do not scatter much as they traverse the inner tracking layers, making separate energy deposits in the discrete sensors hard to resolve and near-by tracks hard to distinguish from each other.  If tracking efficiency is low in events with high track density, mismeasurements are expected in identifying long-lived b-hadron and hadron $\tau$ decays and in calibrating the energy and mass of jets.  These mismeasurements will also cause induced \met, which is an important quantity for this search and many other beyond standard model searches.  

The first step in track reconstruction involves preprocessing Pixel, SCT, and TRT information. Event by event charged track reconstruction in the pixel and SCT detectors starts with clustering groups of pixels and strips in each sensor that respond to an energy deposition above a set threshold and share a common edge or corner.  These clusters form three-dimensional space-points that measure where a particle intersects the active material in the ID.  In the pixel detector, each particle corresponds to one space-point, while in the SCT, clusters must be combined from both sides of a strip layer to obtain a three-dimensional position measurement.  %Charge in the pixel sensors is often collected in more than one adjoining pixels \textcolor{red}{Maybe connect here hit resolution to cluster size and use plots from the IBL paper.. remembering to explain how impact parameter resolution is driven by the resolution of the hit closest to the primary vertex.} 

The next step in tracking is called track finding.  This involves combining Pixel and SCT hits into tracks seeds.  Three consecutive hits are required for a track seed, and seeds with an additional compatible cluster are sent to a Kalman filter.  In the last step, hits from all three of the tracking detectors are fit to make tracks using a global $\chi^2$ function.  These tracks are then given a score based on the fit quality and the number of holes and shared clusters.  Tracks that fall below the minimum allowable score are rejected.


Reconstructed tracks are characterized using five \textit{perigee} parameters at the point of closest approach to the beam axis.  
\begin{itemize}
\item \textbf{\textit{transverse impact parameter}} $d_0$ - track distance to the $z$-axis at the point of closest approach in the $x-y$ plane.
\item \textbf{\textit{longitudinal impact parameter}} $z_0$ - track coordinate along $z$ at the point of closest approach.
\item \textbf{\textit{azimuthal angle}} $\phi_0\equiv\tan^{-1} p_y/p_x$ - track angle to the $x$-axis in the $x$-$y$ plane.
\item \textbf{\textit{polar angle}} $\theta_0$ - track angle to the $z$-axis in the $R-z$ plane.
\item \textbf{\textit{charge over momentum}} $q/p$ - electric charge divided by the track momentum.
\end{itemize}
  \begin{figure}[tbp]
       \includegraphics[width=0.8\columnwidth]{/Users/sheenaschier/Documents/LaFiles/figures/thesis/eventselection/TrkPerigee.png}\\
   \caption{Sketch of ATLAS tracking parameters at the perigee in the $x-y$ plane (left) and the $R-z$ plane (right). }
   \label{fig:trkParam}
 \end{figure}
 
The primary vertex is defined as space position in the detector of the initial pp interaction.  Primary vertices are identified using inner detector tracks that satisfy a set of requirements.  For a track to be considered in the construction of a primary vertex, it must have $ \pt{} > 400\GeV$, $|\eta| < 2.5$, between 9 ($|\eta| \leq 1.65$) and 11 ($|\eta| > 1.65$) silicon hits, at least 1 hit in the IBL or B-Layer, a maximum of one shared pixel hit or two shared SCT hits, no holes in the pixel layers, and no more that one hole in the SCT layers.  Any primary vertex must have at least two associated tracks for reconstruction~\cite{1742-6596-898-4-042056}.  %The primary vertex track selection criteria is summarized in Table~\ref{tab:pvtrk}.
\iffalse
\begin{table}
\tiny
\centering
\begin{tabular}{l|l}
%pg 2 Boutle
  \small Track Kinematics & Track Hit Criteria  \\
  \hline
  $\pt > 400~\MeV$ & \\
  $|d_0|<4~\mathrm{mm}$ & \\
  \hline
\end{tabular}
\caption{Summary of primary vertex track selection}
\label{tab:pvtrk}
\end{table}
\fi
\FloatBarrier

\section{Particle Identification and Reconstruction}
\label{sec:obj:reco}

Reconstructed and identified particles in ATLAS are leptons($e, \mu, \tau$), photons, hadronic jets, which can further be identified as b-jets, and missing transverse momentum \met.  This analysis does not use $\tau$ reconstruction.  There are two categories of reconstructed objects; \textit{baseline}, which is the most inclusive definition of an object and is typically used for preliminary event selection and background modeling, and \textit{signal}, a more exclusive object definition that is a subset of \textit{baseline} and is typically used in defining signal events.  A summary of all the signal and baseline object definitions is given in Table~\ref{tab:objdef}.

    \begin{figure}[tbp]
       \includegraphics[width=.8\columnwidth]{/Users/sheenaschier/Documents/LaFiles/figures/thesis/eventselection/idSketch.png}\\
   \caption{Schematic of an electron in the ATLAS}
   \label{fig:idsketch}
 \end{figure}
Electron likelihood identification is a multivariate technique that uses signal and background probability density functions of discriminating variables to give an overall likelihood of being signal or background.  Figure~\ref{fig:idsketch} depicts an electron in ATLAS moving through the ID detectors and into the calorimeters.  Likelihood variables that discriminate on the tracking include: number of hits on the inner-most pixel layer, hits in the Pixel detector, hits in the SCT+Pixel detectors, transverse impact parameter $d_0$, transverse impact parameter significance ($|d_0/\sigma_{d_0}$), and fractional momentum lost in the detector, likelihood probability based on the transition radiation in the TRT, and track-cluster matching variables.  Likelihood variables that discriminate on calorimeter measurements include: the ratio of transverse energy in the TileCal to the energy in the LAr, the ratio of energy in the last LAr layer to the energy in the full LAr\footnote{This variable is only used for electrons with $\pt<80\GeV$.}, the lateral electromagnetic shower shape in the second LAr layer, shower width in the LAr strip layers.  Signal and background probabilities combine into a single discriminant on which a cut is applied to define a likelihood-based operating point.  Operating points in the electron likelihood identification menu are \textit{VeryLoose, Loose, LooseAndBLayer, Medium}, \textit{Tight}.  \textit{LooseAndBLayer} uses the same likelihood as \textit{Loose} and also requires a hit in the inner-most Pixel layer.  All operating points use the same discriminating variables to ensure tighter operating points are subsets of the more loose operating points.  The electron efficiencies for the \textit{Loose}, \textit{Medium}, and \textit{Tight} LH working points are compared in Figure~\ref{fig:lepEff}.
\begin{figure}[h!]
 \centering
 \includegraphics[width=0.49\columnwidth]{/Users/sheenaschier/Documents/LaFiles/figures/thesis/fakes/fig_01.pdf}
  \includegraphics[width=0.49\columnwidth]{/Users/sheenaschier/Documents/LaFiles/figures/thesis/fakes/fig_02.pdf}
 \caption{Electron efficiency in $\eta$ (left) and \et (right) in the \textit{Loose}, \textit{Medium}, and \textit{Tight} LH identification algorithms}
 \label{fig:lepEff}
 \end{figure}
 
 Muons in this analysis use a cut-based identification technique that first identifies muon tracks in the ID and MS and combines them to form complete muon tracks.  Identification working points are provided based on the muon reconstruction efficiency and background rejection they provide.  Muon ID \textit{Medium} is the default working point used by physics analyses in ATLAS \cite{muonid}.  The \textit{Medium} ID achieves over $95\%$ muon efficiency for muons $4\GeV<\pt<20\GeV$, and over $60\%$ background rejection.  Muon identification efficiencies measured versus muon \pt by the Muon Combined Performance Group in ATLAS are shown in Figure~\ref{fig:emuon}. 
 \begin{figure}[h!]
 \centering
 \includegraphics[width=0.8\columnwidth]{/Users/sheenaschier/Documents/LaFiles/figures/thesis/eventselection/muonEff}
 \caption{Reconstruction efficiency for \textit{Medium} muon identification working point as a function of muon \pt, in the region $0.1<|\eta|<2.5$ \cite{muon}.}
 \label{fig:emuon}
 \end{figure}
  
Lepton isolation is quantified by two main variables, track isolation and calorimeter isolation.  Track isolation is determined by the transverse momenta of tracks in some cone around the track with a radius determined by the lepton \pt.  Calorimeter isolation is dictated by the sum of the transverse energy in the topological clusters (topo clusters), which are cell clusters seeded by calorimeter cells with energy more than four times greater than the noise threshold in the cell.  Topo clusters are then expanded to neighboring cells with energy more than twice above the noise threshold, and finally a last layer of excited calorimeter cells are added to the cluster.  To measure the isolation energy, the lepton energy in the isolation topo cluster is removed and the topo cluster is corrected for pileup and any lepton energy that was not subtracted away.  Final isolation cuts using the track and calorimeter based isolation variables are are classified as either \textit{fixed cut} or \textit{gradient}.  Fixed cut means the working point provides fixed efficiencies across the $\eta-\pt$ plane.  Gradient means the efficiencies are \pt-variant, but still flat in $\eta$.  Like isolation working points are provided for for three grades of isolation: \textit{Loose, Medium,} and \textit{Tight}, and can be based on track isolation, calorimeter isolation, or both.  The \textit{Tight} working points will provide the best rejection of backgrounds, but the lowest efficiencies.
 
Baseline electrons are seeded from energy deposits in the EM calorimeter and reconstructed with algorithms using EM calorimeter clusters that are matched to inner detector tracks.  Baseline electrons must pass \pt{} threshold of 4.5\GeV~and exclusively travel through the central detector region $|\eta | < 2.47$.  A longitudinal impact parameter requirement of $|z_0sin\theta| < 0.5~\mathrm{mm}$ is also applied.  This analysis uses likelihood based identification criteria only.  Baseline electrons are required to satisfy \textit{VeryLooseLLH} identification while signal electrons must pass \textit{Tight} identification plus \textit{GradientLoose} isolation criteria.  Signal electrons also require transverse impact parameter significance $|d_0/\sigma(d_0)| < 5$.  Electron energy deposits in the LAr are generally narrow in $\eta$ and $\phi$ and mostly concentrated in the first two sampling layers.
\begin{figure}[h!]
 \centering
 \includegraphics[width=0.8\columnwidth]{/Users/sheenaschier/Documents/LaFiles/figures/thesis/eventselection/figaux_11.pdf}
 \caption{Signal lepton efficiencies for electrons and muons, averaged over all Higgsino and slepton samples. Efficiencies are shown for leptons within detector acceptance, and with lepton \pt within a factor of 3 of $\Delta~m(\tilde l\tilde\chi_1^0)$ for sleptons samples or within a factor of 3 of $\Delta~m(\tilde \chi_2^0\tilde\chi_1^0)/2$ for Higgsino samples. Uncertainty bands represent the range of efficiencies observed across all signal samples for the given \pt bin.  The $\eta$ dependance consistent with values reported in ATLAS combined performance papers.}
 \label{fig:sigeff}
 \end{figure}
 
Muon information primarily comes from tracks in the muon spectrometer that are often matched charged tracks in the inner detector.  Baseline muons are reconstructed with algorithms that combine tracks from the inner detector and muon spectrometer to form muon candidates.  They must pass \pt{} threshold of 4 \GeV and be in fiducial region $|\eta | < 2.5$. Like with electrons, muon likelihood identification is used, and the discriminating variables are extended to include information from tracks in the muon spectrometer.  Baseline muons are also expected to satisfy \textit{Medium} identification standards  and have a transverse impact parameter $|z_0sin\theta| < 0.5~\mathrm{mm}$.  Signal muons must also satisfy \textit{FixedCutTightTrackOnly} isolation criteria and a transverse impact parameter significance of $|d_0/\sigma(d_0)| < 3$.

Lepton identification, isolation, impact parameter cuts, fiducial acceptance and \pt threshold all effect the lepton efficiencies and result in the efficiencies shown in Figure~\ref{fig:sigeff} that range from roughly $50\%$ for low \pt muons and up to $90\%$ for higher \pt.  For electrons the efficiencies are roughly $20\%$ for low \pt electrons, and increase up to $\sim65\%$.  This is the average over signal samples that fall within some range, where the most compressed signal samples used to evaluate the low \pt leptons and so on.

Baseline jets are built from locally-calibrated three-dimensional topologically clustered calorimeter cells.  Topological clustering here is the same as described in the discussion of lepton isolation.  Jets are constructed using anti-$K_t$ clustering algorithms~\cite{antikt} with radius parameter R = 0.4.   Baseline jets must pass \pt{} threshold of 20 \GeV and be in fiducial region $|\eta | < 4.5$.   Also, jets within $|\eta | < 2.5$ originating from b-hadrons are tagged with the 2-dimensional multivariate b-tagging algorithm \textit{MV2c10} with an 85\% working point.  Signal jets are further restricted to fiducial region $|\eta | < 2.8$, and pileup jets are removed using the jet vertex tagger (JVT) with \textit{Medium} working point efficiency applied to jets with $\pt>60\GeV$ and $|\eta|<2.4$. 

\begin{table}
\tiny
 \centering
  \begin{tabular}{l||c|c|c}
 \hline
\small Selection Criteria & \small \textbf{Electrons} & \small \textbf{Muons} & \small \textbf{Jets}  \\
 \hline
 \hline
\small \textbf{Baseline} &  & & \\ 
 \hline
\small Reco Algorithm &\small \textit{author 16 veto}  &&\\
\small Kinematic&\small $\pt{} > 4.5$ \GeV,  &\small $\pt{} > 4$ \GeV,  &\small $\pt{} > 20$ \GeV,\\
&\small $|\eta | < 2.47$&\small $|\eta | < 2.5$& $|\eta | < 4.5$\\
\small Impact Parameter &\small $|z_0sin\theta|< 0.5$ mm &\small $|z_0sin\theta|< 0.5$ mm &\\
& -- & -- &\\
\small Identification &\small \textit{VeryLooseLLH}  &\small \textit{Medium}  &                 \\
\small Isolation & --    & --  &   \\
\small Clustering & & &\small Anti-$K_t$ R = 0.4 EMTopo\\
\small Jet Vertex Tagging &&& -- \\
\small \textit{b}-tagging &&& -- \\
 \hline
 \hline
 \small \textbf{Signal} &  & \\ 
 \hline
 \small Reco Algorithm &\small \textit{author 16 veto}  &&\\
\small Kinematic&\small $\pt{} > 4.5 \GeV$, &\small $\pt{} > 4 \GeV$,  &\small $\pt{} > 30$ \GeV,\\
&\small $|\eta | < 2.47$&\small $|\eta | < 2.5$&\small $|\eta | < 2.8$\\
\small Impact Parameter &\small $|z_0sin\theta|< 0.5~mm$,&\small $|z_0sin\theta|< 0.5~mm$, &\\
&\small $|d_0/\sigma(d_0)|< 5$&\small $|d_0/\sigma(d_0)|< 3$&\\
\small Identification &\small \textit{Tight} &\small \textit{Medium}   &                 \\
\small Isolation &\small \textit{GradientLoose}     & \small \textit{FixedCutTightTrackOnly} &    \\
\small Clustering & & &\small Anti-$K_t$ R = 0.4 EMTopo\\
\small Jet Vertex Tagging &&&\small \textit{JVT Medium}\\
\small \textit{b}-tagging &&&\small $\pt{} > 20$, $|\eta | < 2.5$ \\
&&& \small MV2c10 FixedCutBeff 85\% \\
  \end{tabular}
  \caption{Summary of object definitions}
  \label{tab:objdef}
\end{table}

Well calibrated energy and momentum measurements of the directly observable objects is important for construction of the particles that traverse the detector without interacting.  These "missing" particles carry away energy and momentum which is recovered with energy and momentum conservation in the plane transverse to the beam pipe.  The vector quantity missing transverse momentum \pt{} is the negative vector sum of the transverse momentum of all the identified physics objects (electrons, muons, jets, photons) plus an additional soft term.  The scalar magnitude of the missing transverse momentum vector gives the missing transverse energy \met. The soft term is constructed from all the tracks not associated with any physics object, but are associated with the primary vertex.  Therefore, \met{} is adjusted for the best possible calibration of the jets and other identified physics objects and still independent of pileup in the soft term.  Pileup jets are removed with a jet vertexing technique that matches jets to primary vertices with track-vertex tagging.


\section{Special Treatment of Reconstructed Objects}
\label{sec:obj:treat}
Once objects are reconstructed and identified, special algorithms often need to be performed before these objects can be used.  For this analysis, these final steps were the removal of overlapping objects and the isolation correction of closely-spaced leptons.

Overlap removal is performed to prevent double counting of physics objects by removing objects based in their separation $\Delta R$ in detector coordinates $\eta$ and $\phi$, given by:
\begin{equation}
\Delta R_{p_1p_2} = \sqrt{(\eta_{p_1}-\eta_{p_2})^2+(\phi_{p_1}-\phi_{p_2})^2}
\end{equation} 
First, jet-electron overlap removal is performed.  If $\Delta R_{jet, electron}$ is less than 0.2 and the the jet is not tagged as a b-jet, the jet is removed and the electron is kept.  If the jet is identified as a b-jet, then the jet is kept and the electron object is removed since the electron is most likely from the semi-leptonic decay of a B-hadron.  If $\Delta R_{jet, electron}$ is less than 0.4, we remove the electron and keep the jet.  Similarly, if the $\Delta R_{jet, muon}$ is less than 0.4, we remove the muon and keep the jet unless the jet has less than three tracks; in which case, the muon will be kept and the jet is discarded.  Lastly, we perform overlap removal on photons and other objects.  It is common that electron and muon objects will also be included in the photon container since they pass the Ecal shower requirements, so typically, overlapping photons and leptons will result in the photon object being removed from the photon container.  If $\Delta R_{photon, electron}$ is less than 0.4 we remove the photon and keep the electron.  If $\Delta R_{photon, muon}$ is less than 0.4, we remove the photon and keep the muon.  If $\Delta R_{photon, jet}$ is less than 0.4, we keep the photon and remove the jet.  

 Soft leptons in a boosted system often have small angular separation, especially when they are products of a low-mass $Z^*$ decay.  These boosted leptons often lie within each others isolation cones, leading to efficiency loss for very small mass-splittings.  The top row of Figures~ \ref{fig:EffRll_ISOCorr} and ~\ref{fig:EffMll_ISOCorr} illustrate the efficiency loss for nearby leptons within $\Delta R < 0.4$ and dilepton invariant mass ($m_{\ell\ell}$) $<5\GeV$ using an electroweakino signal sample with $m_{\tilde\chi_2^0}-m_{\tilde\chi_1^0}=10\GeV$.  This loss is corrected for using a dedicated tool that checks for baseline leptons that fail the isolation criteria to due to another nearby lepton within it's isolation cone and removes tracks associated with the nearby lepton from the track isolation sum.  If the nearby lepton is an electron, the topocluster $E_T$ is also removed from the calorimeter isolation sum.  The corrected isolation variables are then reanalyzed using the original isolation working point.  The bottom rows of Figures~ \ref{fig:EffRll_ISOCorr} and ~\ref{fig:EffMll_ISOCorr} exhibits the recovered dilepton efficiency in simulation after applying the isolation correction tool.  Figure~\ref{fig:nearbylepiso} shows the effect of this correction on low invariant mass dilepton pairs in data.  The data are chosen such that $\Delta\phi(\met, p_{t}^{j_1})<1.5$ to avoid the signal region, which selects $\Delta\phi(\met, p_{t}^{j_1})>2.0$, as explained in Chapter~\ref{ch:sr}.   
  \begin{figure}[tbp]
   % \centering
     \includegraphics[width=0.48\columnwidth]{/Users/sheenaschier/Documents/LaFiles/figures/thesis/eventselection/eff_EE_Rll_110_100_NoOS_NoISO.pdf}
       \includegraphics[width=0.48\columnwidth]{/Users/sheenaschier/Documents/LaFiles/figures/thesis/eventselection/eff_MM_Rll_110_100_GradLoose_NoOS_NoISO.pdf}\\
     \includegraphics[width=0.48\columnwidth]{/Users/sheenaschier/Documents/LaFiles/figures/thesis/eventselection/eff_EE_Rll_110_100_NoOS.pdf}
     \includegraphics[width=0.48\columnwidth]{/Users/sheenaschier/Documents/LaFiles/figures/thesis/eventselection/eff_MM_Rll_110_100_GradLoose_NoOS.pdf}\\
   \caption{Dilepton $\Delta$ R distribution before LepIsoCorrection (top) and after LepIsoCorrection (bottom) for the $ee$-channel (left) and $\mu\mu$-channel (right), using electroweakino signal samples with m($\tilde\chi_2^0\tilde\chi_1^0$) $(110, 100)\GeV$.}
   \label{fig:EffRll_ISOCorr}
 \end{figure}

  \begin{figure}[tbp]
   % \centering
     \includegraphics[width=0.48\columnwidth]{/Users/sheenaschier/Documents/LaFiles/figures/thesis/eventselection/eff_EE_Mll_110_100_NoOS_NoISO.pdf}
       \includegraphics[width=0.48\columnwidth]{/Users/sheenaschier/Documents/LaFiles/figures/thesis/eventselection/eff_MM_Mll_110_100_GradLoose_NoOS_NoISO.pdf}\\
     \includegraphics[width=0.48\columnwidth]{/Users/sheenaschier/Documents/LaFiles/figures/thesis/eventselection/eff_EE_Mll_110_100_NoOS.pdf}
     \includegraphics[width=0.48\columnwidth]{/Users/sheenaschier/Documents/LaFiles/figures/thesis/eventselection/eff_MM_Mll_110_100_GradLoose_NoOS.pdf}\\
   \caption{Dilepton invarient mass distribution before LepIsoCorrection (top) and after LepIsoCorrection (bottom) for the $ee$-channel (left) and $\mu\mu$-channel (right), using electroweakino signal samples with m($\tilde\chi_2^0\tilde\chi_1^0$) $(110, 100)\GeV$.}
   \label{fig:EffMll_ISOCorr}
 \end{figure}
 \begin{figure}[tbp]
  \includegraphics[width=0.48\columnwidth,trim=1.2cm 0cm 1.9cm 0cm,clip]{/Users/sheenaschier/Documents/LaFiles/figures/thesis/nearbylepiso.pdf}
 %  \includegraphics[width=0.48\columnwidth]{/Users/sheenaschier/Documents/LaFiles/figures/thesis/nearbylepiso_signal.pdf}
  \caption{(left) Impact of the \texttt{NearbyLepIsoCorrection} tool on the efficiency of low-mass dilepton pairs in data.  The data are shown in a region with $\Delta\phi(\met, p_{t}^{j1})<1.5$ to avoid the signal region.  Events are triggered with the inclusive-\met{} trigger.  The red trend shows events with two baseline leptons without applying any isolation; the green shows the impact of applying \texttt{GradientLoose} isolation; the blue shows the result of the \texttt{NearbyLepIsoCorrection} applied to the \texttt{GradientLoose} sample.  %(right) Impact of the correction on a Higgsino LSP signal sample with $\Delta m(\chi,\chi)=3\GeV$.
  }
 \label{fig:nearbylepiso}
 \end{figure}
 

 
 


